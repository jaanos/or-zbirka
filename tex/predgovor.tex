Na pobudo izr.~prof.~dr.~Arjane Žitnik,
ki od študijskega leta 2017/18 predava predmet Operacijske raziskave
za študente 2.~letnika Finančne matematike,
sem kot asistent pri tem predmetu začel zbirati
naloge iz vaj, kolokvijev in izpitov ter njihove rešitve.
Pričujočo zbirko tako sestavlja \thetotal{} nalog,
ki sem jih ``podedoval'' od svojih predhodnikov,
in ki sem jih sam sestavil od študijskega leta 2016/17 naprej,
ko sem prevzel vodenje vaj pri tem predmetu.
Naloge so razporejene glede na princip, po katerih jih rešujemo,
kar ravno sledi snovi, kot se obravnava pri predmetu Operacijske raziskave
(pri čemer se v zadnjih letih ne pokrije nalog iz zadnjih dveh razdelkov).
Tako lahko študenti enostavno poiščejo naloge za snov, ki bi jo radi vadili.
S pregledovanjem nalog lahko bralec seveda tudi hitro tudi dobi občutek,
kateri način reševanja bi bil primeren za določen tip naloge.

Na tem mestu bi se rad zahvalil Arjani Žitnik za spodbudo
ter predhodnim predavateljem Vladimirju Batagelju, Sergiu Cabellu, Martinu Juvanu
in mojim predhodnikom na asistentskem mestu
Jerneju Azariji, Ninu Bašiću, Boštjanu Gabrovšku, Davidu Gajserju,
Blažu Jelencu, Matjažu Konvalinki, Alenu Orbaniću in Emilu Žagarju,
ki so sestavili marsikatero nalogo in tudi napisali nekatere rešitve.
Žal za vse naloge ni bilo mogoče določiti avtorja,
zato je ta informacija v sami zbirki izpuščena
-- so pa avtorji, kjer so ti znani ali vsaj verjetni,
zapisani v izvornih datotekah,
ki so javno dostopne na
\href{https://github.com/jaanos/or-zbirka}{repozitoriju zbirke}.
Pri vsaki nalogi je tako naveden vir
-- večinoma gre za vaje, kolokvije in izpite pri Operacijskih raziskavah,
nekaj nalog pa je tudi z vaj pri predmetu Optimizacijske metode
ter iz literature, ki je navedena na koncu zbirke.

Rad bi se zahvalil tudi študentom pri predmetu Operacijske raziskave,
ki so prispevali rešitve nalog in popravke,
in sicer Juretu Babniku, Evi Deželak, Lari Jagodnik, Maksu Perbilu, Tjaši Renko,
Mateju Rojcu, Vitu Rozmanu, Juretu Sternadu, Zali Stopar Špringer,
Nejcu Ševerkarju, Janu Šifrerju in Anji Trobec.
Upam,
da je bila in bo tako njim in njihovim sošolcem
kot tudi bodočim študentom Finančne matematike
ta zbirka v pomoč pri študiju
in morda tudi pri reševanju problemov v resničnem življenju.
