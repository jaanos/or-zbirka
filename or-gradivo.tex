\documentclass[a4paper, 11pt, titlepage]{article}
\usepackage[slovene]{babel}
\usepackage[utf8]{inputenc}
\usepackage[slovene]{babelbib}
\usepackage{xspace}
\usepackage{lmodern}
\usepackage[LSBC3,T1]{fontenc}
\usepackage{chessboard}
\usepackage{microtype}
\usepackage{chngcntr}
\usepackage{environ}
\usepackage{needspace}
\usepackage[small]{caption}
%%%%%%%%%%%%%%%%%%%%%%%%%%%%%%%%%%%%%%%%%%%%%%%%%%%%%%%%
\usepackage{amsmath,amssymb,amstext,wasysym,mathtools,mathdots}
\usepackage{algorithmicx,algpseudocode}
%%%%%%%%%%%%%%%%%%%%%%%%%%%%%%%%%%%%%%%%%%%%%%%%%%%%%%%%
\usepackage{tabularx}
\usepackage{enumerate}
\usepackage[colorlinks=true]{hyperref}
\usepackage{bookmark}
\usepackage{eurosym}
\usepackage{tikz}
\usetikzlibrary{shapes}
\usetikzlibrary{calc}
\usetikzlibrary{matrix}
\usetikzlibrary{backgrounds}
\usetikzlibrary{decorations.pathmorphing}

\usepackage{fourier}
\usepackage[oznake]{omrezja}
\newcommand{\A}{\ensuremath{\mathcal{A}}}
\newcommand{\N}{\ensuremath{\mathbb{N}}}
\newcommand{\R}{\ensuremath{\mathbb{R}}}
\newcommand{\Z}{\ensuremath{\mathbb{Z}}}
\newcommand{\p}{\ensuremath{\mathcal{P}}}
\newcommand{\set}[2]{\ensuremath{\left\{ #1 \; \middle| \; #2 \right\}}}
\newcommand{\mm}{\ensuremath{\phantom{|}-\phantom{|}}}
\newcommand{\pp}{\ensuremath{\phantom{|}+\phantom{|}}}
\newcommand{\EVPI}{\ensuremath{\operatorname{EVPI}}\xspace}
\newcommand{\EVE}{\ensuremath{\operatorname{EVE}}\xspace}
\newcommand{\length}{\ensuremath{\operatorname{length}}\xspace}
\newcommand{\reverse}{\ensuremath{\operatorname{reverse}}\xspace}
\newcommand{\reverseStart}{\ensuremath{\operatorname{reverseStart}}\xspace}
\newcommand{\add}{\ensuremath{\operatorname{add}}\xspace}
\newcommand{\append}{\ensuremath{\operatorname{append}}\xspace}
\newcommand{\pop}{\ensuremath{\operatorname{pop}}\xspace}
\newcommand{\push}{\ensuremath{\operatorname{push}}\xspace}
\newcommand{\remove}{\ensuremath{\operatorname{remove}}\xspace}
\newcommand{\isEmpty}{\ensuremath{\operatorname{isEmpty}}\xspace}
\newcommand{\visited}{\ensuremath{\text{\sl visited}}\xspace}
\newcommand{\postlabel}{\ensuremath{\text{\sl postlabel}}\xspace}
\newcommand{\pred}{\ensuremath{\text{\sl pred}}\xspace}
\newcommand{\True}{\ensuremath{\text{\sc True}}\xspace}
\newcommand{\False}{\ensuremath{\text{\sc False}}\xspace}
\newcommand{\Null}{\ensuremath{\text{\sc Null}}\xspace}
\newcommand{\NOp}{\ensuremath{\text{\sc NoOp}}\xspace}
\newcommand{\visit}{\ensuremath{\text{\sc Visit}}\xspace}
\newcommand{\previsit}{\ensuremath{\text{\sc Previsit}}\xspace}
\newcommand{\postvisit}{\ensuremath{\text{\sc Postvisit}}\xspace}
\newcommand{\Init}{\ensuremath{\text{\sc Init}}\xspace}
\newcommand{\AddVertex}{\ensuremath{\text{\sc AddVertex}}\xspace}
\newcommand{\AddEdge}{\ensuremath{\text{\sc AddEdge}}\xspace}
\newcommand{\DelVertex}{\ensuremath{\text{\sc DelVertex}}\xspace}
\newcommand{\DelEdge}{\ensuremath{\text{\sc DelEdge}}\xspace}
\newcommand{\Adj}{\ensuremath{\text{\sc Adj}}\xspace}
\newcommand{\PrioritetnaVrsta}{\ensuremath{\text{\sc PrioritetnaVrsta}}\xspace}
\renewcommand{\algorithmicrequire}{{\bf Vhod:}}
\DeclareMathOperator*{\argmax}{arg\,max}
\DeclareMathOperator*{\argmin}{arg\,min}
%%%%%%%%%%%%%%%%%%%%%%%%%%%%%%%%%%%%%%%%%%%%%%%%%%%%%%%%
\newcommand{\opis}[2][2]{\shortintertext{\small\needspace{#1\baselineskip}#2:}}
\newcommand{\namig}[1]{\\ {\small {\bf Namig:} #1}}
\newcommand{\opomba}[1]{\\ {\small {\bf Opomba:} #1}}
\newcommand{\primer}[1]{\\ {\small {\bf Primer:} #1}}
\newcommand{\opozorilo}[1]{\\ {\small {\bf Opozorilo:} #1}}
\newcommand{\nadaljevanje}[1]{{\em Nadaljevanje sledi v nalogi~\nal[#1].}}
\newcommand{\odstraniprostor}{\vspace{-1.5\baselineskip}}
%%%%%%%%%%%%%%%%%%%%%%%%%%%%%%%%%%%%%%%%%%%%%%%%%%%%%%%%
\DeclareUnicodeCharacter{20AC}{\euro}
\DeclareRobustCommand{\officialeuro}{%
  \ifmmode\expandafter\text\fi
  {\fontencoding{U}\fontfamily{eurosym}\selectfont e}}
\newcommand{\ME}{\ensuremath{\operatorname{M€}}}
\newcommand{\MD}{\ensuremath{\operatorname{M\$}}}
%%%%%%%%%%%%%%%%%%%%%%%%%%%%%%%%%%%%%%%%%%%%%%%%%%%%%%%%
%% https://tex.stackexchange.com/questions/51733/global-renewcommand-equivalent-of-global-def/51751#51751
\makeatletter
\def\gnewcommand{\g@star@or@long\new@command}
\def\grenewcommand{\g@star@or@long\renew@command}
\def\g@star@or@long#1{%
  \@ifstar{\let\l@ngrel@x\global#1}{\def\l@ngrel@x{\long\global}#1}}
\makeatother

\newcounter{naloga}
\counterwithin*{naloga}{subsection}
\renewcommand{\thenaloga}{\arabic{subsection}.\arabic{naloga}}
\newcommand{\hlabel}[2][\hlab]{\label{#1:#2}\hypertarget{#1.#2}{}}
\newenvironment{slika}[1][t!]{\setlabel{\thislabel}\begin{figure}[#1]\centering}{\hlabel[fig]{\curlabel}\end{figure}}
\newenvironment{tabela}[1][t!]{\setlabel{\thislabel}\begin{table}[#1]\centering}{\hlabel[tab]{\curlabel}\end{table}}
\newcommand{\setlabel}[1]{\grenewcommand{\curlabel}{\taoznaka{#1}}}
\newcommand{\pgfslika}[1][]{\setlabel{#1}\leavevmode\beginpgfgraphicnamed{pgf/\taoznaka{#1}}\input{slike/\taoznaka{#1}.tikz}\endpgfgraphicnamed}
\newcommand{\nastaviOznako}[1]{\if\relax\detokenize{#1}\relax\grenewcommand{\thislabel}{\thenaloga}\else\grenewcommand{\thislabel}{#1}\fi\grenewcommand{\curlabel}{\thislabel}}
\newcommand{\taoznaka}[1]{\if\relax\detokenize{#1}\relax\thislabel\else#1\fi}
\newcommand{\izpisiNalogo}[1]{\nastaviOznako{#1}\input{naloge/#1.tex}}

\newcommand{\taavtor}{}
\newcommand{\tanaslov}{}
\newcommand{\tavir}{}
\newcommand{\thislabel}{?}
\newcommand{\curlabel}{\thislabel}
\newcommand{\fig}[1][]{\ref{fig:\taoznaka{#1}}\xspace}
\newcommand{\tab}[1][]{\ref{tab:\taoznaka{#1}}\xspace}
\newcommand{\raznal}[1]{\ref{raznal:#1}\xspace}
\newcommand{\razres}[1]{\ref{razres:#1}\xspace}
\newcommand{\podnaslov}[2][]{\caption{#2 za nalogo~\if\relax\detokenize{#1}\relax\del{}\else#1\fi.}}

\allowdisplaybreaks

\pgfrealjobname{or-gradivo}
%%%%%%%%%%%%%%%%%%%%%%%%%%%%%%%%%%%%%%%%%%%%%%%%%%%%%%%%
\newcommand{\nal}[1][]{\ref{nal:\taoznaka{#1}}\xspace}
\newcommand{\res}[1][]{\ref{res:\taoznaka{#1}}\xspace}
\newcommand{\sirina}{\textwidth}

\begin{document}
\title{Rešene naloge pri predmetu \\
{\bf OPERACIJSKE RAZISKAVE} \\[1cm]
Študijsko gradivo za študente \\
2.~letnika Finančne matematike}
\author{\Large Janoš Vidali}
\date{\vfill Ljubljana, 2021}
\maketitle

\begin{description}
\itemsep -2pt
\item{NASLOV:} Rešene naloge pri predmetu OPERACIJSKE RAZISKAVE
\item{PODNASLOV:} Študijsko gradivo za študente 2.~letnika Finančne matematike
\item{AVTOR:} Janoš Vidali
\item{IZDAJA:} 1.~izdaja
\item{ZALOŽNIK:} samozaložba
\item{KRAJ:} Ljubljana
\item{LETO IZIDA:} 2021
\item{AVTORSKE PRAVICE:} Janoš Vidali
\item{CENA:} publikacija je brezplačna
\item{NATIS:} elektronsko gradivo, dostopno na naslovu \\ \url{https://jaanos.github.io/or-zbirka/pdf/or-gradivo.pdf}
\end{description}

To gradivo je ponujeno pod licenco
{\em Creative Commons Priznanje avtorstva -- Deljenje pod enakimi pogoji 4.0 Slovenija}
(ali novejšo različico).
To pomeni, da se tako besedilo, slike, grafi in druge sestavine dela
lahko prosto distribuirajo, reproducirajo, uporabljajo,
priobčujejo javnosti in predelujejo pod pogojem,
da se jasno in vidno navede avtorja in naslov tega dela
in da se v primeru spremembe, preoblikovanja
ali uporabe tega dela v svojem delu
lahko distribuira predelava le pod licenco, ki je enaka tej.
Podrobnosti licence so dostopne na spletni strani \url{https://creativecommons.si}
ali na Inštitutu za intelektualno lastnino, Streliška 1, 1000 Ljubljana.

\bigskip
\centerline{\href{http://creativecommons.org/licenses/by-sa/4.0/}{\includegraphics[width=3cm]{CC-BY-SA.png}}}

\vfill
\fbox{
\begin{minipage}{\textwidth}
Kataložni zapis o publikaciji (CIP) pripravili v Narodni in univerzitetni knjižnici v Ljubljani

\bigskip
COBISS.SI-ID 53559811

\bigskip
ISBN 978-961-07-0416-4 (pdf)
\end{minipage}
}


\clearpage

\setcounter{tocdepth}{2}
\pdfbookmark[1]{Kazalo}{toc}
\tableofcontents

\clearpage

\section*{Predgovor}
\addcontentsline{toc}{section}{Predgovor}

Na pobudo izr.~prof.~dr.~Arjane Žitnik,
ki od študijskega leta 2017/18 predava predmet Operacijske raziskave
za študente 2.~letnika Finančne matematike,
sem kot asistent pri tem predmetu začel zbirati
naloge iz vaj, kolokvijev in izpitov ter njihove rešitve.
Pričujočo zbirko sestavljajo predvsem naloge,
ki sem jih sam sestavil (ali dopolnil) od študijskega leta 2016/17 naprej,
ko sem prevzel vodenje vaj pri tem predmetu.
V pripravi je tudi obsežnejša zbirka, ki bo zajemala tudi naloge,
ki so se pojavile na vajah, kolokvijih in izpitih v prejšnjih letih.

Na tem mestu bi se rad zahvalil prof.~Arjani Žitnik za spodbudo,
mojemu predhodniku na asistentskem mestu Davidu Gajserju,
ki je sestavil nalogi~\nal[brezzobec] in~\nal[palacinke],
ter študentom, ki so prispevali nekatere rešitve,
in sicer Evi Deželak
(rešitve nalog~\res[dectree3], \res[bfs], \res[dijkstra], \res[dfs] in~\res[pot]),
Nejcu Ševerkarju (rešitvi nalog~\res[topo] in~\res[dag]),
Anji Trobec (rešitev naloge~\res[konferenca])
in Janu Šifrerju (popravek rešitve naloge~\res[topo]).
Omenim naj še, da sta nalogi~\nal[blago] in~\nal[dfs] vzeti iz vira~\cite{dpv},
ki je služil tudi kot navdih za nekatere naloge iz razdelka~\raznal{grafi}.


\clearpage

\newcommand{\razdelek}[2]{\novastran\subsection{#1}\hlabel[raz\hlab]{#2}\renewcommand{\novastran}{\cleardoublepage}}
\newenvironment{naloga}[3][]{\refstepcounter{naloga}\avtor{#2}\naslov{#1}\vir{#3}\needspace{3\baselineskip}\addcontentsline{toc}{subsubsection}{Naloga \thenaloga}\subsubsection*{}\odstraniprostor\noindent {\bf Naloga}}{}


\section{Naloge}
{
\newcommand{\del}{\nal}
\newcommand{\avtor}[1]{\renewcommand{\taavtor}{\if\relax\detokenize{#1}\relax\else\hfill {\small Avtor: #1}\fi\par}}
\newcommand{\naslov}[1]{\renewcommand{\tanaslov}{\if\relax\detokenize{#1}\relax\else\noindent {\small\bf (#1)}\par\fi}}
\newcommand{\vir}[1]{\renewcommand{\tavir}{\hfill {\small Vir: #1}\par}}
\newcommand{\novastran}{}
\newenvironment{vprasanje}{\label{nal:\thislabel}{\bf \res.}\tavir\tanaslov\smallskip\noindent\ignorespaces}{}
\NewEnviron{odgovor}{}

\razdelek{Zahtevnost algoritmov}

\izpisiNalogo{zrcaljenje}
\izpisiNalogo{stevec}
\izpisiNalogo{bubblesort}
\izpisiNalogo{mergesort}
\izpisiNalogo{razcep}
\izpisiNalogo{korenski}

\razdelek{Celoštevilsko linearno programiranje}{clp}

\izpisiNalogo{maxprirejanje}
\izpisiNalogo{asistenti}
\izpisiNalogo{opravila}
\izpisiNalogo{youtube}
\izpisiNalogo{sadje}
\izpisiNalogo{dodelitev}
\izpisiNalogo{linija}
\izpisiNalogo{nogomet}
\izpisiNalogo{turnir}
\izpisiNalogo{oblacila}
\izpisiNalogo{festival}
\izpisiNalogo{karantena}

\razdelek{Teorija odločanja}

\izpisiNalogo{avtobus}
\izpisiNalogo{konkurenca}
\izpisiNalogo{poker}
\izpisiNalogo{dectree2}
\izpisiNalogo{podjetnik}
\izpisiNalogo{vrac}
\izpisiNalogo{dectree3}
\izpisiNalogo{sok}
\izpisiNalogo{cokolada}
\izpisiNalogo{dectree4}
\izpisiNalogo{trajekt}

\razdelek{Dinamično programiranje}{dinamicno}

\izpisiNalogo{blago}
\izpisiNalogo{blago2}
\izpisiNalogo{strnjenprodukt}
\izpisiNalogo{aplikacija}
\izpisiNalogo{lobisti}
\izpisiNalogo{domine}
\izpisiNalogo{vlagatelj}
\izpisiNalogo{pocivalisca}
\izpisiNalogo{signal}
\izpisiNalogo{vodovod}
\izpisiNalogo{matskoki}
\izpisiNalogo{vlagatelj2}

\razdelek{Algoritmi na grafih}{grafi}

\izpisiNalogo{bfs}
\izpisiNalogo{dijkstra}
\izpisiNalogo{dfs}
\izpisiNalogo{bf}
\izpisiNalogo{topo}
\izpisiNalogo{zaklad}
\izpisiNalogo{pocitnice}
\izpisiNalogo{prerezna}
\izpisiNalogo{dag}
\izpisiNalogo{pot}
\izpisiNalogo{studij}
\izpisiNalogo{alternirajoca}
\izpisiNalogo{neodvisna}
\izpisiNalogo{operaterji}
\izpisiNalogo{stnajkrajsih}
\izpisiNalogo{jadrnica}

\razdelek{CPM/PERT}

\izpisiNalogo{brezzobec}
\izpisiNalogo{brezzobec-lp}
\izpisiNalogo{palacinke}
\izpisiNalogo{palacinke1}
\izpisiNalogo{palacinke2}
\izpisiNalogo{splet}
\izpisiNalogo{konferenca}
\izpisiNalogo{letalo}
\izpisiNalogo{naselje}
\izpisiNalogo{avtocesta}

\razdelek{Upravljanje zalog}

\izpisiNalogo{marta}
\izpisiNalogo{vulkanizer}
\izpisiNalogo{sedezne}
\izpisiNalogo{maske}


}

\clearpage

\section{Rešitve}
{
\newcommand{\del}{\res}
\newcommand{\avtor}[1]{}
\newcommand{\naslov}[1]{}
\newcommand{\vir}[1]{}
\newcommand{\novastran}{}
\newcommand{\hlab}{res}
\NewEnviron{vprasanje}{\hlabel{\thislabel}{\bf \nal.}}
\newenvironment{odgovor}{\noindent\ignorespaces}{}

\razdelek{Zahtevnost algoritmov}

\izpisiNalogo{zrcaljenje}
\izpisiNalogo{stevec}
\izpisiNalogo{bubblesort}
\izpisiNalogo{mergesort}
\izpisiNalogo{razcep}
\izpisiNalogo{korenski}

\razdelek{Celoštevilsko linearno programiranje}{clp}

\izpisiNalogo{maxprirejanje}
\izpisiNalogo{asistenti}
\izpisiNalogo{opravila}
\izpisiNalogo{youtube}
\izpisiNalogo{sadje}
\izpisiNalogo{dodelitev}
\izpisiNalogo{linija}
\izpisiNalogo{nogomet}
\izpisiNalogo{turnir}
\izpisiNalogo{oblacila}
\izpisiNalogo{festival}
\izpisiNalogo{karantena}

\razdelek{Teorija odločanja}

\izpisiNalogo{avtobus}
\izpisiNalogo{konkurenca}
\izpisiNalogo{poker}
\izpisiNalogo{dectree2}
\izpisiNalogo{podjetnik}
\izpisiNalogo{vrac}
\izpisiNalogo{dectree3}
\izpisiNalogo{sok}
\izpisiNalogo{cokolada}
\izpisiNalogo{dectree4}
\izpisiNalogo{trajekt}

\razdelek{Dinamično programiranje}{dinamicno}

\izpisiNalogo{blago}
\izpisiNalogo{blago2}
\izpisiNalogo{strnjenprodukt}
\izpisiNalogo{aplikacija}
\izpisiNalogo{lobisti}
\izpisiNalogo{domine}
\izpisiNalogo{vlagatelj}
\izpisiNalogo{pocivalisca}
\izpisiNalogo{signal}
\izpisiNalogo{vodovod}
\izpisiNalogo{matskoki}
\izpisiNalogo{vlagatelj2}

\razdelek{Algoritmi na grafih}{grafi}

\izpisiNalogo{bfs}
\izpisiNalogo{dijkstra}
\izpisiNalogo{dfs}
\izpisiNalogo{bf}
\izpisiNalogo{topo}
\izpisiNalogo{zaklad}
\izpisiNalogo{pocitnice}
\izpisiNalogo{prerezna}
\izpisiNalogo{dag}
\izpisiNalogo{pot}
\izpisiNalogo{studij}
\izpisiNalogo{alternirajoca}
\izpisiNalogo{neodvisna}
\izpisiNalogo{operaterji}
\izpisiNalogo{stnajkrajsih}
\izpisiNalogo{jadrnica}

\razdelek{CPM/PERT}

\izpisiNalogo{brezzobec}
\izpisiNalogo{brezzobec-lp}
\izpisiNalogo{palacinke}
\izpisiNalogo{palacinke1}
\izpisiNalogo{palacinke2}
\izpisiNalogo{splet}
\izpisiNalogo{konferenca}
\izpisiNalogo{letalo}
\izpisiNalogo{naselje}
\izpisiNalogo{avtocesta}

\razdelek{Upravljanje zalog}

\izpisiNalogo{marta}
\izpisiNalogo{vulkanizer}
\izpisiNalogo{sedezne}
\izpisiNalogo{maske}


}

\clearpage

\bibliographystyle{bababbrv-fl}
\bibliography{reference}
\addcontentsline{toc}{section}{Literatura}

\end{document}
