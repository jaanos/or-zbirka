\begin{naloga}{Jernej Azarija}{Izpit OR 24.5.2016}
\begin{vprasanje}
Kavarna {\em ma$\phi$ja} je razvila recept za novo torto.
Slaščičarna {\em sladkeoperacijskeraziskave}
je za eksluzivno pravico do recepta pripravljena plačati $15\,002 €$.
Če se kavarna {\em ma$\phi$ja} odloči za samostojno prodajo torte,
jih začetni vložek stane $10\,000 €$,
za vsako prodano torto pa zaslužijo $25 €$.
Po njihovi presoji
je verjetnost uspeha recepture (prodajo $100\,000$ tort) enaka $0.4$,
verjetnost propada (prodali bi zgolj $10\,000$ tort) pa $0.6$.
Kavarna se lahko odloči, da zaprosi za pomoč tudi pod\-jet\-je,
ki na osnovi degustacij torte sestavi mnenje o uspehu recepta.
Svetovanje jih stane $35\,000 €$,
natančnost svetovanja pa opisuje spodnja tabela.
\begin{center}
\begin{tabular}{c|cc}
$P(\text{mnenje svetovalca} \;|\;\ \text{uspeh recepta})$
& Recept uspe & Recept ne uspe \\ \hline
Ugodno   & ${3 \over 4}$ & ${1 \over 2}$ \\
Neugodno & ${1 \over 4}$ & ${1 \over 2}$
\end{tabular}
\end{center}
Kako naj se kavarna {\em ma$\phi$ja} odloči?
Zakaj?
\end{vprasanje}
\begin{odgovor}

Definiramo si spremenljivke, ki jih bomo potrebovali in izpišemo znane podatke.
\begin{itemize}
\item[] $X$ \dots zaslužek
\item[] $A$ \dots recept uspe
\item[] $B$ \dots prosijo za nasvet
\item[] $C$ \dots svetujejo ugodno
\item[] $D$ \dots sami proizvajajo
\end{itemize}
\begin{align*}
&P(A) = \frac{2}{5} && P(\neg A) = \frac{3}{5}\\
&P(C|A) = \frac{3}{4} & &P(C|\neg A) = \frac{1}{2}\\
&P(\neg C|A) = \frac{1}{4} & &P(\neg C| \neg A) = \frac{1}{2}
\end{align*}
Izračunajmo verjetnosti, ki jih bomo v nalogi potrebovali.
\begin{align*}
&P(C) = P(C|A) P(A) + P(C|\neg A) P(\neg A) =\frac{3}{4}\cdot \frac{2}{5} + \frac{1}{2}\cdot \frac{3}{5} = \frac{3}{5}\\
&P(A|C) = \frac{P(C|A)P(A)}{P(C)} = \frac{3}{4}\cdot \frac{2}{5}\cdot \frac{5}{3}=\frac{1}{2}\\
&P(\neg A|C) = \frac{P(C|\neg A)P(\neg A)}{P(C)} = \frac{1}{2}\cdot \frac{3}{5}\cdot \frac{5}{3} = \frac{1}{2}
\end{align*}

\begin{itemize}
\item[kavarna ne prosi za svetovanje]
Če se kavarna odloči za proizvodjo in recept uspe, zasluži $2\; 490 \; 000$, kar se zgodi z verjetnostji $0,4$. Sicer zasluži $240 \; 000$. Če se kavarna odloči, da bo recept raje prodala zasluži $15 \; 002$. Pričakovan zaslužek, če se tovarna odloči za proizvodnjo je
$$
E(X|D) = 0,4 \cdot 2 \;490 \;000 + 0,6 \cdot 240 \; 000 = 1\;140\;000
$$
Če pa se kavarna odloči, da bo recept prodala, zasluži $15 \; 002$. Ker je to manj, kot pa pričakovan zaslužek, če se odloči za proizvodnjo, bo kavarna v primeru, da ne prosi za nasvet odloči, da so začela sama proizvajati.
$$
E(X|\neg B) = 1\;140\;000
$$

\item[kavarna prosi za svetovanje]
\item[svetujejo ugodno]
Izračunajmo sedaj $E(X|B \text{ in } C)$. Če se ne odloči za proizvodnjo, je njen zaslužek $-19\;998$. Če se kavarna odloči za proizvodnjo in uspe, zasluži $2\;455000$. To se zgodi z verjetnostjo $P(A|C) = \frac{1}{2}$. Z enako verjetnostjo recept ne uspe in kavarna zasluži le $205\;000$. Pričakovan zaslužek je torej enak 
$$
E(X|B \text{ in } C) = \frac{1}{2} \cdot 2\;455\;000 + \frac{1}{2} \cdot 205\;000 = 1\;330\;000
$$
Kar je več, kot pa če tovarna ne proizvaja sama. Zato je $E(X|D \text{ in } B)= 1\;330\;000$

\item[ne svetujejo ugodno]
Zaslužki v tem primeru so enaki, kot če kavarna prosi za nasvet in svetujejo ugodno. Prav tako verjetnosti, ki smo jih izračunali zgoraj. Zato je $E(X|B \text{ in } \neg C) = 1\;330\;000$.

\end{itemize}
Sedaj lahko izračunamo pričakovan dobiček, če se kavarna odloči, da bo prosila za nasvet.
$$
E(X|B) = \frac{3}{5}\cdot  1\;330\;000 + \frac{2}{5} \cdot 1\;330\;000 = 1\;330\;000
$$

Vidimo, da velja
$$
 E(X|B) > E(X|\neg B) ,
$$
zato je za kavarno bolje, da prosi za nasvet in se ne glede na napoved odloči za proizvodnjo.
\end{odgovor}
\end{naloga}
