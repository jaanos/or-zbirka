\begin{naloga}{Jernej Azarija}{Izpit OR 24.5.2016}
\begin{vprasanje}
Kavarna {\em ma$\phi$ja} je razvila recept za novo torto.
Slaščičarna {\em sladkeoperacijskeraziskave}
je za eksluzivno pravico do recepta pripravljena plačati $15\,002 €$.
Če se kavarna {\em ma$\phi$ja} odloči za samostojno prodajo torte,
jih začetni vložek stane $10\,000 €$,
za vsako prodano torto pa zaslužijo $25 €$.
Po njihovi presoji
je verjetnost uspeha recepture (prodajo $100\,000$ tort) enaka $0.4$,
verjetnost propada (prodali bi zgolj $10\,000$ tort) pa $0.6$.
Kavarna se lahko odloči, da zaprosi za pomoč tudi pod\-jet\-je,
ki na osnovi degustacij torte sestavi mnenje o uspehu recepta.
Svetovanje jih stane $35\,000 €$,
natančnost svetovanja pa opisuje spodnja tabela.
\begin{center}
\begin{tabular}{c|cc}
$P(\text{mnenje svetovalca} \;|\;\ \text{uspeh recepta})$
& Recept uspe & Recept ne uspe \\ \hline
Ugodno   & ${3 \over 4}$ & ${1 \over 2}$ \\
Neugodno & ${1 \over 4}$ & ${1 \over 2}$
\end{tabular}
\end{center}
Kako naj se kavarna {\em ma$\phi$ja} odloči?
Zakaj?
\end{vprasanje}
\begin{odgovor}

    Definiramo si spremenljivke, ki jih bomo potrebovali in izpišemo znane podatke.
    \begin{align*}
    X &\dots \text{zaslužek}\\
    A &\dots \text{recept uspe}\\
    B &\dots \text{prosijo za nasvet}\\
    C &\dots \text{svetujejo ugodno}\\
    D &\dots \text{sami proizvajajo}
    \end{align*}
    \begin{align*}
    P(A) &= \frac{2}{5} & P(\neg A)& = \frac{3}{5}\\
    P(C|A)& = \frac{3}{4}  &P(C|\neg A) &= \frac{1}{2}\\
    P(\neg C|A) &= \frac{1}{4}  &P(\neg C| \neg A) &= \frac{1}{2}
    \end{align*}
    Izračunajmo verjetnosti in pričakovane vrednosti, ki jih nato vpišemo v odločitveno drevo.
    $$
    P(C) = P(C|A) P(A) + P(C|\neg A) P(\neg A) =\frac{3}{4}\cdot \frac{2}{5} + \frac{1}{2}\cdot \frac{3}{5} = \frac{3}{5}
    $$
    \begin{alignat*}{2}
    P(A|C) &= \frac{P(C|A)P(A)}{P(C)} &&= \frac{3}{4}\cdot \frac{2}{5}\cdot \frac{5}{3}=\frac{1}{2}\\
    P(\neg A|C) &= \frac{P(C|\neg A)P(\neg A)}{P(C)} &&= \frac{1}{2}\cdot \frac{3}{5}\cdot \frac{5}{3}= \frac{1}{2}
    \end{alignat*}
    
    \begin{align*}
    E(X \mid D \text{ in }\neg B \text{ in } A) &= 2\, 490\, 000 € & P(A \mid \neg B) &= 0.4\\
    E(X \mid D \text{ in } \neg B \text{ in } \neg A) &= 240\, 000 € & P(\neg A \mid B)& = 0.6 \\
    E(X \mid \neg D \text{ in } \neg B) &= 15 \, 002 €
    \end{align*}
    
    \begin{align*}
    E(X \mid B \text{ in }C \text{ in } A) &=2\,455\, 000 € & P(A \mid B \text{ in } C) &= 0.5\\
    E(X \mid B \text{ in }C \text{ in }\neg A) &=205\, 000 € & P(\neg A \mid B \text{ in } C) &= 0.5\\
    E(X \mid \neg D \text{ in } B \text{ in } C) &= -19\, 998 €
    \end{align*}
    
    \begin{align*}
    E(X \mid B \text{ in } \neg C \text{ in } A) &=2\, 455\, 000 € & P(A \mid B \text{ in } \neg C) &= 0.5\\
    E(X \mid B \text{ in } \neg C \text{ in }\neg A) &=205\, 000 € & P(\neg A \mid B \text{ in } \neg C) &= 0.5\\
    E(X \mid \neg D \text{ in } B \text{ in } \neg C) &= -19\, 998 €
    \end{align*}
    
    %%%%%%odločitveno drevo
    
    Ko s pomočjo odločitvenega drevesa s slike~\fig izračunamo preostale pričakovane vrednosti, lahko vidimo, da
    $$
     E(X|B) > E(X|\neg B) ,
    $$
    zato je za kavarno bolje, da prosi za nasvet in se ne glede na napoved odloči za proizvodnjo.

    \begin{slika}
        \makebox[\textwidth][c]{
            \pgfslika
        }
        \podnaslov{Odločitveno drevo}
    \end{slika}
\end{odgovor}
\end{naloga}
