\begin{naloga}{Gabrovšek, Konvalinka}{Kolokvij OR 25.11.2010}
\begin{vprasanje}
Nov in neizkušen predavatelj Operacijskih raziskav se odloči,
da bo poenostavil Ford-Fulkersonov algoritem.
Študentom pove,
naj za iskanje maksimalnega pretoka na omrežju s končnimi kapacitetami
uporabijo naslednji algoritem:
\begin{small}
\begin{algorithmic}
\Function{Greedyflow}{$G = (V, E), C, s, t$}
    \State $F \gets$ slovar z vrednostjo $0$ za vsako povezavo iz $E$
    \For{$e \in E$}
        \State $F[e] \gets 0$
    \EndFor
    \While{obstaja pot od $s$ do $t$}
        \State $P \gets$ poljubna pot od $s$ do $t$
        \State $\gamma \gets$ minimalna kapaciteta povezav poti $P$
        \For{$e \in P$}
            \State $F[e] \gets F[e] + \gamma$
            \If{$C[e] = \gamma$}
                \State odstrani $e$ iz $G$
            \Else
                \State $C[e] \gets C[e] - \gamma$
            \EndIf
        \EndFor
    \EndWhile
    \State \Return $F$
\EndFunction
\end{algorithmic}
\end{small}

\begin{enumerate}[(a)]
\item Dokaži,
da lahko na grafu s slike~\fig
z uporabo algoritma {\sc Greedyflow} dobimo pravilno rešitev.

\item Poišči primer omrežja in izbire poti,
tako da algoritem {\sc Greedyflow} ne da pravilnega rezultata.

\item Ali za vsako omrežje obstaja taka izbira poti,
da bo algoritem {\sc Greedyflow} dal maksimalni pretok?

\item Dokaži, da je razlika med maksimalnim pretokom in pretokom,
ki ga vrne algoritem {\sc Greedyflow}, lahko poljubno velika.
\end{enumerate}

\begin{slika}
\pgfslika
\podnaslov[\nal{}(a)]{Graf}
\end{slika}
\end{vprasanje}

\begin{odgovor}
\begin{enumerate}[(a)]
\item Pravilno rešitev dobimo,
če algoritem zaporedoma izbere povečujoče poti $s \to u \to y \to t$
(pretok $3$, odstranita se povezavi $u \to y$ in $y \to t$),
$s \to u \to x \to t$ (pretok $3$, odstrani se povezava $u \to x$
in $s \to v \to z \to t$ (pretok $2$, odstrani se povezava $s \to v$).
Dobimo maksimalni pretok $3+3+2 = 8$, glej sliko~\fig[greedyflow-resitev].

\item Na sliki~\fig[greedyflow-primer]
je prikazano omrežje z maksimalnim pretokom $1+1 = 2$.
Če algoritem {\sc Greedyflow} na tem omrežju
v prvem koraku izbere pot $s \to a \to b \to t$ z minimalno kapaciteto $1$,
potem vse povezave iz te poti odstrani in tako ne najde druge povečujoče poti.

\item Za vsako omrežje obstaja izbira poti,
da bo algoritem {\sc Greedyflow} dal maksimalni pretok.
Zapišimo postopek, ki za dani graf $G = (V, E)$,
slovar kapacitet povezav $C$ in vozlišči $s, t \in V$
vrne ustrezno zaporedje poti skupaj s povečanjem pretoka,
ki ga prinese posamezna pot.
V postopku bomo uporabili dvostrano vrsto ({\em double-ended queue}),
na kateri bomo uporabili metode $\append(x)$ (dodajanje $x$ na konec),
$\push(x)$ (dodajanje $x$ na začetek) in $\pop()$ (odstranjevanje z začetka).

\begin{small}
\begin{algorithmic}
\Function{ZaporedjePoti}{$G = (V, E), C, s, t$}
    \State $Q \gets$ prazna dvostrana vrsta
    \If{$s = t$} \hfill če je izvor enak ponoru,
        \State $Q.\append((s, \infty))$
            \hfill lahko prepeljemo poljubno količino
        \State \Return $Q$
    \EndIf
    \State $M \gets$
        minimalni $(s, t)$-prerez omrežja $G$ s kapacitetami povezav $C$
    \State $S, T \gets$ množici z
        $V = S \cup T$, $S \cap T = \emptyset$, $s \in S$ in $t \in T$,
        določeni s prerezom $M$
    \State $G_s, G_t \gets$ podgrafa grafa $G$,
        inducirana z množico vozlišč $S$ oziroma $T$
    \For{$e = u \to v \in M$}
        \State $Q_s \gets \text{\sc{ZaporedjePoti}}(G_s, C, s, u)$
            \hfill rekurzivna klica dobita kopijo $C$,
        \State $Q_t \gets \text{\sc{ZaporedjePoti}}(G_t, C, v, t)$
            \hfill tako da ostane slovar nespremenjen
        \While{$C[e] > 0$}
            \State $P_s, \gamma_s \gets Q_s.\pop()$
            \State $P_t, \gamma_t \gets Q_t.\pop()$
            \State $\gamma \gets \min\{\gamma_s, c[e], \gamma_t\}$
            \For{$e_s \in P_s$} \hfill popravimo povezave v $G_s$
                \If{$C[e_s] = \gamma$}
                    \State odstrani $e_s$ iz $G_s$
                \Else
                    \State $C[e_s] \gets C[e_s] - \gamma$
                \EndIf
            \EndFor
            \State $C[e] \gets C[e] - \gamma$
            \For{$e_t \in P_t$} \hfill popravimo povezave v $G_t$
                \If{$C[e_t] = \gamma$}
                    \State odstrani $e_t$ iz $G_t$
                \Else
                    \State $C[e_t] \gets C[e_t] - \gamma$
                \EndIf
            \EndFor
            \State $Q.\append((P_s \| e \| P_t, \gamma))$
                \hfill dodamo novo pot v $G$
            \If{$\gamma < \gamma_s$}
                    \hfill če nobene povezave v $P_s$ ne zasitimo,
                \State $Q_s.\push((P_s, \gamma_s - \gamma))$
                    \hfill jo vrnemo na začetek vrste
            \EndIf \hfill z zmanjšano kapaciteto
            \If{$\gamma < \gamma_t$} \hfill podobno še za $P_t$
                \State $Q_t.\push((P_t, \gamma_t - \gamma))$
            \EndIf
        \EndWhile
    \EndFor
    \State \Return $Q$ \hfill vrnemo vrsto poti v $G$
\EndFunction
\end{algorithmic}
\end{small}
Algoritem deluje rekurzivno,
pri čemer si pomaga z izračunom minimalnega prereza omrežja,
ki tako graf razdeli na dva dela.
Za vsako povezavo $e = u \to v$ iz prereza
se tako izračunata ustrezni zaporedji poti (kot vrsti)
za pretok od $s$ do $u$ oziroma od $v$ do $t$ v ustreznem podomrežju.
Zaporedje poti za omrežje $G$ se potem dopolnjuje tako,
da se preko povezave $e$ združita poti z vrha obeh vrst,
pri čemer se lahko pot vrne na vrh vrste,
če se nobena povezava ne zasiti;
ko se pa katera od povezav zasiti, se odstrani iz ustreznega grafa.
Ker tak postopek posnema obnašanje algoritma {\sc Greedyflow},
se nikoli ne zgodi, da bi sestavili pot,
ki vsebuje katero že odstranjeno povezavo.
Obravnava povezave $e$ se zaključi, ko se ta zasiti
-- tedaj vrsti poti v ustreznih podomrežjih nista nujno prazni,
se pa ob predpostavki,
da rekurzivni klici vračajo ustrez\-na zaporedja,
tudi predhodno ne izpraznijo.
Postopek se nato nadaljuje z naslednjo povezavo iz $M$.
Ker se zaporedji poti za podomrežji ponovno izračunata,
najdene poti ne bodo vsebovale predhodno odstranjenih povezav.
Do robnega primera pride, ko je izvor enak ponoru
-- to se zgodi v rekurzivnih klicih,
ki obravnavajo povezavo iz minimalnega prereza s krajiščem $s$ oziroma $t$.
Ker je v tem primeru zaporedje poti ustrezno,
lahko po indukciji sklepamo, da algoritem deluje pravilno.

\item Naj bo $G_n = (V_n, E_n)$ $(n \ge 2)$ omrežje, kjer je
\begin{align*}
V_n &= \{s, t\} \cup \set{a_i, b_i}{1 \le i \le n} \quad \text{in} \\
E_n &= \set{s \to a_i, a_i \to b_i, b_i \to t}{1 \le i \le n}
  \cup \set{b_i \to a_{i+1}}{1 \le i \le n-1},
\end{align*}
pri čemer imajo vse povezave kapaciteto $1$.
Maksimalen pretok $n$ dobimo tako,
da peljemo količino $1$ po poteh $s \to a_i \to b_i \to t$ ($1 \le i \le n$)
-- minimalni prerez tedaj sestoji iz povezav $s \to a_i$ ($1 \le i \le n$).
Če pa v prvem koraku algoritma {\sc Greedyflow} izberemo pot
$$
s \to a_1 \to b_1 \to a_2 \to b_2 \to \cdots \to a_n \to b_n \to t,
$$
postanejo vse povezave v tej poti zasičene in se zato odstranijo iz grafa,
zaradi česar algoritem druge povečujoče poti ne najde
in tako konča s pretokom $1$.
Tako dobimo razliko $n-1$ in razmerje $n$ za poljubno velik $n$.
\end{enumerate}

\begin{slika}
\pgfslika[greedyflow-resitev]
\podnaslov[\res{}(a)]{Maksimalni pretok in minimalni prerez}
\end{slika}
\begin{slika}
\pgfslika[greedyflow-primer]
\podnaslov[\res{}(b)]{Primer omrežja in povečujoče poti}
\end{slika}
\end{odgovor}
\end{naloga}
