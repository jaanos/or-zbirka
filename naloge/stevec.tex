\begin{naloga}{Janoš Vidali}{Vaje OR 21.2.2018}
\begin{vprasanje}
Naj bo $\ell[1 \dots n]$ seznam,
ki ima na začetku vse vrednosti nastavljene na $0$.
Dan je spodnji program:

\begin{small}
\begin{algorithmic}
\State $i \gets 1$
\While{$i \le n$}
    \State $\ell[i] \gets 1 - \ell[i]$
    \If{$\ell[i] = 1$}
        \State $i \gets 1$
    \Else
        \State $i \gets i+1$
    \EndIf
\EndWhile
\end{algorithmic}
\end{small}

\begin{enumerate}[(a)]
\item Kaj se dogaja, ko teče zgornji program?
\item Oceni število korakov, ki jih opravi zgornji program,
v odvisnosti od parametra $n$.
\end{enumerate}
\end{vprasanje}

\begin{odgovor}
\begin{enumerate}[(a)]
\item V vsakem obhodu zanke {\bf while}
se vrednost $\ell[i]$ spremeni iz $0$ v $1$ ali obratno.
Če se vrednost spremeni na $1$, se $i$ nastavi na $1$,
sicer se pa poveča za $1$.

Izpišimo si vrednosti v seznamu $\ell$ in spremenljivke $i$
ob koncu vsakega obhoda zanke {\bf while} tekom izvajanja algoritma,
recimo za $n = 4$:
$$
\begin{array}{c|c|ccc|c|c}
\text{obhod} & \ell[4 \dots 1] & i &\qquad&
\text{obhod} & \ell[4 \dots 1] & i \\ \cline{1-3} \cline{5-7}
 1 & 0001 & 1 && 16 & 1001 & 1 \\
 2 & 0000 & 2 && 17 & 1000 & 2 \\
 3 & 0010 & 1 && 18 & 1010 & 1 \\
 4 & 0011 & 1 && 19 & 1011 & 1 \\
 5 & 0010 & 2 && 20 & 1010 & 2 \\
 6 & 0000 & 3 && 21 & 1000 & 3 \\
 7 & 0100 & 1 && 22 & 1100 & 1 \\
 8 & 0101 & 1 && 23 & 1101 & 1 \\
 9 & 0100 & 2 && 24 & 1100 & 2 \\
10 & 0110 & 1 && 25 & 1110 & 1 \\
11 & 0111 & 1 && 26 & 1111 & 1 \\
12 & 0110 & 2 && 27 & 1110 & 2 \\
13 & 0100 & 3 && 28 & 1100 & 3 \\
14 & 0000 & 4 && 29 & 1000 & 4 \\
15 & 1000 & 1 && 30 & 0000 & 5 \\
\end{array}
$$
Če pogledamo samo tiste obhode, na koncu katerih velja $i = 1$, opazimo,
da vrednosti v seznamu $\ell$ predstavljajo
dvojiške zapise števil od $1$ do $2^n - 1$
v ostalih pa se najmanj pomembna $1$ zamenja z $0$.
Algoritem torej simulira dvojiški števec z $n$ mesti.

\item Algoritem obišče vseh $2^n - 1$ števil,
poleg tega pa mora vsakič poskrbeti za zamenjavo vseh enic
za najmanj pomembno ničlo.
Ob upoštevanju, da obstaja $2^{n-i}$ števil
z najmanj pomembno ničlo na $i$-tem mestu,
za vrednost $2^n - 1$ pa je potrebno nadomestiti vseh $n$ mest,
je skupno število korakov enako
\begin{multline*}
n + \sum_{i=1}^n (i \cdot 2^{n-i}) =
\sum_{j=1}^n \left(1 + \sum_{i=j}^n 2^{n-i}\right) = \\
= \sum_{j=1}^n \left(1 + \sum_{i=0}^{n-j} 2^i\right) =
\sum_{j=1}^n 2^{n-j+1} = 2^{n+1} - 2 .
\end{multline*}
Časovna zahtevnost algoritma je torej $O(2^n)$.
\end{enumerate}
\end{odgovor}
\end{naloga}
