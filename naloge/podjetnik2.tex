\begin{naloga}{Janoš Vidali}{Kolokvij OR 2.6.2022}
\begin{vprasanje}
Podjetnik se odloča, ali naj začne z novim projektom.
Če se zanj odloči, bo moral vložiti $20\,000 €$.
Podjetnik ocenjuje,
da bo projekt z verjetnostjo $0.4$ uspešen
in mu bo prinesel $50\,000 €$ zaslužka
(končni dobiček bo tedaj torej $30\,000 €$),
z verjetnostjo $0.6$ pa projekt ne bo uspešen
in mu bo prinesel le $10\,000 €$ zaslužka
(v tem primeru bo torej na koncu imel $10\,000 €$ izgube).

Podjetnik lahko za pomoč prosi vlagatelja,
ki za svoje mnenje pričakuje plačilo $1\,000 €$
in mož\-nost partnerstva pri projektu.
Vlagatelj se lahko odloči odstopiti od sodelovanja
(tedaj podjetniku vrne plačanih $1\,000 €$),
lahko pa si vzame čas, da preuči ponudbo,
in bodisi podjetniku odsvetuje, da bi se lotil projekta,
ali pa sprejme partnerstvo, kar pomeni,
da bo vlagatelj sam pokril $10\%$ začetnega vložka,
povrnjen pa mu bo prav takšen delež zaslužka od projekta
(v slednjem primeru je podjetnik zavezan k temu, da začne s projektom).
Z analizo predhodnih dejanj vlagatelja
je podjetnik izračunal sledeče pogojne verjetnosti:
\begin{center}
\begin{tabular}{c|cc}
$P(\text{vlagateljeva odločitev} \mid \text{uspeh projekta})$
& uspešen & neuspešen \\ \hline
partnerstvo & $0.6$ & $0.1$ \\
odstop & $0.3$ & $0.4$ \\
odsvetuje & $0.1$ & $0.5$
\end{tabular}
\end{center}
Kako naj podjetnik ravna, da bo njegov pričakovani končni dobiček čim večji?
Nariši odločitveno drevo in ga uporabi pri sprejemanju odločitev.
Izračunaj tudi pričakovani dobiček.
\end{vprasanje}

\begin{odgovor}
Če se podjetnik odloči, da bo sam začel s projektom,
bo pričakovani dobiček enak
$$
-20\,000 € + 0.4 \cdot 50\,000 € + 0.6 \cdot 10\,000 € = 6\,000 €.
$$
Brez pomoči vlagatelja
se podjetniku torej izplača začeti s projektom.

Izračunajmo še verjetnosti za vlagateljevo odločitev
ter uspeh projekta glede na slednjo.
\begin{align*}
P(\text{partnerstvo}) &= 0.6 \cdot 0.4 + 0.1 \cdot 0.6 = 0.30 \\
P(\text{odstop}) &= 0.3 \cdot 0.4 + 0.4 \cdot 0.6 = 0.36 \\
P(\text{odsvetuje}) &= 0.1 \cdot 0.4 + 0.5 \cdot 0.6 = 0.34 \\
P(\text{projekt uspešen} \mid \text{partnerstvo})
&= {0.6 \cdot 0.4 \over 0.30} = {4 \over 5} \\
P(\text{projekt neuspešen} \mid \text{partnerstvo})
&= {0.1 \cdot 0.6 \over 0.30} = {1 \over 5} \\
P(\text{projekt uspešen} \mid \text{odstop})
&= {0.3 \cdot 0.4 \over 0.36} = {1 \over 3} \\
P(\text{projekt neuspešen} \mid \text{odstop})
&= {0.4 \cdot 0.6 \over 0.36} = {2 \over 3} \\
P(\text{projekt uspešen} \mid \text{odsvetuje})
&= {0.1 \cdot 0.4 \over 0.34} = {2 \over 17} \\
P(\text{projekt neuspešen} \mid \text{odsvetuje})
&= {0.5 \cdot 0.6 \over 0.34} = {15 \over 17}
\end{align*}
S pomočjo zgoraj izračunanih verjetnosti
lahko narišemo odločitveno drevo s slike~\fig.
Opazimo,
da je pričakovani dobiček ob pomoči vlagatelja enak $6\,500 €$,
zato naj podjetnik zaprosi za pomoč vlagatelja
-- če se slednji odloči za partnerstvo, potem bosta projekt morala izpeljati,
če vlagatelj odstopi, naj podjetnik sam izpelje projekt,
če pa vlagatelj podjetniku odsvetuje, da bi se lotil projekta,
pa naj ga upošteva in ne začne s projektom.

\begin{slika}
\makebox[\textwidth][c]{
\pgfslika
}
\podnaslov{Odločitveno drevo}
\end{slika}\end{odgovor}
\end{naloga}
