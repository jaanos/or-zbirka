\begin{naloga}{Janoš Vidali}{Kolokvij OR 22.4.2024}
\begin{vprasanje}
Barbara razmišlja,
da bi stavila $100€$ na zmago svoje priljubljene športne ekipe.
Na spletni strani stavnice vidi, da je kvota za njeno stavo enaka $2$,
kar pomeni, da bi v primeru, da bi stavo dobila, dobila izplačanih $200€$
(njen dobiček je tedaj $100€$),
v nasprotnem primeru pa nič (tedaj torej izgubi $100 €$).
Lahko se seveda tudi odloči, da stave ne vplača
-- v tem primeru seveda nima dobička ali izgube.
Dobro je znano, da ima stavnica v povprečju $10\%$ dobiček od vplačanih stav.
Iz tega lahko Barbara izračuna verjetnost, da bi svojo stavo dobila.

Barbara se lahko tudi odloči za mnenje povprašati svojega prijatelja Jureta,
ki lahko poda ugod\-no ali neugodno mnenje glede želene stave.
V primeru ugodnega mnenja Jure zahteva $20\%$ morebitnega dobitka od stave
(v našem primeru torej $40€$, tako da ima Barbara tedaj $60€$ dobička),
v primeru neugodnega mnenja pa ne zahteva ničesar
(tedaj se lahko Barbara še vedno odloči, ali bo stavila ali ne)\footnote{%
Lahko predpostaviš, da Barbara plača Juretu,
da po svoji presoji in njenih navodilih vplača stavo ali ji vrne denar,
svojo odločitev pa ji sporoči šele po izteku roka za vplačilo stav.
Tako Barbara ne bo mogla uporabiti morebitnega Juretovega ugodnega mnenja,
da bi še sama stavila mimo njega in tako morda povečala svoj dobiček.
}.
Znane so sledeče pogojne verjetnosti
$P(\text{Juretovo mnenje} \mid \text{stava bo dobila})$:
\begin{center}
\begin{tabular}{c|cc}
& dobi & izgubi \\ \hline
ugodno mnenje & 0.5 & 0.1 \\
neugodno mnenje & 0.5 & 0.9
\end{tabular}
\end{center}
Kako naj se Barbara odloči?
Nariši odločitveno drevo in odločitve sprejmi na podlagi izračunanih verjetnosti
ter izračunaj pričakovani dobiček.
\end{vprasanje}

\begin{odgovor}
Ker ima stavnica $10\%$ dobiček od vplačanih stav, Barbara pričakuje,
da bo ob vplačani stavi imela izgubo $100 € \cdot 0.1 = 10 €$.
Brez Juretovega mnenja se ji torej ne bo izplačalo staviti.

Naj bo $p$ verjetnost, da bi Barbarina stava dobila.
Velja torej
$$
p \cdot 200 € - 100 € = -10 €,
$$
iz česar lahko izračunamo $p = {9 \over 20} = 0.45$.
Izračunajmo še verjetnosti za Juretovo mnenje
ter uspeh stave glede na njega.
\needspace{2\baselineskip}
\begin{align*}
P(\text{ugodno}) &= 0.5 \cdot 0.45 + 0.1 \cdot 0.55 = 0.28 \\
P(\text{neugodno}) &= 0.5 \cdot 0.45 + 0.9 \cdot 0.55 = 0.72 \\
P(\text{stava dobi} \mid \text{ugodno})
&= {0.5 \cdot 0.45 \over 0.28} \approx 0.8036 \\
P(\text{stava izgubi} \mid \text{ugodno})
&= {0.1 \cdot 0.55 \over 0.28} \approx 0.1964 \\
P(\text{stava dobi} \mid \text{neugodno})
&= {0.5 \cdot 0.45 \over 0.72} = 0.3125 \\
P(\text{stava izgubi} \mid \text{neugodno})
&= {0.9 \cdot 0.55 \over 0.72} = 0.6875
\end{align*}
S pomočjo zgoraj izračunanih verjetnosti
lahko narišemo odločitveno drevo s slike~\fig.
Opazimo,
da je pričakovani dobiček, če vpraša Jureta za mnenje, enak $8 €$,
zato naj se Barbara torej odloči za to možnost.
V primeru, da je mnenje neugodno, naj Barbara ne stavi.

\begin{slika}
\makebox[\textwidth][c]{
\pgfslika
}
\podnaslov{Odločitveno drevo}
\end{slika}\end{odgovor}
\end{naloga}
