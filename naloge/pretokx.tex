\begin{naloga}{Alen Orbanić}{Izpit OR 28.6.2010}
\begin{vprasanje}
Na sliki~\fig je podano enoparametrično omrežje
za problem največjega pretoka.
Parameter $x$ je poljubno nenegativno število.

\begin{enumerate}[(a)]
\item Kakšno je pridruženo omrežje,
če začnemo z ničelnim $(s, t)$-tokom
in ga najprej povečamo vzdolž povečujoče poti $s \to a \to c \to d \to t$
ter nato še vzdolž povečujoče poti $s \to c \to d \to b \to t$?

\item Poišči maksimalni $(s, t)$-tok in minimalni $(s, t)$-prerez.

\item Kapaciteto ene povezave vhodnega omrežja lahko povečamo za $1$.
Kateri povezavi naj povečamo kapaciteto,
da bo vrednost maksimalnega $(s, t)$-toka v spremenjenem omrežju čim večja?
Odgovor je načeloma odvisen od vred\-no\-sti parametra $x$.
\end{enumerate}

\begin{slika}
\pgfslika
\podnaslov{Omrežje}
\end{slika}
\end{vprasanje}

\begin{odgovor}
\begin{enumerate}[(a)]
\item V poti $s \to a \to c \to d \to t$
nas najbolj omejujeta povezavi $s \to a$ in $d \to t$ s kapaciteto $2$,
v poti $s \to c \to d \to b \to t$ pa povezava $d \to b$ s kapaciteto $1$.
Ker ima edina skupna povezava $c \to d$ kapaciteto $5$,
bo skupen pretok po povečanju po teh dveh poteh enak $3$.
Pridruženo omrežje je prikazano na sliki~\fig[pretokx-resitev1].

\item Na omrežju s slike~\fig[pretokx-resitev1]
je $s \to c \gets a \to b \to t$ edina možna povečujoča pot.
Pri $0 \le x < 2$ nas s količino $x$ najbolj omejuje povezava $a \to b$
-- tedaj dobimo maksimalni pretok $3 \le 3+x < 5$
in minimalni prerez $\{a \to b, d \to b, d \to t\}$,
glej sliko~\fig[pretokx-resitev2].

Pri $x \ge 2$ nas s količino $2$ najbolj omejuje povezava $c \gets a$
-- tedaj dobimo maksimalni pretok $5$
in minimalni prerez $\{s \to a, d \to b, d \to t\}$,
glej sliko~\fig[pretokx-resitev3].

\item Iz slik~\fig[pretokx-resitev2] in~\fig[pretokx-resitev3] je razvidno,
da lahko po poti $s \to c \to d$ povečamo pretok za največ $2$,
tako da lahko s povečanjem kapacitete povezave $d \to t$ za $1$
za toliko povečamo maksimalni pretok omrežja.
Ker lahko tudi po povezavi $b \to t$ povečamo pretok za vsaj $2$,
enako dosežemo tudi s povečanjem kapacitete povezave $d \to b$ za $1$.

Ker lahko po povezavah $s \to a$ in $c \gets a$
povečamo pretok v vozlišče $a$ za največ $2-x$,
sledi, da lahko s povečanjem kapacitete povezave $a \to b$ za $1$
za toliko povečamo maksimalni pretok omrežja le v primeru,
kadar velja $0 \le x \le 1$.
\end{enumerate}

\begin{slika}[p]
\pgfslika[pretokx-resitev1]
\podnaslov[\res{}(a) in povečujoča pot pri $x > 0$]{Pridruženo omrežje}
\end{slika}
\begin{slika}[p]
\pgfslika[pretokx-resitev2]
\podnaslov[\res{}(b)]{Maksimalni pretok pri $0 \le x < 2$}
\end{slika}
\begin{slika}[p]
\pgfslika[pretokx-resitev3]
\podnaslov[\res{}(b)]{Maksimalni pretok pri $x \ge 2$}
\end{slika}
\end{odgovor}
\end{naloga}
