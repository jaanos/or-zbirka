\begin{naloga}{Janoš Vidali}{Kolokvij OR 5.6.2023}
\begin{vprasanje}
Naj bo $G = (V, E)$ neusmerjen graf s cenami povezav $t_e \in \R$ ($e \in E$),
tako da v grafu ni negativnih ciklov
(tj., za vsak cikel $C$ v grafu $G$ velja $\sum_{e \in E(C)} t_e \ge 0$,
kjer je $E(C)$ množica povezav v ciklu $C$).

Za razdaljo $\partial_t(u, v)$ med vozliščema $u, v \in V$ vzamemo
ceno najcenejše poti med $u$ in $v$
(tj.,
$\partial_t(u, v) = \min\left\{\sum_{e \in E(P)} t_e
\mid \text{$P$ pot med $u$ in $v$}\right\}$,
kjer je $E(P)$ množica povezav v poti $P$).
{\em Utežena ekscentričnost} $\epsilon_t(u)$ vozlišča $u$
je največja razdalja od $u$ do ostalih vozlišč
(tj., $\epsilon_t(u) = \max\left\{\partial_t(u, v) \mid v \in V\right\}$).
{\em Uteženi polmer} $r_t(G)$ grafa $G$ potem definiramo kot
najmanjšo uteženo ekscentričnost,
torej $r_t(G) = \min\left\{\epsilon_t(u) \mid u \in V\right\}$.

\begin{enumerate}[(a)]
\item Predlagaj čim bolj učinkovit algoritem
za določitev uteženega polmera danega grafa.
Kolikšna je njegova časovna zahtevnost?
Pojasni tudi, kako bo algoritem deloval,
če ni izpolnjen pogoj glede negativnih ciklov.

\item S svojim algoritmom izračunaj uteženi polmer grafa s slike~\fig.
Iz odgovora naj bo jasen potek algoritma
(tj., za vsako spremembo naj bo jasno, v katerem koraku se zgodi).
\end{enumerate}

\begin{slika}
\pgfslika
\podnaslov[\nal{}(b)]{Graf}
\end{slika}
\end{vprasanje}

\begin{odgovor}
\begin{enumerate}[(a)]
\item Na grafu $G$ in cenah povezav $t$
uporabimo algoritem {\sc FloydWarshall} iz re\-šit\-ve naloge~\res[fw].
Ta nam vrne matriki $d$ in $\pred$,
ki nam povesta razdalje med pari vozlišč
in predzadnje vozlišče na poti med vsakima dvema vozliščema.
Iz vsake vrstice matrike $d$ vzamemo največjo vrednost
-- tako dobimo utežene ekscentričnosti vseh vozlišč.
Uteženi polmer potem dobimo tako,
da izmed dobljenih uteženih ekscentričnosti vzamemo najmanjšo vrednost.

Algoritem {\sc FloydWarshall} teče v času $O(n^3)$,
kjer je $n$ število vozlišč v grafu $G$.
Za izračun uteženih ekscentričnosti potem potrebujemo še $O(n^2)$ korakov,
za izračun uteženega polmera pa še $O(n)$ korakov.
Skupna časovna zahtevnost algoritma je torej $O(n^3)$.

V primeru, da ima vhodni graf negativen cikel,
bo to zaznal že algoritem {\sc FloydWarshall},
tako da se lahko v tem primeru tudi naš algoritem ustavi.

\item Potek izvajanja algoritma {\sc FloydWarshall}
skupaj z izračuni uteženih ekscentričnosti je prikazan v tabeli~\tab.
Prikazano je tako začetno stanje kot tudi stanje po vsakem koraku glavne zanke,
pri čemer je v zgornjem levem kotu vsake matrike zapisano vozlišče,
skozi katerega dodajamo poti v ustreznem koraku zanke,
spremembe pa so podčrtane.
Iz uteženih ekscentričnosti lahko potem razberemo,
da je uteženi polmer grafa s slike~\fig enak $3$.
\end{enumerate}

\begin{tabela}
$$
\begin{gathered}
\begin{array}{c|cccc}
  & a & b & c & d \\ \hline
a & 0 & 1_a & 4_a & 5_a \\
b & 1_b & 0 & 2_b & \infty \\
c & 4_c & 2_c & 0 & 1_c \\
d & 5_d & \infty & 1_d & 0
\end{array}
\qquad
\begin{array}{c|cccc}
a & a & b & c & d \\ \hline
a & 0 & 1_a & 4_a & 5_a \\
b & 1_b & 0 & 2_b & \underline{6_a} \\
c & 4_c & 2_c & 0 & 1_c \\
d & 5_d & \underline{6_a} & 1_d & 0
\end{array}
\qquad
\begin{array}{c|cccc}
b & a & b & c & d \\ \hline
a & 0 & 1_a & \underline{3_b} & 5_a \\
b & 1_b & 0 & 2_b & 6_a \\
c & \underline{3_b} & 2_c & 0 & 1_c \\
d & 5_d & 6_a & 1_d & 0
\end{array}
\\
\begin{array}{c|cccc}
c & a & b & c & d \\ \hline
a & 0 & 1_a & 3_b & \underline{4_c} \\
b & 1_b & 0 & 2_b & \underline{3_c} \\
c & 3_b & 2_c & 0 & 1_c \\
d & \underline{4_b} & \underline{3_c} & 1_d & 0
\end{array}
\qquad
\begin{array}{c|cccc}
d & a & b & c & d \\ \hline
a & 0 & 1_a & 3_b & 4_c \\
b & 1_b & 0 & 2_b & 3_c \\
c & 3_b & 2_c & 0 & 1_c \\
d & 4_b & 3_c & 1_d & 0
\end{array}
\qquad
\begin{array}{c|c}
u & \epsilon_t(u) \\ \hline
a & 4 \\
b & 3 \\
c & 3 \\
d & 4
\end{array}
\end{gathered}
$$
\podnaslov[\res{}(b)]{Potek izvajanja algoritma}
\end{tabela}
\end{odgovor}
\end{naloga}
