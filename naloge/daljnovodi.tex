\begin{naloga}{Janoš Vidali}{Izpit OR 23.8.2023}
\begin{vprasanje}
V redko poseljeni pokrajini se gradi $n$ tovarn,
ki bodo razporejene na krožnici,
v središču krož\-ni\-ce pa elektrarna, ki bo oskrbovala te tovarne.
Zgraditi želimo omrežje daljnovodov
-- to bo imelo obliko drevesa, kjer so listi elektrarna in tovarne,
notranja vozlišča pa so transformatorske postaje,
kjer se dovodni tok iz smeri elektrarne razdeli na dva dela,
ki oskrbujeta (morda preko nadaljnjih transformatorskih postaj)
zaporedne tovarne na krožnici
(tako bo mogoče daljnovode postaviti brez sekanja).

Za $i$-to zaporedno tovarno na krožnici $(1 \le i \le n)$
poznamo njeno maksimalno porabo $p_i$.
Cena postavitve ene transformatorske postaje
je sorazmerna vsoti maksimalnih porab vseh tovarn, ki jih oskrbuje.
Zanima nas,
kako postaviti transformatorske postaje
(tj., katere tovarne naj vsaka oskrbuje -- natančen položaj nas ne zanima),
da bo skup\-na cena njihove postavitve čim manjša.

Na sliki~\fig je prikazan primer dopustne (ne nujno optimalne!) rešitve.
Prva transformatorska postaja $a$ tok iz elektrarne (vozlišče $E$)
razdeli na daljnovod proti tovarni $t_2$
in na daljnovod proti transformatorski postaji $b$,
ki oskrbuje ostale štiri tovarne.
Od tam se tok razdeli na daljnovoda
proti transformatorskima postajama $c$ in $d$,
ki oskrbujeta tovarne $t_1$ in $t_5$ oziroma $t_3$ in $t_4$.

\begin{enumerate}[(a)]
\item Zapiši rekurzivne enačbe,
s katerimi poiščemo najcenejšo postavitev transformatorskih postaj.
Razloži, kaj predstavljajo spremenljivke,
v kakšnem vrstnem redu jih računamo, ter kako dobimo optimalno rešitev.
Kakšna je časovna zahtevnost algoritma, ki sledi iz tvojih rekurzivnih enačb?

\item Reši nalogo pri podatkih $n = 5$ in $(p_i)_{i=1}^5 = (64, 35, 38, 80, 23)$.
\end{enumerate}
%
\begin{slika}
\pgfslika
\podnaslov{Primer postavitve transformatorskih postaj}
\end{slika}
\end{vprasanje}

\begin{odgovor}
\begin{enumerate}[(a)]
\item Uvedimo oznaki
$$
i^- = \begin{cases}
n & \text{če $i = 1$}, \\
i-1 & \text{sicer}
\end{cases}
$$
in
$$
\llbracket i, j \rrbracket = \begin{cases}
\{i, i+1, \dots, j-1, j\} & \text{če $i \le j$}, \\
\{i, i+1, \dots, n, 1, \dots, j-1, j\} & \text{če $i > j$}.
\end{cases}
$$
Vidimo,
da je $i \mapsto i^-$ bijektivna preslikava
na množici $\llbracket 1, n \rrbracket$,
tako da vrednost $i^-$ enolično določa $i$.

Definirajmo spremenljivko $v_i$ ($0 \le i \le n$)
kot vsoto maksimalnih porab prvih $i$ tovarn
in $c_{ij}$ ($1 \le i, j \le n$)
kot najmanjšo ceno postavitve transformatorske postaje,
ki oskrbuje tovarne iz množice $\llbracket i, j \rrbracket$.
Pišimo še
$$
v_{ij} = \begin{cases}
v_j - v_{i-1} & \text{če $i \le j$}, \\
v_n + v_j - v_{i-1} & \text{če $i > j$}.
\end{cases}
$$

Zapišimo začetne pogoje in rekurzivne enačbe.
\begin{align*}
v_0 &= c_{ii} = 0 & (1 \le i &\le n) \\
v_i &= v_{i-1} + p_i & (1 \le i &\le n) \\
c_{ij} &= v_{ij} + \min\{c_{ih^-} + c_{hj} \mid h^- \in \llbracket i, j^- \rrbracket\}
 & (1 \le i, j &\le n, i \ne j \ne i^-)
\end{align*}
Vrednosti $v_i$ ($0 \le i \le n$) računamo naraščajoče po indeksu $i$,
vrednosti $c_{ij}$ pa v leksikografskem vrstnem redu
glede na $(|\llbracket i, j \rrbracket|, i)$.
Cena celotne postavitve je
$c^* = v_n + \min\{c_{ij^-} + c_{ji^-} \mid 1 \le i < j \le n\}$.
Za izračun porabimo $O(n^3)$ časa.

\item Najprej izračunajmo vrednosti $v_i$ ($1 \le i \le 5$).
\begin{alignat*}{3}
v_1 &=\ &   0 &+ 64 &&=  64 \\
v_2 &=\ &  64 &+ 35 &&=  99 \\
v_3 &=\ &  99 &+ 38 &&= 137 \\
v_4 &=\ & 137 &+ 80 &&= 217 \\
v_5 &=\ & 217 &+ 23 &&= 240
\end{alignat*}
Sedaj izračunajmo še vrednosti $c_{ij}$ ($1 \le i, j \le 5$, $i \ne j \ne i^-$).
\begin{alignat*}{4}
c_{12} &=\ &{}        99 &-   0 &&+ 0 + 0 &&=  99 \\
c_{23} &=\ &{}       137 &-  64 &&+ 0 + 0 &&=  73 \\
c_{34} &=\ &{}       217 &-  99 &&+ 0 + 0 &&= 118 \\
c_{45} &=\ &{}       240 &- 137 &&+ 0 + 0 &&= 103 \\
c_{51} &=\ &{} 240 +  64 &- 217 &&+ 0 + 0 &&=  87 \\
c_{13} &=\ &{}       137 &-   0 &&+ \min\{\underline{0+73}, 99+0\}   &&= 210 \\
c_{24} &=\ &{}       217 &-  64 &&+ \min\{0+118, \underline{73+0}\}  &&= 226 \\
c_{35} &=\ &{}       240 &-  99 &&+ \min\{\underline{0+103}, 118+0\} &&= 244 \\
c_{41} &=\ &{} 240 +  64 &- 137 &&+ \min\{\underline{0+87}, 103+0\}  &&= 254 \\
c_{52} &=\ &{} 240 +  99 &- 217 &&+ \min\{0+99, \underline{87+0}\}   &&= 209 \\
c_{14} &=\ &{}       217 &-   0 &&+ \min\{0+226, 99+118, \underline{210+0}\} &&= 427 \\
c_{25} &=\ &{}       240 &-  64 &&+ \min\{0+244, \underline{73+103}, 226+0\} &&= 352 \\
c_{31} &=\ &{} 240 +  64 &-  99 &&+ \min\{0+254, \underline{118+87}, 244+0\} &&= 410 \\
c_{42} &=\ &{} 240 +  99 &- 137 &&+ \min\{0+209, \underline{103+99}, 254+0\} &&= 404 \\
c_{53} &=\ &{} 240 + 137 &- 217 &&+ \min\{0+210, \underline{87+73}, 209+0\}  &&= 320
\end{alignat*}
Cena optimalne postavitve transformatorskih postaj je potem
\begin{align*}
c^* = 240 + \min\{&{} 0+352, 99+244, \underline{210+103}, 427+0, 0+410, \\
&{} 73+254, \underline{226+87}, 0+404, 118+209, 0+320\} = 553 .
\end{align*}
Poglejmo, kako naj jih postavimo.
\begin{align*}
c^* &= v_5 + c_{13} + c_{45} & \text{delimo na $1$ do $3$ in $4, 5$} &\quad (a) \\
c_{13} &= v_3 - v_0 + c_{11} + c_{23} & \text{delimo na $1$ in $2, 3$} &\quad (b) \\
c_{45} &= v_5 - v_3 + c_{44} + c_{55} & \text{delimo na $4$ in $5$} &\quad (c) \\
c_{23} &= v_3 - v_2 + c_{22} + c_{33} & \text{delimo na $2$ in $3$} &\quad (d)
\end{align*}
%
Postavitev je prikazana na sliki~\fig[daljnovodi-resitev].
Druga optimalna postavitev bi bila taka,
da bi najprej delili tok na tovarni $5, 1$ in tovarne $2$ do $4$,
tok za slednje pa bi nadalje delili na tovarni $2, 3$ in tovarno $4$.
\begin{slika}
\pgfslika[daljnovodi-resitev]
\podnaslov[\res{}(b)]{Postavitev transformatorskih postaj}
\end{slika}
\end{enumerate}
\end{odgovor}
\end{naloga}
