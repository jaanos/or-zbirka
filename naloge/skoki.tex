\begin{naloga}{?}{Izpit OR 24.5.2016}
\begin{vprasanje}
Dan je seznam pozitivnih celih števil $S = [a_1, a_2, \dots, a_n]$,
ki ga interpretiramo na naslednji način.
Element $a_i$ pomeni,
da se lahko iz $i$-te pozicije v seznamu
premaknemo na pozicije $a_{i+1}, a_{i+2}, \dots, a_{i+a_i}$.
Naj bo $f(S)$ minimalno število korakov,
ki so potrebni, da se iz elementa $a_1$ premaknemo v element $a_n$.
Na primer, če je $S = [1, 3, 5, 8, 9, 2, 6, 7, 6, 8, 9]$,
potem je $f(S) = 3$,
saj lahko opravimo skoke $a_1 = 1 \to a_2 = 3 \to a_4 = 8 \to a_{11} = 9$.

\begin{enumerate}[(a)]
\item Napiši rekurzivne enačbe za računanje funkcije $f(S)$.

\item S pomočjo dinamičnega programiranja napiši algoritem,
ki rešuje dani problem.
Kakšna je njegova časovna zahtevnost?
\end{enumerate}
\end{vprasanje}
\begin{odgovor}
\end{odgovor}
\end{naloga}
