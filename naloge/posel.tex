\begin{naloga}{Janoš Vidali}{Kolokvij OR 14.4.2025}
\begin{vprasanje}
Tinetu se je odprla priložnost za mamljiv posel,
s katerim lahko zasluži $5\,000 €$.
Da bi ga sklenil, pa mora že pojutrišnjem na pot,
sicer bo zamudil priložnost in tako ne bo imel dobička.
Žal pa je imel v zadnjem času težave s svojim avtom,
drugega prevoza pa si v tako kratkem času ne bo mogel priskrbeti.
Računa, da bo z verjetnostjo $0.4$ njegov avto zdržal pot,
z verjetnostjo $0.3$ se bodo težave pojavile na poti naprej,
z verjetnostjo $0.3$ pa se bodo težave pojavile na poti nazaj.
V primeru, da bo pravočasno prišel do cilja
(tj., ne bo imel težav na poti naprej),
bo torej zaslužil $5\,000 €$;
v primeru, da se pa pojavijo težave z avtom
(bodisi na poti naprej ali nazaj),
pa bo imel $1\,200 €$ stroškov z odvozom.

Tine se lahko odloči, da še jutri odpelje avto na servis,
kjer bodo ugotovili, ali je avto izpraven ali ne.
Znane so sledeče pogojne verjetnosti
$P(\text{avto je izpraven} \mid \text{pot bo uspela})$:
\begin{center}
	\begin{tabular}{c|cc}
		& uspe & težave (naprej ali nazaj) \\ \hline
		izpraven & $0.9$ & $0.2$ \\
		ni izpraven & $0.1$ & $0.8$
	\end{tabular}
\end{center}
V primeru, da na servisu ugotovijo, da avto ni izpraven,
ga bodo z verjetnostjo $0.5$ uspeli popraviti
-- tedaj bo Tine imel $500 €$ stroškov s servisom in zagotovilo,
da bo pot uspela.
V nasprotnem primeru (ne glede na ugotovitve)
mu bodo na servisu zaračunali $100 €$ za pregled.

Kako naj se Tine odloči?
Nariši odločitveno drevo in odločitve sprejmi na podlagi izračunanih verjetnosti
ter izračunaj pričakovani dobiček.
\end{vprasanje}

\begin{odgovor}
Izračunajmo verjetnosti za oceno izpravnosti
in za uspešnost poti glede na oceno.
\begin{align*}
P(\text{izpraven}) &= 0.9 \cdot 0.4 + 0.2 \cdot (0.3 + 0.3) = 0.48 \\
P(\text{ni izpraven}) &= 0.1 \cdot 0.4 + 0.8 \cdot (0.3 + 0.3) = 0.52 \\
P(\text{pot uspe} \mid \text{izpraven})
&= {0.9 \cdot 0.4 \over 0.48} = 0.75 \\
P(\text{težave naprej} \mid \text{izpraven})
&= {0.2 \cdot 0.3 \over 0.48} = 0.125 \\
P(\text{težave nazaj} \mid \text{izpraven})
&= {0.2 \cdot 0.3 \over 0.48} = 0.125 \\
P(\text{pot uspe} \mid \text{ni izpraven})
&= {0.1 \cdot 0.4 \over 0.52} \approx 0.077 \\
P(\text{težave naprej} \mid \text{ni izpraven})
&= {0.8 \cdot 0.3 \over 0.52} \approx 0.462 \\
P(\text{težave nazaj} \mid \text{ni izpraven})
&= {0.8 \cdot 0.3 \over 0.52} \approx 0.462
\end{align*}
Odločitveno drevo je prikazano na sliki~\fig.
Tine naj pelje avto na servis,
nato pa naj gre na pot ne glede na izid servisa.
Pričakovani dobiček je tedaj $3\,464 €$.
\begin{slika}[p]
\makebox[\textwidth][c]{
\pgfslika
}
\podnaslov{Odločitveno drevo}
\end{slika}
\end{odgovor}
\end{naloga}
