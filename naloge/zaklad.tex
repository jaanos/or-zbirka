\begin{naloga}{Janoš Vidali}{Izpit OR 15.12.2016}
\begin{vprasanje}
Lovec na zaklade se z bogatim ulovom
vrača iz Kalifornije nazaj domov v Chicago,
pri čemer mora seveda prečkati Divji zahod.
Potoval bo s kočijo,
pri čemer bo vsak dan potoval med dvema mestoma in nato prespal.
Zaradi varnosti se bo držal samo državnih cest, ki so varne.
Toda mesta, kjer bo prespal, niso povsem varna.
Za vsako mesto pozna verjetnosti,
da ga tam ne bodo oropali (te so med seboj neodvisne).
Tako bi želel načrtovati najvarnejšo pot domov
-- torej pot z največjo verjetnostjo,
da ga pri nobenem postanku ne bodo oropali.

\begin{enumerate}[(a)]
\item Mesta in ceste med njimi lahko predstavimo z vozlišči in povezavami
v ne\-usme\-rje\-nem grafu $G$, verjetnosti pa kot teže vozlišč.
Opiši, kako lahko za dani graf $G$ z uteženimi vozlišči
učinkovito poiščemo ustrezno pot med danima vozliščema $s$ in $t$
z uporabo variante Dijkstrovega algoritma,
ter utemelji njegovo ustreznost.
Lahko predpostaviš, da sta teži začetnega in končnega vozlišča enaki $1$.

\item Reši problem za graf s slike~\fig,
pri čemer naj se pot začne v LA in konča v CH.
Zadostovalo bo, če verjetnosti računaš na $3$ decimalke natančno.
\end{enumerate}

\begin{slika}
\pgfslika
\podnaslov{Graf}
\end{slika}
\end{vprasanje}

\begin{odgovor}
\begin{enumerate}[(a)]
\item Iz danega neusmerjenega grafa $G = (V, E)$
in funkcije uteži vozlišč $c : V \to [0, 1]$
bomo konstruirali usmerjen graf $G' = (V, E')$,
kjer je $E' = \set{uv, vu}{uv \in E}$
(tj., vsako neusmerjeno povezavo iz $G$
nadomestimo z nasprotno usmerjenima povezavama med krajiščema),
in funkcijo uteži povezav $L : E' \to [0, 1]$ s predpisom $L(uv) = c(u)$
(tj., cena povezave je enaka ceni začetnega vozlišča v $G$).
Na grafu $G'$ s funkcijo uteži povezav $L$
nato poženemo varianto algoritma {\sc Dijkstra} iz naloge~\res[dijkstra].
\begin{small}
\begin{algorithmic}
\Function{DijkstraProb}{$G = (V, E), s \in V, L : E \rightarrow [0, 1]$}
    \State $d \gets$ slovar z vrednostjo $0$ za vsako vozlišče $v \in V$
    \State $\pred \gets$ slovar z vrednostjo $\Null$
        za vsako vozlišče $v \in V$
	\State $Q \gets$ prednostna vrsta
        s prioriteto $0$ za vsako vozlišče $v \in V$
        (na vrhu je vozlišče z največjo prioriteto)
	\State $Q[s] \gets 1$
	\While{$\lnot Q.\isEmpty()$}
		\State $u, d[u] \gets Q.\pop()$
		\For{$v \in \Adj(G, u)$}
			\If{$v \in Q \land Q[v] < d[u] \cdot L_{uv}$}
				\State $Q[v] \gets d[u] \cdot L_{uv}$
                \State $\pred[v] \gets u$
			\EndIf
		\EndFor
	\EndWhile
    \State \Return $(d, \pred)$
\EndFunction
\end{algorithmic}
\end{small}
V primerjavi z algoritmom {\sc Dijkstra}
smo torej zamenjali seštevanje z množenjem in obrnili primerjave,
poleg tega pa smo še ustrezno spremenili začetne vrednosti.
Preslikava $x \mapsto -\log x$ pošlje uteži na interval $[0, \infty]$,
pri čemer se urejenost obrne, množenje se pa nadomesti s seštevanjem
-- zgornji algoritem je torej ekvivalenten
algoritmu {\sc Dijkstra} s tako preslikanimi utežmi,
in ima tako isto časovno zahtevnost
(tj., $O(n^2)$, če za prednostno vrsto uporabimo običajen slovar,
in $O(m \log n)$, če za prednostno vrsto vzamemo kopico,
kjer je $n$ število vozlišč in $m$ število povezav v grafu).

\item Potek izvajanja zgornjega algoritma je prikazan v tabeli~\tab,
iz katere razberemo,
da je najvarnejša pot LA -- SF -- RE -- SLC -- DE -- KC -- SL -- CH,
z verjetnostjo, da nas ne oropajo, enako $0.31752$.
\end{enumerate}
%
\begin{tabela}
\makebox[\textwidth][c]{
\small
\begin{tabular}{cccccccccccccccc}
LA & SF & PH & LV & RE & EP & AQ & DE & SLC & DA & OC & KC & OM & ME & SL & CH \\ \hline
$1$ & $0$ & $0$ & $0$ & $0$ & $0$ & $0$ & $0$ & $0$ & $0$ & $0$ & $0$ & $0$ & $0$ & $0$ & $0$ \\
* & $1_{\text{LA}}$ & $1_{\text{LA}}$ & $1_{\text{LA}}$ &&&&&&&&&&&& \\
& * &&& $1_{\text{SF}}$ &&&&&&&&&&& \\
&& * &&& $0.8_{\text{PH}}$ & $0.8_{\text{PH}}$ &&&&&&&&& \\
&&& * &&&& $0.5_{\text{LV}}$ & $0.5_{\text{LV}}$ &&&&&&& \\
&&&& * &&&& $0.7_{\text{RE}}$ &&&&&&& \\
&&&&& * &&&& $0.48_{\text{EP}}$ &&&&&& \\
&&&&&& * & $0.56_{\text{AQ}}$ &&& $0.56_{\text{AQ}}$ &&&&& \\
&&&&&&& $0.63_{\text{SLC}}$ & * &&&&&&& \\
&&&&&&& * &&&& $0.504_{\text{DE}}$ & $0.504_{\text{DE}}$ &&& \\
&&&&&&&&&& * &&& $0.336_{\text{OC}}$ && \\
&&&&&&&&&&& * &&& $0.3528_{\text{KC}}$ & \\
&&&&&&&&&&&& * &&& $0.2016_{\text{OM}}$ \\
&&&&&&&&& * &&&& $0.384_{\text{DA}}$ && \\
&&&&&&&&&&&&& * && \\
&&&&&&&&&&&&&& * & $0.31752_{\text{SL}}$ \\
&&&&&&&&&&&&&&& *
\end{tabular}
}
\podnaslov[\res{}(b)]{Potek izvajanja algoritma}
\end{tabela}
\end{odgovor}
\end{naloga}
