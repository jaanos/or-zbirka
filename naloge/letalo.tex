\begin{naloga}{Janoš Vidali}{Izpit OR 28.8.2018}
\begin{vprasanje}
Pri izdelavi letala imamo faze, opisane v tabeli~\tab.

\begin{enumerate}[(a)]
\item S pomočjo topološke ureditve ustreznega grafa
določi kritična opravila in čas izdelave.
Uporabi podatke iz stolpca ``trajanje''.

\item Katero opravilo je najmanj kritično?
Najmanj kritično jo opravilo,
katerega trajanje lahko najbolj podaljšamo,
ne da bi vplivali na trajanje izdelave.

\item Naročnik bi rad, da letalo izdelamo v $75$ dneh.
Posamezna opravila lahko skrajšamo tako, da zanje zadolžimo več delavcev.
Seveda prinese tako krajšanje dodatne stroške,
poleg tega pa za vsako opravilo poznamo najmanjše možno trajanje
(glej zadnja dva stolpca v zgornji tabeli).
Zapiši linearni program,
s katerim modeliramo iskanje razporeda opravil,
ki nam prinese čim manjše stroške.
Linearnega programa ne rešuj.
\end{enumerate}

\begin{tabela}
\makebox[\textwidth][c]{
\begin{tabular}{c|l|c|c||c|c}
opravilo & opis & trajanje & pogoji & najmanjše trajanje & cena za dan manj \\
\hline
$a$ & izgradnja nosu              & 40 dni & /      & 36 dni & 1\,000 \\
$b$ & izgradnja trupa s krili     & 50 dni & /      & 48 dni & 1\,500 \\
$c$ & izgradnja repa              & 35 dni & /      & 31 dni &    800 \\
$d$ & vgraditev pilotske kabine   & 30 dni & $a$    & 28 dni & 1\,200 \\
$e$ & opremljanje potniške kabine & 18 dni & $f, g$ & 16 dni &    500 \\
$f$ & povezovanje nosu s trupom   & 10 dni & $a, b$ &  9 dni & 1\,100 \\
$g$ & povezovanje repa s trupom   & 12 dni & $b, c$ & 10 dni & 1\,300 \\
$h$ & vgraditev motorjev          & 20 dni & $b$    & 18 dni & 1\,400
\end{tabular}
}
\podnaslov{Podatki}
\end{tabela}
\end{vprasanje}

\begin{odgovor}
\begin{enumerate}[(a)]
\item Projekt lahko predstavimo z uteženim grafom s slike~\fig,
iz katerega je raz\-vid\-na topološka ureditev $s, a, b, c, d, f, g, h, e, t$.
V tabeli~\tab[letalo-resitev]
so podani naj\-zgod\-nej\-ši začetki opravil in najpoznejši začetki,
da se celotno trajanje izdelave ne podaljša.
Izdelava bo trajala najmanj $88$ dni,
edina kritična pot pa je $s - b - h - e - t$,
ki tako vsebuje vsa kritična opravila.

\item Iz tabele~\tab[letalo-resitev] je razvidno,
da je najmanj kritično opravilo $c$,
saj lahko njegovo trajanje podaljšamo za $41$ dni,
ne da bi vplivali na trajanje celotnega projekta.

\item Zapišimo linearni program,
ki bo za vsako opravilo imel spremenljivki $x_u$ in $y_u$,
ki določata število dni, za katerega skrajšamo trajanje opravila $u$,
oziroma čas začetka izvajanja opravila $u$.
\begin{alignat*}{3}
\min &&&&&{} 1\,000 x_a + 1\,500 x_b + 800 x_c + 1\,200 x_d \\[-1mm]
&&&&+\,&{} 500 x_e + 1\,100 x_f + 1\,300 x_g + 1\,400 x_h
\opis{Opravila brez pogojev}
y_a, &&{} y_b, &&{} y_c &\ge 0
\opis{Odvisnosti med opravili}
y_d &\,-& y_a &\,+& x_a &\ge 40 \\
y_e &\,-& y_f &\,+& x_f &\ge 10 \\
y_e &\,-& y_g &\,+& x_g &\ge 12 \\
y_e &\,-& y_c &\,+& x_c &\ge 90 \\
y_f &\,-& y_a &\,+& x_a &\ge 40 \\
y_f &\,-& y_b &\,+& x_b &\ge 50 \\
y_g &\,-& y_b &\,+& x_b &\ge 50 \\
y_g &\,-& y_c &\,+& x_c &\ge 35 \\
y_h &\,-& y_b &\,+& x_b &\ge 50
\opis{Želeni čas trajanja}
&& y_d &\,-& x_d &\le 45 \\
&& y_e &\,-& x_e &\le 57 \\
&& y_g &\,-& x_g &\le 63
\opis{Minimalno trajanje opravil}
&& 0 &\le&\, x_a &\le 4 \\
&& 0 &\le&\, x_b &\le 2 \\
&& 0 &\le&\, x_c &\le 4 \\
&& 0 &\le&\, x_d &\le 2 \\
&& 0 &\le&\, x_e &\le 2 \\
&& 0 &\le&\, x_f &\le 1 \\
&& 0 &\le&\, x_g &\le 2 \\
&& 0 &\le&\, x_h &\le 2
\end{alignat*}
\end{enumerate}
%
\begin{slika}
\pgfslika
\podnaslov{Graf odvisnosti med opravili in kritična pot}
\end{slika}
%
\begin{tabela}
\setlabel{letalo-resitev}
$$
\begin{array}{c|cccccccccc}
& s & a & b & c & d & f & g & h & e & t \\ \hline
\text{najzgodnejši začetek} &
0 & 0_s & 0_s & 0_s & 40_a & 50_b & 50_b & 50_b & 70_h & 88_e \\
\text{najpoznejši začetek} &
0 & 18_d & 0_h & 41_g & 58_t & 60_e & 76_t & 50_e & 70_t & 88 \\
\text{razlika} &
0 & 18 & 0^* & 41 & 18 & 10 & 26 & 0^* & 0^* & 0
\end{array}
$$
\podnaslov{Razporejanje opravil}
\end{tabela}
\end{odgovor}
\end{naloga}
