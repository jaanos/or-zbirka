\razdelek{Pretoki in prerezi}

\begin{naloga}{?}{Kolokvij OR 19.11.2009}
\begin{vprasanje}[minmaxpretok]
Poišči minimalno vrednost parametra $a$,
da bo pretok skozi omrežje s slike~\fig{} maksimalen.
Utemelji izbiro.

\begin{slika}
\pgfslika
\podnaslov{Omrežje}
\end{slika}
\end{vprasanje}
\begin{odgovor}
\end{odgovor}
\end{naloga}


\begin{naloga}{?}{Kolokvij OR 26.1.2010}
\begin{vprasanje}
Imamo podatke za problem maksimalnega pretoka.
Ali je povezava z najmanjšo kapaciteto
(predpostavimo, da je natanko ena taka)
vedno vsebovana v kakem minimalnem prerezu?
Ali je morda vsebovana celo v vsakem minimalnem prerezu?
\end{vprasanje}
\begin{odgovor}
\end{odgovor}
\end{naloga}


\begin{naloga}{?}{Kolokvij OR 5.2.2010}
\begin{vprasanje}[lenority]
Na univerzi v mestu Lenority so asistenti res nekoliko razvajeni
in imajo močen sindikat.
(To ni nič nenavadnega, če pa univerza zahteva,
da se izpiti začnejo že ob sedmih zjutraj!)
Bliža se izpitno obdobje in potrebno je paziti izpite.
Za vsak dan so izpiti dani vnaprej.
Znano je, da asistenti ne pazijo izpitov, če so vmes ``luknje'',
saj si je to pravico izboril njihov sindikat.
Vodstvu univerze tako ne preostane nič drugega, kot da to upošteva.
Asistenti torej pridejo na faks ob neki uri,
pazijo poljubno dolgo zaporedje izpitov brez odmorov, in potem gredo domov.
Izpiti so podani v obliki časovnih intervalov,
ki se začnejo in končajo ob polni uri.
Upoštevajoč te pogoje mora vodstvo univerze venomer prositi asistente,
naj pridejo pazit izpite.
Toda tudi vodstvo si želi minimizirati ``bolečino'' ob moledovanju
in hoče vsak dan prositi kar se da malo asistentov,
naj vendarle pridejo pazit izpite.

\begin{enumerate}[(a)]
\item Modeliraj problem s pomočjo usmerjenih acikličnih grafov,
pri katerih so intervali predstavljeni z vozlišči.
Identificiraj, kaj je tvoja optimizacijska naloga na takem grafu.

\item Opiši, kako problem prevedeš na problem maksimalnih pretokov.
Natančno utemelji,
zakaj so maksimalni pretoki v tvojem modelu v bijektivni korespondenci
z optimalnimi rešitvami tvoje optimizacijske naloge.

\item S pomočjo algoritma iz prejšnje točke reši problem za petkove izpite,
ki so podani z intervali $(7, 8)$, $(8, 9)$, $(9, 10)$, $(8, 12)$,
$(10, 13)$, $(12, 15)$, $(13, 15)$, $(9, 12)$, $(7, 10)$.
\end{enumerate}
\end{vprasanje}
\begin{odgovor}
\end{odgovor}
\end{naloga}

\begin{naloga}{?}{Izpit OR 28.6.2010}
\begin{vprasanje}[pretokx]
Na sliki~\fig{} je podano enoparametrično omrežje
za problem največjega pretoka.
Parameter $x$ je poljubno nenegativno število.

\begin{enumerate}[(a)]
\item Kakšno je pridruženo omrežje,
če začnemo z ničelnim $(s, t)$-tokom
in ga najprej povečamo vzdolž povečujoče poti $s-a-c-d-t$
ter nato še vzdolž povečujoče poti $s-c-d-b-t$?

\item Poišči maksimalni $(s, t)$-tok in minimalni $(s, t)$-prerez.

\item Kapaciteto ene povezave vhodnega omrežja lahko povečamo za $1$.
Kateri povezavi naj povečamo kapaciteto,
da bo vrednost maksimalnega $(s, t)$-toka v spremenjenem omrežju čim večja?
Odgovor je načeloma odvisen od vrednosti parametra $x$.
\end{enumerate}

\begin{slika}
\pgfslika
\podnaslov{Omrežje}
\end{slika}
\end{vprasanje}
\begin{odgovor}
\end{odgovor}
\end{naloga}


\begin{naloga}{?}{Izpit OR 15.9.2010}
\begin{vprasanje}[terminali]
Na sliki~\fig{} je podano omrežje enosmernih prehodov
med letališkimi terminali skupaj s prepustnostmi (v $1000$ ljudeh na uro).

\begin{enumerate}[(a)]
\item Načrtovalce zanima,
koliko največ potnikov na uro lahko preide med terminaloma $F$ in $B$.
Predlagaj ustrezen algoritem in ga izvedi na grafu.

\item V izrednih razmerah dva terminala proglasijo za vstopna
(na njih letala le pristajajo),
ostale terminale pa za izstopne (letala le vzletajo).
Tako morajo potniki nujno prehajati od vstopnih k izstopnim terminalom.
Strateška odločitev je, da je vstopni terminal $F$,
zaradi redundance pa želijo, da bi bila vstopna terminala dva.
Ostali terminali so torej izstopni.
Katerega izmed terminalov (poleg $F$) se splača še proglasiti za vstopnega,
da bo pretočnost med vhodnimi in izhodnimi terminali kar največja?
\end{enumerate}

\begin{slika}
\pgfslika
\podnaslov{Omrežje}
\end{slika}
\end{vprasanje}
\begin{odgovor}
\end{odgovor}
\end{naloga}
