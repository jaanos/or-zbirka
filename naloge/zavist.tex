\begin{naloga}{Janoš Vidali}{Izpit OR 18.6.2021}
\begin{vprasanje}
Imamo $m$ predmetov, ki jih želimo prodati $n$ strankam.
Za $i$-ti predmet ($1 \le i \le m$) in $j$-to stranko ($1 \le j \le n$)
poznamo znesek $c_{ij}$,
ki ga je $j$-ta stranka pripravljena plačati za $i$-ti predmet.
Ker pa želimo, da si stranke med seboj ne zavidajo,
se lahko odločimo, da vsaki damo nekolikšen popust.
Smatramo, da $j$-ta stranka zavida $k$-ti stranki,
če je razlika med plačano ceno in zneskom,
ki bi ga bila pripravljena plačati $j$-ta stranka,
večja pri $k$-ti kot pri $j$-ti stranki
-- torej,
če $j$-ta in $k$-ta stranka dobita predmete iz množic $A_j$ oziroma $A_k$
in za to plačata ceni $p_j$ oziroma $p_k$,
potem $j$-ta stranka zavida $k$-ti stranki,
če velja $\sum_{i \in A_j} c_{ij} - p_j < \sum_{i \in A_k} c_{ij} - p_k$.
Vsak predmet želimo prodati natanko eni stranki,
nobena stranka pa pri tem ne sme zaslužiti (tj., plačati manj kot $0$).
Zaslužek od prodaje želimo maksimizirati.

Zapiši celoštevilski linearni program, ki modelira zgoraj opisani problem.
\end{vprasanje}

\begin{odgovor}
Za $i$-ti predmet ($1 \le i \le m$)
in $j$-to stranko ($1 \le j \le n$)
bomo uvedli spremenljivki $x_j$ in $y_{ij}$,
kjer je $x_j$ cena, ki jo plača $j$-ta stranka,
vrednost $y_{ij}$ pa interpretiramo kot
$$
y_{ij} = \begin{cases}
1, & \text{$j$-ti stranki prodamo $i$-ti predmet, in} \\
0  & \text{sicer.}
\end{cases}
$$
Zapišimo celoštevilski linearni program.
\begin{alignat*}{2}
&& \max \ \sum_{j=1}^n x_j \quad \text{p.p.} \\
\forall i \in \{1, \dots, n\} \ \forall j \in \{1, \dots, m\}: &\ &
0 \le y_{ij} &\le 1, \quad y_{ij} \in \Z
\opis{Vsak predmet prodamo natanko eni stranki}
\forall i \in \{1, \dots, m\}: &\ & \sum_{j=1}^n y_{ij} &= 1
\opis{Nobena stranka ne zasluži}
\forall j \in \{1, \dots, n\}: &\ &
x_j &\ge 0
\opis{Nobena stranka ne plača več, kot je pripravljena}
\forall j \in \{1, \dots, n\}: &\ & \sum_{i=1}^m c_{ij} y_{ij} &\ge x_j
\opis{Stranke si med seboj ne zavidajo}
\forall j \in \{1, \dots, n\} \ \forall k \in \{1, \dots, n\}: &\ & \sum_{i=1}^m c_{ij} (y_{ij} - y_{ik}) &\ge x_j - x_k
\end{alignat*}
\end{odgovor}
\end{naloga}
