\begin{naloga}{Janoš Vidali}{Izpit OR 29.8.2017}
\begin{vprasanje}
Dano je odločitveno drevo s slike~\fig,
pri čemer velja $0 \le p \le 1/4$.
Pričakovano vred\-nost želimo maksimizirati.
Poišči optimalne odločitve in pričakovano vrednost
v odvisnosti od vrednosti parametra $p$.

\begin{slika}
\pgfslika
\podnaslov{Odločitveno drevo}
\end{slika}
\end{vprasanje}

\begin{odgovor}
Izračunajmo najprej pričakovane vrednosti
v vozliščih $F$, $G$ in $H$ odločitvenega drevesa s slike~\fig.
\begin{alignat*}{2}
F &= 25 \cdot 2p - 10 \cdot (1-2p) &&= 70p - 10 \\
G &= 50 \cdot p - 10 \cdot (1-p) &&= 60p - 10 \\
H &= 30 \cdot 4p - 30 \cdot (1-4p) &&= 240p - 30
\end{alignat*}
Obravnavajmo najprej odločitve v vozliščih $C$, $D$ in $E$.
Za pot od vozlišča $C$ k vozlišču $F$ se odločimo,
če velja $70p - 10 \ge 0$ oziroma $p \ge 1/7$.
Za pot od vozlišča $D$ k vozlišču $G$ se odločimo,
če velja $60p - 10 \ge -5$ oziroma $p \ge 1/12$.
Za pot od vozlišča $E$ k vozlišču $H$ se odločimo,
če velja $240p - 30 \ge -5$ oziroma $p \ge 5/48$.
Sedaj lahko izračunamo pričakovano vrednost v vozlišču $B$
v odvisnosti od parametra $p$.
\begin{alignat*}{4}
0 &\le p < {1 \over 12} &\quad&\Rightarrow\quad
& B &= 0.6 \cdot (-5) + 0.4 \cdot (-5) &&= -5 \\
{1 \over 12} &\le p < {5 \over 48} &\quad&\Rightarrow\quad
& B &= 0.6 \cdot (60p - 10) + 0.4 \cdot (-5) &&= 36p - 8 \\
{5 \over 48} &\le p \le {1 \over 4} &\quad&\Rightarrow\quad
& B &= 0.6 \cdot (60p - 10) + 0.4 \cdot (240p - 30) &&= 132p - 18
\end{alignat*}
Obravnavajmo sedaj še odločitev v vozlišču $A$
-- za pot k vozlišču $B$ se odločimo, če velja:
\begin{alignat*}{4}
0 &\le p < {1 \over 12}: &\quad -5 &\ge 0 &\quad&\Rightarrow\quad &&\bot \\
{1 \over 12} &\le p < {5 \over 48}: &\quad 36p - 8 &\ge 0
&\quad&\Rightarrow\quad &&\bot \\
{5 \over 48} &\le p < {1 \over 7}: &\quad 132p - 18 &\ge 0
&\quad&\Rightarrow\quad & {3 \over 22} &\le p < {1 \over 7} \\
{1 \over 7} &\le p \le {1 \over 4}: &\quad 132p - 18 &\ge 70p - 10
&\quad&\Rightarrow\quad & {1 \over 7} &\le p \le {1 \over 4}
\end{alignat*}
Odločamo se torej po sledečem pravilu.
\begin{itemize}
\item Če je $0 \le p < 3/22$,
se odločimo za pot preko vozlišča $C$ v list z vrednostjo $0$.
\item Če je $1/7 \le p < 3/22$,
se odločimo za pot preko vozlišča $B$.
    \begin{itemize}
    \item Če pridemo v vozlišče $D$, se odločimo za pot preko vozlišča $G$.
    \item Če pridemo v vozlišče $E$, se odločimo za pot preko vozlišča $H$.
    \end{itemize}
Pričakovana vrednost je $132p - 18 \in [0, 15]$.
\end{itemize}
Proces odločanja je prikazan na sliki~\fig[dectree2-resitev].

\begin{slika}
\pgfslika[dectree2-resitev]
\podnaslov[\res v odvisnosti od vrednosti parametra $p$]{Odločitveno drevo}
\end{slika}
\end{odgovor}
\end{naloga}
