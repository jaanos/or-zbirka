\begin{naloga}{Janoš Vidali}{Kolokvij OR 2.6.2022}
\begin{vprasanje}
Dan je neusmerjen graf $G = (V, E)$.
Poiskati želimo {\em ožino} grafa,
tj., zanima nas dolžina najkrajšega cikla v grafu $G$.
Tukaj kot cikel razumemo zaporedje različnih vozlišč $u_1 u_2 \dots u_k$,
kjer je $k \ge 3$ in $u_{i-1} u_i \in E$ ($1 \le i \le k$, $u_0 = u_k$);
dolžina takega cikla je potem $k$.
Pri reševanju tega problema
si bomo pomagali s sledečo različico iskanja v širino:
\needspace{3\baselineskip}
\begin{small}
\begin{algorithmic}
\Function{bfs2}{$G = (V, E),\ s \in V$}
	\State $z \gets \False$ \hfill zastavica za sode cikle
	\State $S \gets \{s\}$ \hfill množica obiskanih vozlišč
	\State $T \gets \{s\}$ \hfill trenutni nivo
	\State $i \gets 1$ \hfill trenutna globina
	\While{$\lnot T.\isEmpty()$} \hfill ponavljamo, dokler nivo ni prazen
		\State $N \gets$ prazna množica \hfill naslednji nivo
		\For{$u \in T$} \hfill gremo čez vozlišča na trenutnem nivoju
			\For{$v \in$ $\Adj(G, u)$} \hfill pregledamo sosede $u$
				\If{$v \not\in S$} \hfill če soseda prvič obiščemo,
					\State $S.\add(v)$ \hfill ga označimo za obiskanega
					\State $N.\add(v)$ \hfill in ga dodamo na naslednji nivo
				\ElsIf{$v \in T$} \hfill če je sosed na istem nivoju,
					\State \Return $2i - 1$ \hfill smo našli lih cikel
				\ElsIf{$v \in N$} \hfill če je sosed na naslednjem nivoju,
					\State $z \gets \True$ \hfill smo našli sod cikel
				\EndIf
			\EndFor
		\EndFor
		\If{$z$} \hfill če smo našli sod cikel, končamo
			\State \Return $2i$
		\EndIf
		\State $T \gets N$ \hfill naslednji nivo postane trenutni
		\State $i \gets i + 1$ \hfill povečamo globino
	\EndWhile
	\State \Return $\infty$ \hfill nismo našli cikla
\EndFunction
\end{algorithmic}
\end{small}

\begin{enumerate}[(a)]
\item Naj bo $C = u_1 u_2 \dots u_k$ najkrajši cikel v grafu $G$.
Dokaži, da za poljubna $i, j$ ($1 \le i \le j \le k$) velja,
da sta $u_i u_{i+1} \dots u_{j-1} u_j$
in $u_i u_{i-1} \dots u_1 u_k \dots u_{j+1} u_j$
najkrajši dve poti od $u_i$ do $u_j$ v grafu $G$.

\item Naj bo $g$ dolžina najkrajšega cikla v grafu $G$
in $s$ vozlišče grafa $G$.
Utemelji sledeči trditvi:
\begin{itemize}
\itemsep 0mm
\item Klic funkcije {\sc bfs2}$(G, s)$ vrne vrednost, ki je večja ali enaka $g$.
\item Če vozlišče $s$ leži na najkrajšem ciklu v grafu $G$,
potem klic funkcije {\sc bfs2}$(G, s)$ vrne $g$.
\end{itemize}

\item Natančno opiši algoritem (z besedami ali psevdokodo),
ki izračuna ožino danega neusmerjenega grafa $G = (V, E)$.
Tvoj algoritem lahko kliče funkcijo {\sc bfs2}.
Upoštevaj, da veljajo trditve iz točk (a) in (b)
(tudi, če nista rešeni).

\item Oceni časovno zahtevnost algoritma iz točke (c)
v odvisnosti od števila vozlišč in povezav grafa $G = (V, E)$
(lahko uporabiš standardne oznake $n = |V|$, $m = |V| + |E|$).
Svojo oceno utemelji.

\item Na grafu $G = (V, E)$ s slike~\fig
izvedi klic funkcije {\sc bfs2}$(G, a)$,
pri čemer upoštevaj abecedni vrstni red sosedov vsakega vozlišča.
Iz odgovora naj bo jasen potek algoritma
(tj., za vsako spremembo naj bo jasno, v katerem koraku se zgodi).
\end{enumerate}

\begin{slika}
\pgfslika
\podnaslov[\nal{}(e)]{Graf}
\end{slika}
\end{vprasanje}

\begin{odgovor}
\begin{enumerate}[(a)]
\item Označimo poti $P_1 := u_i u_{i+1} \dots u_{j-1} u_j$
in $P_2 := u_i u_{i-1} \dots u_1 u_k \dots u_{j+1} u_j$.
Brez škode za splošnost lahko predpostavimo, da je dolžina poti $P_1$
manjša ali enaka dolžini poti $P_2$.
Denimo, da v grafu $G$ obstaja pot $P' = v_0 v_1 v_2 \dots v_\ell \ne P_1$,
kjer je $v_0 = u_i$ in $v_\ell = u_j$, ki je krajša od poti $P_2$.
Potem je $u_i u_{i+1} \dots u_j$ $v_{\ell-1} v_{\ell-2} \dots v_1$
obhod v grafu $G$,
ki sestoji iz poti $P_1$ in $P'$ in je torej krajši od $C$.
V grafu $G$ torej obstaja krajši cikel od $C$,
kar je v protislovju s predpostavko, da je $C$ najkrajši cikel v $G$.
Poti $P_1$ in $P_2$ sta torej najkrajši dve poti od $u_i$ do $u_j$ v grafu $G$.

\item Funkcija {\sc bfs2} se ustavi,
ko najde povezavo med vozliščema na razdalji $i-1$ od $s$ (tedaj vrne $2i-1$),
ali pa take povezave ne najde,
najde pa dve poti dolžine $i$ od $s$ do nekega vozlišča (tedaj vrne $2i$).
Če ne najde ničesar od tega,
potem povezana komponenta grafa $G$, ki vsebuje vozlišče $s$,
ne vsebuje ciklov (tedaj funkcija vrne $\infty$).
Funkcija torej vrne vrednost,
ki je večja ali enaka dolžini nekega cikla v grafu
(in torej večja ali enaka $g$),
oziroma $\infty$, če graf nima ciklov.

Denimo sedaj, da vozlišče $s$ leži na najkrajšem ciklu v grafu $G$,
in naj bo $t$ vozlišče, ki povzroči končanje algoritma
-- očitno mora vozlišče $t$ ležati na nekem najkrajšem ciklu $C$,
ki vsebuje vozlišče $s$.
Potem bodisi obstajata dve najkrajši poti od $s$ do $t$,
ali pa obstajata enolično določeni najkrajši poti enake dolžine
od $s$ do $t$ in vozlišča $t'$, ki je sosedno vozlišču $t$.
V obeh primerih morata biti obe poti po povezavah disjunktni,
sicer bi obstajal cikel, krajši od $C$.
Funkcija {\sc bfs2} v obeh primerih vrne ravno dolžino cikla $C$.

\needspace{\baselineskip}
\item Zapišimo psevdokodo za funkcijo {\sc Ožina}.
\begin{small}
\begin{algorithmic}
\Function{Ožina}{$G = (V, E)$}
    \State \Return $\min\{${\sc bfs2}$(G, s) \mid s \in V\}$
\EndFunction
\end{algorithmic}
\end{small}
Funkcija torej vrne ravno najmanjšo vrednost,
ki jo {\sc bfs2} vrne za graf $G$ in neko njegovo vozlišče,
ta pa je dosežena ravno za vozlišča na najkrajšem ciklu
in je enaka njegovi dolžini.

\item Pri klicu funkcije {\sc bfs2}
je vsako vozlišče največ enkrat v trenutnem nivoju
-- za vsakega od njih se pregledajo vsi njegovi sosedi.
Vsaka povezava grafa se torej obravnava največ dvakrat,
vse operacije z množicami pa tečejo v konstantnem času.
Funkcija {\sc bfs2} torej teče v času $O(m)$.

Funkcija {\sc Ožina} kliče funkcijo {\sc bfs2} za vsako vozlišče grafa
in vrne najmanjšo dobljeno vrednost.
Časovna zahtevnost funkcije {\sc Ožina} je torej $O(mn)$.

\item Potek izvajanja algoritma je prikazan v tabeli~\tab,
kjer so zabeležene spremembe vrednosti spremenljivk
(izpuščena je le množica $S$, ki sestoji iz vozlišč,
ki so se do tedaj pojavila kot $u$ ali $v$).
Zabeleženo je tudi, kdaj vsakič se zgodi pogoj,
ki vrednost $z$ nastavi na $\True$
(tj., tudi, ko že ima to vrednost).
Ker se to prvič zgodi v tretjem koraku (tj., pri $i = 3$)
in algoritem ne najde povezave med dvema vozliščema na istem nivoju,
se algoritem ustavi na koncu tretjega obhoda zanke {\bf while}
in vrne vrednost $2i = 6$.
\end{enumerate}

\begin{tabela}[htbp]
$$
\begin{array}{c|c|c|c|c|c}
i & T & u & v & N & z \\ \hline
1 & \{a\} & a && \emptyset & \False \\
&&& b & \{b\} \\
&&& h & \{b, h\} \\
&&& j & \{b, h, j\} \\
2 & \{b, h, j\} & b && \emptyset \\
&&& a & \emptyset \\
&&& c & \{c\} \\
&&& k & \{c, k\} \\
&& h && \{c, k\} \\
&&& a & \{c, k\} \\
&&& g & \{c, k, g\} \\
&&& q & \{c, k, g, q\} \\
&& j && \{c, k, g, q\} \\
&&& a & \{c, k, g, q\} \\
&&& m & \{c, k, g, q, m\} \\
&&& o & \{c, k, g, q, m, o\} \\
3 & \{c, k, g, q, m, o\} & c && \emptyset \\
&&& b & \emptyset \\
&&& d & \{d\} \\
&&& \ell & \{d, \ell\} \\
&& k && \{d, \ell\} \\
&&& b & \{d, \ell\} \\
&&& n & \{d, \ell, n\} \\
&&& p & \{d, \ell, n, p\} \\
&& g && \{d, \ell, n, p\} \\
&&& f & \{d, \ell, n, p, f\} \\
&&& h & \{d, \ell, n, p, f\} \\
&&& p & \{d, \ell, n, p, f\} & \True \\
&& q && \{d, \ell, n, p, f\} \\
&&& h & \{d, \ell, n, p, f\} \\
&&& \ell & \{d, \ell, n, p, f\} & \True \\
&&& n & \{d, \ell, n, p, f\} & \True \\
&& m && \{d, \ell, n, p, f\} \\
&&& d & \{d, \ell, n, p, f\} & \True \\
&&& j & \{d, \ell, n, p, f\} \\
&&& p & \{d, \ell, n, p, f\} & \True \\
&& m && \{d, \ell, n, p, f\} \\
&&& f & \{d, \ell, n, p, f\} & \True \\
&&& j & \{d, \ell, n, p, f\} \\
&&& \ell & \{d, \ell, n, p, f\} & \True
\end{array}
$$
\podnaslov[\res{}(e)]{Potek izvajanja algoritma}
\end{tabela}
\end{odgovor}
\end{naloga}
