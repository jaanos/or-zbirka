\begin{naloga}{Sergio Cabello}{Vaje OR 14.12.2016}
\begin{vprasanje}
Zapiši algoritem, ki ugotovi, ali ima graf več kot eno topološko ureditev.
\end{vprasanje}

\begin{odgovor}
Algoritem deluje tako,
da v vrsto najprej doda vozlišča z vhodno stopnjo $0$,
nato pa pobira vozlišča iz vrste in jih odstranjuje iz grafa,
pri čemer se v vrsto dodajajo vozlišča,
ki jim pri tem izhodna stopnja pade na $0$.
Če se zgodi, da je v nekem trenutku v vrsti več kot eno vozlišče,
si algoritem to zapomni
-- če algoritem tako obišče vsa vozlišča
(tj., obstaja topološka ureditev),
potem je topološka ureditev več kot ena in algoritem vrne $\True$.
Če pa je topološka ureditev ena sama ali pa te ni,
algoritem vrne $\False$.
\begin{small}
\begin{algorithmic}
\Function{VečTopo}{$G = (V, E)$}
	\State {\sl stopnja} $\gets$ slovar vhodnih stopenj vozlišč grafa $G$
	\State $Q \gets$ prazna vrsta
	\State {\sl več} $\gets \False$
	\State {\sl prvo} $\gets \False$
	\State $i \gets 0$ \hfill števec obiskanih vozlišč
	\For{$v \in V$} \hfill začetno iskanje izvorov
	    \If{{\sl stopnja}$[v] = 0$}
	        \State $Q.\push(v)$
	        \If{{\sl prvo}} \hfill če je bil predhodno najden izvor,
	            \State {\sl več} $\gets \True$ \hfill rešitev ni enolična
	        \Else
	            \State {\sl prvo} $\gets \True$ \hfill prvi najden izvor
	        \EndIf
	    \EndIf
	\EndFor
	\While{$\lnot Q.\isEmpty()$}
	    \State $u \gets Q.\pop()$
	    \State $i \gets i+1$ \hfill povečamo števec
	    \State {\sl prvo} $\gets \False$
	    \For{$v \in \Adj(G, u)$} \hfill izhodni sosedi vozlišča $u$
	        \State {\sl stopnja}$[v] \gets$ {\sl stopnja}$[v] - 1$
	        \If{{\sl stopnja}$[v] = 0$} \hfill nov izvor po odstranjevanju $u$
		        \State $Q.\push(v)$
		        \If{{\sl prvo}}
		            \State {\sl več} $\gets \True$
		        \Else
		            \State {\sl prvo} $\gets \True$
		        \EndIf
	        \EndIf
	    \EndFor
	\EndWhile
	\State \Return {\sl več}${} \land (i = |V|)$ \hfill vrni $\True$, če obstaja topološka ureditev in ta ni enolična
\EndFunction
\end{algorithmic}
\end{small}
\end{odgovor}
\end{naloga}
