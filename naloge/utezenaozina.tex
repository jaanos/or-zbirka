\begin{naloga}{Janoš Vidali}{Kolokvij OR 9.6.2025}
\begin{vprasanje}
Dan je utežen neusmerjen graf $G = (V, E, c)$
s cenami povezav $c_e > 0$ ($e \in E$).
Poiskati želimo {\em uteženo ožino} grafa,
tj., zanima nas cena najcenejšega cikla v grafu $G$.
Tukaj kot cikel razumemo zaporedje različnih vozlišč $u_1 u_2 \dots u_k$,
kjer je $k \ge 3$ in $u_{i-1} u_i \in E$ ($1 \le i \le k$, $u_0 = u_k$);
cena takega cikla je potem $\sum_{i=1}^k c_{u_{i-1} u_i}$.
Pri reševanju tega problema
si bomo pomagali s sledečo različico Dijkstrovega algoritma,
ki vsako vozlišče obravnava dvakrat:
\needspace{2\baselineskip}
\begin{small}
\begin{algorithmic}
\Function{Dijkstra2}{$G = (V, E, c),\ s \in V$}
	\State $d \gets$ prazen slovar \hfill slovar najmanjših cen poti
	\State $\pred \gets \set{(v, j): (\Null, \Null)}{v \in V, \ j \in \{1, 2\}}$
	\hfill slovar predhodnikov
	\State $Q \gets \PrioritetnaVrsta(\set{(v, j): \infty}{v \in V, \ j \in \{1, 2\}})$
	\hfill prioritetna vrsta \\ \hfill s podvojenimi vozlišči
	\State $Q[s, 1] \gets 0$ \hfill najmanjša cena do prve kopije $s$ je $0$
	\While{$Q$ ni prazna} \hfill ponavljamo, dokler priorietna vrsta ni prazna
		\State $(u, i), \ \ell \gets Q.\pop()$
		\hfill vzamemo kopijo vozlišča \\
		\hfill z najmanjšo ceno poti v prioritetni vrsti
		\State $d[u, i] \gets \ell$ \hfill shranimo dobljeno ceno poti
		\State $w, h \gets \pred[u, i]$
		\hfill dobimo predhodnika trenutne kopije vozlišča
		\For{$v \in \Adj(G, u)$} \hfill pregledamo sosede trenutnega vozlišča
			\If{$v = w$} \hfill do predhodnega vozlišča se ne vračamo
				\State {\bf continue}
			\EndIf
			\State $r \gets \ell + c_{uv}$
			\hfill izračunamo ceno nove poti preko $u$
			\If{$i = 1 \land (v, 1) \in Q \land Q[v, 1] > r$}
				\hfill če smo pri prvi kopiji vozlišča
				\State $Q[v, 2] \gets Q[v, 1]$
				\hfill ter smo dobili manjšo ceno poti do prve kopije soseda,
				\State $\pred[v, 2] \gets \pred[v, 1]$
				\hfill potem premaknemo trenutno ceno poti
				\State $Q[v, 1] \gets r$
				\hfill in predhodnika prve kopije k drugi kopiji
				\State $\pred[v, 1] \gets (u, i)$
				\hfill ter ju ustrezno nastavimo pri prvi
			\ElsIf{$(v, 2) \in Q \land Q[v, 2] > r$}
			\hfill sicer, če smo dobili manjšo ceno poti
				\State $Q[v, 2] \gets r$
				\hfill kot do druge kopije,
				\State $\pred[v, 2] \gets (u, i)$
				\hfill ustrezno popravimo predhodnika in ceno poti
			\EndIf
		\EndFor
	\EndWhile
	\State \Return $(d, \pred)$
	\hfill vrnemo najmanjše cene in predhodnike za obe kopiji
\EndFunction
\end{algorithmic}
\end{small}
Tukaj je $\PrioritetnaVrsta$ podatkovna struktura,
ki hrani pare ključev in vred\-no\-sti,
pri čemer omogoča sledeče operacije:
\begin{itemize}
\itemsep 0mm
\item $Q \gets \PrioritetnaVrsta$({\sl slovar})
v času $O(n)$ konstruira prioritetno vrsto $Q$,
ki hrani ključe in vrednosti iz podanega slovarja;
\item $Q.\isEmpty()$ v času $O(1)$ pove, ali je prioritetna vrsta prazna;
\item $k \in Q$ v času $O(1)$ pove,
ali prioritetna vrsta hrani vnos s ključem $k$;
\item $Q[k]$ v času $O(1)$ vrne vrednost pri ključu $k$;
\item $Q[k] \gets v$ v času $O(\log n)$ zamenja vrednost pri ključu $k$ z $v$;
in
\item $k, v \gets Q.\pop()$ v času $O(\log n)$
iz prioritetne vrste odstrani ter vrne par $(k, v)$,
kjer je $v$ najmanjša hranjena vrednost ter $k$ pripadajoči ključ
(če je takih ključev več, se odstrani in vrne samo en vnos).
\end{itemize}
V zgornjih časovnih zahtevnostih $n$ predstavlja
število vnosov v prioritetni vrsti.
Ključi, uporabljeni pri prioritetni vrsti $Q$ v zgornjem algoritmu
(kot tudi pri slovarjih $d$ in $\pred$),
so urejeni pari iz $V \times \{1, 2\}$.

\begin{enumerate}[(a)]
\item Naj bo $C = u_1 u_2 \dots u_k$ najcenejši cikel v grafu $G$.
Dokaži, da za poljubna $i, j$ ($1 \le i \le j \le k$) velja,
da sta $u_i u_{i+1} \dots u_{j-1} u_j$
in $u_i u_{i-1} \dots u_1 u_k \dots u_{j+1} u_j$
najcenejši dve poti od $u_i$ do $u_j$ v grafu $G$.

\item Naj bo $g$ cena najcenejšega cikla v grafu $G$.
Utemelji sledeči trditvi:
\begin{itemize}
\itemsep 0mm
\item Na koncu izvajanja klica funkcije {\sc Dijkstra2}$(G, s)$
velja $d[s, 2] \ge g$.
\item Če vozlišče $s$ leži na najcenejšem ciklu v grafu $G$,
potem na koncu izvajanja klica funkcije {\sc Dijkstra2}$(G, s)$
velja $d[s, 2] = g$.
\end{itemize}

\item Na grafu $G = (V, E, c)$ s slike~\fig
s cenami povezav $c_e$ ($e \in E$), označenimi na povezavah,
izvedi klic funkcije {\sc Dijkstra2}$(G, a)$.
Iz odgovora naj bo jasen potek algoritma
(tj., za vsako spremembo naj bo jasno, v katerem koraku se zgodi).

\item Natančno opiši algoritem (z besedami ali psevdokodo),
ki izračuna uteženo ožino danega uteženega neusmerjenega grafa $G = (V, E, c)$.
Tvoj algoritem lahko kliče funkcijo {\sc Dijkstra2}.
Upoštevaj, da veljajo trditve iz točk (a) in (b)
(tudi, če nista rešeni).

\item Oceni časovno zahtevnost algoritma {\sc Dijkstra2}
in algoritma iz točke (d)
v odvisnosti od števila vozlišč in povezav grafa $G = (V, E, c)$
(lahko uporabiš standardne oznake $n = |V|$, $m = |V| + |E|$).
Svojo oceno utemelji.
\end{enumerate}

\begin{slika}
\pgfslika
\podnaslov[\nal{}(c)]{Graf}
\end{slika}
\end{vprasanje}

\begin{odgovor}
\begin{enumerate}[(a)]
\item Označimo poti $P_1 := u_i u_{i+1} \dots u_{j-1} u_j$
in $P_2 := u_i u_{i-1} \dots u_1 u_k \dots u_{j+1} u_j$.
Brez škode za splošnost lahko predpostavimo,
da je cena poti $P_1$ manjša ali enaka ceni poti $P_2$.
Denimo, da v grafu $G$ obstaja pot $P' = v_0 v_1 v_2 \dots v_\ell \ne P_1$,
kjer je $v_0 = u_i$ in $v_\ell = u_j$, ki je cenejša od poti $P_2$.
Potem je $u_i u_{i+1} \dots u_j$ $v_{\ell-1} v_{\ell-2} \dots v_1$
obhod v grafu $G$,
ki sestoji iz poti $P_1$ in $P'$ in je torej cenejši od $C$.
Ker so cene povezav pozitivne,
v grafu $G$ torej obstaja cenejši cikel od $C$,
kar je v protislovju s predpostavko, da je $C$ najcenejši cikel v $G$.
Poti $P_1$ in $P_2$ sta torej najcenejši dve poti
od $u_i$ do $u_j$ v grafu $G$.

\item Ob koncu izvajanja funkcije {\sc Dijkstra2}
so vrednosti $d[u, 1]$ ($u \in V$)
enake cenam najcenejših poti (in tako tudi sprehodov) od $s$ do $u$,
vrednosti $d[u, 2]$ ($u \in V$)
pa so enake cenam drugih najkrajših sprehodov od $s$ do $u$,
pri katerih se nikoli ne vrnemo v neposrednega predhodnika nekega vozlišča.
Vrednost $d[s, 2]$ je torej enaka ceni
najcenejšega obhoda preko vozlišča $s$ z vsaj eno povezavo
brez vračanja v neposredne predhodnike
-- ta sestoji iz nekega cikla $C$
in poti med $s$ in nekim vozliščem na $C$ (morda z $0$ povezavami).
Ker so cene povezav pozitivne, tako velja $d[s, 2] \ge g$.

Denimo sedaj, da vozlišče $s$ leži na najcenejšem ciklu v grafu $G$.
Ker so cene povezav pozitivne,
je ta cikel tudi najcenejši obhod skozi vozlišče $s$ z vsaj eno povezavo,
in tako velja $d[s, 2] = g$.

\item Potek izvajanja algoritma je prikazan v tabeli~\tab.
Iz nje razberemo,
da je najcenejši obhod preko vozlišča $a$ z vsaj eno povezavo
brez vračanja v predhodnike $a - d - e - f - a$ s ceno $7$.

\item Zapišimo psevdokodo za funkcijo {\sc UteženaOžina}.
\begin{small}
\begin{algorithmic}
\Function{UteženaOžina}{$G = (V, E, c)$}
	\State $g \gets \infty$
	\For{$u \in V$}
		\State $d, \pred \gets \text{\sc{Dijkstra2}}(G, u)$
		\State $g \gets \min\{g, d[u, 2]\}$
	\EndFor
    \State \Return $g$
\EndFunction
\end{algorithmic}
\end{small}
Funkcija torej vrne ravno najmanjšo vrednost $d[u, 2]$,
ki jo funkcija {\sc Dijkstra2} vrne za graf $G$
in neko njegovo vozlišče $u$,
ta pa je dosežena ravno za vozlišča na najcenejšem ciklu
in je enaka njegovi ceni.

\item Pri klicu funkcije {\sc Dijkstra2}
se vsako vozlišče iz prioritetne vrste vzame dvakrat
-- za vsakega od njih se pregledajo vsi njegovi sosedi.
Vsaka povezava grafa se torej obravnava največ dvakrat,
vse operacije na vrsti pa tečejo v času $O(\log n)$.
Funkcija {\sc Dijkstra2} torej teče v času $O(m \log n)$.

Funkcija {\sc UteženaOžina} kliče funkcijo {\sc Dijkstra2}
za vsako vozlišče $u$ grafa in vrne najmanjšo dobljeno vrednost $d[u, 2]$.
Časovna zahtevnost funkcije {\sc UteženaOžina} je torej $O(mn \log n)$.
\end{enumerate}

\begin{tabela}[htbp]
$$
\begin{array}{cccccccccccc}
(a, 1) & (a, 2) & (b, 1) & (b, 2) & (c, 1) & (c, 2) & (d, 1) & (d, 2) & (e, 1) & (e, 2) & (f, 1) & (f, 2) \\ \hline
0 & \infty & \infty & \infty & \infty & \infty & \infty & \infty & \infty & \infty & \infty & \infty \\
* && 4_{a,1} &&&& 2_{a,1} &&&& 2_{a,1} & \\
&&&& 3_{d,1} && * && 4_{d,1} &&& \\
&&&&& 9_{f,1} &&& 3_{f,1} & 4_{d,1} & * & \\
&&& 15_{c,1} & * &&&&&&& 10_{c,1} \\
&&& 14_{e,1} &&&& 5_{e,1} & * &&& \\
&& * &&&&&&&&& \\
&&&&&&&&& * && 5_{e,2} \\
& 7_{d,2} &&&& 6_{d,2} && * &&&& \\
&&&&&&&&&&& * \\
&&&&& * &&&&&& \\
& * && 11_{a,2} &&&&&&&& \\
&&& * &&&&&&&&
\end{array}
$$
\podnaslov[\res{}(c)]{Potek izvajanja algoritma}
\end{tabela}
\end{odgovor}
\end{naloga}
