\begin{naloga}{Alen Orbanić}{Izpit OR 5.2.2010}
\begin{vprasanje}
Naša naloga je,
da med pet ljudi z uporabo denarnih nadomestil razdelimo pet predmetov,
tako da bo vsak dobil pošten delež in da bo delitev brez zavisti.
V spodnji matriki je podano vrednotenje vsakega izmed 5 predmetov
s strani petih ljudi.
$$
\begin{bmatrix}
11 &  9 & 19 & 14 & 11 \\
 4 & 10 & 15 & 20 & 12 \\
 6 &  5 &  5 & 17 & 9  \\
20 & 23 &  6 &  3 & 18 \\
10 & 18 & 23 &  9 & 10
\end{bmatrix}
$$
Seveda želimo pri tem kot svetovalec zaslužiti čimveč.
Ker ni zunanjega financiranja, lahko naš zaslužek pridobimo le iz nadomestil.
Uporabi kak znan algoritem za delitev pod temi pogoji in razloži,
kako si pridobiš kar največji dobiček.
\end{vprasanje}

\begin{odgovor}
Najprej z madžarsko metodo z utežmi poiščemo tako razdelitev predmetov,
da bo skupna cena čim večja.
\begin{alignat*}{2}
\begin{bmatrix}
11 &  9 & 19 & 14 & 11 \\
 4 & 10 & 15 & 20 & 12 \\
 6 &  5 &  5 & 17 & 9  \\
20 & 23 &  6 &  3 & 18 \\
10 & 18 & 23 &  9 & 10
\end{bmatrix}
&\to
\begin{bmatrix}
 8 & 10 &  0 &  5 &  8 \\
16 & 10 &  5 &  0 &  8 \\
11 & 12 & 12 &  0 &  8 \\
 3 &  0 & 17 & 20 &  5 \\
13 &  5 &  0 & 14 & 13
\end{bmatrix}
&&\to
\begin{bmatrix}
 5 & 10 & \underline{ 0} & \underline{ 5} & 3 \\
13 & 10 & \underline{ 5} & \underline{ 0} & 3 \\
 8 & 12 & \underline{12} & \underline{ 0} & 3 \\
\underline{0} & \underline{0} & \underline{\underline{17}}
& \underline{\underline{20}} & \underline{0} \\
10 &  5 & \underline{ 0} & \underline{14} & 8
\end{bmatrix}
\to \\ \to
\begin{bmatrix}
 2 & 7 & \underline{ 0} & \underline{ 5} & \underline{0} \\
10 & 7 & \underline{ 5} & \underline{ 0} & \underline{0} \\
 5 & 9 & \underline{12} & \underline{ 0} & \underline{0} \\
\underline{0} & \underline{0} & \underline{\underline{20}}
& \underline{\underline{23}} & \underline{\underline{0}} \\
 7 & 2 & \underline{ 0} & \underline{14} & \underline{5}
\end{bmatrix}
&\to
\begin{bmatrix}
\underline{0} & 5 &  0 &  5 & 0 \\
 8 & 5 &  5 & \underline{0} & 0 \\
 3 & 7 & 12 &  0 & \underline{0} \\
 0 & \underline{0} & 22 & 25 & 2 \\
 5 & 0 & \underline{0} & 14 & 5
\end{bmatrix}
&&\Rightarrow
\begin{bmatrix}
\underline{11} &  9 & 19 & 14 & 11 \\
 4 & 10 & 15 & \underline{20} & 12 \\
 6 &  5 &  5 & 17 & \underline{9} \\
20 & \underline{23} &  6 &  3 & 18 \\
10 & 18 & \underline{23} &  9 & 10
\end{bmatrix}
\end{alignat*}
Dobimo razdelitev predmetov s skupno ceno $86$.

Da bomo predmete razdelili brez zavisti,
bomo osebi $j$ ($1 \le j \le 5$) dodelili popust $d_j$.
Smatramo, da oseba $i$ zavida osebi $j$ ($1 \le i, j \le 5$),
če je razlika med vrednostjo dodeljenega predmeta po cenitvi osebe $i$
in plačano vrednostjo večja pri osebi $j$ kot pri osebi $i$.
Definirajmo torej
$$
a_{ij} = b_{ij} - b_{jj} + d_j
\quad (1 \le i, j \le 5),
$$
kjer je $b_{ij}$ cenitev predmeta, dodeljenega osebi $j$, s strani osebe $i$.
Opazimo, da velja $a_{jj} = d_j$ ($1 \le j \le 5$).
Iščemo matriko $A = (a_{ij})_{i,j=1}^5$
z minimalno vsoto diagonalnih elementov,
za katero velja $a_{ij} \le a_{ii}$ ($1 \le i, j \le 5$).

Postopali bomo tako, da bomo na začetku popuste nastavili na $0$,
nato pa za osebi $i, j$ z $a_{ij} > a_{ii}$
povečali popust osebe $i$ za $a_{ij} - a_{ii}$ in postopek ponavljali,
dokler zavisti več ne bo.
\begin{align*}
\begin{bmatrix}
 \mathbf{0} &  -6 & \underline{2} & -14 &  -4 \\
-7 &   \mathbf{0} & \underline{3} & -13 &  -8 \\
-5 &  -3 & \mathbf{0} & -18 & -18 \\
\underline{9} & -17 & \underline{9} &   \mathbf{0} & -17 \\
-1 & -11 & \underline{1} &  -5 &   \mathbf{0}
\end{bmatrix}
&\to
\begin{bmatrix}
 \mathbf{2} &  -3 & 2 & -5 &  -3 \\
-5 &   \mathbf{3} & 3 & -4 &  -7 \\
-3 &   0 & \mathbf{0} & -9 & -17 \\
\underline{11} & -14 & 9 &  \mathbf{9} & -16 \\
 1 &  -8 & 1 &  \underline{4} &   \mathbf{1}
\end{bmatrix}
\to \\ \to
\begin{bmatrix}
 \mathbf{2} &  -3 & 2 & -3 &   0 \\
-5 &   \mathbf{3} & 3 & -2 &  -4 \\
-3 &   0 & \mathbf{0} & -7 & -14 \\
11 & -14 & 9 &  \mathbf{11} & -13 \\
 1 &  -8 & 1 &  \underline{6} &   \mathbf{4}
\end{bmatrix}
&\to
\begin{bmatrix}
 \mathbf{2} &  -3 & 2 & -3 &   2 \\
-5 &   \mathbf{3} & 3 & -2 &  -2 \\
-3 &   0 & \mathbf{0} & -7 & -12 \\
11 & -14 & 9 &  \mathbf{11} & -11 \\
 1 &  -8 & 1 &  6 &   \mathbf{6}
\end{bmatrix}
\end{align*}
Prvi osebi bomo torej dali prvi predmet po ceni $11 - 2 = 9$,
drugi osebi četrti predmet po ceni $20 - 3 = 17$,
tretji osebi peti predmet po ceni $9 - 0 = 9$,
četrti osebi drugi predmet po ceni $23 - 11 = 12$,
peti osebi pa tretji predmet po ceni $23 - 6 = 17$.
Skupni zaslužek iz nadomestil je tako $64$.
\end{odgovor}
\end{naloga}
