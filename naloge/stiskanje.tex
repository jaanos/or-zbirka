\begin{naloga}{Alen Orbanić}{Izpit OR 28.6.2010}
\begin{vprasanje}
Podatke (nize) stiskamo na naslednji način.
Podana je tabela $m$ nizov, dolžina vsakega je kvečjemu $k$.
Radi bi zakodirali podatkovni niz $D$ dolžine $n$
z najkrajšim možnim zaporedjem nizov iz tabele.
Npr., če naša tabela vsebuje nize $(a, ba, abab, b)$
in bi radi zakodirali niz $bababbaababa$,
je najboljši način kodiranja zaporedje $(b, abab, ba, abab, a)$,
torej s petimi kodnimi besedami.

\begin{enumerate}[(a)]
\item S pomočjo strategije dinamičnega programiranja
izpelji čim bolj učinkovit algoritem,
ki poišče najkrajše kodiranje danega niza $D$,
če kodiranje za tak niz obstaja.

\item Demonstriraj uporabo algoritma na podanem nizu $bababbaababa$
s podano tabelo $(a, ba, abab, b)$.

\item Kakšna je časovna zahtevnost tvoje rešitve?
\end{enumerate}
\end{vprasanje}

\begin{odgovor}
\begin{enumerate}[(a)]
\item Denimo,
da imamo v tabeli nize $w_h$ ($1 \le h \le m$) dolžine $\ell_h \le k$,
ter da imamo podan niz $D = (u_i)_{i=1}^n$.
Naj bo $d_j$ dolžina najkrajšega kodiranja niza $D_j = (u_i)_{i=1}^j$
s kodnimi besedami iz tabele.
Zapišimo začetni pogoj in rekurzivne enačbe.
\begin{alignat*}{2}
d_0 &= 0 \\
d_j &= 1 + &{} \min\left(\set{d_{j-\ell_h}}{1 \le h \le m \land
                              \ell_h \le j \land D_j = D_{j-\ell_h} \| w_h}
                         \cup \{\infty\}\right) \qquad \\
&& (1 \le j \le n)
\end{alignat*}
Vrednosti $d_j$ računamo v naraščajočem vrstnem redu glede na indeks $j$.
Če velja $d_n = \infty$, potem ustrezno kodiranje ne obstaja,
sicer pa dolžino najkrajšega kodiranja dobimo kot $d^* = d_n$.

\item Izračunajmo vrednosti $d_j$ ($1 \le j \le n$).
Ker niz $D$ sestoji samo iz znakov $a$ in $b$,
ki se kot niza pojavita v tabeli,
bodo vse vrednosti $d_j$ končne.
\begin{alignat*}{3}
u_1    &= b &\qquad d_1    &= 1 + \min\{\underline{0_b}\} &&= 1 \\
u_2    &= a &\qquad d_2    &= 1 + \min\{1_a, \underline{0_{ba}}\} &&= 1 \\
u_3    &= b &\qquad d_3    &= 1 + \min\{\underline{1_b}\} &&= 2 \\
u_4    &= a &\qquad d_4    &= 1 + \min\{2_a, \underline{1_{ba}}\} &&= 2 \\
u_5    &= b &\qquad d_5    &= 1 + \min\{2_b, \underline{1_{abab}}\} &&= 2 \\
u_6    &= b &\qquad d_6    &= 1 + \min\{\underline{2_b}\} &&= 3 \\
u_7    &= a &\qquad d_7    &= 1 + \min\{3_a, \underline{2_{ba}}\} &&= 3 \\
u_8    &= a &\qquad d_8    &= 1 + \min\{\underline{3_a}\} &&= 4 \\
u_9    &= b &\qquad d_9    &= 1 + \min\{\underline{4_b}\} &&= 5 \\
u_{10} &= a &\qquad d_{10} &= 1 + \min\{5_a, \underline{4_{ba}}\} &&= 5 \\
u_{11} &= b &\qquad d_{11} &= 1 + \min\{5_b, \underline{3_{abab}}\} &&= 4 \\
u_{12} &= a &\qquad d_{12} &= 1 + \min\{\underline{4_a}, 5_{abab}\} &&= 5
\end{alignat*}
Dolžina najkrajšega kodiranja je torej $d^* = d_{12} = 5$.
Rekonstruirajmo še ustrezno kodiranje.
\begin{align*}
d_{12} &= 1 + d_{11} & a \\
d_{11} &= 1 + d_7    & abab \\
d_7    &= 1 + d_5    & ba \\
d_5    &= 1 + d_1    & abab \\
d_1    &= 1 + d_0    & b
\end{align*}
Najkrajše kodiranje je torej res $(b, abab, ba, abab, a)$.

\item V algoritmu izračunamo $n$ spremenljivk,
pri izračunu vsake pa primerjamo $m$ nizov,
pri čemer vsakič primerjamo največ $k$ znakov.
Časovna zahtevnost algoritma je torej $O(nmk)$.
\end{enumerate}
\end{odgovor}
\end{naloga}
