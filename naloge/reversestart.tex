\begin{naloga}{Bašić, Gajser}{Izpit OR 4.9.2012}
\begin{vprasanje}
Dr.~Kaczyński se ukvarja s podatkovno strukturo $A$,
ki je zelo podobna tabeli (angl.~{\em array}) celih števil.
Vrednosti v posameznih celicah ``tabele'' $A$ lahko beremo,
ne moremo pa jih spreminjati.
Elementi ``tabele'' $A$ so indeksirani s celimi števili od $1$ do $n$,
kjer je $n = A.\length()$ indeks zadnjega elementa ``tabele'' $A$.
Edina operacija (poleg dostopanja do posameznih elementov),
ki jo lahko izvajamo nad $A$, je $A.\reverseStart(i)$.
Če ima na začetku $A$ vrednosti
$$
A = [a_1, a_2, \dots, a_{i-1}, a_i, a_{i+1}, \dots, a_{n-1}, a_n],
$$
po klicu ukaza $A.\reverseStart(i)$ izgleda takole:
$$
A = [a_i, a_{i-1}, \dots, a_2, a_1, a_{i+1}, \dots, a_{n-1}, a_n] .
$$
Dr.~Kaczyński se je lotil implementacije algoritmov
nad to podatkovno strukturo.
Najprej je seveda implementiral algoritem za urejanje:
\begin{small}
\begin{algorithmic}
\State $n \gets A.\length()$
\For{$i = n, \dots, 2$}
    \For{$j = 1, \dots, i-1$}
        \If{$A[j] > A[i]$}
            \State $A.\reverseStart(j)$
            \State $A.\reverseStart(i)$
        \EndIf
    \EndFor
\EndFor
\end{algorithmic}
\end{small}

\begin{enumerate}[(a)]
\item Oceni časovno zahtevnost zgornjega algoritma.
Upoštevaj, da se operacija $A.\reverseStart(i)$ izvede v konstantnem času.

\item Algoritem izvedi na ``tabeli'' $A = [5, 9, 12, 7, 15]$.
Ali deluje pravilno?

\item Napiši algoritem za urejanje, ki bo deloval pravilno.
Njegova časovna zahtevnost ne sme biti slabša
od časovne zahtevnosti algoritma dr.~Kaczyńskega.
\end{enumerate}
\end{vprasanje}
\begin{odgovor}
\end{odgovor}
\end{naloga}
