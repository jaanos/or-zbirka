\begin{naloga}{Alen Orbanić}{Izpit OR 15.9.2010}
\begin{vprasanje}
Letalska družba namerava nabaviti že rabljeno letalo, ki stane $170\,000 €$.
Ocenjujejo, da bodo z njim imeli,
če je odlično ohranjeno, $1\,000\,000 €$ dobička,
če je zadovoljivo ohranjeno, $340\,000 €$ dobička,
in če je slabo ohranjeno, le $10\,000 €$ dobička.
Verjetnosti, da je letalo odlično, zadovoljivo ali slabo ohranjeno,
so zaporedoma $0.2$, $0.3$ in $0.5$.
\begin{enumerate}[(a)]
\item Modeliraj problem v okviru teorije odločanja (stanja, odločitve).
Kakšno odločitev svetuješ vodstvu družbe?

\item Družba lahko naroči oceno letala pri izvedenski firmi,
ki zahteva za svoje poročilo $10\,000 €$.
Vodstvo družbe takole ocenjuje pogojne verjetnosti:
\begin{center}
\begin{tabular}{c|ccc}
$P(\text{rezultat poročila} \;|\;\ \text{kakovost letala})$
& odlično & zadovoljivo & slabo \\ \hline
ugodno & 0.9 & 0.6 & 0.1 \\
neugodno & 0.1 & 0.4 & 0.9
\end{tabular}
\end{center}
Kako naj se vodstvo družbe odloči?
\end{enumerate}
\end{vprasanje}

\begin{odgovor}
\begin{enumerate}[(a)]
\item Imamo eno samo odločitev -- ali naj nabavimo letalo.
Če ga ne nabavimo, imamo $0 €$ dobička,
če pa ga, pa preidemo v stanje,
v katerem imamo z verjetnostjo $0.2$
dobiček $1\,000\,000 € - 170\,000 € = 830\,000 €$,
z verjetnostjo $0.3$ dobiček $340\,000 € - 170\,000 € = 170\,000 €$,
z verjetnostjo $0.5$ pa dobiček $10\,000 € - 170\,000 € = -160\,000 €$.
V tem primeru je pričakovani dobiček enak
$$
0.2 \cdot 830\,000 € + 0.3 \cdot 170\,000 € - 0.5 \cdot 160\,000 €
= 137\,000 € .
$$
Letalski družbi se torej nabava letala izplača,
pri tem pa pričakuje dobiček $137\,000 €$.

\item Izračunajmo verjetnosti za primer, da naročimo izvedensko mnenje.
\begin{align*}
P(\text{ugodno}) = 0.9 \cdot 0.2 + 0.6 \cdot 0.3 + 0.1 \cdot 0.5 &= 0.41 \\
P(\text{neugodno}) = 0.1 \cdot 0.2 + 0.4 \cdot 0.3 + 0.9 \cdot 0.5 &= 0.59 \\
P(\text{odlično} \mid \text{ugodno}) = {0.9 \cdot 0.2 \over 0.41}
&\approx 0.439 \\
P(\text{zadovoljivo} \mid \text{ugodno}) = {0.6 \cdot 0.3 \over 0.41}
&\approx 0.439 \\
P(\text{slabo} \mid \text{ugodno}) = {0.1 \cdot 0.5 \over 0.41}
&\approx 0.122 \\
P(\text{odlično} \mid \text{neugodno}) = {0.1 \cdot 0.2 \over 0.59}
&\approx 0.034 \\
P(\text{zadovoljivo} \mid \text{neugodno}) = {0.4 \cdot 0.3 \over 0.59}
&\approx 0.203 \\
P(\text{slabo} \mid \text{neugodno}) = {0.9 \cdot 0.5 \over 0.59}
&\approx 0.763
\end{align*}
S pomočjo zgoraj izračunanih verjetnosti
lahko narišemo odločitveno drevo s slike~\fig.
Opazimo,
da se nam izplača naročiti izvedensko mnenje,
ter letalo nabaviti, če je to ugodno,
in ga ne nabaviti, če mnenje ni ugodno.
\end{enumerate}
%
\begin{slika}
\makebox[\textwidth][c]{
\pgfslika
}
\podnaslov{Odločitveno drevo}
\end{slika}
\end{odgovor}
\end{naloga}
