\begin{naloga}{Janoš Vidali}{Izpit OR 19.6.2019}
\begin{vprasanje}
Dano je odločitveno drevo s slike~\fig,
pri čemer velja $0 \le p \le 1/2$.
Pričakovano vred\-nost želimo minimizirati.
Poišči optimalne odločitve in pričakovano vrednost
v odvisnosti od vrednosti parametra $p$.

\begin{slika}
\pgfslika
\podnaslov{Odločitveno drevo}
\end{slika}
\end{vprasanje}

\begin{odgovor}
Izračunajmo najprej pričakovane vrednosti v vozliščih $F$ in $G$
odločitvenega drevesa s slike~\fig.
\begin{alignat*}{2}
F &= 160 \cdot p + 40 \cdot (1-p) &&= 120 p + 40 \\
G &= 120 \cdot 2p + 20 \cdot (1-2p) &&= 200 p + 20
\end{alignat*}

Obravnavajmo najprej odločitev v vozlišču $D$.
Za pot k vozlišču $F$ se odločimo,
če velja $120p + 40 < 60$ oziroma $p < 1/6$.

Obravnavajmo sedaj še odločitev v vozlišču $E$.
Za pot k vozlišču $G$ se odločimo,
če velja $200p + 20 < 50$ oziroma $p < 3/20$.

\needspace{\baselineskip}
Pričakovana vrednost v vozlišču $B$ bo sledeča.
\begin{align*}
p &< {1 \over 6}: & 0.7 \cdot (120p + 40) + 0.3 \cdot (100p + 30) &= 114p + 37
\\
p &\ge {1 \over 6}: & 0.7 \cdot 60 + 0.3 \cdot (100p + 30) &= 30p + 51
\end{align*}

Izračunajmo še pričakovano vrednost v vozlišču $C$.
\begin{align*}
p &< {3 \over 20}: & 0.6 \cdot (200p + 20) + 0.4 \cdot 70 &= 120p + 40
\\
p &\ge {3 \over 20}: & 0.6 \cdot 50 + 0.4 \cdot 70 &= 58
\end{align*}

Nazadnje obravnavajmo še odločitev v vozlišču $A$.
\begin{align*}
0 \le p &< {3 \over 20}: & \min \{ 114p + 37, 120p + 40 \} &= 114p + 37 \\
{3 \over 20} \le p &< {1 \over 6} : & \min \{ 114p + 37, 58 \} &= 114p + 37 \\
{1 \over 6} \le p &\le {1 \over 2} : & \min \{ 30p + 51, 58 \} &= \begin{cases}
30p + 51 & p < {7 \over 30} \\
58 & p \ge {7 \over 30}
\end{cases}
\end{align*}

Proces odločanja je prikazan na sliki~\fig[dectree4-resitev].
Ker velja ${7 \over 30} > {3 \over 20}$, opazimo,
da v primeru $p \ge {7 \over 30}$ iz vozlišča $E$
nikoli ne izberemo poti proti vozlišču $G$.
%
\begin{slika}
\pgfslika[dectree4-resitev]
\podnaslov[\res v odvisnosti od vrednosti parametra $p$]{Odločitveno drevo}
\end{slika}
\end{odgovor}
\end{naloga}
