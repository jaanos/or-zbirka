\begin{naloga}{Janoš Vidali}{Izpit OR 26.6.2024}
\begin{vprasanje}
Na upravni enoti želijo zaradi povečanega števila zahtevkov občanov
optimizirati delovni proces,
kar bodo storili s prerazporejanjem zaposlenih po oddelkih.
V vsaki izmeni dela $n$ zaposlenih,
ki jih želijo razporediti med $m$ oddelkov,
pri čemer mora biti na vsak oddelek dodeljen vsaj en za\-pos\-le\-ni.
Določili so vrednosti $z_{ij}$ ($1 \le i \le m$, $1 \le j \le n$),
ki ocenjujejo število zahtevkov,
ki jih lahko v eni uri obdelajo na $i$-tem oddelku,
če tja dodelijo $j$ zaposlenih.
Poiskati želijo tako razporeditev zaposlenih med oddelke,
da bodo skupno lahko obdelali čim več zahtevkov.
\begin{enumerate}[(a)]
\item Zapiši rekurzivne enačbe za reševanje danega problema.
Izrazi časovno zahtevnost algoritma, ki sledi iz enačb,
v odvisnosti od parametrov $m$ in $n$.

\item S svojimi rekurzivnimi enačbami reši problem
pri podatkih $n = 8$, $m = 3$ in vred\-no\-stih $z_{ij}$ iz spodnje tabele.
$$
\begin{array}{r|cccccc}
z_{ij} & 1 & 2 & 3 & 4 & 5 & 6 \\ \hline
\text{Oddelek za javni red in promet} &  3 &  6 &  8 & 10 & 12 & 13 \\
\text{Oddelek za matične zadeve}      &  4 &  8 & 11 & 14 & 15 & 16 \\
\text{Oddelek za okolje in prostor}   &  2 &  3 &  4 &  5 &  6 &  7
\end{array}
$$
\end{enumerate}
\end{vprasanje}

\begin{odgovor}
\begin{enumerate}[(a)]
\item Naj bo $p_{ij}$ največje število zahtevkov,
ki jih lahko v eni uri obdelajo v prvih $i$ oddelkih,
če nanje dodelijo $j$ zaposlenih.
Določimo začetne pogoje in rekurzivne enačbe.
\begin{align*}
p_{1j} &= z_{1j} && (1 \le j \le n-m+1) \\
p_{ij} &= \max\set{p_{i-1,h} + z_{i,j-h}}{i-1 \le h \le j-1}
&& (2 \le i \le m, \ i \le j \le n-m+i)
\end{align*}
Vrednosti $p_{ij}$ računamo v leksikografskem vrstnem redu glede na $(i, j)$.
Največje skupno število zahtevkov, ki jih lahko obdelajo v eni uri,
dobimo kot $p^* = p_{mn}$.
Algoritem, ki sledi iz zgornjih enačb, teče v času $O(mn^2)$.

\item Izračunajmo tiste končne vrednosti $p_{ij}$ ($i \ge 2$),
ki jih potrebujemo za reševanje problema.
Pri tem smo oddelke oštevilčili z indeksi $1, 2, 3$
v podanem vrstnem redu v tabeli.
\needspace{2\baselineskip}
\begin{alignat*}{2}
p_{22} &= 3+4 &&= 7 \\
p_{23} &= \max\{\underline{3+8}, 6+4\} &&= 11 \\
p_{24} &= \max\{\underline{3+11}, \underline{6+8}, 8+4\} &&= 14 \\
p_{25} &= \max\{\underline{3+14}, \underline{6+11}, 8+8, 10+4\} &&= 17 \\
p_{26} &= \max\{3+15, \underline{6+14}, 8+11, 10+8, 12+4\} &&= 20 \\
p_{27} &= \max\{3+16, 6+15, \underline{8+14}, 10+11, 12+8, 13+4\} &&= 22 \\
p_{38} &= \max\{7+7, 11+6, 14+5, 17+4, 20+3, \underline{22+2}\} &&= 24
\end{alignat*}
Največ lahko torej v eni uri obdelajo $p^* = p_{38} = 24$ zahtevkov.
Poiščimo še optimalno razporeditev po oddelkih.
\begin{align*}
p_{38} &= p_{27} + z_{31} && \text{$1$ zaposleni na Oddelku za okolje in prostor} \\
p_{27} &= p_{13} + z_{24} && \text{$4$ zaposleni na Oddelku za matične zadeve} \\
p_{13} &= z_{13}          && \text{$3$ zaposleni na Oddelku za javni red in promet}
\end{align*}
\end{enumerate}
\end{odgovor}
\end{naloga}
