\begin{naloga}{Bašić, Gajser}{Izpit OR 9.7.2012}
\begin{vprasanje}
Dan je številski trikotnik
$$
\begin{array}{ccccccccc}
                &&&& a_{11} \\
            &&& a_{21} && a_{22} \\
        && a_{31} && a_{32} && a_{33} \\
     & \iddots && \vdots && \vdots && \ddots \\
a_{n1} && a_{n2} && \cdots && a_{n,n-1} && a_{nn}
\end{array} ,
$$
kjer so $a_{ij} \in \Z$.
{\em Spust} v takem trikotniku je pot od vrha do dna
-- t.j., zaporedje $(a_{i, j_i})_{i=1}^n$,
kjer je $j_1 = 1$ in $j_i \in \{j_{i-1}, j_{i-1}+1\}$
za vsak $i = 2, \dots, n$.
Teža spusta je vsota števil, po katerih poteka spust
(torej $\sum_{i=1}^n a_{i, j_i}$).

S pomočjo dinamičnega programiranja opiši postopek,
ki poišče najtežji spust v danem številskem trikotniku.
\end{vprasanje}

\begin{odgovor}
Naj bo $p_{ij}$ največja teža spusta v trikotniku z vrhom v $a_{ij}$.
Določimo začetne pogoje in rekurzivne enačbe.
\begin{align*}
p_{nj} &= a_{nj} && (1 \le j \le n) \\
p_{ij} &= a_{ij} + \max\{p_{i+1,j}, p_{i+1,j+1}\} && (1 \le j \le i \le n-1)
\end{align*}
Vrednosti $p_{ij}$ računamo
v padajočem leksikografskem vrstnem redu glede na $(i, j)$.
Težo najtežega spusta dobimo kot $p^* = p_{11}$.
Sam spust $(a_{i, j_i})_{i=1}^n$ lahko dobimo tako,
da izračunamo
\begin{align*}
j_1 &= 1 \\
j_i &= \begin{cases}
j_{i-1} & p_{i,j_{i-1}} \ge p_{i,j_{i-1}+1}, \text{in} \\
j_{i-1} + 1 & \text{sicer}
\end{cases}
& (2 \le i \le n)
\end{align*}
v naraščajočem vrstnem redu glede na indeks $i$.
\end{odgovor}
\end{naloga}
