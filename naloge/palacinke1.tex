\begin{naloga}{Janoš Vidali}{Vaje OR 18.5.2020}
\begin{vprasanje}
Določi skupne, proste, varnostne in neodvisne rezerve
opravil iz naloge~\nal[palacinke].
\end{vprasanje}

\begin{odgovor}
Skupna rezerva opravila je največji čas,
za katerega lahko zamaknemo končanje opravila,
ne da bi s tem vplivali na čas končanja projekta.
Izračunamo jo kot minimalno razliko najkasnejšega začetka naslednika
in najzgodnejšega konca predhodnika,
od katere odštejemo trajanje opravila.
Skupne rezerve računamo pri določitvi kritičnih in najmanj kritičnih opravil
ter so za nalogo~\nal[palacinke]
prikazane v vrstici ``razlika'' tabele~\tab[palacinke-resitev].

Prosta rezerva opravila je največji čas,
za katerega lahko zamaknemo končanje opravila,
če z nasledniki začnemo ob najzgodnejšem možnem času.
Izračunamo jo kot minimalno razliko najzgodnejšega začetka naslednika
in najzgodnejšega konca predhodnika,
od katere odštejemo trajanje opravila.

Varnostna rezerva opravila je največji čas,
za katerega lahko zamaknemo končanje opravila,
ne da bi s tem vplivali na čas končanja projekta,
če s predhodniki končamo ob najkasnejšem možnem času.
Izračunamo jo kot minimalno razliko najkasnejšega začetka naslednika
in najkasnejšega konca predhodnika,
od katere odštejemo trajanje opravila.

Neodvisna rezerva opravila je največji čas,
za katerega lahko zamaknemo končanje opravila,
če z nasledniki začnemo ob najzgodnejšem možnem času
in s predhodniki končamo ob najkasnejšem možnem času.
Izračunamo jo kot minimalno razliko najzgodnejšega začetka naslednika
in najkasnejšega konca predhodnika,
od katere odštejemo trajanje opravila.

Tudi proste, varnostne in neodvisne rezerve opravil iz naloge~\nal[palacinke]
so prikazane v tabeli~\tab[palacinke-resitev].
\end{odgovor}
\end{naloga}
