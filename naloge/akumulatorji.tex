\begin{naloga}{W.~L.~Winston}{\cite[\S15, Example~5]{w}}
\begin{vprasanje}
Tovarna avtomobilov za svoje potrebe izdeluje akumulatorje.
Vsako leto proizvedejo $10\,000$ avtomobilov,
izdelajo pa lahko $25\,000$ akumulatorjev letno.
Stroški zagona proizvodnje so $200 €$,
stroški zaloge pa $0.25 €$ letno za vsak akumulator.
Primanjkljaja ne dovolimo.
Kolikokrat na leto naj zaženejo proizvodnjo akumulatorjev?
Koliko časa naj traja proizvodnja?
Koliko akumulatorjev naj takrat izdelajo?
Kako veliko skladišče potrebujejo?
\end{vprasanje}

\begin{odgovor}
Imamo sledeče podatke (pri časovni enoti $1$ leto).
\begin{align*}
\nu &= 10\,000 &
\lambda &= 25\,000 \\
K &= 200 € &
s &= 0.25 €
\end{align*}
Primanjkljaja ne dovolimo, zato velja $p = \infty$ in torej $\beta = 1$.

Izračunajmo optimalno dolžino cikla $\tau^*$
in največjo zalogo $M^*$.
\begin{alignat*}{3}
\alpha &= \nu \left(1 - {\nu \over \lambda}\right)
&&= 10\,000 \cdot (1 - 0.4) &&= 6\,000 \\
\tau^* &= \sqrt{2K \beta \over s \alpha}
&&= \sqrt{400 \over 1\,500} &&\approx 0.516 \\
M^* &= \sqrt{2K \alpha \over s \beta}
&&= \sqrt{2\,400\,000 \over 0.25} &&\approx 3\,098.387
\end{alignat*}
Optimalna dolžina cikla je torej približno $0.516$ leta
oziroma $188.485$ dni,
pri tem pa potrebujejo skladišče za $3\,099$ akumulatorjev.

Izračunajmo še trajanje proizvodnje $t^*$
in število izdelanih akumulatorjev $q^*$ v posameznem ciklu.
\begin{alignat*}{3}
t^* &= {M^* \over \lambda - \nu}
&&\approx {3\,098.387 \over 15\,000} &&\approx 0.207 \\
q^* &= \tau^* \nu = t^* \lambda
&&\approx 0.516 \cdot 10\,000 &&\approx 5\,163.978
\end{alignat*}
Proizvodnja torej traja približno $0.207$ leta
oziroma $75.394$ dni,
skupno pa izdelajo okoli $5\,164$ akumulatorjev.
\end{odgovor}
\end{naloga}
