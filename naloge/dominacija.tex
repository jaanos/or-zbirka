\begin{naloga}{?}{Vaje OR 16.3.2016}
\begin{vprasanje}
Definiraj problem dominacijske množice v grafu
in zapiši celoštevilski linearni program,
ki rešuje opisani problem.
\end{vprasanje}

\begin{odgovor}
Naj bo $G = (V, E)$ neusmerjen graf.
Množica $D \subseteq V$ {\em dominira} graf $G$
če je vsako vozlišče iz $V$ bodisi v $D$, bodisi ima soseda v $D$.
Pri problemu dominacijske množice iščemo najmanjšo množico $D_{\min}$,
ki dominira graf $G$.

Za vsako vozlišče $v \in V$ bomo uvedli spremenljivko $x_v$,
katere vrednost interpretiramo kot
$$
x_v = \begin{cases}
1, & \text{vozlišče $v$ je v množici $D_{\min}$, in} \\
0  & \text{sicer.}
\end{cases}
$$
Zapišimo celoštevilski linearni program.
\begin{alignat*}{2}
&& \min \ \sum_{v \in V} x_v &\quad \text{p.p.} \\
\forall v \in V: &\ & 0 \le x_v &\le 1, \quad x_v \in \Z \\
\forall v \in V: &\ & x_v + \sum_{uv \in E} x_u &\ge 1
\end{alignat*}
\end{odgovor}
\end{naloga}
