\begin{naloga}{David Gajser}{Izpit OR 28.8.2013}
\begin{vprasanje}
Kukavica bo izlegla $16$ jajc in jih podtaknila v $12$ gnezd,
ki pripadajo dvema taščicama, štirim vrtnim penicam, trem travniškim cipam,
dvema belima pastiricama in eni sovi.
V vsako gnezdo lahko izleže največ tri jajca,
pri čemer je verjetnost, da mladiči v gnezdu $i$ preživijo,
enaka $p_{ij}$, kjer je $j$ število podtaknjenih jajc v gnezdu $i$
(preživijo bodisi vsi ali noben mladič v posameznem gnezdu).
Pri vsaki od petih vrst ptic želi izleči vsaj eno jajce,
pri taščicah pa želi izleči strogo več jajc kot pri belih pastiricah.
Poleg tega pri drugi beli pastirici ne bo odložila jajca,
če bo pri prvi taščici odložila dve jajci ali več.
Kukavica želi maksimizirati pričakovano število preživelih mladičev.

Zapiši problem kot celoštevilski linearni program.
\end{vprasanje}

\begin{odgovor}
Oštevilčimo gnezda s števili od $1$ do $12$:
gnezdi taščic naj imata števili $1$ in $2$,
gnezda vrtnih penic naj imajo števila $3$ in $4$, $5$ in $6$,
gnezda travniških cip naj imajo števila $7$ in $8$, in $9$,
gnezdi belih pastiric naj imata števili $10$ in $11$,
sovino gnezdo pa naj ima število $12$.
Za $i$-to gnezdo ($1 \le i \le 12$) in $j \in \{1, 2, 3\}$
bomo uvedli spremenljivko $x_{ij}$,
katere vrednost interpretiramo kot
$$
x_{ij} = \begin{cases}
1, & \text{če naj kukavica v $i$-to gnezdo izleže natanko $j$ jajc, in} \\
0  & \text{sicer.}
\end{cases}
$$
Naj bo $V = \{\{1, 2\}, \{3, 4, 5, 6\}, \{7, 8, 9\}, \{10, 11\}, \{12\}\}$
particija indeksov gnezd glede na vrsto ptice.
Zapišimo celoštevilski linearni program.
\begin{align*}
\max \ \sum_{i=1}^{12} \sum_{j=1}^3 j \ p_{ij} \ x_{ij} &\quad \text{p.p.} \\
\forall i \in \{1, \dots, 12\} \ \forall j \in \{1, 2, 3\}: \ 0 \le x_{ij} &\le 1, \quad x_{ij} \in \Z
\opis{V vsakem gnezdu je določeno število jajc}
\forall i \in \{1, \dots, 12\}: \ \sum_{j=1}^3 x_{ij} &\le 1
\opis{Skupaj izleže $16$ jajc}
\sum_{i=1}^{12} \sum_{j=1}^3 j \ x_{ij} &= 16
\opis{K vsaki vrsti ptice izleže vsaj eno jajce}
\forall S \in V: \ \sum_{i \in S} \sum_{j=1}^3 x_{ij} &\ge 1
\opis{Pri taščicah izleže strogo več jajc kot pri belih pastiricah}
\sum_{i=1}^2 \sum_{j=1}^3 j \ x_{ij} - \sum_{i=10}^{11} \sum_{j=1}^3 j \ x_{ij} &\ge 1
\opis{Pri drugi beli pastirici ne bo odložila jajca,
če bo pri prvi taščici odložila dve jajci ali več}
\sum_{j=2}^3 x_{1j} + \sum_{j=1}^3 x_{11,j} &\le 1
\end{align*}
\end{odgovor}
\end{naloga}
