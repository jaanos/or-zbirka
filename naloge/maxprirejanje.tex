\begin{naloga}{Janoš Vidali}{Vaje OR 12.10.2016}
\begin{vprasanje}
Napiši linearni program,
ki modelira iskanje največjega prirejanja v dvodelnem grafu.
\end{vprasanje}

\begin{odgovor}
Naj bo $G = (V, E)$ dvodelen graf.
Lahko torej zapišemo $V = A \cup B$ tako,
da sta množici $A$ in $B$ disjunktni
ter za vsako povezavo $uv \in E$ velja $u \in A$, $v \in B$.
Množica $M \subset E$ je {\em prirejanje},
če nobeni dve povezavi iz $M$ nimata skupnega krajišča.
Iščemo največje prirejanje $M_{\max}$.

Za vsako povezavo $uv \in E$ bomo uvedli spremenljivko $x_{uv}$,
katere vrednost interpretiramo kot
$$
x_{uv} = \begin{cases}
1, & \text{povezava $uv$ je v množici $M_{\max}$, in} \\
0  & \text{sicer.}
\end{cases}
$$
Zapišimo celoštevilski linearni program.
\begin{alignat*}{2}
\max &\ & \sum_{uv \in E} x_{uv} &\quad \text{p.p.} \\
\forall uv \in E: &\ & 0 \le x_{uv} &\le 1, \quad x_{uv} \in \Z \\
\forall u \in A: &\ & \sum_{uv \in E} x_{uv} &\le 1 \\
\forall v \in B: &\ & \sum_{uv \in E} x_{uv} &\le 1
\end{alignat*}
\end{odgovor}
\end{naloga}
