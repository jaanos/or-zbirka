\begin{naloga}{Janoš Vidali}{Izpit OR 31.1.2017}
\begin{vprasanje}
Imamo $m$ opravil, ki jih želimo razporediti med $n$ strojev.
Vsak stroj lahko hkrati opravlja le eno opravilo,
vsa opravila pa trajajo eno časovno enoto, neodvisno od stroja.
Če stroj $i$ ($1 \le i \le n$) uporabimo za vsaj eno opravilo,
plačamo ceno $c_i$
(cena ostane enaka, če na istem stroju naredimo več opravil).
Skupni stroški ne smejo preseči količine $C$.
Dani sta še množici parov $P$ in $S$,
pri čemer $(j, k) \in P$ pomeni,
da mora biti opravilo $j$ dokončano pred začetkom opravila $k$,
$(j, k) \in S$ pa pomeni, da se opravili $j$ in $k$ ne smeta izvajati hkrati.
Imamo še dodaten pogoj, ki zahteva,
da je lahko v posamezni časovni enoti lahko aktivnih največ $A$ strojev.
Minimizirati želimo čas dokončanja zadnjega stroja.

Zapiši celoštevilski linearni program, ki modelira zgoraj opisani problem.
\end{vprasanje}

\begin{odgovor}
Za stroj $i$ ($1 \le i \le n$) bomo uvedli spremenljivko $x_i$,
dodatno za opravilo $j$ in časovno enoto $h$ ($1 \le h, j \le m$)
pa še spremenljivko $y_{ijh}$.
Njihove vrednosti interpretiramo kot
\begin{align*}
x_i &= \begin{cases}
1, & \text{če stroj $i$ izvede vsaj eno opravilo, in} \\
0  & \text{sicer;}
\end{cases} \\
\text{ter} \quad
y_{ijh} &= \begin{cases}
1, & \text{če stroj $i$ izvede opravilo $j$ v časovni enoti $h$, in} \\
0  & \text{sicer.}
\end{cases}
\end{align*}
Dodatno bomo uvedli še spremenljivko $t$,
katere vrednost bo enaka časovni enoti dokončanja zadnjega stroja.

Zapišimo celoštevilski linearni program.
\begin{alignat*}{3}
&& \min &\ t & \text{p.p.} \\
\forall i \in \{1, \dots, n\}: &\ & 0 \le x_i &\le 1, & x_i &\in \Z \\
\forall i \in \{1, \dots, n\} \ \forall j, h \in \{1, \dots, m\}:
&\ & 0 \le y_{ijh} &\le 1, & y_{ijh} &\in \Z
\opis{Vsako opravilo se izvede natanko enkrat}
\forall j \in \{1, \dots, m\}: &\ & \sum_{i=1}^n \sum_{h=1}^m y_{ijh} &= 1
\opis{Vsako stroj lahko hkrati opravlja največ eno opravilo}
\forall i \in \{1, \dots, n\} \ \forall h \in \{1, \dots, m\}:
&\ & \sum_{j=1}^m y_{ijh} &\le 1
\opis{Omejitev stroškov obratovanja}
\forall i \in \{1, \dots, n\}:
&\ & \sum_{j=1}^m \sum_{h=1}^m y_{ijh} &\le m x_i \\
&& \sum_{i=1}^n c_i x_i &\le C
\opis{Vrstni red opravil}
\forall (j, k) \in P:
&\ & \sum_{i=1}^n \sum_{h=1}^m h (y_{ikh} - y_{ijh}) &\ge 1
\opis{Hkratnost opravil}
\forall (j, k) \in S \ \forall h \in \{1, \dots, m\}:
&\ & \sum_{i=1}^n (y_{ijh} + y_{ikh}) &\le 1
\opis{Omejitev hkrati aktivnih strojev}
\forall h \in \{1, \dots, m\}: &\ & \sum_{i=1}^n \sum_{j=1}^m y_{ijh} &\le A
\opis{Omejitev časa dokončanja}
\forall i \in \{1, \dots, n\} \ \forall j, h \in \{1, \dots, m\}:
&\ & h y_{ijh} &\le t
\end{alignat*}
\end{odgovor}
\end{naloga}
