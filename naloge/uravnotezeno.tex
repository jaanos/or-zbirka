\begin{naloga}{Jernej Azarija}{Izpit OR 29.8.2016}
\begin{vprasanje}
Naj bo $G = (V, E)$ enostaven graf.
Barvanju vozlišč grafa $c_k : V \to \{1, 2, \dots, k\}$
pravimo {\em $k$-uravnoteženo barvanje},
če ima naslednji lastnosti:
\begin{itemize}
\item Če sta vozlišči $u$ in $v$ sosedni v $G$,
potem velja $c_k(u) \ne c_k(v)$.
\item Za vsaka $i, j \in \{1, 2, \dots, k\}$ velja
$$
\Big| |\set{v \in V}{c_k(v) = i}| - |\set{v \in V}{c_k(v) = j}| \Big|
\le 1
$$
-- tj., število vozlišč, pobarvanih z barvama $i$ in $j$,
se razlikuje za največ $1$.
\end{itemize}
Napiši celoštevilski linearni program, ki ima dopustno rešitev natanko tedaj,
ko za vhodni graf $G$ obstaja $k$-uravnoteženo barvanje.
Predpostaviš lahko, da je tudi $k$ vhodni podatek.
Koliko pogojev in koliko spremenljivk
ima dobljeni celoštevilski linearni program?
\end{vprasanje}

\begin{odgovor}
Za vozlišče $u \in V$ in $i$-to barvo ($1 \le i \le k$)
bomo uvedli spremenljivko $x_{ui}$,
katere vrednost interpretiramo kot
$$
x_{ui} = \begin{cases}
1; & \text{vozlišče $u$ pobarvamo z barvo $i$, in} \\
0  & \text{sicer.}
\end{cases}
$$
Zapišimo omejitve za celoštevilski linearni program.
Ker nas zanima le obstoj dopustne rešitve, ciljna funkcija ni pomembna.
\begin{alignat*}{2}
\forall u \in V \ \forall i \in \{1, \dots, k\}: &\ &
0 \le x_{ui} &\le 1, \quad x_{ui} \in \Z
\opis{Vsako vozlišče pobarvamo z eno barvo}
\forall u \in V: &\ & \sum_{i=1}^k x_{ui} &= 1
\opis{Sosedni vozlišči nimata iste barve}
\forall uv \in E \ \forall i \in \{1, \dots, k\}: &\ & x_{ui} + x_{vi} &\le 1
\opis{Število vozlišč dveh različnih barv se razlikuje za največ $1$}
\forall i, j \in \{1, \dots, k\}, \ i < j: &\ &
-1 \le \sum_{u \in V} (x_{ui} - x_{uj}) &\le 1
\end{alignat*}
Naj bo $n$ število vozlišč in $m$ število povezav grafa $G$.
Potem ima zgornji celoštevilski linearni program
$nk$ celoštevilskih spremenljivk in $2nk + n + mk + k(k-1)$ pogojev.
\end{odgovor}
\end{naloga}
