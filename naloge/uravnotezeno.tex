\begin{naloga}{?}{Izpit OR 24.5.2016}
\begin{vprasanje}
Naj bo $G = (V, E)$ enostaven graf.
Barvanju vozlišč grafa $c_k : V \to \{1, 2, \dots, k\}$
pravimo {\em $k$-uravnoteženo barvanje},
če ima naslednji lastnosti:
\begin{itemize}
\item Če sta vozlišči $u$ in $v$ sosedni v $G$,
potem velja $c_k(u) \ne c_k(v)$.
\item Za vsaka $i, j \in \{1, 2, \dots, k\}$ velja
$$
\Big| |\set{v \in V}{c_k(v) = i}| - |\set{v \in V}{c_k(v) = h}| \Big|
\le 1
$$
-- tj., število vozlišč, pobarvanih z barvama $i$ in $j$,
se razlikuje za največ $1$.
\end{itemize}
Napiši celoštevilski linearni program, ki ima dopustno rešitev natanko tedaj,
ko za vhodni graf $G$ obstaja $k$-uravnoteženo barvanje.
Predpostaviš lahko, da je tudi $k$ vhodni podatek.
Koliko pogojev in koliko spremenljivk
ima dobljeni celoštevilski linearni program?
\end{vprasanje}
\begin{odgovor}
\end{odgovor}
\end{naloga}
