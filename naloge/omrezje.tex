\begin{naloga}{David Gajser}{Izpit OR 24.6.2013}
\begin{vprasanje}
Računalniško omrežje sestavlja $n$ računalnikov.
Predstavljeno je z neusmerjenim povezanim grafom $G$,
ki ima na povezavah nenegativne uteži.
Utež na povezavi $uv$ nam pove,
koliko časa (v milisekundah) potrebuje informacija,
da pride neposredno od računalnika $u$ do računalnika $v$.
Vsak računalnik potrebuje natanko $1$ ms, da informacijo pošlje naprej.
Naj bo $T_{uv}$ najkrajši čas,
v katerem lahko računalnik $v$ dobi informacijo od računalnika $u$.
Ta čas vsebuje $1$ ms, ki jo računalnik $u$ porabi za pošiljanje informacije,
in ne vsebuje $1$ ms za računalnik $v$,
saj slednjemu informacije ni več potrebno poslati naprej.
Tudi naš računalnik je priklopljen na to omrežje
-- recimo, da je predstavljen z vozliščem $s$.
Zanima nas, kateri računalnik od nas najkasneje dobi informacije,
tj., pri katerem $v$ je dosežena maksimalna vrednost $T_{sv}$,
in kolikšen je ta maksimum.

\begin{enumerate}[(a)]
\item Opiši algoritem časovne zahtevnosti največ $O(n^2)$,
ki reši dani problem.
Na vhod naj sprejme graf $G$ in vozlišče $s$.

\item Reši nalogo na grafu s slike~\fig tako,
da bo postopek jasen in pravilen.
\end{enumerate}

\begin{slika}
\pgfslika
\podnaslov{Graf}
\end{slika}
\end{vprasanje}

\begin{odgovor}
\begin{enumerate}[(a)]
\item Problem lahko prevedemo
na problem iskanja najcenejših poti od vozlišča $s$ do ostalih vozlišč v grafu,
kjer je vsaka povezava utežena s številom,
ki je za $1$ večje od uteži na ustrezni povezavi v grafu $G$.
Vrnemo vozlišče, do katerega je najdena pot najdražja.

Ker so uteži nenegativne, lahko za reševanje uporabimo Dijkstrov algoritem.
Če za prednostno vrsto uporabimo seznam,
potem ta teče v času $O(n^2)$
(v primeru, ko je število povezav $m = O\left({n^2 \over \log n}\right)$,
potem bo hitrejša različica algoritma,
pri kateri za prednostno vrsto uporabimo kopico,
saj ta teče v času $O(m \log n)$).

\item Postopek izvajanja Dijkstrovega algoritma
na grafu s slike~\fig[omrezje-resitev] z začetkom pri $s$
je prikazan v tabeli~\tab,
iz katere je razvidno, da bo informacijo nakasneje dobil računalnik $z$,
in sicer po $6$ ms.
\end{enumerate}
%
\begin{slika}
\pgfslika[omrezje-resitev]
\podnaslov{Graf in drevo najkrajših poti}
\end{slika}
%
\begin{tabela}[h]
$$
\begin{array}{cccccccc}
s & t & u & v & w & x & y & z \\ \hline
* & 1_s && 3_s &&& 5_s & \\
& * & 3_t &&&&& \\
&& * &&& 4_u && \\
&&& * & 5_v &&& 6_v \\
&&&&& * && \\
&&&& * &&& \\
&&&&&& * & \\
&&&&&&& *
\end{array}
$$
\podnaslov[\res{}(b)]{Potek izvajanja algoritma}
\end{tabela}
\end{odgovor}
\end{naloga}
