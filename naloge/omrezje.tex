\begin{naloga}{David Gajser}{Izpit OR 24.6.2013}
\begin{vprasanje}
Računalniško omrežje sestavlja $n$ računalnikov.
Predstavljeno je z neusmerjenim povezanim grafom $G$,
ki ima na povezavah nenegativne uteži.
Utež na povezavi $ij$ nam pove,
koliko časa (v milisekundah) potrebuje informacija,
da pride neposredno od računalnika $i$ do računalnika $j$.
Vsak računalnik potrebuje natanko $1$ ms, da informacijo pošlje naprej.
Naj bo $T_{i,j}$ najkrajši čas,
v katerem lahko računalnik $j$ dobi informacijo od računalnika $i$.
Ta čas vsebuje $1$ ms, ki jo računalnik $i$ porabi za pošiljanje informacije,
in ne vsebuje $1$ ms za računalnik $j$,
saj slednjemu informacije ni več potrebno poslati naprej.
Tudi naš računalnik je priklopljen na to omrežje
-- recimo, da je predstavljen z vozliščem 0.
Zanima nas, kateri računalnik od nas najkasneje dobi informacije,
tj., pri katerem $j$ je dosežena maksimalna vrednost $T_{0,j}$,
in kolikšen je ta maksimum.

\begin{enumerate}[(a)]
\item Reši nalogo na grafu s slike~\fig tako,
da bo postopek jasen in pravilen.

\item Opiši algoritem časovne zahtevnosti največ $O(n^2)$,
ki reši dani problem.
Na vhod naj sprejme graf $G$ in vozlišče $0$.
\end{enumerate}

\begin{slika}
\pgfslika
\podnaslov{Graf}
\end{slika}
\end{vprasanje}
\begin{odgovor}
\end{odgovor}
\end{naloga}
