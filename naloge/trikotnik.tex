\begin{naloga}{?}{Vaje OR 20.4.2016}
\begin{vprasanje}
Napiši algoritem, ki za vhodni graf $G$ določi, ali ima trikotnik.
Katero podatkovno strukturo za grafe boš uporabil?
\end{vprasanje}

\begin{odgovor}
Uporabili bomo predstavitev grafa s seznamom sosedov
(glej nalogo~\res[sosgraf]).
\begin{small}
\begin{algorithmic}
\Function{Trikotnik}{$G$}
    \For{$u \in G.E$}
        \For{$v \in \Adj(G, u)$}
            \For{$w \in \Adj(G, v)$}
                \If{$u \in \Adj(G, w)$}
                    \State \Return $\True$
                \EndIf
            \EndFor
        \EndFor
    \EndFor
    \State \Return $\False$
\EndFunction
\end{algorithmic}
\end{small}
Da pregledamo vse pare $(u, v)$, porabimo $O(n)$ korakov,
da obiščemo vsa vozlišča,
in še $O(m)$ korakov, da obiščemo vse povezave.
Tukaj je $n$ število vozlišč in $m$ število povezav grafa.
Za vsak tak par pregledamo še sosede $w$ vozlišča $v$
-- teh je vsakič največ $\Delta$ (največja stopnja grafa $G$).
Nazadnje še v času $O(1)$ preverimo, ali sta vozlišči $u$ in $w$ sosedni.
Skupna časovna zahtevnost algoritma je torej $O(n + m\Delta)$.
\end{odgovor}
\end{naloga}
