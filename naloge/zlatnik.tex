\begin{naloga}{Blaž Jelenc}{Izpit OR 25.8.2014}
\begin{vprasanje}
V škatli so štirje ponarejeni kovanci, vredni $0 €$,
in en zlat kovanec, vreden $100 €$.
Na slepo lahko izbereš po en kovanec,
pri čemer moraš za vsako izbiranje plačati $30 €$.
Določi pričakovani profit, ki ga dosežeš pri igranju te igre.
\end{vprasanje}

\begin{odgovor}
Očitno se nam po uspešnem poskusu ne izplača več poskušati.
Tako naj slučajni spremenljivki $X_i$ in $Y_i$ ($0 \le i \le 4$)
predstavljata profit v primerih,
ko se odločimo za $(i+1)$-ti poskus
oziroma se odločimo končati igro po $i$ poskusih,
ob predpostavki, da so bili prejšnji poskusi neuspešni.
Imamo torej
\begin{align*}
E(X_i) &= {1 \over 5-i} \cdot (70 € - i \cdot 30 €)
    + {4-i \over 5-i} \cdot \max\{E(X_{i+1}), E(Y_{i+1})\}
\quad \text{in} \\
E(Y_i) &= -i \cdot 30 € .
\end{align*}
Izračunajmo ustrezne vrednosti.
\begin{align*}
E(X_4) &= -50 € & E(Y_4) &= -120 € \\
E(X_3) &= -35 € & E(Y_3) &= -90 € \\
E(X_2) &= -20 € & E(Y_2) &= -60 € \\
E(X_1) &=  -5 € & E(Y_1) &= -30 € \\
E(X_0) &=  10 € & E(Y_0) &=   0 €
\end{align*}
Vidimo, da za vse $i$ ($0 \le i \le 4$) velja $E(X_i) > E(Y_i)$,
kar pomeni, da je optimalna strategija taka,
da igramo, dokler ne dobimo zlatnika.
Pričakovani profit je potem $10 €$.
\end{odgovor}
\end{naloga}
