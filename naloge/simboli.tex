\begin{naloga}{Gabrovšek, Konvalinka}{Izpit OR 28.6.2011}
\begin{vprasanje}
Dano imamo zaporedje simbolov $\top$ (resnično) in $\bot$ (neresnično).
Med vsakima simboloma imamo veznik $\land$ (prvi in drugi),
$\lor$ (prvi ali drugi) ali $\oplus$ (prvi ali drugi, ne pa tudi oba),
npr.~$\top \land \top \oplus \bot$.
V zaporedje postavljamo oklepaje,
in sicer tako, da se znotraj vsakega para oklepajev
(tj., oklepaj in ustrezni zaklepaj)
pojavita dva podizraza, ločena z veznikom.
Podizraz je lahko posamezen simbol ali par oklepajev s svojo vsebino.
Izračunaj število postavitev oklepajev,
da je logična vrednost celotnega izraza $\top$,
in število postavitev oklepajev, da je rezultat $\bot$.
\namig{definiraj $T_{ij}$ in $F_{ij}$
kot število takih postavitev oklepajev med $i$-tim in $j$-tim simbolom.}
\end{vprasanje}

\begin{odgovor}
Denimo, da imamo zaporedje simbolov $(s_h)_{h=0}^n$
in zaporedje veznikov $(v_h)_{h=1}^n$,
pri čemer se veznik $v_h$ pojavi med simboloma $s_{h-1}$ in $s_h$.
Naj bosta $T_{ij}$ in $F_{ij}$ števili takih postavitev oklepajev
v zaporedje simbolov $(s_h)_{h=i}^j$ z vezniki $(v_h)_{h=i+1}^j$,
da ima ustrezni izraz vrednost $\top$ oziroma $\bot$.
Zapišimo začetne pogoje in rekurzivne enačbe.
\begin{align*}
T_{ii} &= \begin{cases}
1 & s_i = \top \\
0 & s_i = \bot
\end{cases} & (0 \le i \le n) \\
F_{ii} &= 1 - T_{ii} & (0 \le i \le n) \\
T_{ij} &= \sum_{h=i+1}^j \begin{cases}
T_{i,h-1} T_{hj} & v_h = \land \\
T_{i,h-1} T_{hj} + F_{i,h-1} T_{hj} + T_{i,h-1} F_{hj} & v_h = \lor \\
F_{i,h-1} T_{hj} + T_{i,h-1} F_{hj} & v_h = \oplus
\end{cases} & (0 \le i < j \le n) \\
F_{ij} &= \sum_{h=i+1}^j \begin{cases}
F_{i,h-1} T_{hj} + T_{i,h-1} F_{hj} + F_{i,h-1} F_{hj} & v_h = \land \\
F_{i,h-1} F_{hj} & v_h = \lor \\
T_{i,h-1} T_{hj} + F_{i,h-1} F_{hj} & v_h = \oplus
\end{cases} & (0 \le i < j \le n)
\end{align*}
Vrednosti $T_{ij}$ in $F_{ij}$ računamo
v leksikografskem vrstnem redu glede na $(j-i, i)$,
število postavitev oklepajev,
da je vrednost celotnega izraza enaka $\top$ ali $\bot$,
pa dobimo kot $T^* = T_{0n}$ oziroma $F^* = F_{0n}$.
Algoritem, ki sledi iz zgornjih rekurzivnih enačb, teče v času $O(n^3)$.
\end{odgovor}
\end{naloga}
