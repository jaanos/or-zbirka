\begin{naloga}{Gabrovšek, Konvalinka}{Izpit OR 28.6.2011}
\begin{vprasanje}
Dano imamo zaporedje simbolov $\top$ (resnično) in $\bot$ (neresnično).
Med vsakima izrazoma imamo $\land$ (prvi in drugi),
$\lor$ (prvi ali drugi) ali $\oplus$ (prvi ali drugi, ne pa tudi oba),
npr.~$\top \land \top \oplus \bot$.
Izračunaj število postavitev oklepajev, da je rezultat $\top$,
in število postavitev oklepajev, da je rezultat $\bot$.
\namig{definiraj $T_{i,j}$ in $F_{i,j}$
kot število takih postavitev oklepajev med $i$-tim in $j$-tim izrazom.}
\end{vprasanje}
\begin{odgovor}
\end{odgovor}
\end{naloga}
