\begin{naloga}{Blaž Jelenc}{Izpit OR 5.6.2014}
\begin{vprasanje}
Dan je graf mest in cestnih povezav med njimi, prikazan na sliki~\fig.
Predvidena je obnova nekaterih cest,
pri čemer je strošek obnove sorazmeren z dolžino ceste.

\begin{enumerate}[(a)]
\item Določi ceste, ki jih je potrebno obnoviti,
da bo strošek obnove čim manjši,
ter da bo med poljubnima dvema mestoma
obstajala povezava preko samih obnovljenih cest.

\item Zaradi nižanja stroškov preuči še naslednjo možnost.
Določi ceste, ki jih je potrebno obnoviti, da bo strošek obnove čim manjši,
ter da bo med poljubnima dvema mestoma obstajala povezava
preko samih obnovljenih cest in največ ene neobnovljene ceste.
\end{enumerate}

\begin{slika}
\pgfslika
\podnaslov{Graf}
\end{slika}
\end{vprasanje}

\begin{odgovor}
\begin{enumerate}[(a)]
\item Na grafu s slike~\fig s Primovim algoritmom poiščemo najcenejše vpeto drevo.
Potek izvajanja algoritma je prikazan v tabeli~\tab,
najcenejše vpeto drevo s ceno $87$
pa je prikazano na sliki~\fig[obnova-resitev].

\item Poiskati želimo najcenejši gozd $F$ na povezavah danega grafa,
tako da v grafu obstaja povezava med vsakim parom povezanih komponent gozda $F$.
Če v danem grafu skrčimo vse povezave iz $F$,
bomo torej dobili poln graf s to\-lik\-šnim številom vozlišč,
kolikor ima $F$ povezanih komponent
-- tak poln graf je torej minor v danem grafu.

Ker je graf s slike~\fig ravninski in torej ne vsebuje minorja $K_5$,
bo iskani gozd imel največ $4$ povezane komponente.
Iskanja optimalne rešitve se torej lahko lotimo tako,
da največ trem povezavam spremenimo utež na $0$
ter s Primovim algoritmom poiščemo najcenejše vpeto drevo v takih grafih
(če se v njem pojavi povezava, ki smo ji utež nastavili na $0$,
smatramo, da te ceste ne bomo prenovili).
Izkaže se, da je optimalna rešitev taka kot na sliki~\fig[obnova-resitev],
pri čemer izpustimo povezave $cd$, $eh$ in $ij$.
Skupna cena obnove je v tem primeru $54$.
\end{enumerate}
%
\begin{tabela}
$$
\begin{array}{c|c|l}
\text{dodano vozlišče} & \text{teža drevesa} & \text{vrsta povezav} \\ \hline
a &  0 & ac: 2, ab: 7, ad: 10 \\
c &  2 & ab: 7, cd: 8, ad: 10, cg: 14 \\
b &  9 & bf: 6, cd: 8, ad: 10, be: 10, cg: 14 \\
f & 15 & cd: 8, fe: 8, ad: 10, be: 10, cg: 14, fj: 16 \\
d & 23 & de: 5, fe: 8, be: 10, cg: 14, fj: 16 \\
e & 28 & eh: 10, cg: 14, fj: 16, ej: 18 \\
h & 38 & hi: 6, hg: 12, hk: 12, cg: 14, fj: 16, ej: 18 \\
i & 44 & ik: 8, hg: 12, hk: 12, cg: 14, ij: 15, fj: 16, ej: 18 \\
k & 52 & kg: 10, hg: 12, cg: 14, ij: 15, k\ell: 15, fj: 16, ej: 18 \\
g & 62 & ij: 15, k\ell: 15, fj: 16, ej: 18 \\
j & 77 & j\ell: 10, k\ell: 15 \\
\ell & 87 &
\end{array}
$$
\podnaslov[\res{}(a)]{Potek izvajanja algoritma}
\end{tabela}
%
\begin{slika}
\pgfslika[obnova-resitev]
\podnaslov[\res{}]{Najcenejše vpeto drevo}
\end{slika}
\end{odgovor}
\end{naloga}
