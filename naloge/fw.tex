\begin{naloga}{Janoš Vidali}{Vaje OR 17.5.2021}
\begin{vprasanje}
S pomočjo Floyd-Warshallovega algoritma poišči
najkrajše poti med vsemi pari vozlišč v grafu s slike~\fig[bf].
\end{vprasanje}

\begin{odgovor}
Sledili bomo naslednjemu algoritmu.
\begin{small}
\begin{algorithmic}
\Function{FloydWarshall}{$G = (V, E), L : E \to \R$}
    \State $d \gets$ slovar z vrednostjo $\infty$ za vsak par vozlišč $u, v \in V$
    \State $\pred \gets$ slovar z vrednostjo $\Null$
        za vsak par vozlišč $u, v \in V$
    \For{$u \in V$}
        \State $d[u, u] \gets 0$
    \EndFor
    \For{$uv \in E$}
        \State $d[u, v] \gets L_{uv}$
        \State $\pred[u, v] \gets u$
    \EndFor
    \For{$w \in V$}
        \For{$u \in V$}
            \If{$d[u, w] + d[w, u] < 0$}
                \State \Return graf ima negativen cikel
            \EndIf
            \For{$v \in V$}
                \State $\ell \gets d[u, w] + d[w, v]$
                \If{$\ell < d[u, v]$}
                    \State $d[u, v] \gets \ell$
                    \State $\pred[u, v] \gets \pred[w, v]$
                \EndIf
            \EndFor
        \EndFor
    \EndFor
    \State \Return $(d, \pred)$
\EndFunction
\end{algorithmic}
\end{small}
Po vsakem obhodu zunanje zanke {\bf for}
slovar $d$ vsebuje dolžine tistih najkrajših poti,
ki v svoji notranjosti (tj., brez začetnega in končnega vozlišča)
obiščejo le tista vozlišča,
ki jih je ta zanka že obravnavala.
Algoritem pri tem preverja, ali je našel zanko negativne dolžine
-- tedaj sklene, da v grafu obstaja negativen cikel.
Če graf nima negativnih ciklov,
po izhodu iz zunanje zanke
slovar $d$ vsebuje dolžine najkrajših poti med vsemi pari vozlišč.
Časovna zahtevnost algoritma je $O(n^3)$,
kjer je $n$ število vozlišč grafa.

Potek zgornjega algoritma na grafu s slike~\fig[bf] je prikazan v tabeli~\tab
(prikazani so samo koraki, ko se katera od vrednosti v slovarju $d$ spremeni).
Izračunane razdalje so povzete v tabeli~\tab[fw-resitev],
v kateri je prikazano tudi,
preko katerega vozlišča gre pot, preden obišče ciljno vozlišče
(prazni vnosi pomenijo, da ustrezna pot ne obstaja).

\begin{tabela}
$$
\begin{array}{c|c|c|c|ccc|c|c|c|c}
w & u & v & d[uv] & \pred[u, v] & $\qquad$ & w & u & v & d[uv] & \pred[u, v] \\ \cline{1-5} \cline{7-11}
  & a & b &     3 & a && e & a & g &    13 & e \\
  & a & f &     6 & a && e & b & g &    10 & e \\
  & b & c &     7 & b && e & c & g &    12 & e \\
  & b & e &     9 & b && e & d & g &    19 & e \\ \cline{7-11}
  & c & d &    -7 & c && g & b & f &     9 & g \\
  & d & b &     9 & d && g & c & f &    11 & g \\
  & d & h &     4 & d && g & d & f &    18 & g \\
  & e & d &     1 & e && g & e & f &     0 & g \\
  & e & g &     1 & e && g & h & f &    -6 & g \\ \cline{7-11}
  & f & g &     8 & f && h & a & f &     1 & g \\
  & g & f &    -1 & g && h & a & g &     2 & h \\
  & h & g &    -5 & h && h & b & f &    -2 & g \\ \cline{1-5}
b & a & c &    10 & b && h & b & g &    -1 & h \\
b & a & e &    12 & b && h & c & f &    -9 & g \\
b & d & c &    16 & b && h & c & g &    -8 & h \\
b & d & e &    18 & b && h & d & f &    -2 & g \\ \cline{1-5}
c & a & d &     3 & c && h & d & g &    -1 & h \\
c & b & d &     0 & c && h & e & f &    -1 & g \\ \cline{1-5}
d & a & h &     7 & d && h & e & g &     0 & h \\
d & b & h &     4 & d \\
d & c & b &     2 & d \\
d & c & e &    11 & b \\
d & c & h &    -3 & d \\
d & e & b &    10 & d \\
d & e & c &    15 & b \\
d & e & h &     5 & d 


















\end{array}
$$
\podnaslov{Potek izvajanja algoritma}
\end{tabela}

\begin{tabela}
\setlabel{fw-resitev}
$$
\begin{array}{c|cccccccc}
& a & b & c & d & e & f & g & h \\ \hline
a & 0 & 3_a & 10_b & 3_c & 12_b &  1_g &  2_h &  7_d \\
b & & 0 & 7_b & 0_c & 9_b & -2_g & -1_h &  4_d \\
c & & 2_d & 0 & -7_d & 11_b & -9_g & -8_h & -3_d \\
d & & 9_d & 16_b & 0 & 18_b & -2_g & -1_h & 4_d \\
e & & 10_d & 15_b & 1_e & 0 & -1_g &  0_h & 5_d \\
f & & & & & & 0 & 8_f & \\
g & & & & & & -1_g & 0 & \\
h & & & & & & -6_g & -5_h & 0 \\
\end{array}
$$
\podnaslov{Najkrajše razdalje}
\end{tabela}

\end{odgovor}
\end{naloga}
