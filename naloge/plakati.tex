\begin{naloga}{?}{Izpit OR 15.9.2010}
\begin{vprasanje}
Na avtocestni odsek dolžine $M$ kilometrov
želimo postaviti oglasne plakate.
Dovoljene lokacije plakatov določa urad za oglaševanje
in so predstavljene s števili $x_1, x_2, \dots x_n$,
kjer $x_i$ ($1 \le i \le n$)
predstavlja oddaljenost od začetka odseka v kilometrih.
Profitabilnost oglasa na lokaciji $x_i$ določa vrednost $v_i$
($1 \le i \le n$).
Urad za oglaševanje podaja tudi omejitev,
da mora biti razdalja med oglasi vsaj $d$ kilometrov.
Oglase želimo postaviti tako, da bodo čim bolj profitabilni.
\begin{enumerate}[(a)]
\item Reši problem za parametre $M = 20$, $d = 5$, $n = 8$,
$(x_i)_{i=1}^n = (1, 2, 8, 10, 12,$ $14, 17, 20)$ in
$(v_i)_{i=1}^n = (8, 8, 12, 10, 7, 5, 6, 10)$.
\item Napiši rekurzivne enačbe za opisani problem.
\item Napiši algoritem,
ki poišče najbolj profitabilno postavitev oglasov za dane parametre.
Kakšna je njegova časovna zahtevnost?
\end{enumerate}

\end{vprasanje}
\begin{odgovor}
\end{odgovor}
\end{naloga}
