\begin{naloga}{Janoš Vidali}{Kolokvij OR 17.4.2023}
\begin{vprasanje}
Jože je ravnokar dobil službo v mestu in načrtuje,
po kateri poti se bo vsako jutro peljal od doma do službe.
Lahko se odloči za pot po avtocesti:
od doma do avtocestnega priključka potrebuje $5$ minut,
trajanje poti po avtocesti pa je odvisno od gneče.
Statistični podatki kažejo,
da se z verjetnostjo $0.6$ na avtocesti pojavi zastoj
in tedaj traja pot od priključka do mesta $30$ minut;
z verjetnostjo $0.4$ pa zastoja ni
in pot po avtocesti traja $10$ minut
(skupno trajanje poti je torej $35$ ali $15$ minut).

Lahko se pa Jože zapelje na vrh hriba, do koder ima $5$ minut,
in od tam oceni gostoto prometa na avtocesti.
Izkušnje kažejo, da veljajo sledeče pogojne verjetnosti
$P(\text{gostota prometa} \mid \text{zastoj na avtocesti})$:
\begin{center}
\begin{tabular}{c|cc}
& zastoj & ni zastoja \\
\hline
gost promet & $0.9$ & $0.4$ \\
redek promet & $0.1$ & $0.6$
\end{tabular}
\end{center}
Jože se potem lahko odloči, ali bo pot nadaljeval po avtocesti
(do priključka potrebuje še dodatnih $5$ minut
-- skupaj bo torej potreboval $40$ ali $20$ minut),
ali pa bo pot nadaljeval po lokalni cesti,
po kateri do mesta potrebuje $23$ minut
(skupaj torej $28$ minut).

\begin{enumerate}[(a)]
\item Kako naj se Jože odloči, da bo pričakovano trajanje poti čim krajše?
Nariši odločitveno drevo in odločitve sprejmi na podlagi izračunanih verjetnosti
ter izračunaj pričakovano trajanje poti.

\item Jože želi minimizirati tudi trajanje popoldanske poti nazaj domov iz službe,
pri čemer bo upošteval tudi to,
da se bodo v naslednjem mesecu v popoldanskem času
na avtocesti izvajala vzdrževalna dela.
Tudi sedaj se lahko odloči za pot lokalni cesti,
po kateri celotna pot traja $28$ minut,
ali pa za pot po avtocesti, pri čemer je njeno trajanje odvisno od gneče.
V času vračanja je verjetnost, da se pojavi zastoj, enaka $0.4$
-- tedaj celotna pot po avtocesti od mesta do doma traja $35 + 2a$ minut;
z verjetnostjo $0.6$ pa zastoja ni in pot traja $15 + a$ minut.
Tukaj je $a \in [0, 10]$ parameter,
ki se vsak dan spreminja v odvisnosti od obsega del.
V primeru, ko se je Jože zjutraj pripeljal po avtocesti,
pa lahko bolje oceni verjetnost zastoja na poti domov,
saj pozna tudi pogojne verjetnosti
$P(\text{zastoj zjutraj} \mid \text{zastoj popoldne})$
(te so neodvisne od gostote prometa,
kot jo lahko Jože zjutraj oceni z vrha hriba):
\begin{center}
\begin{tabular}{c|cc}
& zastoj popoldne & ni zastoja popoldne \\
\hline
zastoj zjutraj & $0.8$ & $0.3$ \\
ni zastoja zjutraj & $0.2$ & $0.7$
\end{tabular}
\end{center}
Nariši odločitveno drevo
ter določi optimalne odločitve in pričakovano trajanje poti
v odvisnosti od parametra $a$.
\end{enumerate}
\end{vprasanje}

\begin{odgovor}
\begin{enumerate}[(a)]
\item Izračunajmo verjetnosti za oceno gostote prometa
in za nastanek zastoja glede na oceno.
\begin{align*}
P(\text{gost promet}) &= 0.9 \cdot 0.6 + 0.4 \cdot 0.4 = 0.7 \\
P(\text{redek promet}) &= 0.1 \cdot 0.6 + 0.6 \cdot 0.4 = 0.3 \\
P(\text{zastoj zjutraj} \mid \text{gost promet})
&= {0.9 \cdot 0.6 \over 0.7} \approx 0.77 \\
P(\text{ni zastoja zjutraj} \mid \text{gost promet})
&= {0.4 \cdot 0.4 \over 0.7} \approx 0.23 \\
P(\text{zastoj zjutraj} \mid \text{redek promet})
&= {0.1 \cdot 0.6 \over 0.3} = 0.2 \\
P(\text{ni zastoja zjutraj} \mid \text{redek promet})
&= {0.6 \cdot 0.4 \over 0.3} = 0.8
\end{align*}
Odločitveno drevo je prikazano na sliki~\fig[zastoj-a].
Jože naj gre vsako jutro na vrh hriba
-- če je promet gost, naj nadaljuje po lokalni cesti,
sicer pa naj gre po avtocesti.
Pričakovano trajanje poti je tedaj $26.8$ minut.

\item Iz slike~\fig[zastoj-a] je razvidno,
da se bo Jože z verjetnostjo $0.7$ zjutraj peljal po lokalni cesti
in tako ne bo vedel,
ali je zjutraj nastal zastoj.
Izračunajmo še verjetnosti,
da se je zjutraj peljal po avtocesti in je ali ni nastal zastoj,
ter pogojne verjetnosti za zastoj popoldne v odvisnosti od zastoja zjutraj.%
\footnote{%
Opazimo lahko,
da pri izračunu pogojnih verjetnosti izračunamo,
da je verjetnost gneče zjutraj enaka $0.5$,
kar se sicer ne sklada z oceno iz točke (a).
}
\begin{align*}
P(\text{po avtocesti in zastoj zjutraj}) = 0.3 \cdot 0.2 &= 0.06 \\
P(\text{po avtocesti in ni zastoja zjutraj}) = 0.3 \cdot 0.8 &= 0.24 \\
P(\text{zastoj popoldne} \mid \text{zastoj zjutraj}) = {0.8 \cdot 0.4 \over 0.8 \cdot 0.4 + 0.3 \cdot 0.6} &= 0.64 \\
P(\text{ni zastoja popoldne} \mid \text{zastoj zjutraj}) = {0.3 \cdot 0.6 \over 0.8 \cdot 0.4 + 0.3 \cdot 0.6} &= 0.36 \\
P(\text{zastoj popoldne} \mid \text{ni zastoja zjutraj}) = {0.2 \cdot 0.4 \over 0.2 \cdot 0.4 + 0.7 \cdot 0.6} &= 0.16 \\
P(\text{zastoj popoldne} \mid \text{ni zastoja zjutraj}) = {0.7 \cdot 0.6 \over 0.2 \cdot 0.4 + 0.7 \cdot 0.6} &= 0.84
\end{align*}
Naj bo $X$ trajanje popoldanske poti po avtocesti v minutah.
Izračunajmo pričakovane vrednosti v odvisnosti od jutranje poti.
\begin{align*}
E(X \mid \text{po lokalni cesti zjutraj})
&= 0.4 \cdot (35 + 2a) + 0.6 \cdot (15 + a) \\ &= 23 + 1.4a \\
E(X \mid \text{po avtocesti in zastoj zjutraj})
&= 0.64 \cdot (35 + 2a) + 0.36 \cdot (15 + a) \\ &= 27.8 + 1.64a \\
E(X \mid \text{po avtocesti in ni zastoja zjutraj})
&= 0.16 \cdot (35 + 2a) + 0.84 \cdot (15 + a) \\ &= 18.2 + 1.16a
\end{align*}
Pot po avtocesti izberemo,
če je njeno pričakovano trajanje manjše od $28$ minut,
kolikor traja pot po lokalni cesti.
Poglejmo, pri katerih vrednostih $a$ bi izbrali pot po avtocesti:
\begin{align*}
% BUG: \opis doda prazno vrstico
\shortintertext{\small\needspace{2\baselineskip}%
%\opis{
Po lokalni cesti zjutraj%
:%
}
a < {28 - 23 \over 1.4} &\approx 3.57
\opis{Po avtocesti in zastoj zjutraj}
a < {28 - 27.8 \over 1.64} &\approx 0.12
\opis{Po avtocesti in ni zastoja zjutraj}
a < {28 - 18.2 \over 1.16} &\approx 8.45
\end{align*}
Z zgornjimi podatki lahko narišemo
odločitveno drevo s slike~\fig[zastoj-b].

Naj bo $Y$ trajanje izbrane popoldanske poti v minutah.
Da določimo pričakovano trajanje poti,
bomo najprej povzeli njene pričakovane vrednosti
v odvisnosti od jutranje poti.
\begin{align*}
u := E(Y \mid \text{po lokalni cesti zjutraj})
&= \begin{cases}
23 + 1.4a & a < 3.57 \\
28 & a \ge 3.57
\end{cases} \\
v := E(Y \mid \text{po avtocesti in zastoj zjutraj})
&= \begin{cases}
27.8 + 1.64a & a < 0.12 \\
28 & a \ge 0.12
\end{cases} \\
w := E(Y \mid \text{po avtocesti in ni zastoja zjutraj})
&= \begin{cases}
18.2 + 1.16a & a < 8.45 \\
28 & a \ge 8.45
\end{cases}
\end{align*}
Sedaj lahko izračunamo pričakovano trajanje poti.
\begin{multline*}
E(Y) = 0.7 u + 0.06 v + 0.24 w = \\
= \begin{cases}
22.136 + 1.3568 a \in [22.136, 22.301) & 0 \le a < 0.12 \\
22.148 + 1.2584 a \in [22.301, 26.642) & 0.12 \le a < 3.57 \\
25.648 + 0.2784 a \in [26.642, 28) & 3.57 \le a < 8.45 \\
28 & 8.45 \le a \le 10
\end{cases}
\end{multline*}
\end{enumerate}

\begin{slika}
\makebox[\textwidth][c]{
\pgfslika[zastoj-a]
}
\podnaslov[\res{}(a)]{Odločitveno drevo}
\end{slika}

\begin{slika}
\makebox[\textwidth][c]{
\pgfslika[zastoj-b]
}
\podnaslov[\res{}(b) v odvisnosti od vrednosti parametra $a$]%
{Odločitveno drevo}
\end{slika}
\end{odgovor}
\end{naloga}
