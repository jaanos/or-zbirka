\begin{naloga}{Janoš Vidali}{Kolokvij OR 14.4.2025}
\begin{vprasanje}
Krt je ravnokar izkopal nov rov in se odloča,
kje naj iz rova odkoplje krtine, da bi se vanje ujelo čim več črvov.
Rov je razdelil na $n$ zaporednih delov, iz katerih lahko odkoplje krtino
-- če se jo odloči odkopati na nekem delu,
bo odvečno zemljo odmetal na prejšnji ali naslednji del,
tako da tam ne bo mogel izkopati krtine
(če izkoplje krtino na prvem ali zadnjem delu,
potem bo moral zemljo odmetati na drugi oziroma predzadnji del,
saj ga z druge strani omejujejo drevesa).
Prav tako lahko na posamezen del odmeče zemljo samo iz ene krtine,
sicer bi zemlje bilo preveč na kupu in bi se usipala nazaj.
Za $i$-ti del ($1 \le i \le n$) pozna vrednost $a_i$,
ki pove, koliko črvov bi se vsak dan ujelo na krtino na tem delu.


\begin{enumerate}[(a)]
\item Zapiši rekurzivne enačbe,
s katerimi poiščemo ustrezno postavitev krtin,
v katere se bo ujelo največ črvov.
Razloži, kaj predstavljajo spremenljivke,
v kakšnem vrstnem redu jih računamo, ter kako dobimo optimalno rešitev.

\item Kakšna je časovna zahtevnost algoritma,
ki sledi iz tvojih rekurzivnih enačb?

\item Reši nalogo pri podatkih $n = 8$
in $(a_i)_{i=1}^8 = (94, 81, 47, 57, 73, 84, 6, 14)$.
\end{enumerate}
\end{vprasanje}

\begin{odgovor}
\begin{enumerate}[(a)]
\item Definirajmo spremenljivko $v_i$ ($0 \le i \le n$)
kot največje število črvov,
ki se lahko ujame v krtine na delih od $1$ do $i$,
če je tudi vsa odmetana zemlja na teh delih.
Zapišimo začetne pogoje in rekurzivne enačbe.
\begin{align*}
v_0 &= v_1 = 0 \\
v_i &= \max\{v_{i-1}, v_{i-2} + a_{i-1}, v_{i-2} + a_i\}
 & (2 \le i \le n)
\end{align*}
Tukaj smo upoštevali, da lahko krt dela $i$ ne uporabi,
ali pa da izkoplje krtino na delu $i-1$ in odmeče zemljo na del $i$,
oziroma izkoplje krtino na delu $i$ in odmeče zemljo na del $i-1$.

Vrednosti $v_i$ ($0 \le i \le n$) računamo naraščajoče po indeksu $i$.
Največje število črvov, ki se lahko dnevno ujamejo v krtine,
je tako $v^* = v_n$.

\item Posamezno vrednost $v_i$ ($0 \le i \le n$)
lahko izračunamo v konstantnem času.
Ker moramo izračunati $O(n)$ vrednosti,
tudi za celoten izračun porabimo $O(n)$ časa.

\item Izračunajmo vrednosti $v_i$ ($2 \le i \le 8$).
\begin{alignat*}{2}
v_2 &= \max\{0, \underline{0 + 94}, 0+ 81\} &&= 94 \\
v_3 &= \max\{\underline{94}, 0 + 81, 0 + 47\} &&= 94 \\
v_4 &= \max\{94, 94 + 47, \underline{94 + 57}\} &&= 151 \\
v_5 &= \max\{151, 94 + 57, \underline{94 + 73}\} &&= 167 \\
v_6 &= \max\{167, 151 + 73, \underline{151 + 84}\} &&= 235 \\
v_7 &= \max\{235, \underline{167 + 84}, 167 + 6\} &&= 251 \\
v_8 &= \max\{\underline{251}, 235 + 6, 235 + 14\} &&= 251
\end{alignat*}
V krtine se bo torej vsak dan ujelo $v^* = v_8 = 251$ črvov.
Poglejmo, kako naj jih krt postavi.
\begin{align*}
v^* = v_8 &= v_7 & \text{ne uporabi dela $8$} \\
v_7 &= v_5 + a_6 & \text{krtina na delu $6$, zemlja na delu $7$} \\
v_5 &= v_3 + a_5 & \text{krtina na delu $5$, zemlja na delu $4$} \\
v_3 &= v_2 & \text{ne uporabi dela $3$} \\
v_2 &= v_0 + a_1 & \text{krtina na delu $1$, zemlja na delu $2$}
\end{align*}
\end{enumerate}
\end{odgovor}
\end{naloga}
