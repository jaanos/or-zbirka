\begin{naloga}{Janoš Vidali}{Izpit OR 26.6.2024}
\begin{vprasanje}
Dobavitelj ima na zalogi $m$ različnih surovin,
za katere se zanima $n$ kupcev.
Naj bo $c_{ij}$ ($1 \le i \le m$, $1 \le j \le n$) cena,
po kateri je $j$-ti kupec pripravljen kupiti celotno zalogo $i$-te surovine,
ter $d_j$ ($1 \le j \le n$) kapital,
ki ga ima na voljo $j$-ti kupec za nakup surovin.
Dobavitelj se lahko odloči, katerim kupcem bo prodal katero surovino,
pri čemer bo vsak kupec dobil enak delež posamezne surovine
in plačal sorazmerno ceno zanjo
(npr. če se dobavitelj odloči prodati neko surovino trem kupcem,
bo vsak od njih dobil tretjino zaloge te surovine
in zanjo plačal tretjino ponujene cene).
Dobavitelj želi odprodati celotno zalogo vseh surovin
in pri tem čim več zaslužiti,
pri čemer mora vsak kupec dobiti takšne količine surovin,
da jih bo lahko plačal po ponujeni ceni
(tj., skupni stroški posameznega kupca ne smejo preseči njegovega kapitala).

Zapiši celoštevilski linearni program, ki modelira zgoraj opisani problem.
\namig{uvedi spremenljivke, ki beležijo število kupcev,
ki bodo dobili posamezno surovino.}
\end{vprasanje}

\begin{odgovor}
Za $i$-to surovino ($1 \le i \le m$),
$j$-tega kupca ($1 \le j \le n$) in število $k$ ($1 \le k \le n$)
bomo uvedli spremenljivki $x_{ijk}$ in $y_{ik}$,
katerih vrednosti interpretiramo kot
\begin{align*}
x_{ijk} &= \begin{cases}
1 & \text{$j$-temu kupcu prodamo ${1 \over k}$ zaloge $j$-te surovine, in} \\
0 & \text{sicer; ter}
\end{cases} \\
y_{ik} &= \begin{cases}
1 & \text{$i$-to surovino prodamo $k$ kupcem, in} \\
0 & \text{sicer.}
\end{cases}
\end{align*}
Zapišimo celoštevilski linearni program.
\begin{alignat*}{3}
& \max &\ \sum_{i=1}^m \sum_{j=1}^n \sum_{k=1}^n {c_{ij} \over k} &{} x_{ijk} && \text{p.p.} \\
\forall i \in \{1, \dots, m\} \ \forall j, k \in \{1, \dots, n\}: &\ &
0 \le x_{ijk} &\le 1, & x_{ijk} &\in \Z \\
\forall i \in \{1, \dots, m\} \ \forall k \in \{1, \dots, n\}: &\ &
0 \le y_{ik} &\le 1, & y_{ik} &\in \Z
\opis{Surovine prodamo ustreznemu številu kupcev}
\forall i \in \{1, \dots, n\} \ \forall k \in \{1, \dots, n\}: &\ &
\sum_{j=1}^n x_{ijk} &= k y_{ik}
\opis{Za vsako surovino je določeno število kupcev}
\forall i \in \{1, \dots, m\}: &\ &
\sum_{k=1}^n y_{ik} &= 1
\opis{Kupci ne presežejo svojega kapitala}
\forall j \in \{1, \dots, n\}: &\ &
\sum_{i=1}^m \sum_{k=1}^n {c_{ij} \over k} x_{ijk} &\le d_j
\end{alignat*}
\end{odgovor}
\end{naloga}

