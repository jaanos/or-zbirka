\begin{naloga}{Boštjan Gabrovšek}{Izpit OR 9.2.2011}
\begin{vprasanje}
Selektor košarkaške reprezentance bi rad sestavil petčlansko začetno postavo, ki bo imela povprečno
višino kar se da visoko. Na voljo ima sledeče igralce:
\begin{center}
\begin{tabular}{lll}
Igralec & Višina & Pozicija \\ \hline
Anže    & 213    & Center   \\
Borut   & 208    & Center   \\
Ciril   & 203    & Krilo    \\
David   & 192    & Krilo    \\
Emil    & 196    & Krilo    \\
Filip   & 197    & Obramba  \\
Gorazd  & 200    & Obramba  \\
Hugo    & 195    & Obramba  \\
\end{tabular}
\end{center}
Pri izbiri mora selektor upoštevati naslednje pogoje:
\begin{itemize}
\item v začetni postavi morajo biti zastopane vse tri pozicije,
\item v rezervi mora biti bodisi Ciril bodisi Filip, ne pa oba,
\item v začetni postavi je lahko največ en center,
\item če začne Borut ali David, mora Hugo ostati v rezervi.
\end{itemize}
Formuliraj problem kot celoštevilski linearni program.
\end{vprasanje}
\begin{odgovor}
\end{odgovor}
\end{naloga}
