\begin{naloga}{Janoš Vidali}{Kolokvij OR 12.4.2021}
\begin{vprasanje}
V večjem podjetju so se odločili za nabavo licenc za programsko opremo,
in sicer bodo za vsakega zaposlenega iz množice $Z$ nabavili natanko eno licenco.
Razvijalec svojo programsko opremo ponuja v različicah $1, 2, \dots, n$ z različnimi nabori funkcionalnosti.
V podjetju so identificirali množico $F$ funkcionalnosti, ki jih zanimajo,
ter določili vrednosti $p_{zf}$ in $q_{fi}$
($z \in Z$, $f \in F$, $1 \le i \le n$),
ki povedo sledeče:
\begin{align*}
p_{zf} &= \begin{cases}
1 & \text{če zaposleni $z$ potrebuje funkcionalnost $f$, in} \\
0 & \text{sicer,}
\end{cases} \\
\shortintertext{ter}
q_{fi} &= \begin{cases}
1 & \text{če različica $i$ ponuja funkcionalnost $f$, in} \\
0 & \text{sicer.}
\end{cases}
\end{align*}
Cena, ki jo v podjetju plačajo za posamezno licenco,
je odvisna od števila licenc,
ki jih kupijo za posamezno različico programske opreme:
za posamezno licenco za različico $i$ plačajo ceno $c_{ij}$,
če kupijo vsaj $r_j$ takih licenc ($1 \le j \le m$).
Lahko predpostaviš, da velja $1 = r_1 < r_2 < \cdots < r_m$
in $c_{i1} \ge c_{i2} \ge \cdots \ge c_{im} > 0$ ($1 \le i \le n$).

\begin{enumerate}[(a)]
\item Zapiši celoštevilski linearni program,
ki modelira iskanje najcenejšega izbora licenc,
tako da bo vsak zaposleni dobil različico programske opreme,
ki ponuja vse funkcionalnosti, ki jih potrebuje.

\item Celoštevilskemu linearnemu programu iz točke (a)
dodaj omejitve za sledeče dodatne pogoje.

\begin{itemize}
\item Zaposlena $a, b \in Z$ morata dobiti licenco za enako različico programske opreme.
\item Vsako funkcionalnost, ki jo ima na voljo katerikoli zaposleni,
mora v svoji različici programske opreme imeti na voljo tudi direktor $d \in Z$.
\item Če ima katerikoli zaposleni licenco za tako različico,
ki ponuja funkcionalnost $e \in F$,
potem mora funkcionalnost $e$ biti na voljo vsem zaposlenim.
\end{itemize}
\end{enumerate}
\end{vprasanje}

\begin{odgovor}
Za zaposlenega $z \in Z$ in $1 \le i \le n$, $1 \le j \le m$
bomo uvedli spremenljivko $x_{zij}$,
katere vrednost interpretiramo kot
$$
x_{zij} = \begin{cases}
1 & \text{zaposlenemu $z$ kupijo licenco za različico $i$ po ceni $c_{ij}$, in} \\
0 & \text{sicer.}
\end{cases}
$$

\begin{enumerate}[(a)]
\item Zapišimo celoštevilski linearni program.
\begin{alignat*}{2}
\min &\ &\sum_{z \in Z} \sum_{i=1}^n \sum_{j=1}^m c_{ij} x_{zij} &\quad \text{p.p.} \\
\forall z \in Z \ \forall i \in \{1, \dots, n\} \ \forall j \in \{1, \dots, m\}: &\ &
0 \le x_{zij} &\le 1, \quad x_{zij} \in \Z
\opis{Vsak zaposleni dobi natanko eno licenco}
\forall z \in Z: &\ &
\sum_{i=1}^n \sum_{j=1}^m x_{zij} &= 1
\opis{Vsak zaposleni ima na voljo vse funkcionalnosti, ki jih potrebuje}
\forall z \in Z \ \forall f \in F \ \forall i \in \{1, \dots, n\}: &\ & \sum_{j=1}^m x_{zij} &\le q_{fi} - p_{zf} + 1
\opis{Ustrezna cena glede na število licenc}
\forall z \in Z \ \forall i \in \{1, \dots, n\} \ \forall j \in \{1, \dots, m\}: &\ & \sum_{y \in Z} x_{yij} &\ge r_j x_{zij}
\end{alignat*}

\item Zapišimo še dodatne omejitve.
\odstraniprostor
\begin{alignat*}{2}
\opis{$a$ in $b$ dobita enako licenco}
\forall i \in \{1, \dots, n\}: &\ & \sum_{j=1}^m (x_{aij} - x_{bij}) &= 0
\opis{Direktor ima na voljo vsako funkcionalnost, ki jo imajo na voljo ostali zaposleni}
\forall z \in Z \ \forall f \in F \ \forall h, i \in \{1, \dots, n\} \ \forall j, k \in \{1, \dots, m\}: &\ & x_{zij} + x_{dhk} &\le q_{fh} - q_{fi} + 2
\opis{Če ima kdorkoli funkcionalnost $e$, potem jo imajo vsi}
\forall y, z \in Z \ \forall h, i \in \{1, \dots, n\} \ \forall j, k \in \{1, \dots, m\}: &\ & x_{zij} + x_{yhk} &\le q_{eh} - q_{ei} + 2
\end{alignat*}
\end{enumerate}
\end{odgovor}
\end{naloga}

