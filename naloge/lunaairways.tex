\begin{naloga}{Jernej Azarija}{Izpit OR 12.5.2016}
\begin{vprasanje}
Smo v letu 2500 in turistične rakete že potujejo na Luno.
LunaAirways oglašuje let, na katerem je lahko $100$ potnikov.
LunaAirways računa, da bodo vsa mesta zasedena.
Z leti izkušenj vedo,
da nekatere potnike zagrabi strah in tako ne pridejo na potovanje.
Ocenjujejo, da je verjetnost $p_i$,
da natanko $i$ potnikov ($0 \le i \le 5$) ne pride na potovanje,
naslednje:
\begin{center}
\begin{tabular}{c|cc}
$i$ & od $0$ do $3$ & od $4$ do $5$ \\ \hline
$p_i$ & $0.2$ & $0.1$
\end{tabular}
\end{center}

S prodajo karte ima družba $300$ denot dobička.
Za vsakega potnika, ki bo moral menjati let,
ima LunaAirways $400$ denot stroškov.
Koliko kart naj proda LunaAirways, če želi maksimizirati pričakovani dobiček?
\end{vprasanje}

\begin{odgovor}
    Naj bo $X_k$ pričakovani dobiček, če družba LunaAirways proda $k$ kart.\\
    \begin{alignat*}{2}
        E(X_{100}) &= 100 \cdot 300\;\text{denot} &&= 30\,000\;\text{denot} \\
        E(X_{101}) &= 101 \cdot 300\;\text{denot} - 0.2 \cdot 400\;\text{denot} &&= 30\,220\;\text{denot}\\
        E(X_{102}) &= 102 \cdot 300\;\text{denot} - 0.2 \cdot (2+1) \cdot 400\;\text{denot} &&= 30\,360\;\text{denot}\\
        E(X_{103}) &= 103 \cdot 300\;\text{denot} - 0.2 \cdot (3+2+1) \cdot 400\;\text{denot} &&= 30\,420\;\text{denot}\\
        E(X_{104}) &= 104 \cdot 300\;\text{denot} - 0.2 \cdot (4+3+2+1) \cdot 400\;\text{denot} &&= 30\,400\;\text{denot}\\
        E(X_{105}) &= 105 \cdot 300\;\text{denot} - (0.2 \cdot (5+4+3+2) + 0.1) \cdot 400\;\text{denot} &&= 30\,340\;\text{denot}\\
        E(X_{106}) &= 106 \cdot 300\;\text{denot} - (0.2 \cdot (6+5+4+3) + 0.1 \cdot (2+1)) \cdot 400\;\text{denot}&&= 30\,240\;\text{denot}\\
    \end{alignat*}
Poglejmo si malo bolj podrobno primer, ko družba proda $103$ karte.
Najprej izračunamo kolikšen dobiček ustvarimo s prodajo $103$ kart, nato pa upoštevamo, da z verjetnostjo $0.2$ nobenega ne zagrabi strah,
torej pride na potovanje vseh $103$ ljudi, in bodo morali sedaj trije ljudje zamenjati let. Podoben razmislek naredimo tudi za primere, če strah zagrabi
1 ali 2 človeka, takrat bosta morala let zamenjati 2 oz. 1 človek, in ker se vsi trije primeri zgodijo z isto verjetnostjo, smo izpostavili to verjetnost.\\
Družba pa naj, za maksimizacijo svojega dobička, proda $103$ karte.
\end{odgovor}
\end{naloga}
