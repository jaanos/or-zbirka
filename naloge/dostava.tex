\begin{naloga}{Janoš Vidali}{Izpit OR 5.6.2023}
\begin{vprasanje}
Preko storitve za dostavo želimo naročiti $m$ različnih artiklov,
ki jih lahko naročimo iz $n$ različnih trgovin.
Za vsaka $i, j$ ($1 \le i \le m$, $1 \le j \le n$) imamo sledeče podatke:
$a_i$ je število kosov $i$-tega artikla, ki jih želimo naročiti,
$b_{ij}$ je število kosov $i$-tega artikla, ki so na voljo v $j$-ti trgovini,
$c_{ij}$ je cena za kos $i$-tega artikla v $j$-ti trgovini,
$d_j$ je strošek dostave iz $j$-te trgovine
(plačamo ga, če iz nje naročimo vsaj en artikel)
in $e_j$ je minimalni znesek naročila (brez stroškov dostave) iz $j$-te trgovine
(naročila ne moremo opraviti, če naročimo za manj kot minimalni znesek).
Določiti želimo,
koliko kosov posameznega artikla bomo naročili iz vsake trgovine,
da bodo skupni stroški čim manjši.

Zapiši celoštevilski linearni program,
ki modelira iskanje optimalne rešitve zgornjega problema.
\end{vprasanje}

\begin{odgovor}
Za $j$-to trgovino ($1 \le j \le n$) bomo uvedli spremenljivko $x_j$,
katere vrednost interpretiramo kot
$$
x_j = \begin{cases}
1 & \text{naročamo iz $j$-te trgovine, in} \\
0 & \text{sicer.}
\end{cases}
$$
Poleg tega bomo za $i$-ti artikel in $j$-to trgovino
($1 \le i \le m$, $1 \le j \le n$)
uvedli še spremenljivko $y_{ij}$,
ki pove, koliko kosov $i$-tega artikla bomo naročili iz $j$-te trgovine.

Zapišimo celoštevilski linearni program.
\needspace{2\baselineskip}
\begin{alignat*}{3}
\min &\ & \sum_{j=1}^n \Big(d_j x_j &+ \sum_{i=1}^m c_{ij} y_{ij}\Big)
&\quad \text{p.p.} \\
\forall j \in \{1, \dots, n\}: &\ & 0 \le x_j &\le 1, & x_j &\in \Z
\opis{Število kosov je omejeno z zalogo in naročilom}
\forall i \in \{1, \dots, m\} \ \forall j \in \{1, \dots, n\}: &\ &
0 \le y_{ij} &\le b_{ij} x_j, & y_{ij} &\in \Z
\opis{Naročimo želeno število kosov artikla}
\forall i \in \{1, \dots, m\}: &\ & \sum_{j=1}^n y_{ij} &= a_i
\opis{Če naročamo iz trgovine, je dosežena minimalna vrednost naročila}
\forall j \in \{1, \dots, n\}: &\ & \sum_{i=1}^m c_{ij} y_{ij} &\ge e_j x_j
\end{alignat*}
\end{odgovor}
\end{naloga}

