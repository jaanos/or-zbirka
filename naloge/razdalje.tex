\begin{naloga}{Sergio Cabello}{Izpit OR 15.3.2017}
\begin{vprasanje}
Dan je usmerjen graf $G = (V, E)$ s pozitivnimi dolžinami povezav.
Dano imamo še vozlišče $s \in V$
ter vrednost $\delta_v$ za vsako vozlišče $v \in V$.
Natančno opiši algoritem (z besedami ali psevdokodo),
ki v času $O(m)$ (kjer je $m = |V| + |E|$) preveri,
ali velja $\delta_v = d_G(s, v)$ za vse $v \in V$
-- t.j., v linearnem času preveri,
ali so vrednosti $\delta_v$
enake razdalji med vozliščema $s$ in $v$ v grafu $G$.

Lahko predpostaviš, da so vsa vozlišča grafa $G$ dosegljiva iz vozlišča $s$.
Utemelji pravilnost meje za časovno zahtevnost algoritma.
\end{vprasanje}
\begin{odgovor}

Ideja algoritma je tesno povezana z algoritmom {\sc dijkstra} iz naloge~\nal[dijkstra].
Lahko predpostavimo, da v primeru pravilnosti vseh vhodnih razdalj, so bile te poiskane z dijkstrovim algoritmom,
torej, če sledimo poti, ki jo je ta za nas pustil, bomo lahko preverili prav vse nastavljene razdalje v $\delta$.

Zapišimo algoritem

\begin{small}
\begin{algorithmic}
\Function{razdalje}{$G = (V, E), s, \delta$}
	\If{$\delta_s \neq 0$}
		\State \Return \False
	\EndIf
	\State $d_v \gets \infty$ za  $\forall v \in V$
	\State $d_s \gets 0$
	\State dosezeni$[v] \gets \False$ za $\forall v \in V$
	\State sklad $\gets [s]$
	\While{$\neg\isEmpty($sklad$)$}
		\State $u \gets$ sklad.$\pop()$
		\State dosezeni$[u] \gets \True$
		\For{$v \in \Adj(G, u)$}
			\If{$d_u + \ell_{uv} = \delta_v$}
				\State $d_v \gets d_u + \ell_{uv}$
				\If{$\neg\text{dosezeni}[v]$}
					\State stack$.\append(v)$
				\EndIf
			\ElsIf{$d_u + \ell_{uv} < \delta_v$}
				\State \Return \False
			\EndIf
		\EndFor
	\EndWhile
	\State \Return $\operatorname{all}$(dosezeni)
\EndFunction
\end{algorithmic}
\end{small}

V primeru, ko je $\delta$ pravilno nastavljen, bo algoritem deloval pravilno, 
za dokaz se lahko sklicujemo na dijkstrov algoritem in predpostavko, da so vsa vozlišča dosegljiva iz $s$.

Če bo neko vozlišče $u$ v $\delta$ imelo manjšo razdaljo kot ta s katero lahko do njega pridemo iz $s$,
potem ga algoritem ne bo označil, saj bi moral biti $u$, v primeru označitve dosegljiv s to dolžino iz $s$.

Preostane nam še primer, ko ima vozlišče v $\delta$ previsoko razdaljo.
Tu ga lahko označimo, a le če pridemo do njega z manj kot optimalno potjo, 
torej je to vozlišče dosegljivo še iz vsaj ene druge poti.
Pravilno delovanje algoritma v tem primeru ni povsem očitno, torej si poglejmo dokaz.
Predpostavimo, da je bilo previsoko nastavljeno vozlišče $t$ in primer, ko smo, vključno s $t$,
označili vsa vozlišča v grafu (sicer bo algoritem vrnil \False).
Ker je $t$ bil označeno smo do njega lahko prišli z neko potjo 
$$u_1 \rightarrow u_2 \rightarrow \dots \rightarrow u_{n-1} \rightarrow u_n = t,$$
na kateri so {\bf vsa} vozlišča nastavljena previsoko.
Ker je $s$ nastavljen pravilno in v grafu obstaja še vsaj ena pot do $t$ in ta seka prejšnjo, je ta pot oblike
$$s \rightarrow v_1 \rightarrow v_2 \rightarrow \dots \rightarrow v_m \rightarrow u_i,$$
kjer $i \in {1, 2, \dots, n}$.
Ker se pot začne v $s$ (označen pravilno) in so vsa vozlišča na njej označena po predpostavki, so nastavljena pravilno,
torej bi morala biti napaka zaznana.\bigskip

Algoritem pregleda vsako vozlišče največ enkrat in pri vsakem njegovem sosedu porabi konstantno časa.
Časovno zahtevnost torej lahko izračunamo kot 
$$T(n)  = \sum_{u \in V}\left(deg_G(u) \times O(1) + O(1)\right) = O(|E|) + O(|V|) = O(|V| + |E|)).$$
\bigskip

Poglejmo si algoritem na primeru~\fig, pri čemer vzamemo $\delta_v = d_g(v)$ za $ \forall v \in V \setminus \{T\}$ in $\delta_T = 16$.
Izvajanje je predstavljeno s tabelo~\tab.
Na zadnjem koraku so vsa vozlišča označena, a velja $d_u + \ell_{uv} < \delta_v$, torej algoritem vrne $\False$.
\begin{slika}
\pgfslika
\podnaslov[\res{}]{Primer}
\end{slika}
\begin{tabela}
\makebox[\textwidth][c]{
\begin{tabular}{cc|cccc|c}
$u$ & $v$ & $\delta_v$ & $d_u$ & $d_v$ & $\ell_{uv}$ & dosezen$[v]$ \\ \hline
$S$ & $U$ & $1$ & $0$ & $\infty$ & $1$ & $\False$\\
$S$ & $V$ & $2$ & $0$ & $\infty$ & $2$ & $\False$\\
$V$ & $Z$ & $6$ & $2$ & $\infty$ & $4$ & $\False$\\
$Z$ & $T$ & $16$ & $6$ & $\infty$ & $10$ & $\False$\\
$U$ & $W$ & $2$ & $1$ & $\infty$ & $1$ & $\False$\\
$W$ & $T$ & $16$ & $2$ & $16$ & $1$ & $\True$
\end{tabular}
}
\podnaslov[\res{}]{potek izvajanja algoritma {\sc razdalje}}
\end{tabela}

\end{odgovor}
\end{naloga}

