\begin{naloga}{Sergio Cabello}{Izpit OR 15.3.2017}
\begin{vprasanje}
Dan je usmerjen graf $G = (V, E)$ s pozitivnimi dolžinami povezav.
Dano imamo še vozlišče $s \in V$
ter vrednost $\delta_v$ za vsako vozlišče $v \in V$.
Natančno opiši algoritem (z besedami ali psevdokodo),
ki v času $O(m)$ (kjer je $m = |V| + |E|$) preveri,
ali velja $\delta_v = d_G(s, v)$ za vse $v \in V$
-- t.j., v linearnem času preveri,
ali so vrednosti $\delta_v$
enake razdalji med vozliščema $s$ in $v$ v grafu $G$.

Lahko predpostaviš, da so vsa vozlišča grafa $G$ dosegljiva iz vozlišča $s$.
Utemelji pravilnost meje za časovno zahtevnost algoritma.
\end{vprasanje}

\begin{odgovor}
Ideja algoritma je tesno povezana z algoritmom {\sc Dijkstra}
iz naloge~\res[dijkstra].
Če so vse vhodne razdalje pravilne,
lahko predpostavimo, da so bile poiskane z Dijkstrovim algoritmom.
Če torej sledimo poti, ki jo ta opravi,
bomo lahko preverili prav vse nastavljene razdalje $\delta_v$.
\begin{small}
\begin{algorithmic}
\Function{Razdalje}{$G = (V, E), s \in V, L : E \rightarrow \R_{+},
                     (\delta_v)_{v \in V}$}
	\If{$\delta_s \neq 0$}
		\State \Return \False
	\EndIf
    \State {\sl doseženi} $\gets$ slovar z vrednostjo $\False$
                                  za vsak $v \in V$
	\State $\ell \gets [s]$
	\While{$\neg\ell.\isEmpty()$}
		\State $u \gets \ell.\pop()$
		\State {\sl doseženi}$[u] \gets \True$
		\For{$v \in \Adj(G, u)$}
			\If{$\delta_u + L_{uv} = \delta_v$}
				\If{$\neg${\sl doseženi}$[v]$}
					\State $\ell.\append(v)$
				\EndIf
			\ElsIf{$\delta_u + L_{uv} < \delta_v$}
				\State \Return \False
			\EndIf
		\EndFor
	\EndWhile
	\State \Return $\forall v \in V :$ {\sl doseženi}$[v]$
\EndFunction
\end{algorithmic}
\end{small}

V primeru, ko vrednosti $\delta_v$ ustrezajo razdaljam od $s$,
bo algoritem deloval pravilno,
kar sledi iz pravilnosti Dijkstrovega algoritma in predpostavke,
da so vsa vozlišča dosegljiva iz $s$.

Če bo za neko vozlišče $v$ vrednost $\delta_v$ manjša od razdalje $d_G(s, v)$,
potem algoritem vozlišča $v$ ne bo označil.
Če pa je vrednost $\delta_v$ večja od razdalje $d_G(s, v)$,
algoritem morda lahko označi vozlišče $v$,
a se to lahko zgodi le, če pridemo do njega z manj kot optimalno potjo
-- to vozlišče je torej dosegljivo še iz vsaj ene druge poti.
Predpostavimo, da je $t$ najvišje vozlišče v drevesu najkrajših poti,
ki ga algoritem označi, čeprav velja $\delta_t > d_G(s, t)$,
in naj bo $r$ njegov predhodnik v drevesu.
Ker velja $\delta_r = d_G(s, r)$,
bo algoritem pri pregledovanju sosedov vozlišča $r$ ugotovil,
da velja $\delta_r + L_{rt} < \delta_t$, in torej vrnil $\False$.

Algoritem pregleda vsako vozlišče največ enkrat in pri vsakem njegovem sosedu porabi konstantno časa.
Časovno zahtevnost lahko torej izračunamo kot 
$$
\sum_{v \in V}\left(O(1) + \deg_G(u) \cdot O(1)\right)
= O(|V|) + O(|E|) = O(|V| + |E|).
$$

Poglejmo si, kako algoritem deluje na grafu $G = (V, E)$ s slike~\fig,
pri čemer vzamemo $\delta_v = d_G(a, v)$ za vsak $v \in V \setminus \{f\}$
in $\delta_f = 16$.
Izvajanje algoritma z začetnim vozliščem $a$ je prikazano v tabeli~\tab.
Na zadnjem koraku so vsa vozlišča označena, a velja $d_u + L_{uv} < \delta_v$,
tako da algoritem vrne $\False$.

\begin{slika}
\pgfslika
\podnaslov{Primer grafa}
\end{slika}

\begin{tabela}
\makebox[\textwidth][c]{
\begin{tabular}{cc|ccc|c}
$u$ & $v$ & $\delta_u$ &  $\delta_v$ & $L_{uv}$ & {\sl dosežen}$[v]$ \\
\hline
$a$ & $b$ & $0$ & $2$  & $2$  & $\False$ \\
$a$ & $c$ & $0$ & $1$  & $1$  & $\False$ \\
$b$ & $d$ & $2$ & $6$  & $4$  & $\False$ \\
$c$ & $e$ & $1$ & $2$  & $1$  & $\False$ \\
$d$ & $f$ & $6$ & $16$ & $10$ & $\False$ \\
$e$ & $f$ & $2$ & $16$ & $1$  & $\True$
\end{tabular}
}
\podnaslov{Potek izvajanja algoritma na primeru s slike~\fig}
\end{tabela}

\end{odgovor}
\end{naloga}
