\begin{naloga}{David Gajser}{Izpit OR 24.6.2013}
\begin{vprasanje}
Potujoči trgovec Marco načrtuje pot, na kateri bo imel kar največji dobiček.

Dan je usmerjen acikličen graf $G = (V, E)$ z utežmi na povezavah.
Uteži predstavljajo dobiček na povezavi (oz.~izgubo, če so negativne).
Marco bo začel svojo pot v enem izmed vozlišč grafa
in se po usmerjenih povezavah odpravil do končnega vozlišča
(začetek in konec nista znana, moramo ju še določiti).
Pomagaj Marcu najti najboljšo pot (vključno z začetnim in končnim vozliščem)!
Kakšen dobiček si lahko obeta?

\begin{enumerate}[(a)]
\item Opiši algoritem, ki reši dani problem,
ter obravnavaj njegovo časovno zahtevnost.
\namig{uporabi dinamično programiranje,
da bo tvoj algoritem tekel v času $O(|E| + |V|)$}.

\item Reši nalogo za graf s slike~\fig
(razloži, kako si prišel do rezultata, oz.~jasno označi vmesne rezultate).
\end{enumerate}

\begin{slika}
\pgfslika
\podnaslov{Graf}
\end{slika}
\end{vprasanje}

\begin{odgovor}
\begin{enumerate}[(a)]
\item Denimo, da za vsako povezavo $e \in E$
poznamo njeno utež $w_e$.
Naj bo $x_u$ ($u \in V$) največji dobiček,
ki ga lahko Marco doseže, če svojo pot konča v vozlišču $u$.
Zapišimo rekurzivne enačbe.
$$
x_u = \max\left(\{0\} \cup \set{x_v + w_{vu}}{vu \in E}\right)
\qquad (u \in V)
$$
Vrednosti $x_u$ računamo glede na topološko ureditev grafa.
Optimalni dobiček dobimo kot $x^* = \max\set{x_u}{u \in V}$,
optimalno pot pa lahko rekonstruiramo glede na vrednosti,
ki smo jih uporabili pri računanju $x^*$.
Časovna zahtevnost algoritma je $O(n + m)$,
kjer je $n$ število vozlišč, $m$ pa število povezav podanega grafa.

\item Izračunajmo vrednosti $x_u$.
Uporabili bomo abecedni vrstni red vozlišč,
za katerega opazimo, da ustreza topološki ureditvi.
\begin{alignat*}{2}
x_a &&&= 0 \\
x_b &= \max\{\underline{0}, 0+(-1)\} &&= 0 \\
x_c &= \max\{0, \underline{0+1}\} &&= 1 \\
x_d &= \max\{0, \underline{0+6}\} &&= 6 \\
x_e &= \max\{0, 0+10, \underline{6+5}\} &&= 11 \\
x_f &= \max\{0, \underline{0+6}, \underline{1+5}\} &&= 6 \\
x_g &= \max\{0, \underline{1+10}, \underline{6+5}\} &&= 11 \\
x_h &= \max\{0, 6+5, \underline{11+2}\} &&= 13 \\
x_i &= \max\{0, 11+(-2), \underline{6+5}\} &&= 11 \\
x_j &= \max\{0, 13+(-5), \underline{11+(-2)}\} &&= 9 \\
x_k &= \max\{0, \underline{9+8}\} &&= 17 \\
x_\ell &= \max\{0, \underline{11+3}, 11+1\} &&= 14
\end{alignat*}
Največji dobiček, ki si ga lahko obeta Marco, je torej $x^* = 17$.
Rekonstruirajmo pot, ki nam da tak dobiček.
\begin{align*}
x^* = x_k &= x_j + w_{jk} & \text{gre po povezavi $jk$} \\
x_j &= x_i + w_{ij} & \text{gre po povezavi $ij$} \\
x_i &= x_f + w_{fi} & \text{gre po povezavi $fi$} \\
x_f &= x_c + w_{cf} & \text{gre po povezavi $cf$} \\
x_c &= x_a + w_{ac} & \text{gre po povezavi $ac$} \\
x_a &= 0 & \text{začne v vozlišču $a$}
\end{align*}
Optimalna pot je torej $a \to c \to f \to i \to j \to k$.
Ker velja tudi $x_f = x_b + w_{bf}$ in $x_b = 0$,
je druga optimalna pot $b \to f \to i \to j \to k$.
\end{enumerate}
\end{odgovor}
\end{naloga}
