\begin{naloga}{Janoš Vidali}{Kolokvij OR 11.6.2018}
\begin{vprasanje}
Vulkanizer v svoji delavnici izdeluje in prodaja avtomobilske pnevmatike.
Glede na trenutno povpraševanje vsak teden proda $60$ pnevmatik,
v istem času pa jih lahko izdela $90$.
Cena zagona proizvodnje je $180 €$,
cena skladiščenja posamezne pnevmatike pa je $0.5 €$ na teden.

\begin{enumerate}[(a)]
\item Denimo, da primanjkljaja ne dovolimo.
Izračunaj dolžino cikla pro\-iz\-vod\-nje in prodaje pnevmatik,
pri kateri so stroški najmanjši.
Kako veliko skladišče mora vulkanizer imeti?
Izračunaj tudi enotske stroške.

\item Kako dolgo naj pri zgornji rešitvi traja proizvodnja?
Koliko pnevmatik naj vulkanizer naredi v vsakem ciklu?

\item Vulkanizer je ocenil,
da bi ga primanjkljaj ene pnevmatike stal $2 €$ na teden.
Kakšna naj bosta dolžina cikla in velikost skladišča,
da bodo stroški čim manjši?
Izračunaj tudi enotske stroške in določi največji primanjkljaj.
\end{enumerate}
\end{vprasanje}

\begin{odgovor}
Imamo sledeče podatke (pri časovni enoti $1$ teden).
\begin{align*}
\nu &= 60 &
\lambda &= 90 \\
K &= 180 € &
s &= 0.5 €
\end{align*}
Pri teh podatkih izračunajmo še
$$
\alpha = \nu \left(1 - {\nu \over \lambda}\right)
= 60 \cdot \left(1 - {2 \over 3}\right) = 20 .
$$

\begin{enumerate}[(a)]
\item Če primanjkljaja ne dovolimo, imamo $p = \infty$ in torej $\beta = 1$.
Izračunajmo optimalno dolžino cikla $\tau^*$,
največjo zalogo $M^*$ in enotske stroške $S^*$.
\begin{alignat*}{3}
\tau^* &= \sqrt{2K \beta \over s \alpha}
&&= \sqrt{360 \over 10} &&= 6 \\
M^* &= \sqrt{2K \alpha \over s \beta}
&&= \sqrt{7\,200 \over 0.5} &&= 120 \\
S^* &= \sqrt{2Ks \alpha \over \beta}
&&= \sqrt{3\,600} € &&= 60 €
\end{alignat*}
Optimalna dolžina cikla je torej $6$ tednov,
pri tem pa mora vulkanizer imeti skladišče za $120$ pnevmatik.
Tedenski stroški proizvodnje in skladiščenja so $60 €$.

\item Izračunajmo še trajanje proizvodnje $t^*$
in število izdelanih pnevmatik $q^*$ v posameznem ciklu.
\begin{alignat*}{3}
t^* &= {M^* \over \lambda - \nu} &&= {120 \over 30} &&= 4 \\
q^* &= \tau^* \nu = t^* \lambda &&= 6 \cdot 60 &&= 360
\end{alignat*}
Proizvodnja torej traja $4$ tedne,
skupno pa v tem času izdela $360$ pnevmatik.

\item Vzamemo $p = 2 €$ in torej $\beta = 1 + s/p = 1.25$.
Izračunajmo optimalno dolžino cikla,
potrebno velikost skladišča in enotske stroške še v tem primeru.
\begin{alignat*}{3}
\tau^* &= \sqrt{2K \beta \over s \alpha}
&&= \sqrt{450 \over 10} &&\approx 6.708 \\
M^* &= \sqrt{2K \alpha \over s \beta}
&&= \sqrt{7\,200 \over 0.625} &&\approx 107.331 \\
S^* &= \sqrt{2Ks \alpha \over \beta}
&&= \sqrt{3\,600 \over 1.25} € &&\approx 53.666 €
\end{alignat*}
Optimalni interval naročanja je v tem primeru torej $6.708$ tednov,
pri tem pa mora vulkanizer imeti skladišče za $108$ pnevmatik.
Tedenski stroški pro\-iz\-vod\-nje in skladiščenja so $53.666 €$.

Izračunajmo še največji primanjkljaj $m^*$ v posameznem ciklu.
$$
m^* = \tau^* \alpha - M^* \approx 6.708 \cdot 20 - 107.331 \approx 26.833
$$
Največji primanjkljaj torej ne preseže $27$ pnevmatik.
\end{enumerate}

\end{odgovor}
\end{naloga}
