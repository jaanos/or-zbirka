\begin{naloga}{Alen Orbanić}{Izpit OR 15.9.2010}
\begin{vprasanje}
Z usmerjenim grafom je podano omrežje enosmernih prehodov
med letališkimi terminali (glej npr.~sliko~\fig[terminali]).
Načrtovalci niso čisto gotovi, ali je možno prehajati med vsemi terminali.
Predlagaj algoritem, ki bi ugotovil,
ali je vedno možen prehod med poljubnima izbranima terminaloma.
Kakšna je časovna zahtevnost predlaganega algoritma?

\nadaljevanje{terminali}
\end{vprasanje}

\begin{odgovor}
Na grafu lahko ponovimo iskanje v širino ali iskanje v globino
z začetkom v vsakem vozlišču,
ter preverimo, ali smo vsakič dosegli vsa vozlišča.
V obeh primerih je časovna zahtevnost postopka $O(mn)$,
kjer je $m$ število povezav in $n$ število vozlišč v grafu.
\end{odgovor}
\end{naloga}
