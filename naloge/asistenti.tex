\begin{naloga}{Janoš Vidali}{Izpit OR 15.12.2016}
\begin{vprasanje}
Na oddelku za matematiko je zaposlenih $n$ asistentov,
ki jim moramo dodeliti vaje pri $m$ predmetih.
Za asistenta $i$ ($1 \le i \le n$) naj bosta $a_i$ in $b_i$
najmanjše in največje število ur, ki jih lahko izvaja,
ter $N_i \subseteq \{1, 2, \dots, m\}$ množica predmetov,
ki jih ne želi izvajati.
Za predmet $j$ ($1 \le j \le m$)
naj bo $c_j$ število skupin za vaje pri predmetu,
ter $u_j$ število ur vaj na skupino.
Poleg tega vemo, da sta asistenta $p$ in $q$ skregana,
zato pri nobenem predmetu ne smeta oba izvajati vaj.

Predmete želimo asistentom dodeliti tako,
da bomo ob upoštevanju njihovih želja
minimizirali največje število različnih predmetov,
ki smo jih dodelili posameznemu asistentu.

Zapiši celoštevilski linearni program, ki modelira zgoraj opisani problem.
\namig{napiši program s spremenljivko $t$, ki je dopusten natanko tedaj,
ko vsak asistent dobi največ $t$ različnih predmetov,
in potem minimimiziraj $t$.}
\end{vprasanje}

\begin{odgovor}
Za $i$-tega asistenta ($1 \le i \le n$)
in $j$-ti predmet ($1 \le j \le m$)
bomo uvedli spremenljivki $x_{ij}$ in $y_{ij}$,
kjer je $x_{ij}$ število skupin,
ki jih $i$-temu asistentu dodelimo pri $j$-tem predmetu,
vrednost $y_{ij}$ pa interpretiramo kot
$$
y_{ij} = \begin{cases}
1, & \text{$i$-ti asistent bo izvajal vaje pri $j$-tem predmetu, in} \\
0  & \text{sicer.}
\end{cases}
$$
Poleg tega bomo uvedli še spremenljivko $t$,
s katero omejimo največje število različnih predmetov
pri posameznem asistentu.

Zapišimo celoštevilski linearni program.
\begin{alignat*}{2}
&& \min &\ t \quad \text{p.p.} \\
\forall i \in \{1, \dots, n\} \ \forall j \in \{1, \dots, m\}: &\ &
x_{ij} &\ge 0, \quad x_{ij} \in \Z \\
\forall i \in \{1, \dots, n\} \ \forall j \in \{1, \dots, m\}: &\ &
0 \le y_{ij} &\le 1, \quad y_{ij} \in \Z
\opis{$x_{ij} = 0$ natanko tedaj, ko $y_{ij} = 0$}
\forall i \in \{1, \dots, n\} \ \forall j \in \{1, \dots, m\}: &\ &
y_{ij} \le x_{ij} &\le c_j y_{ij}
\opis{Omejimo število ur za vsakega asistenta}
\forall i \in \{1, \dots, n\}: &\ & a_i \le \sum_{j=1}^m u_j x_{ij} &\le b_i
\opis{Neželenih predmetov ne dodelimo}
\forall i \in \{1, \dots, n\}: &\ & \sum_{j \in N_i} y_{ij} &= 0
\opis{Za vsak predmet dodelimo potrebno število skupin}
\forall j \in \{1, \dots, m\}: &\ & \sum_{i=1}^n x_{ij} &= c_j
\opis{Nobenega predmeta ne dodelimo asistentoma $p$ in $q$}
\forall j \in \{1, \dots, m\}: &\ & y_{pj} + y_{qj} &\le 1
\opis{Število različnih predmetov na asistenta omejimo s $t$}
\forall i \in \{1, \dots, n\}: &\ & \sum_{j=1}^n y_{ij} &\le t
\end{alignat*}
\end{odgovor}
\end{naloga}
