\begin{naloga}{Jernej Azarija}{Izpit OR 24.5.2016}
\begin{vprasanje}
Dan je graf $G = (V, E)$.
Razbitju $V_1, V_2, \dots, V_k$ množice vozlišč $V$
pravimo {\em $k$-barvno pakiranje},
če velja $d_G(u, v) > i$ za vsaki različni vozlišči $u, v \in V_i$
($1 \le i \le k$) --
tj., poljubni dve vozlišči iz $V_i$ sta v grafu $G$ oddaljeni za več kot $i$.
Minimalnemu številu $k$, za katero obstaja $k$-barvno pakiranje,
pravimo {\em mavrično pakirno število} in ga označimo s $\chi_p(G)$.
Napiši celoštevilski linearni program,
ki za dani graf $G$ izračuna $\chi_p(G)$.
\end{vprasanje}

\begin{odgovor}
Naj bo $n$ število vozlišč grafa $G$.
Očitno bomo lahko množico njegovih vozlišč razbili na največ $n$ podmnožic,
zato bomo za vozlišče $u \in V$ in $i \in \{1, \dots, n\}$
uvedli spremenljivko $x_{ui}$,
katere vrednost interpretiramo kot
$$
x_{ui} = \begin{cases}
1; & u \in V_i, \\
0; & u \not\in V_i.
\end{cases}
$$
Poleg tega bomo za vozlišči $u, v \in V$ uvedli še spremenljivko $y_{uv}$,
ki bo pred\-stav\-lja\-la razdaljo med njima.
Nazadnje bomo uvedli še spremenljivko $t$, ki bo štela,
koliko podmnožic potrebujemo.

Zapišimo celoštevilski linearni program.
\begin{alignat*}{3}
&& \min &\ t &\quad \text{p.p.} \\
\forall u \in V \ \forall i \in \{1, \dots, n\}: &\ &
0 \le x_{ui} &\le 1, & x_{ui} &\in \Z \\
\forall u, v \in V: &\ & y_{uv} &\ge 0, & y_{uv} &\in \Z \\
&& t &\ge 0, & t &\in \Z
\opis{Vsako vozlišče postavimo v eno podmnožico}
\forall u \in V: &\ & \sum_{i=1}^n x_{ui} &= 1
\opis{Štetje podmnožic}
\forall u \in V \ \forall i \in \{1, \dots, n\}: &\ & i x_{ui} &\le t
\opis{Razdalje v podmnožicah}
\forall u, v \in V, \ u \ne v: &\ &
(i+1) (x_{ui} + x_{vi} - 1) &\le y_{uv}
\opis{Razdalje med vozlišči}
\forall u \in V: &\ & y_{uu} &= 0 \\
\forall u \in V \ \forall vw \in E: &\ & -1 \le y_{uv} - y_{uw} &\le 1
\end{alignat*}
Mavrično pakirno število $\chi_p(G)$ dobimo kot vrednost ciljne funckije
zgornjega celoštevilskega linearnega programa.

Opomnimo,
da vrednosti $y_{uv}$ v dopustnih rešitvah zgornjega programa
ne predstavljajo nujno dejanskih razdalj med vozlišči
-- velja le $y_{uv} \le d_G(u, v)$.
Ker pa za $u, v \in V_i$ velja $y_{uv} \ge i+1$,
sledi $d_G(u, v) \ge i+1$, kar ustreza pogojem naloge.
\end{odgovor}
\end{naloga}
