\begin{naloga}{David Gajser}{Kolokvij OR 9.5.2013}
\begin{vprasanje}
V turistični agenciji Zobček že vrsto let organizirajo izlete po Sloveniji.
Zaradi velikega povpraševanja so se odločili,
da bodo ponudbo razširili tudi na Madžarsko.
Izbrali so destinacije in vodiče,
sedaj pa želijo slednje razporediti po destinacijah tako,
da bi vsako destinacijo pokrivala natanko dva vodiča
in bi vsak vodič pokrival natanko eno destinacijo.
Vsak vodič je podal seznam destinacij, urejenih po priljubljenosti,
začenši s tisto, po kateri bi najraje vodil turiste.
Agencija je za vsako destinacijo naredila urejen seznam vodičev,
tako da je na prvem mestu tisti vodič,
ki bi predvidoma najbolje vodil turiste po destinaciji,
na zadnjem pa tisti, ki bi to predvidoma počel najslabše.

Če bi bilo destinacij in vodičev malo,
bi agencija sama nekako razdelila vodiče med destinacije, a temu ni tako.
Še več, seznam destinacij in vodičev nameravajo z leti spreminjati,
zato nujno potrebujejo postopek,
ki bi jim pomagal razdeliti vodiče po destinacijah.

\begin{enumerate}[(a)]
\item Danih je $n$ destinacij in $2n$ vodičev,
pri čemer za vsako destinacijo poznamo preference agencije glede vodičev
in za vsakega vodiča poznamo njegove preference glede destinacij.
Opiši algoritem časovne zahtevnosti $O(n^2)$,
ki izračuna {\em stabilno razdelitev} vodičev med destinacije oz.~nam pove,
da stabilna razdelitev ne obstaja.

Razdelitev $2n$ vodičev med n destinacij je {\em nestabilna},
če obstajata vodič $v$ in destinacija $d$,
tako da bi $v$ raje vodil po destinaciji $d$ kakor po trenutni destinaciji
in bi agencija raje imela vodiča $v$ na destinaciji $d$
namesto enega izmed trenutnih vodičev na tej destinaciji.
Razdelitev $2n$ vodičev med $n$ destinacij je {\em stabilna},
če sta vsaki destinaciji dodeljena natanko dva vodiča in če ni nestabilna.

\item Reši nalogo za naslednje podatke ($n = 3$):
\begin{center}
\begin{tabular}{c|c}
Destinacija & preference agencije glede vodičev \\ \hline
Budimpešta  & Anisa, Bela, Črt, Ema, Dana, Corina \\
Győr        & Črt, Bela, Corina, Dana, Ema, Anisa \\
Szeged      & Ema, Anisa, Bela, Dana, Črt, Corina
\end{tabular}

\bigskip
\begin{tabular}{c|c}
Vodič  & preference vodičev \\ \hline
Anisa  & Szeged, Budimpešta, Győr \\
Bela   & Budimpešta, Szeged, Győr \\
Corina & Győr, Szeged, Budimpešta \\
Črt    & Budimpešta, Győr, Szeged \\
Dana   & Budimpešta, Szeged, Győr \\
Ema    & Győr, Budimpešta, Szeged
\end{tabular}
\end{center}
\end{enumerate}
\end{vprasanje}

\begin{odgovor}
\begin{enumerate}[(a)]
\item
Naj bo $v[i][j]$ ($1 \le i \le 2n$, $1 \le j \le n$)
$j$-ta preferenca $i$-tega vodiča,
$d[j][i]$ ($1 \le i \le 2n$, $1 \le j \le n$)
rang $i$-tega vodiča pri preferencah agencije za $j$-to destinacijo.
Uporabili bomo varianto Gale-Shapleyevega algoritma,
ki bo vsaki destinaciji dodelila dva vodiča.

\begin{small}
\begin{algorithmic}
\Function{StabilnaDodelitev}%
{$v[1 \dots 2n][1 \dots n], d[1 \dots n][1 \dots 2n]$}
    \State $Q \gets [1, 2, \dots, 2n]$
    \State $k \gets$ seznam dolžine $2n$ z vrednostmi $1$
    \State $M \gets$ matrika velikosti $n \times 2$ z vrednostmi $\Null$
    \While{$\lnot Q.\isEmpty()$}
        \State $i \gets Q.\pop()$
        \State $j \gets v[k[i]]$
        \State $k[i] \gets k[i] + 1$
        \State $i' \gets M[j][2]$
        \If{$i' = \Null \lor d[i] < d[i']$}
            \If{$d[i] < d[M[j][1]]$}
                \State $M[j][2] \gets M[j][1]$
                \State $M[j][1] \gets i$
            \Else
                \State $M[j][2] \gets i$
            \EndIf
            \If{$i' \ne \Null$}
                \State $Q.\append(i')$
            \EndIf
        \Else
            \State $Q.\append(i)$
        \EndIf
    \EndWhile
    \State \Return $M$
\EndFunction
\end{algorithmic}
\end{small}

Ker se vsak od $2n$ vodičev v vrsto $Q$ doda največ $n$-krat,
algoritem teče v času $O(n^2)$.

\item Izvedba algoritma {\sc StabilnaDodelitev} pri danih podatkih
je prikazana v tabeli~\tab.
Budimpešto torej dodelimo Beli in Črtu,
Győr Corini in Emi, Szeged pa Anisi in Dani.
\end{enumerate}

\begin{tabela}
\makebox[\textwidth][c]{
\begin{tabular}{c|cccccc|ccc}
$Q$ & \multicolumn{6}{c|}{$k$} & \multicolumn{3}{c}{$M$} \\
& A & B & C & Č & D & E & Budimpešta & Győr & Szeged \\ \hline
[A, B, C, Č, D, E] & 1 & 1 & 1 & 1 & 1 & 1
& [\Null, \Null{}] & [\Null, \Null{}] & [\Null, \Null{}] \\{}
[B, C, Č, D, E] & 2 & 1 & 1 & 1 & 1 & 1
& [\Null, \Null{}] & [\Null, \Null{}] & [A, \Null{}] \\{}
[C, Č, D, E] & 2 & 2 & 1 & 1 & 1 & 1
& [B, \Null{}] & [\Null, \Null{}] & [A, \Null{}] \\{}
[Č, D, E] & 2 & 2 & 2 & 1 & 1 & 1
& [B, \Null{}] & [C, \Null{}] & [A, \Null{}] \\{}
[D, E] & 2 & 2 & 2 & 2 & 1 & 1
& [B, Č] & [C, \Null{}] & [A, \Null{}] \\{}
[E, D] & 2 & 2 & 2 & 2 & 2 & 1
& [B, Č] & [C, \Null{}] & [A, \Null{}] \\{}
[D] & 2 & 2 & 2 & 2 & 2 & 2
& [B, Č] & [C, E] & [A, \Null{}] \\{}
[] & 2 & 2 & 2 & 2 & 3 & 2
& [B, Č] & [C, E] & [A, D]
\end{tabular}
}
\podnaslov[\res{}(b)]{Potek izvajanja algoritma}
\end{tabela}
\end{odgovor}
\end{naloga}
