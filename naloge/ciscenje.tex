\begin{naloga}{Janoš Vidali}{Kolokvij OR 9.6.2025}
\begin{vprasanje}
Novakovi načrtujejo spomladansko čiščenje hiše in okolice.
Opravila, ki jih želijo opraviti, so zbrana v tabeli~\tab.

\begin{enumerate}[(a)]
\item Topološko uredi ustrezni graf in ga nariši.
Za trajanja opravil vzemi pričakovana trajanja po modelu PERT.

\item Določi kritično pot in čas trajanja čiščenja,
če za trajanja opravil uporabimo pričakovana trajanja po modelu PERT.

\item Katero opravilo je (ob zgornjih predpostavkah) najmanj kritično?
Najmanj kritično je opravilo, katerega trajanje lahko najbolj podaljšamo,
ne da bi vplivali na celotno trajanje čiščenja.

\item Določi variance trajanj opravil in oceni ve\-rjet\-nost,
da bo čiščenje trajalo manj kot $550$ minut.
\end{enumerate}
%
\begin{tabela}
\makebox[\textwidth][c]{
\begin{tabular}{c|cc|b{2cm}b{2.7cm}b{2cm}}
& Opravilo & Pogoji & Minimalno trajanje
& Najbolj verjetno trajanje & Maksimalno trajanje \\ \hline
$a$ & snemanje zaves           & /         &  24 minut &  30 minut &  36 minut \\
$b$ & pranje zaves             & $a$       & 168 minut & 192 minut & 204 minute \\
$c$ & sušenje zaves            & $b, f$    & 240 minut & 300 minut & 420 minut \\
$d$ & ponovno nameščanje zaves & $c, e$    &  36 minut &  48 minut &  66 minut \\
$e$ & čiščenje oken            & $a$       & 210 minut & 300 minut & 330 minut \\
$f$ & čiščenje dvorišča        & /         & 120 minut & 150 minut & 210 minut \\
$g$ & košnja trate             & /         &  60 minut &  75 minut & 105 minut \\
$h$ & odvoz smeti              & $e, f, g$ &  30 minut &  45 minut &  45 minut
\end{tabular}
}
\podnaslov{Podatki}
\end{tabela}
\end{vprasanje}

\begin{odgovor}
\begin{enumerate}[(a)]
\item Projekt lahko predstavimo z uteženim grafom s slike~\fig,
iz katerega je raz\-vid\-na topološka ureditev
$s, a, f, g, b, e, c, h, d, t$.
Uteži povezav ustrezajo pričakovanim trajanjem opravil,
ki so prikazana v tabeli~\tab[ciscenje-resitev].

\item V tabeli~\tab[ciscenje-resitev]
so podani najzgodnejši začetki opravil in najpoznejši začetki,
da se celotno trajanje projekta ne podaljša.
Pričakovano trajanje projekta je torej $\mu = 330$ dni,
edina kritična pot pa je $s - a - b - c - d - t$,
ki tako vsebuje tudi vsa kritična opravila.

\item Iz tabele~\tab[ciscenje-resitev] je razvidno,
da je najmanj kritično opravilo $g$,
saj lahko njegovo trajanje
podaljšamo za $459$ minut v primerjavi s pričakovanim,
ne da bi vplivali na trajanje celotnega projekta.

\item Izračunajmo standardni odklon trajanja pričakovane kritične poti,
pri čemer uporabimo variance iz tabele~\tab[ciscenje-resitev].
$$
\sigma = \sqrt{4 + 36 + 900 + 25} \approx 31.06
$$
Naj bo $X$ slučajna spremenljivka,
ki meri čas izvajanja pričakovane kritične poti.
Izračunajmo verjetnost končanja v roku $T = 550$ minut.
$$
P(X \le 550) = \Phi\left({T - \mu \over \sigma}\right) = \Phi(-0.934) = 0.1752
$$
\end{enumerate}
%
\begin{slika}
\makebox[\textwidth][c]{
\pgfslika
}
\podnaslov{Graf odvisnosti med opravili in kritična pot}
\end{slika}
%
\begin{tabela}
\setlabel{ciscenje-resitev}
\makebox[\textwidth][c]{
\begin{tabular}{c|cccccccccc}
& $s$ & $a$ & $f$ & $g$ & $b$ & $e$ & $c$ & $h$ & $d$ & $t$ \\ \hline
pričakovano && $30$ & $155$ & $77.5$ & $190$ & $290$ & $310$ & $42.5$ & $49$ \\
varianca && $4$ & $225$ & $56.25$ & $36$ & $400$ & $900$ & $6.25$ & $25$ \\
\hline
najprej & $0$ & $0_s$ & $0_s$ & $0_s$ & $30_a$ & $30_a$ & $220_b$ & $320_e$ & $530_c$ & $579_d$ \\
najkasneje & $0_a$ & $0_b$ & $65_c$ & $459_h$ & $30_c$ & $240_d$ & $220_d$ & $536.5_t$ & $530_t$ & $579$ \\
\text{razlika} & $0$ & $0^*$ & $65$ & $459$ & $0^*$ & $210$ & $0^*$ & $216.5$ & $0^*$ & $0$
\end{tabular}
}
\podnaslov{Razporejanje opravil}
\end{tabela}
\end{odgovor}
\end{naloga}
