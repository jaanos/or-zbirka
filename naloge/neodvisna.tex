\begin{naloga}{Janoš Vidali}{Kolokvij OR 3.6.2019}
\begin{vprasanje}
{\em Neodvisna množica} vozlišč grafa $G = (V, E)$
je taka množica $S \subseteq V$,
da sta poljubni vozlišči iz množice $S$ nesosedni v $G$,
torej $uv \not\in E$ za vsaka $u, v \in S$.

Dano je drevo $T = (V, E)$ in uteži vozlišč $c_v$ ($v \in V$).
V drevesu $T$ želimo najti {\em najtežjo neodvisno množico}
-- torej tako množico vozlišč $S \subseteq V$,
ki maksimizira vsoto njihovih uteži, torej vrednost $\sum_{u \in S} c_u$.

\begin{enumerate}[(a)]
\item Denimo, da je drevo $T$ podano kot neusmerjen graf,
predstavljen s seznami sosedov.
Razloži, kako lahko sestaviš tabelo $\pred$,
ki za vsako vozlišče $v \in V$ določa njegovega prednika,
če za koren izbereš vozlišče $r \in V$.
Koren $r$ je lahko izbran poljubno, zanj pa velja $\pred[r] = \Null$.
Kako iz seznamov sosedov in tabele $\pred$ ugotovimo,
katera vozlišča so neposredni nasledniki danega vozlišča v drevesu?

\item Napiši rekurzivne enačbe za reševanje problema
najtežje neodvisne množice v drevesu $T$.
Razloži, kaj pred\-stav\-lja\-jo spremenljivke,
v kakšnem vrstnem redu jih računamo, ter kako dobimo optimalno rešitev.
Predpostaviš lahko,
da imaš poleg seznamov sosedov in uteži vozlišč drevesa $T$
na voljo tudi tabelo $\pred$ kot v točki (a).
\namig{za vsako vozlišče uporabi dve spremenljivki
-- eno za primer, ko je vozlišče izbrano, in eno za primer, ko ni.}

\item Natančno opiši postopek (z besedami ali psevdokodo),
ki iz zgoraj izračunanih vred\-no\-sti
sestavi najtežjo neodvisno množico v $T$.

\item Oceni časovno zahtevnost algoritma iz točk (b) in (c).

\item S svojim algoritmom poišči najtežjo neodvisno množico
na drevesu s slike~\fig.
Iz re\-šit\-ve naj bo jasno, kako poteka izvajanje algoritma.
\end{enumerate}
%
\begin{slika}
\makebox[\textwidth][c]{
\pgfslika
}
\podnaslov[\nal{}(e)]{Drevo}
\end{slika}
\end{vprasanje}

\begin{odgovor}
\end{odgovor}
\end{naloga}
