\begin{naloga}{Jernej Azarija}{Izpit OR 24.5.2016}
\begin{vprasanje}
Dana je matrika $A$ dimenzij $n \times n$, katere elementi so $0$ in $1$.
{\em Strnjena podmatrika} $A'$ matrike $A$ je matrika,
določena z zaporednimi vrsticami in stolpci matrike $A$.
{\em Velikost} matrike je število elementov v matriki.
Velikost podmatrike dimenzij $n_1 \times n_2$ je torej $n_1 n_2$.
Naj bo $N(A)$ največja velikost take podmatrike $A'$ matrike $A$,
da $A'$ vsebuje samo enice.
Na primer:
$$
N\left(\begin{bmatrix}
1 & 0 & 0 \\
1 & 1 & 0 \\
1 & 1 & 1
\end{bmatrix}\right) = 4, \quad
N\left(\begin{bmatrix}
1 & 0 & 1 \\
1 & 0 & 0 \\
1 & 1 & 1
\end{bmatrix}\right) = 3, \quad \text{in} \quad
N\left(\begin{bmatrix}
1 & 0 & 1 \\
0 & 0 & 0 \\
0 & 1 & 1
\end{bmatrix}\right) = 2.
$$

\begin{enumerate}[(a)]
\item Napiši rekurzivne enačbe za računanje $N(A)$.
\item Napiši algoritem za računanje $N(A)$
s časovno zahtevnostjo največ $O(n^4)$.
\end{enumerate}
\end{vprasanje}

\begin{odgovor}
\begin{enumerate}[(a)]
\item Naj bo $p_{hik}$ število stolpcev
v največji strnjeni podmatriki s samimi enicami
z vrsticami od $h$-te do $i$-te vrstice matrike $A$,
katere zadnji stolpec je $k$-ti stolpec matrike $A$.
Poleg tega naj bo $s_{ij}$ vsota prvih $i$ komponent (tj., število enic)
v $j$-tem stolpcu matrike $A$.
Določimo začetne pogoje in rekurzivne enačbe.
\begin{align*}
s_{0j}  &= 0 & (1 &\le j \le n) \\
s_{ij}  &= A_{ij} + s_{i-1,j} & (1 &\le i, j \le n) \\
p_{hi0} &= 0 & (1 &\le h \le i \le n) \\
p_{hik} &= \begin{cases}
p_{h,i,k-1} + 1 & s_{ik} - s_{h-1,k} = i-h+1, \text{in} \\
0 & \text{sicer}
\end{cases} &
(1 &\le h \le i \le n, \\[-5mm]
&& 1 &\le k \le n)
\end{align*}
Najprej za vsak $j$ ($1 \le j \le n$)
izračunamo vrednosti $s_{ij}$ naraščajoče po indeksu $i$,
nato pa za vsaka $h$ in $i$ ($1 \le h \le i \le n$)
izračunamo vrednosti $p_{hik}$ naraščajoče po indeksu $k$.
Največjo velikost strnjene podmatrike samih enic dobimo kot
$$
N(A) = \max\set{(i-h+1) p_{hik}}{1 \le h \le i \le m, \ 0 \le k \le n} .
$$

\item Zapišimo algoritem,
ki bo vrnil velikost največje strnjene podmatrike samih enic skupaj s paroma,
ki določata prvo in zadnjo vrstico oziroma stolpec.
Za razliko od zgornjih rekurzivnih enačb
bo tukaj $p_{hik}$ označeval velikost matrike in ne samo števila stolpcev.
\begin{small}
\begin{algorithmic}
\Function{Enice}{$A \in \{0, 1\}^{n \times n}$}
	\For{$j = 1, \dots, n$} \hfill računanje $s_{ij}$
		\State $s_{0j} \gets 0$ \hfill začetni pogoj
		\For{$i = 1, \dots, n$}
			\State $s_{ij} \gets A_{ij} + s_{i-1,j}$ \hfill rekurzivna enačba
		\EndFor
	\EndFor
	\State $p^*, h^*, i^*, j^*, k^* \gets 0, 0, 0, 0, 0$
		\hfill ničelna rešitev
	\For{$h = 1, \dots, n$} \hfill računanje $p_{hik}$
		\For{$i = h, \dots, n$}
			\State $\ell \gets i - h + 1$ \hfill število vrstic
			\State $p_{hi0} \gets 0$ \hfill začetni pogoj
			\State $j \gets 1$ \hfill podmatriko začenjamo s prvim stolpcem
			\For{$k = 1, \dots, n$}
				\If{$s_{ik} - s_{h-1,k} = \ell$} \hfill preverimo, ali imamo
					\State $p_{hik} \gets p_{h,i,k-1} + \ell$
						\hfill v trenutnem stolpcu same enice
					\If{$p_{hik} > p^*$}
							\hfill preverimo, ali imamo boljšo rešitev
						\State $p^*, h^*, i^*, j^*, k^* \gets p_{hik}, h, i, j, k$
					\EndIf
				\Else \hfill če nimamo samih enic, bomo naslednjo
					\State $j \gets k+1$
						\hfill podmatriko začeli v naslednjem stolpcu
					\State $p_{hik} \gets 0$
				\EndIf
			\EndFor
		\EndFor
	\EndFor
	\State \Return $p^*, (h^*, i^*), (j^*, k^*)$
\EndFunction
\end{algorithmic}
\end{small}
Za računanje $s_{ij}$ potrebujemo $O(n^2)$ korakov,
za računanje $p_{hik}$ pa $O(n^3)$ korakov.
Skupna časovna zahtevnost algoritma je torej $O(n^3)$.
\end{enumerate}
\end{odgovor}
\end{naloga}
