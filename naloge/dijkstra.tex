\begin{naloga}{Janoš Vidali}{Vaje OR 7.12.2016}
\begin{vprasanje}
S pomočjo Dijkstrovega algoritma
določi razdalje od vozlišča $a$ do ostalih vozlišč
v grafu s slike~\fig.

\begin{slika}
\pgfslika
\podnaslov{Graf}
\end{slika}
\end{vprasanje}


\begin{odgovor}
Sledili bomo naslednjemu algoritmu.
Predpostavili bomo,
da imamo na voljo podatkovno strukturo za prednostno vrsto,
ki za vsak element hrani njegovo prioriteto (to lahko tudi spreminjamo).
Podatkovna struktura ima metodo $\pop()$,
ki vrne in odstrani element z najmanjšo prioriteto
skupaj z njegovo prioriteto.
Tukaj bodo elementi vozlišča grafa,
prioritete pa dolžine najkrajših najdenih poti od začetnega vozlišča.
\begin{small}
\begin{algorithmic}
\Function{Dijkstra}{$G = (V, E), s \in V, L : E \rightarrow \R_{+}$}
    \State $d \gets$ slovar z vrednostjo $\infty$ za vsako vozlišče $v \in V$
    \State $\pred \gets$ slovar z vrednostjo $\Null$
        za vsako vozlišče $v \in V$
	\State $Q \gets$ prednostna vrsta
        s prioriteto $\infty$ za vsako vozlišče $v \in V$
        (na vrhu je vozlišče z najmanjšo prioriteto)
	\State $Q[s] \gets 0$
	\While{$\lnot Q.\isEmpty()$}
		\State $u, d[u] \gets Q.\pop()$
		\For{$v \in \Adj(G, u)$}
			\If{$v \in Q \land Q[v] > d[u] + L_{uv}$}
				\State $Q[v] \gets d[u] + L_{uv}$
                \State $\pred[v] \gets u$
			\EndIf
		\EndFor
	\EndWhile
    \State \Return $(d, \pred)$
\EndFunction
\end{algorithmic}
\end{small}
Če za prednostno vrsto uporabimo običajen slovar,
je časovna zahtevnost metode $\pop()$ linearna v številu elementov slovarja,
spreminjanje vrednosti v slovarje pa se opravi v konstantnem času.
Časovna zahtevnost zgornjega algoritma je tako $O(n^2)$,
kjer je $n$ število vozlišč v grafu.

Če pa za prednostno vrsto vzamemo kopico,
sta časovni zahtevnosti metode $\pop()$ in spreminjanja vrednosti v kopici
logaritemski v velikosti kopice.
Časovna zahtevnost zgornjega algoritma je v tem primeru $O((m+n) \log n)$,
kjer je $m$ število povezav v grafu.

Potek zgornjega algoritma na grafu s slike~\fig,
pri čemer sledimo abecednemu vrstnemu redu vozlišč,
je prikazan v tabeli~\tab.
Drevo najkrajših poti je prikazano na sliki~\fig[dijkstra-resitev].

\begin{tabela}[p]
$$
\begin{array}{c|c|c|c}
u & v & Q & \text{razdalje in predhodniki} \\ \hline
  &   & [(a,0)]  & [0, \infty, \infty, \infty, \infty, \infty, \infty, \infty] \\
a & b & [(b, 3)]  & [0, 3_a, \infty, \infty, \infty, \infty, \infty, \infty] \\
a & f & [(b, 3), (f, 6)]  & [0, 3_a, \infty, \infty, \infty, 6_a, \infty, \infty] \\
b & c & [(f, 6), (c, 10)]  & [0, 3_a, 10_b, \infty, \infty, 6_a, \infty, \infty] \\
b & e & [(f, 6), (c, 10), (e, 12)]  & [0, 3_a, 10_b, \infty, 12_b, 6_a, \infty, \infty] \\
f & g & [(c, 10), (e, 12), (g, 14)]  & [0, 3_a, 10_b, \infty, 12_b, 6_a, 14_g, \infty] \\
c & d & [(e, 12), (g, 14), (d, 17)]  & [0, 3_a, 10_b, 17_c, 12_b, 6_a, 14_g, \infty] \\
e & d & [(d, 13), (g, 14), (d, 17)]  & [0, 3_a, 10_b, 13_e, 12_b, 6_a, 14_g, \infty] \\
e & g & [(g, 13), (d, 13), (g, 14), (d, 17)]  & [0, 3_a, 10_b, 13_e, 12_b, 6_a, 13_e, \infty] \\
g & f & [(d, 13), (g, 14), (d, 17)]  & [0, 3_a, 10_b, 13_e, 12_b, 6_a, 13_e, \infty] \\
d & b & [(g, 14), (d, 17)]  & [0, 3_a, 10_b, 13_e, 12_b, 6_a, 13_e, \infty] \\
d & h & [(g, 14), (d, 17), (h, 17)]  & [0, 3_a, 10_b, 13_e, 12_b, 6_a, 13_e, 17_d] \\
g & f & [(d, 17), (h, 17)]  & [0, 3_a, 10_b, 13_e, 12_b, 6_a, 13_e, 17_d] \\
d & b & [(h, 17)]  & [0, 3_a, 10_b, 13_e, 12_b, 6_a, 13_e, 17_d] \\
d & h & [(h, 17)]  & [0, 3_a, 10_b, 13_e, 12_b, 6_a, 13_e, 17_d] \\
h & g & [\ ]  & [0, 3_a, 10_b, 13_e, 12_b, 6_a, 13_e, 17_d]
\end{array}
$$
\podnaslov{Potek izvajanja algoritma}
\end{tabela}

\begin{slika}[p]
\pgfslika[dijkstra-resitev]
\podnaslov{Drevo najkrajših poti}
\end{slika}
\end{odgovor}
\end{naloga}
