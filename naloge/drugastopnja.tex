\begin{naloga}{Gabrovšek, Konvalinka}{Kolokvij OR 24.1.2011}
\begin{vprasanje}
Študent tretjega letnika finančne matematike se mora odločiti,
ali bi nadaljeval študij na drugi stopnji.
Ocenjuje, da bo, če študij uspešno zaključi,
v življenju zaslužil $200\,000 €$ več,
kot če študij zaključi že po prvi stopnji.
Če pa študija ne zaključi uspešno,
bo imel zaradi stroškov študija in izgubljenega dohodka $40\,000 €$ izgube.
Verjetnost, da bo študij na drugi stopnji uspešno zaključil, je $80 \%$.
\begin{enumerate}[(a)]
\item Modeliraj problem v okviru teorije odločanja (stanja, odločitve).
Kakšno odločitev svetuješ študentu?

\item Matematični oddelek ponuja dodatno testiranje,
ki študentom pomaga pri odločitvi, ali naj nadaljujejo študij.
Test stane $500$ evrov,
iz izkušenj kolegov pa študent ocenjuje,
da so pogojne verjetnosti naslednje:
\begin{center}
\begin{tabular}{c|cc}
$P(\text{rezultat testa} \;|\; \text{uspešnost študija})$
& uspešen & neuspešen \\ \hline
pozitiven & $19/20$ & $1/10$ \\
negativen & $1/20$ & $9/10$
\end{tabular}
\end{center}
Ali naj se študent prijavi na dodatno testiranje?
\end{enumerate}
\end{vprasanje}

\begin{odgovor}
\begin{enumerate}[(a)]
\item Študent ima eno samo odločitev -- ali naj nadaljuje študij.
Če ga ne nadaljuje, ima $0 €$ dobička,
če pa ga, pa preide v stanje,
v katerem ima z verjetnostjo $0.8$ dobiček $200\,000 €$,
z verjetnostjo $0.2$ pa $-40\,000 €$.
V tem primeru je pričakovani dobiček enak
$$
0.8 \cdot 200\,000 € - 0.2 \cdot 40\,000 € = 152\,000 € .
$$
Študentu se torej nadaljevanje študija izplača,
pri tem pa pričakuje dobiček $152\,000 €$.

\item Izračunajmo verjetnosti za primer,
da se študent odloči za testiranje.
\begin{align*}
P(\text{test pozitiven}) =
{19 \over 20} \cdot 0.8 + {1 \over 10} \cdot 0.2 &= 0.78 \\
P(\text{test negativen}) =
{1 \over 20} \cdot 0.8 + {9 \over 10} \cdot 0.2 &= 0.22 \\
P(\text{uspešen študij} \mid \text{test pozitiven}) =
{19/20 \cdot 0.8 \over 0.78} &\approx 0.974 \\
P(\text{neuspešen študij} \mid \text{test pozitiven}) =
{1/10 \cdot 0.2 \over 0.78} &\approx 0.026 \\
P(\text{uspešen študij} \mid \text{test negativen}) =
{1/20 \cdot 0.8 \over 0.22} &\approx 0.182 \\
P(\text{neuspešen študij} \mid \text{test negativen}) =
{9/10 \cdot 0.2 \over 0.22} &\approx 0.818
\end{align*}
S pomočjo zgoraj izračunanih verjetnosti
lahko narišemo odločitveno drevo s slike~\fig.
Opazimo, da se študentu ne izplača prijaviti na testiranje,
saj mu to ne da zadostne informacije za morebitno spremembo odločitve.
\end{enumerate}
%
\begin{slika}
\makebox[\textwidth][c]{
\pgfslika
}
\podnaslov{Odločitveno drevo}
\end{slika}
\end{odgovor}
\end{naloga}
