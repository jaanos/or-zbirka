\begin{naloga}{David Gajser}{Izpit OR 28.8.2015}
\begin{vprasanje}
V podjetju Hierarhija so delavci razvrščeni hierarhično tako,
da za vsaka dva delavca vemo, kdo je komu nadrejen.
Če povemo še bolj natančno, delavce lahko postavimo v vrsto tako,
da so za vsakega delavca levo od njega podrejeni, desno pa nadrejeni.

Recimo, da je v podjetju $n$ delavcev, označimo jih z $d_1, d_2, \dots, d_n$.
Predpostavimo tudi, da so že urejeni hierarhično,
tj., delavec $d_i$ je podrejen delavcem $d_{i+1}, d_{i+2}, \dots, d_n$.
Vsak delavec se boji prvih nekaj delavcev, ki so mu podrejeni,
da bi ga prehiteli v hierarhiji.
Naj bo $k_i \in \N_0$ stopnja ogroženosti delavca $d_i$ ($1 \le i \le n$).
Stopnja ogroženosti nam pove,
koliko delavcev, ki so podrejeni delavcu $d_i$, se ta delavec ``boji''.
Z drugimi besedami,
delavec $d_i$ se boji delavcev $d_{i_1}, d_{i-2}, \dots, d_{i-k_i}$.

Z dinamičnim programiranjem poišči tako največjo skupino delavcev,
da se nihče v tej skupini ne bo bal drugega iz te skupine.
Dovolj je zapisati rekurzivno enačbo z začetnimi pogoji.
Kako lahko izvemo, kateri delavci so v tej (največji) skupini?
\end{vprasanje}
\begin{odgovor}
\end{odgovor}
\end{naloga}
