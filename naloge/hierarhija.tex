\begin{naloga}{David Gajser}{Izpit OR 28.8.2015}
\begin{vprasanje}
V podjetju Hierarhija so delavci razvrščeni hierarhično tako,
da za vsaka dva delavca vemo, kdo je komu nadrejen.
Če povemo še bolj natančno, delavce lahko postavimo v vrsto tako,
da so za vsakega delavca levo od njega podrejeni, desno pa nadrejeni.

Recimo, da je v podjetju $n$ delavcev, označimo jih z $d_1, d_2, \dots, d_n$.
Predpostavimo tudi, da so že urejeni hierarhično,
tj., delavec $d_i$ je podrejen delavcem $d_{i+1}, d_{i+2}, \dots, d_n$.
Vsak delavec se boji prvih nekaj delavcev, ki so mu podrejeni,
da bi ga prehiteli v hierarhiji.
Naj bo $k_i \in \{0, 1, \dots, i-1\}$
stopnja ogroženosti delavca $d_i$ ($1 \le i \le n$).
Stopnja ogroženosti nam pove,
koliko delavcev, ki so podrejeni delavcu $d_i$, se ta delavec ``boji''.
Z drugimi besedami,
delavec $d_i$ se boji delavcev $d_{i-1}, d_{i-2}, \dots, d_{i-k_i}$.

Z dinamičnim programiranjem poišči tako največjo skupino delavcev,
da se nihče v tej skupini ne bo bal drugega iz te skupine.
Dovolj je zapisati rekurzivno enačbo z začetnimi pogoji.
Kako lahko izvemo, kateri delavci so v tej (največji) skupini?
\end{vprasanje}

\begin{odgovor}
Naj bo $p_i$ velikost največje skupine delavcev izmed $d_1, d_2, \dots, d_i$,
da se nihče v tej skupini ne bo bal drugega v tej skupini.
Določimo začetni pogoj in rekurzivne enačbe.
\begin{align*}
p_0 &= 0 \\
p_i &= \max\{p_{i-1}, p_{i-k_i-1} + 1\} \quad (1 \le i \le n)
\end{align*}
Vrednosti $p_i$ računamo naraščajoče glede na indeks $i$.
Maksimalno velikost skupine dobimo kot $p^* = p_n$.

Iskano skupino lahko iz vrednosti
$k_i$ ($1 \le i \le n$) in $p_i$ ($1 \le i \le n$)
rekonstruiramo s sledečim algoritmom.
\begin{small}
\begin{algorithmic}
\Function{RekonstruirajSkupino}{$(k_i)_{i=1}^n, (p_i)_{i=1}^n$}
	\State $S \gets$ prazna množica
	\State $i \gets n$
	\While{$i > 0$}
		\If{$p_i = p_{i-1}$} \hfill upoštevamo $p_0 = 0$
			\State $i \gets i-1$
		\Else
			\State $S.\add(i)$
			\State $i \gets i-k_i-1$
		\EndIf
	\EndWhile
	\State \Return $S$
\EndFunction
\end{algorithmic}
\end{small}
\end{odgovor}
\end{naloga}
