\begin{naloga}{Alen Orbanić}{Kolokvij OR 26.1.2010}
\begin{vprasanje}
Upravljalec nove moderne visoko tehnološko opremljene kongresne dvorane
v Kongresnem centru Grebka Pohleparja zaračunava za najem dvorane $5 €$/min.
Povpraševanje za najem nove dvorane je ogromno,
a se na žalost ponudbe časovno prekrivajo,
dvorano pa seveda lahko najame le ena stranka naenkrat.
Tako se je upravljalec že prvi dan znašel pod goro ponudb,
med katerimi je napol v paniki izbiral,
bodimo pošteni, tako bolj ``čez palec''.
A za kongresni center, ki nosi ime takega velikana ekonomije, se spodobi,
da si uredi vse segmente poslovanja čim bolj optimalno
in je kot tak zgled drugim.
Predpostavimo,
da ima upravljalec v trenutku odločanja podatke v obliki intervalov
$(z_i, k_i)$, $i = 1, \dots, n$,
kjer $z_i$ in $k_i$ predstavljata začetni in končni čas ponudbe $i$.
Ponudbi se prekrivata, če je njun presek neprazen časovni interval.

\begin{enumerate}[(a)]
\item Denimo, da uredimo intervale ponudb
glede na začetne čase ponudb od najmanjšega do največjega.
Potem izberemo prvi interval, vse intervale, ki ga prekrivajo, pa zavržemo,
ter postopek ponovimo rekurzivno.
Upravljalcu se ta postopek zdi optimalen. Kaj pa meniš ti?

\item Denimo, da so intervali ponudb urejeni kot v prejšnji točki
ter da je interval $I$ zadnji interval v neki optimalni izbiri OPT.
Premisli,
kateri intervali so lahko kandidati za predzadnji interval v izbiri OPT.
Na podlagi tega premisleka sestavi postopek,
ki s pomočjo dinamičnega programiranja najde optimalno rešitev problema.
Postopek utemelji.

\item Kakšna je časovna zahtevnost tvojega algoritma v najslabšem primeru?

\item Z uporabo razvitega algoritma reši problem za podatke
$(15, 60)$, $(60, 180)$, $(30, 90)$, $(120, 210)$,
$(165, 255)$, $(75, 135)$, $(30, 105)$, $(90, 150)$.
\end{enumerate}
\end{vprasanje}

\begin{odgovor}
Ker je zaslužek premo sorazmeren času najema,
bomo tukaj optimizirali skupen čas najema dvorane.
Če želimo iz časa (predpostavimo, da je podan v minutah) izračunati zaslužek,
ga lahko pomnožimo s $5 €$.

\begin{enumerate}[(a)]
\item Postopek ni optimalen,
saj lahko najdemo protiprimer, kjer opisani postopek ne da optimalne rešitve.
Denimo, da imamo ponudbe $(0, 2), (1, 6), (5, 7)$
-- opisani postopek izbere prvi in zadnji interval s skupnim trajanjem $4$,
medtem ko je optimalna rešitev izbira drugega intervala s trajanjem $5$.

\item Če je $I = (z_t, k_t)$ zadnji interval v optimalni izbiri OPT,
potem so kandidati za predzadnji interval vsi intervali oblike $(z_i, k_i)$,
kjer je $k_i \le z_t$.
Tako naj bo $x_i$ največji čas najema dvorane,
če je zadnji uporabljeni interval $(z_i, k_i)$.
Brez škode za splošnost lahko predpostavimo,
da za vsak $i$ $(1 \le i \le n)$ velja $0 \le z_i < k_i$
ter da so intervali $(z_i, k_i)$ urejeni naraščajoče po $k_i$.
Zapišimo začetne pogoje in rekurzivne enačbe.
\begin{align*}
x_0 &= k_0 = 0 \\
x_i &= k_i - z_i + \max\set{x_j}{0 \le j < i, k_j \le z_i}
\end{align*}
Urejenost intervalov nam zagotovi,
da velja $k_j > z_i$ za vse $j \ge i$,
tako da nam teh primerov ni potrebno obravnavati.
Največji skupni čas najema potem dobimo kot
$x^* = \max\set{x_i}{1 \le i \le n}$.

\item Izračunati je potrebno $n$ vrednosti $x_i$,
za izračun $x_i$ pa je potrebno pregledati največ $i$ vrednosti $x_j$.
Skupna časovna zahtevnost je torej $O(n^2)$.

\item Najprej bomo podatke uredili po časih končanje
-- imeli bomo torej
\begin{align*}
((z_i, k_i))_{i=1}^8 = (&{} (15, 60), (30, 90), (30, 105), (75, 135), \\
&{} (90, 150), (60, 180), (120, 210), (165, 255)) .
\end{align*}
Izračunajmo sedaj vrednosti $x_i$ ($1 \le i \le 8$).
\begin{alignat*}{2}
x_1 &= 45 + 0 &&= 45 \\
x_2 &= 60 + 0 &&= 60 \\
x_3 &= 75 + 0 &&= 75 \\
x_4 &= 60 + \max\{0, \underline{45}\} &&= 105 \\
x_5 &= 60 + \max\{0, 45, \underline{60}\} &&= 120 \\
x_6 &= 120 + \max\{0, \underline{45}\} &&= 165 \\
x_7 &= 90 + \max\{0, 45, 60, \underline{75}\} &&= 165 \\
x_8 &= 90 + \max\{0, 45, 60, 75, 105, \underline{120}\} &&= 210
\end{alignat*}
Največji skupni čas najema je torej $x^* = 210$.
Poiščimo še intervale, v katerih naj upravljalec odda dvorano.
\begin{align*}
x^* = x_8 &= k_8-z_8 + x_5 && \text{interval $(165, 255)$} \\
	  x_5 &= k_5-z_5 + x_2 && \text{interval $(90, 150)$} \\
	  x_2 &= k_2-z_2 + x_0 && \text{interval $(30, 90)$} \\
\end{align*}
\end{enumerate}
\end{odgovor}
\end{naloga}
