\begin{naloga}{Alen Orbanić}{Kolokvij OR 19.11.2009}
\begin{vprasanje}
Poišči minimalno vrednost parametra $\alpha$,
da bo pretok skozi omrežje s slike~\fig maksimalen.
Utemelji izbiro.

\begin{slika}
\pgfslika
\podnaslov{Omrežje}
\end{slika}
\end{vprasanje}

\begin{odgovor}
Predpostavimo, da je vrednost parametra $\alpha$ dovolj velika,
da ne vpliva na pretok skozi omrežje.
Ob tej predpostavki poiščemo maksimalni pretok
-- postopek reševanja je prikazan na slikah~%
\fig[minmaxpretok-resitev1],~\fig[minmaxpretok-resitev2],~%
\fig[minmaxpretok-resitev3] in~\fig[minmaxpretok-resitev4].
Dobimo maksimalni pretok $10+2+4 = 2+9+5 = 16$,
najmanjša vrednost $\alpha$, pri kateri ga lahko dosežemo,
pa bo enaka najmanjšemu možnemu pretoku po povezavi $b \to c$.
Ta pretok bi lahko zmanjšali,
če bi našli cikel iz premih povezav z neničelnimi pretoki
in nezasičenih nasprotnih povezav,
ki vsebuje povezavo $b \to c$.
Toda iz slike~\fig[minmaxpretok-resitev4] vidimo,
da lahko iz vozlišča $c$
nadaljujemo le kot $c \to t \to c$ ali $c \to e \to t \to c$
in tako ne moremo doseči vozlišča $b$.
Pretoka po povezavi $b \to c$ torej ni mogoče zmanjšati,
tako da je iskana vrednost enaka $\alpha = 7$.

\begin{slika}
\pgfslika[minmaxpretok-resitev1]
\podnaslov{Prvi korak}
\end{slika}
\begin{slika}
\pgfslika[minmaxpretok-resitev2]
\podnaslov{Drugi korak}
\end{slika}
\begin{slika}
\pgfslika[minmaxpretok-resitev3]
\podnaslov{Tretji korak}
\end{slika}
\begin{slika}
\pgfslika[minmaxpretok-resitev4]
\podnaslov{Maksimalni pretok in minimalni prerez}
\end{slika}
\end{odgovor}
\end{naloga}
