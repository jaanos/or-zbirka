\begin{naloga}{?}{Kolokvij OR 31.5.2012}
\begin{vprasanje}
Rexhep Bajrami bi se rad naslednja štiri leta
ukvarjal s prodajo sadja in zelenjave
(po štirih letih mu poteče delovna viza).
Rad bi najel parcelo za stojnico, ki bo stala $6000 €$.
Če je lokacija dobra, bo imel $12000 €$ dobička,
če pa je lokacija slaba, bo imel le $3000 €$ dobička.
Ocenjuje, da je z verjetnostjo $2/3$ lokacija dobra,
z verjetnostjo $1/3$ pa slaba.
\begin{enumerate}[(a)]
\item Z odločitvenim drevesom opiši njegove možnosti in ugotovi,
kako naj se odloči ter kakšen dobiček naj pričakuje.
\item Za nasvet lahko vpraša znanca Seada, ki ``ima nos'' za tovrstne posle.
Sead mu lahko da nasvet, a zanj zahteva $1200 €$.
Dobro je znano, da ima Sead naslednje pogojne verjetnosti
$P(\text{Seadovo mnenje} \; | \; \text{kakovost parcele})$:
\begin{center}
\begin{tabular}{c|cc}
& dobra & slaba \\
\hline
priporoča & $2/3$ & $1/2$ \\
odsvetuje & $1/3$ & $1/2$
\end{tabular}
\end{center}
Ali naj vpraša Seada za nasvet?
Kakšen je pričakovani dobiček?
\end{enumerate}
\end{vprasanje}
\begin{odgovor}
\end{odgovor}
\end{naloga}
