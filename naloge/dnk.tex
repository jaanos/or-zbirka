\begin{naloga}{?}{Izpit OR 9.2.2011}
\begin{vprasanje}
Znanstveniki proučujejo novo vrsto bakterije.
V DNK te bakterije so našli na\-sled\-nje zaporedje nukleotidov:
\begin{center}
{\tt AACAGTTA}.
\end{center}
Sumijo, da je to različica gena že znane bakterije, ki ima zaporedje
\begin{center}
{\tt ACCATGTA}.
\end{center}
Če sta zaporedji dovolj podobni,
obstaja verjetnost, da imata gena podobni funkciji.
Dovoljene so naslednje operacije:
\begin{itemize}
\item substitucija ({\tt ACT} $\to$ {\tt AGT}),
\item vstavljanje ({\tt AC} $\to$ {\tt AGC}),
\item izbris ({\tt AGC} $\to$ {\tt AC}) in
\item transpozicija ({\tt AT} $\to$ {\tt TA}).
\end{itemize}
\begin{enumerate}[(a)]
\item Pomagajte znanstvenikom in napišite postopek,
ki bo preštel najmanše število operacij,
s katerim pridemo od enega DNK zaporedja do drugega.
\item Postopek iz prejšnje točke izvedite nad podanima zaporedjema
in ugotovite število razlik.
\end{enumerate}
\end{vprasanje}
\begin{odgovor}
\end{odgovor}
\end{naloga}
