\begin{naloga}{Gabrovšek, Konvalinka}{Izpit OR 9.2.2011}
\begin{vprasanje}
Znanstveniki proučujejo novo vrsto bakterije.
V DNK te bakterije so našli na\-sled\-nje zaporedje nukleotidov:
\begin{center}
{\tt AACAGTTA}.
\end{center}
Sumijo, da je to različica gena že znane bakterije, ki ima zaporedje
\begin{center}
{\tt ACCATGTA}.
\end{center}
Če sta zaporedji dovolj podobni,
obstaja verjetnost, da imata gena podobni funkciji.
Dovoljene so naslednje operacije:
\begin{itemize}
\item substitucija ({\tt ACT} $\to$ {\tt AGT}),
\item vstavljanje ({\tt AC} $\to$ {\tt AGC}),
\item izbris ({\tt AGC} $\to$ {\tt AC}) in
\item transpozicija ({\tt AT} $\to$ {\tt TA}).
\end{itemize}
\begin{enumerate}[(a)]
\item Pomagaj znanstvenikom in napiši postopek,
ki bo preštel najmanjše število operacij,
s katerim pridemo od enega zaporedja nukleotidov do drugega.
\item Postopek iz prejšnje točke izvedi nad podanima zaporedjema
in ugotovi število razlik.
\end{enumerate}
\end{vprasanje}

\begin{odgovor}
\begin{enumerate}[(a)]
\item Naj bosta $(a_i)_{i=1}^m$ in $(b_j)_{j=1}^n$
podani zaporedji nukleotidov,
spremenljivka $r_{uv}$ ($0 \le u \le m, 0 \le v \le n$)
pa naj predstavlja najmanjše število operacij,
s katerim pridemo od zaporedja $(a_i)_{i=1}^u$
do zaporedja $(b_j)_{j=1}^v$.
Zapišimo začetne pogoje in rekurzivne enačbe.
\begin{alignat*}{3}
r_{0v} &= v &&&\quad (0 &\le v \le n) \\
r_{u0} &= u &&&\quad (0 &\le u \le m) \\
r_{uv} &= 1 &+ \min\big(
&{} \{r_{u-1,v-1}, r_{u-1,v}, r_{u,v-1}\} \cup \set{r_{u-1,v-1} - 1}{a_u = b_v}
&\quad (1 &\le u \le m, \\
&& \cup &{} \set{r_{u-2,v-2}}{u, v \ge 2
                              \land a_{u-1} = b_v \land a_u = b_{v-1}}\big)
&\quad 1 &\le v \le n)
\end{alignat*}
Vrednosti $r_{uv}$ računamo v leksikografskem vrstnem redu
glede na indeksa $(u, v)$,
najmanjše potrebno število operacij pa dobimo kot $r^* = r_{mn}$.
Algoritem, ki sledi iz zgornjih rekurzivnih enačb, teče v času $O(mn)$.

\item Zapišimo matriko vrednosti $(r_{uv})_{u,v=0}^8$.
$$
\begin{array}{c|ccccccccc}
r_{uv} & \epsilon & \text{\tt A} & \text{\tt C} & \text{\tt C} & \text{\tt A} & \text{\tt T} & \text{\tt G} & \text{\tt T} & \text{\tt A} \\ \hline
\epsilon     & 0 & 1 & 2 & 3 & 4 & 5 & 6 & 7 & 8 \\
\text{\tt A} & 1 & 0 & 1 & 2 & 3 & 4 & 5 & 6 & 7 \\
\text{\tt A} & 2 & 1 & 1 & 2 & 2 & 3 & 4 & 5 & 6 \\
\text{\tt C} & 3 & 2 & 1 & 1 & 2 & 3 & 4 & 5 & 6 \\
\text{\tt A} & 4 & 3 & 2 & 2 & 1 & 2 & 3 & 4 & 5 \\
\text{\tt G} & 5 & 4 & 3 & 3 & 2 & 2 & 2 & 3 & 4 \\
\text{\tt T} & 6 & 5 & 4 & 4 & 3 & 2 & 2 & 2 & 3 \\
\text{\tt T} & 7 & 6 & 5 & 5 & 4 & 3 & 3 & 2 & 3 \\
\text{\tt A} & 8 & 7 & 6 & 6 & 5 & 4 & 4 & 3 & 2
\end{array}
$$
Najmanjše število operacij je tako $r_{88} = 2$, dosežemo pa ga tako,
da opravimo substitucijo {\tt A} $\to$ {\tt C} na drugem mestu
in transpozicijo {\tt GT} $\to$ {\tt TG} na šestem in sedmem mestu.
\end{enumerate}
\end{odgovor}
\end{naloga}
