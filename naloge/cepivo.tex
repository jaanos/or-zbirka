\begin{naloga}{Janoš Vidali}{Izpit OR 25.8.2020}
\begin{vprasanje}
V laboratoriju bodo razvijali cepivo za COVID-19.
V tabeli~\tab so zbrana opravila, ki so jih identificirali,
 in odvisnosti med njimi.
\begin{enumerate}[(a)]
\item Topološko uredi ustrezni graf in ga nariši.
\item Določi kritična opravila in kritično pot ter čas izdelave.
\item Katero opravilo je najmanj kritično?
Najmanj kritično je opravilo,
katerega trajanje lahko najbolj podaljšamo,
ne da bi vplivali na trajanje izdelave.
\end{enumerate}

\begin{tabela}
\begin{tabular}{c|ccc}
    & Opravilo                         & Pogoji & Trajanje \\ \hline
$a$ & pregled literature               & /      & 45 dni \\
$b$ & pridobivanje vzorcev virusa      & /      & 20 dni \\
$c$ & genska analiza vzorcev           & $b$    & 40 dni \\
$d$ & priprava prototipa cepiva        & $a, b$ & 50 dni \\
$e$ & izboljševanje cepiva             & $c, f$ & 30 dni \\
$f$ & laboratorijsko testiranje        & $d$    & 21 dni \\
$g$ & proizvodnja cepiva               & $k$    & 35 dni \\
$h$ & iskanje prostovoljcev            & /      & 25 dni \\
$i$ & testiranje s prostovoljci        & $e, h$ & 60 dni \\
$j$ & pridobivanje certifikacije       & $i$    & 14 dni \\
$k$ & sklepanje prednaročniških pogodb & $i$    &  7 dni
\end{tabular}
\podnaslov{Podatki}
\end{tabela}
\end{vprasanje}

\begin{odgovor}
\begin{enumerate}[(a)]
\item Projekt lahko predstavimo z uteženim grafom s slike~\fig,
iz katerega je raz\-vid\-na topološka ureditev
$s, a, b, h, c, d, f, e, i, j, k, g, t$.

\item V tabeli~\tab[cepivo-resitev]
so podani najzgodnejši začetki opravil in najpoznejši začetki,
da se celotno trajanje priprave ne podaljša.
Izdelava bo torej trajala najmanj $248$ dni,
edina kritična pot pa je $s - a - d - f - e - i - k - g - t$,
ki tako vsebuje tudi vsa kritična opravila.

\item Iz tabele~\tab[cepivo-resitev] je razvidno,
da je najmanj kritično opravilo $h$,
saj lahko njegovo trajanje podaljšamo za $121$ dni,
ne da bi vplivali na trajanje celotnega projekta.
\end{enumerate}
%
\begin{slika}
\makebox[\textwidth][c]{
\pgfslika
}
\podnaslov{Graf odvisnosti med opravili in kritična pot}
\end{slika}
%
\begin{tabela}
\setlabel{cepivo-resitev}
\makebox[\textwidth][c]{
\begin{tabular}{r|cccccccccccccc}
& $s$ & $a$ & $b$ & $h$ & $c$ & $d$ & $f$ & $e$ & $i$ & $j$ & $k$ & $g$ & $t$ \\ \hline
najprej &
$0$ & $0_s$ & $0_s$ & $0_s$ & $20_b$ & $45_a$ & $95_d$ & $116_f$ & $146_e$ & $206_i$ & $206_i$ & $213_k$ & $248_g$ \\
najkasneje &
$0_a$ & $0_d$ & $25_d$ & $121_i$ & $76_e$ & $45_f$ & $95_e$ & $116_i$ & $146_k$ & $234_t$ & $206_g$ & $213_t$ & $248$ \\
razlika &
$0$ & $0^*$ & $25$ & $121$ & $56$ & $0^*$ & $0^*$ & $0^*$ & $0^*$ & $28$ & $0^*$ & $0^*$ & $0$
\end{tabular}
}
\podnaslov{Razporejanje opravil}
\end{tabela}
\end{odgovor}
\end{naloga}
