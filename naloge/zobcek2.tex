\begin{naloga}{David Gajser}{Izpit OR 24.6.2013}
\begin{vprasanje}
V turistični agenciji Zobček že vrsto let
organizirajo izlete po Sloveniji in na Madžarskem.
Zaradi velikega povpraševanja so se odločili,
da bodo ponudbo razširili tudi na države območja bivše Jugoslavije.
Ker bodo tako nudili izlete po veliko novih državah,
želijo postaviti štiri najboljše dosedanje vodiče za nove šefe,
tako da bo vsak odgovoren za nekaj držav.
Naša naloga je šefe stabilno razporediti po območjih.
Tabeli preferenc sta naslednji:
\begin{center}
\makebox[\textwidth][c]{
\begin{small}
\begin{tabular}{c|c}
Šefi & preference šefov \\ \hline
Črt  & Srbija in Črna gora, Slovenija in Madžarska,
       Hrvaška in BiH, Makedonija in Kosovo \\
Žan  & Slovenija in Madžarska, Srbija in Črna gora,
       Hrvaška in BiH, Makedonija in Kosovo \\
Šime & Slovenija in Madžarska, Makedonija in Kosovo,
       Hrvaška in BiH, Srbija in Črna gora \\
Mujo & Hrvaška in BiH, Slovenija in Madžarska,
       Makedonija in Kosovo, Srbija in Črna gora
\end{tabular}
\end{small}
}

\bigskip
\begin{tabular}{c|c}
Območje                & preference agencije glede šefov \\ \hline
Slovenija in Madžarska & Črt, Mujo, Žan, Šime \\
Hrvaška in BiH         & Šime, Žan, Črt \\
Srbija in Črna gora    & Šime, Žan, Mujo, Črt \\
Makedonija in Kosovo   & Šime, Mujo, Črt, Žan
\end{tabular}
\end{center}
Opazimo, da agencija za šefa območja Hrvaške in BiH ne želi postaviti Muja.

Poišči stabilno prirejanje, v katerem Mujo ni šef območja Hrvaške in BiH,
ali utemelji, da le-ta ne obstaja\footnote{
Slovenija in Madžarska je eno nedeljivo območje.
Enako velja za ostale pare držav.
}.
\end{vprasanje}

\begin{odgovor}
Stabilno prirejanje, v katerem Mujo ni šef območja Hrvaške in BiH, ne obstaja.
To lahko dokažemo tako, da predpostavimo, da imamo tako prirejanje,
ter analiziramo možne primere in izpeljemo protislovje.

Denimo, da je Mujo šef območja Slovenije in Madžarske.
Ker bi pri agenciji na tem mestu raje imeli Črta,
bo ta moral biti šef območja z večjo preferenco, torej Srbije in Črne gore.
V agenciji bi na tem mestu raje videli Žana,
ki sta mu pa ostali le območji z nižjo preferenco.
Prišli smo do protislovja.

Mujo torej ni šef območja Slovenije in Madžarske,
pač pa območja z nižjo preferenco.
Šef območja Slovenije in Madžarske je tako nekdo,
ki ima pri agenciji višjo preferenco kot Mujo -- tak je edino Črt.
Za šefa območja Makedonije in Kosova bi na agenciji raje kot Muja imeli Šimeta
-- če je Mujo šef tega območja, je Šime šef območja z nižjo preferenco,
protislovje.
Mujo je torej šef območja Srbije in Črne gore,
a bi v agenciji na tem mestu raje videli Žana,
ki pa je potem šef območja z nižjo preferenco.
Tako spet pridemo do protislovja,
iz česar sklepamo, da stabilno prirejanje,
v katerem Mujo ni šef območja Hrvaške in BiH, ne obstaja.
\end{odgovor}
\end{naloga}
