\begin{naloga}{Janoš Vidali}{Kolokvij OR 17.4.2023}
\begin{vprasanje}
Danih je $n$ enakokrakih pravokotnih trikotnikov,
pri čemer je $\ell_i$ ($1 \le i \le n$) dolžina katete $i$-tega trikotnika.
Vsak trikotnik se z enim od krajišč hipotenuze dotika osi $x$,
pri čemer se nanj priležna kateta dviga navpično nad njim
(tj., pravokotno na os $x$),
nasprotna kateta pa je na desni strani priležne katete.
Trikotnike želimo postaviti tako,
da se med seboj ne prekrivajo (lahko se dotikajo),
območje, ki ga zavzemajo, pa ima čim manjšo skupno širino
(tj., minimiziramo razliko v $x$-koordinatah levega roba skrajno levega trikotnika
in najbolj desnega oglišča kateregakoli od trikotnikov),
pri čemer smemo trikotnike le premikati vzdolž osi $x$
(tj., ne smemo jih odmakniti od osi $x$ oziroma jih rotirati ali zrcaliti
-- vsi trikotniki torej gledajo v isto smer).
Lahko se prepričaš, da mora torej za poljubna različna $i, j$
veljati $|x_i - x_j| \ge \min\{\ell_i, \ell_j\}$,
kjer sta $x_i$ in $x_j$ $x$-koordinati
levih robov $i$-tega in $j$-tega trikotnika.
Primer je podan na sliki~\fig.

\begin{enumerate}[(a)]
\item Zapiši celoštevilski linearni program, ki modelira zgornji problem.
Iz rešitve naj bo mogoče odčitati tudi ustrezno postavitev trikotnikov,
torej $x$-koordinate njihovih levih robov.
\namig{za vsak urejen par trikotnikov uvedi spremenljivko,
ki pove, kateri od njiju je bolj na levi,
ter minimiziraj zgornjo mejo za širino.
Lahko tudi predpostaviš, da so vsi trikotniki desno od izhodišča.}

\item K svojemu celoštevilskemu linearnemu programu dodaj pogoje za sledeči omejitvi.
\begin{itemize}
\item Trikotnik $a$ mora ležati med trikotnikoma $b$ in $c$
(pri čemer ni pomembno, kateri od njiju je bolj na levi).
\item Desno oglišče trikotnika $d$ je najbolj desno oglišče
kateregakoli od tri\-kot\-ni\-kov
(pri čemer dovolimo, da kak trikotnik leži v celoti pod trikotnikom $d$).
\end{itemize}
\end{enumerate}

\begin{slika}
\pgfslika
\caption{Primer dopustne (ne nujno optimalne) rešitve za nalogo~\nal.}
\end{slika}
\end{vprasanje}

\begin{odgovor}
Za $i$-ti trikotnik ($1 \le i \le n$) bomo uvedli spremenljivko $x_i$,
ki poda $x$-koordinato njegovega levega roba.
Za $i$-ti in $j$-ti trikotnik ($1 \le i, j \le n$, $i \ne j$)
bomo uvedli še spremenljivko $y_{ij}$,
katere vrednost interpretiramo kot
$$
y_{ij} = \begin{cases}
1 & \text{$i$-ti trikotnik leži levo od $j$-tega trikotnika, in} \\
0 & \text{sicer.}
\end{cases}
$$
Poleg tega bomo uvedli še spremenljivko $t$,
ki predstavlja zgornjo mejo za $x$-koordinate desnih oglišč trikotnikov.
Nazadnje uvedemo še konstanto $L = \sum_{i=1}^n \ell_i$,
ki predstavlja zgornjo mejo za širino območja, ki ga zavzemajo trikotniki.

\begin{enumerate}[(a)]
\item Zapišimo celoštevilski linearni program.
\begin{alignat*}{2}
&& \min \ t &\quad \text{p.p.} \\
\forall i \in \{1, \dots, n\}: &\ &
0 \le x_i &\le t - \ell_i \\
\forall i, j \in \{1, \dots, n\}, \ i \ne j: &\ &
0 \le y_{ij} &\le 1, \quad y_{ij} \in \Z
\opis{Trikotniki se ne prekrivajo}
\forall i, j \in \{1, \dots, n\}, \ i \ne j: &\ &
x_i - x_j + L y_{ij} &\ge \min\{\ell_i, \ell_j\}
\opis{Relativni položaj dveh trikotnikov}
\forall i, j \in \{1, \dots, n\}, \ i < j: &\ &
y_{ij} + y_{ji} &= 1
\end{alignat*}

\item Zapišimo še dodatne omejitve.
\odstraniprostor
\begin{align*}
\opis{Trikotnik $a$ leži med trikotnikoma $b$ in $c$}
y_{ab} + y_{ac} &= 1
\opis{Trikotnik $d$ se dotika desnega roba območja}
x_d + \ell_d &= t
\end{align*}
\end{enumerate}
\end{odgovor}
\end{naloga}

