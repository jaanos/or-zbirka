\begin{naloga}{Janoš Vidali}{Izpit OR 10.7.2017}
\begin{vprasanje}
Za hitrejše nalaganje posnetkov na YouTubu se uporabljajo krajevni strežniki,
do katerih lahko uporabniki na določeni lokaciji hitreje dostopajo
kakor do glavnega strežnika, ki vsebuje vse posnetke.
Vendar pa imajo krajevni strežniki omejen prostor,
zato je potrebno ugotoviti,
kateri posnetki naj se naložijo na katere krajevne strežnike.

Denimo, da imamo poleg glavnega strežnika še $k$ krajevnih strežnikov,
$m$ internetnih po\-nud\-ni\-kov in $n$ posnetkov.
Naj bo $c_h$ ($1 \le h \le k$) prostor v megabajtih,
ki je na voljo na krajevnem strežniku $h$.
Z indeksom $0$ označimo glavni strežnik
-- lahko torej predpostavljaš $c_0 = \infty$.
Nadalje naj bo $s_j$ ($1 \le j \le n$) velikost posnetka $j$,
prav tako v megabajtih,
$\ell_{hi}$ ($0 \le h \le k$, $1 \le i \le m$)
latenca (zakasnitev pri prenosu v milisekundah)
ponudnika $i$ do strežnika $h$,
in $r_{ij}$ ($1 \le i \le m$, $1 \le j \le n$)
število zahtevkov za posnetek $j$, ki jih pričakujejo od ponudnika $i$.
Vsi parametri so cela števila.

Pri vsakem zahtevku bo posnetek poslan iz strežnika z najmanjšo latenco,
ki vsebuje želeni posnetek.
Lahko predpostavljaš,
da ima za vsakega ponudnika glavni strežnik največjo latenco
(torej $\ell_{0i} \ge \ell_{hi}$ za vsaka $1 \le h \le k$, $1 \le i \le m$).
Določiti želimo, katere posnetke naj naložimo na posamezen krajevni strežnik,
da minimiziramo vsoto pričakovanih latenc pri vseh zahtevkih.
Posamezen posnetek lahko seveda naložimo tudi na več krajevnih strežnikov,
ali pa na nobenega (v tem primeru bo poslan iz glavnega strežnika).

Zapiši celoštevilski linearni program, ki modelira zgoraj opisani problem.
\end{vprasanje}

\begin{odgovor}
Za vsak strežnik $h$ ($0 \le h \le k$),
ponudnika $i$ ($1 \le i \le m$) in posnetek $j$ ($1 \le j \le n$)
bomo uvedli spremenljivko $x_{hij}$,
za strežnik $h$ ($1 \le h \le k$) in posnetek $j$ ($1 \le j \le n$)
pa dodatno še spremenljivko $y_{hj}$.
Njihove vrednosti interpretiramo kot
\begin{align*}
x_{hij} &= \begin{cases}
1, & \text{ponudniku $i$ pošiljamo posnetek $j$ iz strežnika $h$, in} \\
0  & \text{sicer;}
\end{cases} \\
\text{ter} \quad
y_{hj} &= \begin{cases}
1, & \text{posnetek $j$ naložimo na strežnik $h$, in} \\
0  & \text{sicer.}
\end{cases}
\end{align*}
Zapišimo celoštevilski linearni program.
\begin{alignat*}{3}
\min\ \sum_{h=0}^k \sum_{i=1}^m \sum_{j=1}^n r_{ij} \ell_{hi} x_{hij}
&\quad \text{p.p.} \\
\forall h \in \{0, \dots, k\} \ \forall i \in \{1, \dots, m\}
\ \forall j \in \{1, \dots, n\}:
&\ & 0 \le x_{hij} &\le 1, & x_{hij} &\in \Z \\
\forall h \in \{1, \dots, k\} \ \forall j \in \{1, \dots, n\}:
&\ & 0 \le y_{hj} &\le 1, & y_{hj} &\in \Z
\opis{Omejitev prostora}
\forall h \in \{1, \dots, k\}: &\ & \sum_{j=1}^n s_j y_{hj} &\le c_h
\opis{Za dostavo mora biti posnetek prisoten na strežniku}
\forall h \in \{0, \dots, k\} \ \forall i \in \{1, \dots, m\}
\ \forall j \in \{1, \dots, n\}: &\ & x_{hij} &\le y_{hj}
\opis{Strežnika za posnetek ne uporabimo,
če se nahaja na strežniku z manjšo latenco}
\forall h, t \in \{0, \dots, k\} \ \forall i \in \{1, \dots, m\}
\ \forall j \in \{1, \dots, n\}:
&\ & (\ell_{hi} - \ell_{ti}) (x_{hij} + y_{tj}) &\le 1
\opis{Vsak posnetek pošiljamo ponudniku iz natanko enega strežnika}
\forall h \in \{0, \dots, k\} \ \forall j \in \{1, \dots, n\}:
&\ & \sum_{i=1}^m x_{hij} &= 1
\end{alignat*}
\end{odgovor}
\end{naloga}
