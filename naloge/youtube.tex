\begin{naloga}{Janoš Vidali}{Izpit OR 10.7.2017}
\begin{vprasanje}
Za hitrejše nalaganje posnetkov na YouTubu se uporabljajo krajevni strežniki,
do katerih lahko uporabniki na določeni lokaciji hitreje dostopajo
kakor do glavnega strežnika, ki vsebuje vse posnetke.
Vendar pa imajo krajevni strežniki omejen prostor,
zato je potrebno ugotoviti,
kateri posnetki naj se naložijo na katere krajevne strežnike.

Denimo, da imamo poleg glavnega strežnika še $k$ krajevnih strežnikov,
$m$ internetnih po\-nud\-ni\-kov in $n$ posnetkov.
Naj bo $c_h$ ($1 \le h \le k$) prostor v megabajtih,
ki je na voljo na krajevnem strežniku $h$.
Z indeksom $0$ označimo glavni strežnik
-- lahko torej predpostavljaš $c_0 = \infty$.
Nadalje naj bo $s_j$ ($1 \le j \le n$) velikost posnetka $j$,
prav tako v megabajtih,
$\ell_{hi}$ ($0 \le h \le k$, $1 \le i \le m$)
latenca (zakasnitev pri prenosu v milisekundah)
ponudnika $i$ do strežnika $h$,
in $r_{ij}$ ($1 \le i \le m$, $1 \le j \le n$)
število zahtevkov za posnetek $j$, ki jih pričakujejo od ponudnika $i$.
Vsi parametri so cela števila.

Pri vsakem zahtevku bo posnetek poslan iz strežnika z najmanjšo latenco,
ki vsebuje želeni posnetek.
Lahko predpostavljaš,
da ima za vsakega ponudnika glavni strežnik največjo latenco
(torej $\ell_{0i} \ge \ell_{hi}$ za vsaka $1 \le h \le k$, $1 \le i \le m$).
Določiti želimo, katere posnetke naj naložimo na posamezen krajevni strežnik,
da minimiziramo vsoto pričakovanih latenc pri vseh zahtevkih.
Posamezen posnetek lahko seveda naložimo tudi na več krajevnih strežnikov,
ali pa na nobenega (v tem primeru bo poslan iz glavnega strežnika).

Zapiši celoštevilski linearni program, ki modelira zgoraj opisani problem.
\namig{pri določitvi latence ponudnika za dan posnetek
si pomagaj s spremenljivko, ki šteje strežnike z latenco,
ki ne presega izračunane latence.}
\end{vprasanje}
\begin{odgovor}
\end{odgovor}
\end{naloga}
