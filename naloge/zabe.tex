\begin{naloga}{?}{Izpit OR 4.9.2012}
\begin{vprasanje}
Žabec Rok in žabica Neli živita v mlaki,
na kateri plavajo lokvanjevi listi.
Na enem listu sedi Rok,
na drugi strani mlake pa je list, na katerem počiva Neli.
Rok bi rad po listih priskakljal k Neli.
Z enim skokom lahko premosti razdaljo največ $d$.
Čisto vseeno mu je, koliko skokov naredi
in koliko je skupna preskakana razdalja.
Ali jo lahko obišče?
Lokvanjevi listi so podani kot seznam koordinat v $\R^2$.
Listi so majhni v primerjavi z razdaljami med njimi,
zato jih lahko obravnavaš kot točke.

\begin{enumerate}[(a)]
\item Formuliraj zgornji problem kot problem na grafih in predlagaj algoritem,
s katerim ga lahko rešiš.

\item Reši problem za $d = 3$ in lokvanje na koordinatah
$$
(0, 0), \ (2, 1), \ (4, 1), \ (2, 4), \ (7, 4),
\ (1, 6), \ (5, 6), \ (3, 8), \ (8, 8),
$$
pri čemer Rok sedi na lokvanju $(0, 0)$, Neli pa na $(8, 8)$.
\end{enumerate}
\end{vprasanje}
\begin{odgovor}
\end{odgovor}
\end{naloga}
