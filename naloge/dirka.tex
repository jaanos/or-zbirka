\begin{naloga}{Janoš Vidali}{Izpit OR 27.8.2019}
\begin{vprasanje}
Izdelati želimo terminski plan
za pripravo in izvedbo kolesarske dirke po Sloveniji.
V tabeli~\tab so zbrana opravila pri projektu.

\begin{enumerate}[(a)]
\item Topološko uredi ustrezni graf in ga nariši.
Za trajanja opravil vzemi pričakovana trajanja po modelu PERT.

\item Določi pričakovano kritično pot in čas izdelave.

\item Katero opravilo je (ob zgornjih predpostavkah) najmanj kritično?
Najmanj kritično je opravilo, katerega trajanje lahko najbolj podaljšamo,
ne da bi vplivali na celotno trajanje izvedbe.
\end{enumerate}
%
\begin{tabela}
\makebox[\textwidth][c]{
\begin{tabular}{c|cc|b{2cm}b{2.7cm}b{2cm}}
& Opravilo & Pogoji & Minimalno trajanje
& Najbolj verjetno trajanje & Maksimalno trajanje \\ \hline
$a$ & določitev trase                   & /         & 24 dni & 30 dni & 36 dni \\
$b$ & priprava trase za kronometer      & $a$       &  5 dni &  6 dni & 10 dni \\
$c$ & priprava trase za gorsko etapo    & $a$       & 20 dni & 20 dni & 26 dni \\
$d$ & priprava trase za zaključno etapo & $a$       &  6 dni &  7 dni &  8 dni \\
$e$ & pridobivanje tujih udeležencev    & $a$       & 16 dni & 17 dni & 21 dni \\
$f$ & izvedba regionalnih kvalifikacij  & /         & 15 dni & 18 dni & 24 dni \\
$g$ & izvedba kronometra                & $b, e, f$ &  1 dan &  1 dan &  1 dan \\
$h$ & izvedba gorske etape              & $c, g$    &  1 dan &  1 dan &  1 dan \\
$i$ & izvedba zaključne etape           & $d, h$    &  1 dan &  1 dan &  1 dan \\
$j$ & priprava zaključne prireditve     & $e, f$    &  4 dni &  5 dni &  6 dni
\end{tabular}
}
\caption{Podatki za nalogi~\nal in~\nal[dirka2].}
\end{tabela}
\end{vprasanje}

\begin{odgovor}
\begin{enumerate}[(a)]
\item Projekt lahko predstavimo z uteženim grafom s slike~\fig,
iz katerega je razvidna topološka ureditev
$s, a, f, b, c, d, e, g, j, h, i, t$.
Uteži povezav ustrezajo pričakovanim trajanjem opravil,
ki so prikazana v tabeli~\tab[dirka-resitev].

\item V tabeli~\tab[dirka-resitev]
so podani najzgodnejši začetki opravil in najpoznejši začetki,
da se celotno trajanje projekta ne podaljša.
Pričakovano trajanje projekta je torej $\mu = 53$ dni,
edina kritična pot pa je $s - a - c - h - i - t$,
ki tako vsebuje tudi vsa kritična opravila.

\item Iz tabele~\tab[dirka-resitev] je razvidno,
da je najmanj kritično opravilo $f$,
saj lahko njegovo trajanje
podaljšamo za $31.5$ dni v primerjavi s pričakovanim,
ne da bi vplivali na trajanje celotnega projekta.
\end{enumerate}
%
\begin{slika}
\makebox[\textwidth][c]{
\pgfslika
}
\podnaslov{Graf odvisnosti med opravili in kritična pot}
\end{slika}
%
\begin{tabela}
\setlabel{dirka-resitev}
$$
\begin{array}{c|cccccccccccc}
& s & a & f & b & c & d & e & g & j & h & i & t \\ \hline
\text{pričakovano} && 30 & 18.5 & 6.5 & 21 & 7 & 17.5 & 1 & 5 & 1 & 1 \\
\text{varianca} && 4 & 2.25 & 0.69 & 1 & 0.11 & 0.69 & 0 & 0.11 & 0 & 0 \\
\hline
\text{najprej} & 0 & 0_s & 0_s & 30_a & 30_a & 30_a & 30_a & 47.5_e & 47.5_e & 51_c & 52_h & 53_i \\
\text{najkasneje} & 0_a & 0_c & 29.5_j & 43.5_g & 30_h & 45_i & 30.5_j & 50_h & 48_t & 51_i & 52_t & 53 \\
\text{razlika} & 0 & 0^* & 29.5 & 13.5 & 0^* & 15 & 0.5 & 2.5 & 0.5 & 0^* & 0^* & 0
\end{array}
$$
\caption{Razporejanje opravil za nalogi~\nal in~\nal[dirka2].}
\end{tabela}
\end{odgovor}
\end{naloga}
