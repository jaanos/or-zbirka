\begin{naloga}{Bašić, Gajser}{Kolokvij OR 5.4.2012}
\begin{vprasanje}
Iz zajetja $s$ želimo preko treh čistilnih postaj
(vsako predstavljajo $4$ vozlišča)
napeljati vodo do tovarne $t$.
Opravka imamo z vodovodnim omrežjem,
pred\-stav\-lje\-nim z grafom na sliki~\fig.
Na povezavah so kapacitete cevi (v litrih na sekundo),
cevi z oznakami $\alpha_1$, $\alpha_2$ in $\alpha_3$ pa še niso zgrajene.

\begin{enumerate}[(a)]
\item Koliko vode lahko po danem omrežju dobi tovarna na sekundo?

\item Trenutni dotok vode v tovarno za potrebe proizvodnje ni dovolj velik,
zato razmišljajo o izgradnji novih cevi.
Možne lokacije novih cevi so $\alpha_1$, $\alpha_2$ in $\alpha_3$.
Uprava tovarne se je odločila,
da bodo s postavitvijo novih cevi na nekaterih možnih lokacijah
povečali pretok kar največ, kot je mogoče.
Cev kapacitete $K$ stane $K \cdot 100 €$.
Za inštalacijo ene cevi morajo plačati dodatnih $10\,000 €$.
Koliko najmanj jih bo stal projekt?
\end{enumerate}

\begin{slika}
\makebox[\textwidth][c]{
\pgfslika
}
\podnaslov{Omrežje}
\end{slika}
\end{vprasanje}

\begin{odgovor}
\begin{enumerate}[(a)]
\item Po vsaki od poti $s \to a \to g \to m \to t$,
$s \to b \to h \to m \to t$, $s \to c \to i \to m \to t$
$s \to d \to j \to m \to t$, $s \to e \to k \to m \to t$
in $s \to f \to \ell \to m \to t$ lahko povečamo pretok za $5$.
Skupni pretok je torej $30$, glej sliko~\fig[zajetje-resitev1].

\item Na sliki~\fig[zajetje-resitev2] vidimo,
kako lahko povečujemo pretok skozi čistilne naprave ob predpostavki,
da so vrednosti $\alpha_i$ ($i \in \{1, 2, 3\}$) dovolj velike.
Prva povečujoča pot je speljana skozi postajo $\{a, b, g, h\}$
-- po njej lahko pretok povečamo za največ $5$.
Druga povečujoča pot je speljana skozi postajo $\{c, d, i, j\}$
-- po njej lahko pretok prav tako povečamo za največ $5$.
Maksimalen pretok skozi čistilno postajo je prikazan za $\{e, f, k, \ell\}$.
Tako lahko skozi posamezno postajo pretok povečamo za največ $10$.

Iz slike~\fig[zajetje-resitev1] je razvidno,
da lahko pretok čez povezavo $m \to t$ povečamo za največ $20$.
Tako bomo postavili cevi skupne kapacitete $20$,
kar nas bo stalo $20 \cdot 100 € = 2\,000 €$.
To lahko dosežemo z inštalacijo dveh cevi kapacitete $10$,
kar nas dodatno stane $2 \cdot 10\,000 € = 20\,000 €$.
Skupna cena projekta je torej $22\,000 €$.
\end{enumerate}

\begin{slika}
\makebox[\textwidth][c]{
\pgfslika[zajetje-resitev1]
}
\podnaslov[\res{}(a)]{Maksimalni pretok in minimalni prerez}
\end{slika}
\begin{slika}
\makebox[\textwidth][c]{
\pgfslika[zajetje-resitev2]
}
\podnaslov[\res{}(b)]{Povečevanje pretokov}
\end{slika}
\end{odgovor}
\end{naloga}
