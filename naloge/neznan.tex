\begin{naloga}{Juvan, Orbanić}{Kolokvij OR 26.1.2010}
\begin{vprasanje}
Pri neznanem problemu, ki smo se ga lotili z dinamičnim programiranjem,
smo prišli do naslednje rekurzivne zveze:
$$
c_{ij} = \max\{c_{i-1,j-1}, c_{i+1,j-1} + 2c_{i-1,j+1}\} .
$$
Izračunati želimo vrednost $c_{mn}$.
Katere vrednosti $c_{ij}$,
pri čemer je $(i, j)$ zunaj $\{1, \dots, m\} \times \{1, \dots, n\}$,
moramo poznati, da bo $c_{mn}$ enolično določen?
Poišči čim manjšo takšno množico.
\end{vprasanje}

\begin{odgovor}
Opazimo, da za izračun vrednosti $c_{ij}$
potrebujemo samo take vrednosti $c_{i'j'}$,
za katere velja $i+j \equiv i'+j' \pmod{2}$.
V primeru $m = 1$ tako za izračun $c_{1j}$
potrebujemo le $c_{0,j-1}, c_{2,j-1}, c_{0,j+1}$,
v primeru $n = 1$ pa za izračun $c_{i1}$
potrebujemo le $c_{i-1,0}, c_{i+1,0}, c_{i-1,2}$.

V primeru, ko velja $m, n \ge 2$,
bo potrebno poznati vse vrednosti
$c_{0j}$ ($0 \le j \le n+1$, $j \equiv m+n \pmod{2}$),
$c_{i0}$ ($0 \le i \le m+1$, $i \equiv m+n \pmod{2}$),
$c_{m+1,j}$ ($1 \le j \le n-1$, $j+1 \equiv n \pmod{2}$)
in $c_{n+1,i}$ ($1 \le i \le n-1$, $i+1 \equiv m \pmod{2}$).
Ker pa v podani rekurzivni zvezi pride do zanke
(npr. za izračun $c_{ij}$ potrebujemo $c_{i+1,j-1}$,
za izračun slednje vrednosti pa spet potrebujemo $c_{ij}$),
morajo na teh začetnih vrednostih veljati določeni pogoji,
da bo rešitev sploh obstajala in da bo enolično določena.
\end{odgovor}
\end{naloga}
