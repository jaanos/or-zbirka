\begin{naloga}{Janoš Vidali}{Kolokvij OR 19.4.2019}
\begin{vprasanje}
V manjšem podjetju so razvili nov okus sadnega soka.
Proizvedli so ga že $100\,000$ litrov,
ko so prejeli ponudbo multinacionalke,
da odkupijo vso razpoložljivo količino po ceni $1 €$ na liter.
Če se ne odločijo za odprodajo,
bodo sok sami pakirali in ponudili na trgu po ceni $3 €$ na liter.
Zagon pakiranja stane $1\,000 €$, pakiranje enega litra soka pa $0.5 €$.
V podjetju ocenjujejo,
da bo z verjetnostjo $0.7$ produkt uspešen in ga bodo vsega prodali,
z verjetnostjo $0.3$ pa bo neuspešen in ga bodo prodali le $10\,000$ litrov.

Pri podjetju imajo na voljo ravno dovolj časa,
da pred odločitvijo pakirajo in na regionalnem trgu
ponudijo $1\,000$ litrov soka po akcijski ceni $2.5 €$ na liter.
Ocenjujejo,
da bi v primeru uspeha produkta z verjetnostjo $0.8$ tudi akcija uspela
in bi tako razprodali akcijske zaloge,
v primeru neuspeha produkta pa bi se to zgodilo le z verjetnostjo $0.1$.
Če akcija ne bi uspela, bi prodali le $100$ litrov soka.
Če se po akcijski ponudbi odločijo za samostojno prodajo,
bodo seveda morali še enkrat plačati stroške zagona pakiranja,
v nasprotnem primeru pa bo multinacionalka
odkupila preostalih $99\,000$ litrov še nepakiranega soka.
\opomba{količina, prodana v akcijski ponudbi,
je vključena v končno prodano količino.
Če torej v akciji prodajo vseh $1\,000$ litrov soka po ceni $2.5 €$ na liter,
ga bodo v primeru neuspeha produkta v redni prodaji
prodali le še $9\,000$ litrov po ceni $3 €$ na liter,
v primeru uspeha pa še $99\,000$ litrov.}

\smallskip

Kako naj se pri podjetju odločijo, da bo pričakovani zaslužek čim večji?
Nariši od\-lo\-čit\-ve\-no drevo in ga uporabi pri sprejemanju odločitev.
\end{vprasanje}

\begin{odgovor}
Če se podjetje odloči za samostojno prodajo (brez predhodne akcijske ponudbe),
bo ob uspehu imelo
$100\,000 \cdot (3 € - 0.5 €) - 1\,000 € = 249\,000 €$ dobička,
ob ne\-uspe\-hu pa
$10\,000 \cdot 3 € - 100\,000 \cdot 0.5 € - 1\,000 € = -21\,000 €$,
torej $21\,000 €$ izgube.
Ker je pričakovani dobiček enak
$$
0.7 \cdot 249\,000 € - 0.3 \cdot 21\,000 € = 168\,000 €,
$$
kar je več kot $100\,000 \cdot 1 € = 100\,000 €$,
kolikor bi zaslužili ob prodaji multinacionalki,
se podjetju v primeru, da se ne odločijo za akcijsko ponudbo,
bolj izplača samostojna prodaja.

\needspace{3\baselineskip}
Izračunajmo še dobičke, če se odločijo za akcijsko ponudbo:
\odstraniprostor
\begin{align*}
\opis{Akcija uspe, produkt uspe}
1\,000 \cdot 2.5 € + 99\,000 \cdot 3 € -
100\,000 \cdot 0.5 € - 2 \cdot 1\,000 € &= 247\,500 €
\opis{Akcija uspe, produkt ne uspe}
1\,000 \cdot 2.5 € + 9\,000 \cdot 3 € -
100\,000 \cdot 0.5 € - 2 \cdot 1\,000 € &= -22\,500 €
\opis{Akcija uspe, prodajo multinacionalki}
1\,000 \cdot (2.5 € - 0.5 €) + 99\,000 \cdot 1 € - 1\,000 € &= 100\,000 €
\opis{Akcija ne uspe, produkt uspe}
100 \cdot 2.5 € + 99\,900 \cdot 3 € -
100\,000 \cdot 0.5 € - 2 \cdot 1\,000 € &= 247\,950 €
\opis{Akcija ne uspe, produkt ne uspe}
100 \cdot 2.5 € + 9\,900 \cdot 3 € -
100\,000 \cdot 0.5 € - 2 \cdot 1\,000 € &= -22\,050 €
\opis{Akcija ne uspe, prodajo multinacionalki}
100 \cdot 2.5 € - 1\,000 \cdot 0.5 € - 1\,000 € + 99\,000 \cdot 1 €
&= 97\,750 €
\end{align*}

Izračunajmo še verjetnosti uspeha akcije
ter uspešnosti produkta ob vsaki napovedi.
\begin{align*}
P(\text{akcija uspešna}) &= 0.7 \cdot 0.8 + 0.3 \cdot 0.1 = 0.59 \\
P(\text{akcija neuspešna}) &= 0.7 \cdot 0.2 + 0.3 \cdot 0.9 = 0.41 \\
P(\text{produkt uspe} \;|\; \text{akcija uspešna})
&= {0.7 \cdot 0.8 \over 0.59} = {56 \over 59} \approx 0.949 \\
P(\text{produkt ne uspe} \;|\; \text{akcija uspešna})
&= {0.3 \cdot 0.1 \over 0.59} = {3 \over 59} \approx 0.051 \\
P(\text{produkt uspe} \;|\; \text{akcija neuspešna})
&= {0.7 \cdot 0.2 \over 0.41} = {14 \over 41} \approx 0.341 \\
P(\text{produkt ne uspe} \;|\; \text{akcija neuspešna})
&= {0.3 \cdot 0.9 \over 0.41} = {27 \over 41} \approx 0.659
\end{align*}
S pomočjo zgoraj izračunanih verjetnosti
lahko narišemo odločitveno drevo s slike~\fig.
Opazimo, da se podjetju izplača izvesti akcijo,
saj je pričakovani dobiček v tem primeru $178\,002.5 €$.
Če akcija uspe, naj se odločijo za samostojno prodajo,
če pa ne uspe, pa naj zaloge prodajo multinacionalki.

\begin{slika}
\makebox[\textwidth][c]{
\pgfslika
}
\podnaslov{Odločitveno drevo}
\end{slika}
\end{odgovor}
\end{naloga}
