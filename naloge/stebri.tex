\begin{naloga}{Janoš Vidali}{Kolokvij OR 12.4.2021}
\begin{vprasanje}
Zgraditi želimo avtocestni viadukt dolžine $\ell$ metrov,
ki bo prečkal dolino s težavnim terenom.
Geologi in arhitekti so že pregledali teren in identificirali $n$ mest,
kamor bi lahko postavili podporne stebre.
Vsako mesto je predstavljeno s številom $x_i$ ($1 \le i \le n$),
ki predstavlja oddaljenost v metrih od začetka viadukta
(lahko predpostaviš,
da velja $0 < x_i < x_j < \ell$ za vsaka $1 \le i < j \le n$).
Gradnja stebra na mestu $x_i$ nas stane $c_i$,
razdalja do sosednjih stebrov
(oziroma začetne ali končne točke viadukta pri prvem in zadnjem stebru)
ne sme preseči $r_i$
-- t.j., če zgradimo stebra na mestih $x_i$ in $x_j$ ter nobenega drugega vmes,
mora veljati $|x_i - x_j| \le r_i$ in $|x_i - x_j| \le r_j$
(začetna in končna točka nimata take omejitve,
tako da je dolžina odsekov od začetka do prvega stebra
ter od zadnjega stebra do konca
odvisna samo od omejitev prvega in zadnjega stebra).
Poiskati želimo čim cenejšo postavitev stebrov,
da bo mogoče v celoti zgraditi viadukt.

\begin{enumerate}[(a)]
\item Zapiši rekurzivne enačbe za reševanje danega problema.
Razloži, kaj pred\-stav\-lja\-jo spremenljivke,
v kakšnem vrstnem redu jih računamo,
ter kako dobimo optimalno rešitev.

\item Oceni časovno zahtevnost algoritma, ki sledi iz zgoraj zapisanih enačb.

\needspace{3\baselineskip}
\item S svojim algoritmom poišči optimalno postavitev stebrov pri podatkih $\ell = 1000$, $n = 10$ in
\begin{alignat*}{11}
(x_i)_{i=1}^{10} &= \{&{} 49, &&{} 60, &&{} 119, &&{} 258, &&{} 309, &&{} 509, &&{} 688, &&{} 753, &&{} 857, &&{} 887\} &, \\
(c_i)_{i=1}^{10} &= \{&{} 8, &&{} 9, &&{} 32, &&{} 14, &&{} 18, &&{} 9, &&{} 27, &&{} 47, &&{} 16, &&{} 22\} &, \\
(r_i)_{i=1}^{10} &= \{&{} 76, &&{} 75, &&{} 217, &&{} 251, &&{} 277, &&{} 273, &&{} 181, &&{} 168, &&{} 193, &&{} 137\} &.
\end{alignat*}
\end{enumerate}
\end{vprasanje}

\begin{odgovor}
\begin{enumerate}[(a)]
\item Naj bo $p_i$ najmanjša cena postavitve stebrov do $i$-tega mesta,
če zadnji steber stoji na $i$-tem mestu
(pri čemer upoštevamo,
da moramo ``stebra'' postaviti tudi na začetno in končno točko
brez dodatnih stroškov).
Določimo začetne pogoje in rekurzivne enačbe.
\begin{align*}
p_0 &= x_0 = c_{n+1} = 0 \\
r_0 &= r_{n+1} = \infty \\
x_{n+1} &= \ell \\
p_i &= c_i + \min\{p_j \mid 0 \le j \le i-1, x_i - x_j \le \min\{r_i, r_j\}\}\} && (1 \le i \le n+1)
\end{align*}
Vrednosti $p_i$ računamo naraščajoče po indeksu $i$ ($0 \le i \le n+1$).
Najmanjšo ceno postavitve baznih postaj dobimo s $p^* = p_{n+1}$.

\item Za izračun vrednosti $p_i$ za posamezen $i$
potrebujemo $O(i)$ časa.
Ker ta izračun opravimo $n$-krat,
je torej časovna zahtevnost ustreznega algoritma $O(n^2)$.

\item Izračunajmo vrednosti $p_i$ ($1 \le i \le 11$).
\begin{alignat*}{2}
p_1    &=  8 + 0                           &&= 8  \\
p_2    &=  9 + \min\{\underline{0}, 8\}    &&= 9  \\
p_3    &= 32 + \min\{\underline{0}, 8, 9\} &&= 32  \\
p_4    &= 14 + 32                          &&= 46 \\
p_5    &= 18 + \min\{\underline{32}, 46\}  &&= 50 \\
p_6    &=  9 + \min\{\underline{46}, 50\}  &&= 55 \\
p_7    &= 27 + 55                          &&= 82 \\
p_8    &= 47 + \min\{\underline{55}, 82\}  &&= 102 \\
p_9    &= 16 + \min\{\underline{82}, 102\} &&= 98 \\
p_{10} &= 22 + \min\{102, \underline{98}\} &&= 120 \\
p_{11} &=  0 + \min\{\underline{98}, 120\} &&= 98
\end{alignat*}
Najmanjša cena postavitve stebrov je torej $p^* = p_{11} = 98$.
Poglejmo, kam moramo postaviti stebre.
\begin{align*}
p_{11} &= c_{11} + p_9 && \text{steber na $x_9 = 857$} \\
p_9    &= c_9    + p_7 && \text{steber na $x_7 = 688$} \\
p_7    &= c_7    + p_6 && \text{steber na $x_6 = 509$} \\
p_6    &= c_6    + p_4 && \text{steber na $x_4 = 258$} \\
p_4    &= c_4    + p_3 && \text{steber na $x_3 = 119$} \\
p_3    &= c_3    + p_0
\end{align*}
\end{enumerate}
\end{odgovor}
\end{naloga}
