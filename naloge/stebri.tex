\begin{naloga}{Janoš Vidali}{Kolokvij OR 12.4.2021}
\begin{vprasanje}
Zgraditi želimo avtocestni viadukt dolžine $\ell$ metrov,
ki bo prečkal dolino s težavnim terenom.
Geo\-lo\-gi in arhitekti so že pregledali teren in identificirali $n$ mest,
kamor bi lahko postavili podporne stebre.
Vsako mesto je predstavljeno s številom $x_i$ ($1 \le i \le n$),
ki predstavlja oddaljenost v metrih od začetka viadukta,
pri čemer velja $0 = x_1 < x_2 < \cdots < x_n = \ell$.
Gradnja stebra na mestu $x_i$ nas stane $c_i$
(pri čemer velja $c_1 = c_n = 0$,
saj prvi in zadnji steber predstavljata začetno in končno točko viadukta),
razdalja do sosednjih stebrov ne sme preseči $r_i$
-- t.j., če zgradimo stebra na mestih $x_i$ in $x_j$ ter nobenega drugega vmes,
mora veljati $|x_i - x_j| \le r_i$ in $|x_i - x_j| \le r_j$.
Poiskati želimo čim cenejšo postavitev stebrov,
da bo mogoče v celoti zgraditi viadukt.

\begin{enumerate}[(a)]
\item Zapiši rekurzivne enačbe za reševanje danega problema.
Razloži, kaj pred\-stav\-lja\-jo spremenljivke,
v kakšnem vrstnem redu jih računamo,
ter kako dobimo optimalno rešitev.

\item Zapiši psevdokodo algoritma, ki sledi iz zgoraj zapisanih enačb,
in oceni njegovo časovno zahtevnost.
Algoritem naj vrne samo najmanjšo ceno postavitve stebrov,
ne pa tudi same izbire mest za postavitev.

\needspace{3\baselineskip}
\item S svojim algoritmom poišči optimalno postavitev stebrov pri podatkih $\ell = 1000$, $n = 12$ in
\begin{alignat*}{13}
(x_i)_{i=1}^{12} &= \{&{} 0, &&{} 49, &&{} 60, &&{} 119, &&{} 258, &&{} 309, &&{} 509, &&{} 688, &&{} 753, &&{} 857, &&{} 887, &&{} 1000\} &, \\
(c_i)_{i=1}^{12} &= \{&{} 0, &&{} 8, &&{} 9, &&{} 32, &&{} 14, &&{} 18, &&{} 9, &&{} 27, &&{} 47, &&{} 16, &&{} 22, &&{} 0\} &, \\
(r_i)_{i=1}^{12} &= \{&{} 300, &&{} 76, &&{} 75, &&{} 217, &&{} 251, &&{} 277, &&{} 273, &&{} 181, &&{} 168, &&{} 193, &&{} 137, &&{} 300\} &.
\end{alignat*}
\end{enumerate}
\end{vprasanje}

\begin{odgovor}
\begin{enumerate}[(a)]
\item Naj bo $p_i$ najmanjša cena postavitve stebrov do $i$-tega mesta,
če zadnji steber stoji na $i$-tem mestu
(pri čemer upoštevamo,
da moramo ``stebra'' postaviti tudi na začetno in končno točko).
Določimo začetne pogoje in rekurzivne enačbe.
\begin{align*}
p_1 &= c_1 \\
p_i &= c_i + \min\{p_j \mid 1 \le j \le i-1, x_i - x_j \le \min\{r_i, r_j\}\}\} && (2 \le i \le n)
\end{align*}
Vrednosti $p_i$ računamo naraščajoče po indeksu $i$ ($1 \le i \le n$).
Najmanjšo ceno postavitve baznih postaj dobimo s $p^* = p_n$.

\item Zapišimo psevdokodo algoritma.
\begin{small}
\begin{algorithmic}
\Function{Stebri}{$(x_i)_{i=1}^n, (c_i)_{i=1}^n, (r_i)_{i=1}^n$}
	\State $p_1 \gets c_1$
	\For{$i = 2, 3, \dots, n$}
	    \State $p_i \gets \infty$
	    \State $j \gets i-1$
	    \While{$j > 0 \land x_i - x_j \le r_i$}
	        \If{$p_j < p_i	\land x_i - x_j \le r_j$}
	            \State $p_i \gets p_j$
	        \EndIf
	        \State $j \gets j-1$
	    \EndWhile
	    \State $p_i \gets p_i + c_i$
	\EndFor
	\State \Return $p_n$
\EndFunction
\end{algorithmic}
\end{small}

Za izračun vrednosti $p_i$ za posamezen $i$
potrebujemo $O(i)$ časa.
Ker ta izračun opravimo za $i = 1, 2, \dots, n+1$,
je torej časovna zahtevnost ustreznega algoritma $O(n^2)$.

\item Izračunajmo vrednosti $p_i$ ($2 \le i \le 12$).
\begin{alignat*}{2}
p_2    &=  8 + 0                           &&= 8  \\
p_3    &=  9 + \min\{\underline{0}, 8\}    &&= 9  \\
p_4    &= 32 + \min\{\underline{0}, 8, 9\} &&= 32  \\
p_5    &= 14 + 32                          &&= 46 \\
p_6    &= 18 + \min\{\underline{32}, 46\}  &&= 50 \\
p_7    &=  9 + \min\{\underline{46}, 50\}  &&= 55 \\
p_8    &= 27 + 55                          &&= 82 \\
p_9    &= 47 + 82                          &&= 129 \\
p_{10} &= 16 + \min\{\underline{82}, 129\} &&= 98 \\
p_{11} &= 22 + \min\{129, \underline{98}\} &&= 120 \\
p_{12} &=  0 + \min\{\underline{98}, 120\} &&= 98
\end{alignat*}
Najmanjša cena postavitve stebrov je torej $p^* = p_{12} = 98$.
Poglejmo, kam moramo postaviti stebre.
\begin{align*}
p_{12} &= c_{12} + p_{10} & \text{steber na } x_{10} &= 857 \\
p_{10} &= c_{10} + p_8    & \text{steber na } x_8    &= 688 \\
p_8    &= c_8    + p_7    & \text{steber na } x_7    &= 509 \\
p_7    &= c_7    + p_5    & \text{steber na } x_5    &= 258 \\
p_5    &= c_5    + p_4    & \text{steber na } x_4    &= 119 \\
p_4    &= c_4    + p_1
\end{align*}
\end{enumerate}
\end{odgovor}
\end{naloga}
