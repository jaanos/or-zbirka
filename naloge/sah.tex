\begin{naloga}{Janoš Vidali}{Kolokvij OR 22.4.2024}
\begin{vprasanje}
Dana je šahovnica velikosti $n \times n$
in množica (tipov) šahovskih figur $\mathcal{F}$.
Polja šahovnice indeksiramo s pari $(i, j)$ ($1 \le i, j \le n$),
kjer je $i$ številka (navpične) {\em linije}\footnote{%
Linije običajno sicer označujemo s črkami $a - h$.
},
$j$ pa številka (vodoravne) {\em vrste}.
Za vsako figuro $f \in \mathcal{F}$ je dana vrednost $c_f$,
poleg tega pa za vsako polje $(i, j)$ in figuro $f \in \mathcal{F}$
poznamo še množico $N_{fij}$,
ki vsebuje tista polja na šahovnici, ki jih napada figura $f$,
če jo postavimo na polje $(i, j)$.

Na šahovnico želimo postaviti figure tako,
da je njihova skupna vrednost čim večja
in da se med seboj ne napadajo --
če je na polju $(i, j)$ figura $f$,
mora torej vsako polje $(h, k) \in N_{fij}$ biti prazno.
Na posamezno polje lahko seveda postavimo največ eno figuro,
posameznih figur pa je lahko na šahovnici več.

Na sliki~\fig je podan primer dopustne postavitve figur
(pri običajnih šahovskih pravilih),
kjer kralj stoji na polju $(5, 1)$.
Z zeleno so označena prazna polja, ki jih ne napada nobena figura.
Vsako od ostalih praznih polj napada vsaj ena od figur na šahovnici.

\begin{enumerate}[(a)]
\item Zapiši celoštevilski linearni program, ki modelira zgornji problem.
\item Denimo, da velja $\mathcal{F} = \{K, Q, B, N, R, P\}$,
kjer oznake zaporedoma predstavljajo
kralja, kraljico, lovca, skakača, topa in kmeta\footnote{%
Standardne oznake izhajajo iz angleških imen za figure: {\em king}, {\em queen},
{\em bishop}, {\em k\underline{n}ight}, {\em rook}, {\em pawn}.
}.
K svojemu celoštevilskemu linearnemu programu dodaj pogoje,
ki modelirajo sledeče omejitve.
\begin{itemize}
\item Na šahovnici je natanko en kralj.
\item Na šahovnici je največ $n$ kmetov.
\item Noben kmet ni v prvi ali zadnji vrsti
(tj., na poljih $(i, j)$ z $j = 1$ ali $j = n$).
\item Če sta na šahovnici vsaj dva lovca,
potem ne smejo vsi zasedati polj iste barve
(tj., obstajati morata lovca na poljih $(i, j)$ in $(h, k)$,
kjer je eden od $i+j$ in $h+k$ lih, drugi pa sod).
\end{itemize}
\end{enumerate}

\begin{slika}
\setboardfontencoding{LSBC3}
\setchessboard{
showmover=false,
setpieces={Ke1,Pa2,Pa3,Pa5,Pc2,Pf3,Qd4,Bb5,Rg6,Rh7,Na8,Nb8,Nc8},
labelbottomformat=\arabic{filelabel}
}
\chessboard[
margintopwidth=0pt,
coloremph,
fieldmaskcolor=green,
emphareas={b1-c1,f5-f5,f8-f8}]
\podnaslov{Primer dopustne postavitve figur (pri običajnih šahovskih pravilih)}
\end{slika}
\end{vprasanje}

\begin{odgovor}
\begin{enumerate}[(a)]
\item Za figuro $f \in \mathcal{F}$ in polje $(i, j)$ ($1 \le i, j \le n$)
bomo uvedli spremenljivko $x_{fij}$,
katere vrednost interpretiramo kot
$$
x_{fij} = \begin{cases}
1 & \text{na polju $(i, j)$ je figura $f$, in} \\
0 & \text{sicer.}
\end{cases}
$$

Zapišimo celoštevilski linearni program.
\needspace{2\baselineskip}
\begin{alignat*}{3}
\min &\ & \sum_{f \in \mathcal{F}} \sum_{i=1}^n \sum_{j=1}^n c_f x_{fij}
&\quad \text{p.p.} \\
\forall f \in \mathcal{F} \ \forall i, j \in \{1, \dots, n\}: &\ &
0 \le x_{fij} \le 1, &\quad x_{fij} \in \Z
\opis{Na vsakem polju je največ ena figura}
\forall i, j \in \{1, \dots, n\}: &\ &
\sum_{f \in \mathcal{F}} x_{fij} &\le 1
\opis{Napadena polja so prazna}
\forall f \in \mathcal{F} \ \forall i, j \in \{1, \dots, n\}: &\ &
n^2 x_{fij} + \sum_{g \in \mathcal{F}} \sum_{(h, k) \in N_{fij}} x_{ghk}
&\le n^2
\end{alignat*}

\item Zapišimo še dodatne omejitve.
\odstraniprostor
\begin{align*}
\opis{Na šahovnici je natanko en kralj}
\sum_{i=1}^n \sum_{j=1}^n x_{Kij} &= 1
\opis{Na šahovnici je največ $n$ kmetov}
\sum_{i=1}^n \sum_{j=1}^n x_{Pij} &\le n
\opis{Noben kmet ni v prvi ali zadnji vrsti}
\sum_{i=1}^n (x_{Pi1} + x_{Pin}) &= 0
\opis{Če imamo vsaj dva lovca, morajo zasedati tako bela kot črna polja}
\sum_{i=1}^n \sum_{\substack{j=1 \\ \text{$i+j$ sod}}}^n x_{Bij} -
n^2 \sum_{i=1}^n \sum_{\substack{j=1 \\ \text{$i+j$ lih}}}^n x_{Bij} &\le 1 \\
\sum_{i=1}^n \sum_{\substack{j=1 \\ \text{$i+j$ lih}}}^n x_{Bij} -
n^2 \sum_{i=1}^n \sum_{\substack{j=1 \\ \text{$i+j$ sod}}}^n x_{Bij} &\le 1
\end{align*}
\end{enumerate}
\end{odgovor}
\end{naloga}

