\begin{naloga}{Janoš Vidali}{Izpit OR 22.6.2020}
\begin{vprasanje}
V šiviljski delavnici so začeli proizvajati zaščitne maske.
Vsak dan lahko izdelajo $400$ mask, prodajo pa jih $300$.
Cena zagona proizvodnje je $240 €$,
cena skladiščenja ene maske za en dan pa je $0.1 €$.

\begin{enumerate}[(a)]
\item Denimo, da primanjkljaj ni dovoljen.
Kako pogosto naj zaženejo proizvodnjo in koliko časa naj ta teče?
Koliko mask naj izdelajo v posameznem ciklu
ter kako veliko skladišče potrebujejo?

\item Obravnavajmo še primer, ko dovolimo primanjkljaj,
ki nas stane $0.4 €$ na dan za eno masko.
Izračunaj podatke iz prejšnje točke še za ta primer.
Kolikšen je največji dovoljeni primanjklaj?
\end{enumerate}
\end{vprasanje}

\begin{odgovor}
Imamo sledeče podatke (pri časovni enoti $1$ dan).
\begin{align*}
\nu &= 300 &
K &= 240 € \\
\lambda &= 400 &
s &= 0.1 €
\end{align*}
Pri teh podatkih izračunajmo še
$$
\alpha = \nu \left(1 - {\nu \over \lambda}\right) = 300 \cdot {1 \over 4} = 75 .
$$

\begin{enumerate}[(a)]
\item Ker primanjkljaj ni dovoljen,
vzamemo $p = \infty$ in torej $\beta = 1$.
Izračunajmo optimalno dolžino cikla $\tau^*$,
največjo zalogo $M^*$, trajanje proizvodnje $t^*$
ter število izdelanih mask $q^*$ v posameznem ciklu.
\begin{alignat*}{3}
\tau^* &= \sqrt{2K \beta \over s \alpha}
&&= \sqrt{480 \over 7.5} &&= 8 \\
M^* &= \sqrt{2K \alpha \over s \beta}
&&= \sqrt{36\,000 \over 0.1} &&= 600 \\
t^* &= {M^* \over \lambda - \nu}
&&= {600 \over 100} &&= 6 \\
q^* &= \tau^* \nu = t^* \lambda
&&= 8 \cdot 300 &&= 2\,400
\end{alignat*}
Optimalna dolžina cikla je torej $8$ dni, od tega proizvodnja teče $6$ dni.
V tem času proizvedejo $2\,400$ mask,
v skladišču pa mora biti prostora za $600$ mask.

\item Sedaj imamo $p = 0.4 €$ in torej $\beta = 1 + s/p = 1.25$.
Izračunajmo optimalno dolžino cikla $\tau^*$,
največjo zalogo $M^*$, največji primanjkljaj $m^*$,
trajanje pro\-iz\-vod\-nje $t^*$
ter število izdelanih mask $q^*$ v posameznem ciklu.
\begin{alignat*}{3}
\tau^* &= \sqrt{2K \beta \over s \alpha}
&&= \sqrt{600 \over 7.5} &&\approx 8.944 \\
M^* &= \sqrt{2K \alpha \over s \beta}
&&= \sqrt{36\,000 \over 0.125} &&\approx 536.656 \\
m^* &= \tau^* \alpha - M^*
&&\approx 670.820 - 536.656 &&\approx 134.164 \\
t^* &= {M^* + m^* \over \lambda - \nu}
&&\approx {670.820 \over 100} &&= 6.708 \\
q^* &= \tau^* \nu
&&\approx 8.944 \cdot 300 &&= 2\,683.282
\end{alignat*}
V tem primeru je torej optimalna dolžina cikla $8.944$ dni,
od tega proizvodnja teče $6.708$ dni.
V tem času proizvedejo približno $2\,683$ mask,
v skladišču pa mora biti prostora za $537$ mask.
Največji primanjkljaj ne preseže $135$ mask.
\end{enumerate}
\end{odgovor}
\end{naloga}
