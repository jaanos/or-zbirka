\begin{naloga}%
{Dasgupta, Papadimitriou, Vazirani}{\cite[Exercise~6.14]{dpv}}
\begin{vprasanje}
Imamo pravokoten kos blaga dimenzij $m \times n$,
kjer sta $m$ in $n$ pozitivni celi števili,
ter seznam $k$ izdelkov,
pri čemer potrebujemo za izdelek $h$
pravokoten kos blaga dimenzij $a_h \times b_h$
($a_h, b_h$ sta pozitivni celi števili),
ki ga prodamo za ceno $c_h > 0$.
Imamo stroj, ki lahko poljuben kos blaga razreže na dva dela
bodisi vodoravno, bodisi navpično.
Začetni kos blaga želimo razrezati tako,
da bomo lahko naredili izdelke,
ki nam bodo prinašali čim večji dobiček.
Pri tem smemo izdelati poljubno število kosov posameznega izdelka.
Kose blaga lahko seveda tudi obračamo
(tj., za izdelek $h$ lahko narežemo kos velikosti
$a_h \times b_h$ ali $b_h \times a_h$).

Zapiši rekurzivne enačbe za reševanje danega problema.
Razloži, kaj predstavljajo spremenljivke,
v kakšnem vrstnem redu jih računamo,
ter kako dobimo optimalno rešitev.
\end{vprasanje}

\begin{odgovor}
Naj bo $p_{ij}$ največji dobiček,
ki ga lahko iztržimo z rezanjem kosa blaga dimenzij $i \times j$
($1 \le i \le m$, $1 \le j \le n$).
Določimo rekurzivne enačbe.
\begin{alignat*}{3}
p_{ij} &= \max
& \big( &{} \set{c_h}{1 \le h \le k \land (a_h, b_h) \in \{(i, j), (j, i)\}\}} \\
&& \cup &{} \set{p_{\ell j} + p_{i-\ell, j}}{1 \le \ell \le i/2} \\
&& \cup &{} \set{p_{i \ell} + p_{i, j-\ell}}{1 \le \ell \le j/2} \cup \{0\}\big)
\end{alignat*}
Prva množica tukaj obravnava primere, ko imamo kos blaga želene velikosti,
naslednji dve pa obravnavata vodoravne oziroma navpične reze.
Zadnja množica poskrbi, da je vrednost $p_{11}$ definirana tudi v primeru,
ko nam kos blaga velikosti $1 \times 1$ ne prinese dobička.

Vrednosti $p_{ij}$ računamo v leksikografskem vrstnem redu indeksov
(npr.~najprej naraščajoče po $i$, nato pa naraščajoče po $j$).
Maksimalni dobiček dobimo kot $p^* = p_{mn}$.
\end{odgovor}
\end{naloga}
