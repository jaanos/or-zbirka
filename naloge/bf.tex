\begin{naloga}{Janoš Vidali}{Vaje OR 21.5.2018}
\begin{vprasanje}
S pomočjo Bellman-Fordovega algoritma
določi razdalje od vozlišča $a$ do ostalih vozlišč
v grafu s slike~\fig.

\begin{slika}
\pgfslika
\caption{Graf za nalogi~\nal in~\nal[fw].}
\end{slika}
\end{vprasanje}

\begin{odgovor}
Sledili bomo naslednjemu algoritmu.
\begin{small}
\begin{algorithmic}
\Function{BellmanFord}{$G = (V, E), s \in V, L : E \to \R$}
    \State $d \gets$ slovar z vrednostjo $\infty$ za vsako vozlišče $v \in V$
    \State $\pred \gets$ slovar z vrednostjo $\Null$
        za vsako vozlišče $v \in V$
    \State $d[s] \gets 0$
    \State $i \gets 0$
    \State $\text{\sl trenutna} \gets \{s\}$
    \While{$\lnot \text{\sl trenutna}.\isEmpty()$}
        \State $i \gets i+1$
        \If{$i > |V|$}
            \State \Return graf ima negativen cikel
        \EndIf
        \State $\text{\sl{naslednja}} \gets$ prazna množica
        \For{$u \in \text{\sl trenutna}$}
            \For{$v \in \Adj(G, u)$}
                \If{$d[v] > d[u] + L_{uv}$}
                    \State $d[v] \gets d[u] + L_{uv}$
                    \State $\pred[v] \gets u$
                    \State $\text{\sl{naslednja}}.\add(v)$
                \EndIf
            \EndFor
        \EndFor
        \State $\text{\sl{trenutna}} \gets \text{\sl{naslednja}}$
    \EndWhile
    \State \Return $(d, \pred)$
\EndFunction
\end{algorithmic}
\end{small}
V $i$-tem obhodu zanke {\bf while}
slovar $d$ vsebuje dolžine tistih najkrajših poti
od $s$ do vsakega drugega vozlišča,
ki vsebujejo največ $i$ povezav.
Če graf nima negativnih ciklov,
je v vsaki najkrajši poti največ $n-1$ povezav
(kjer je $n$ število vozlišč)
in zato algoritem konča najkasneje po $n$-tem obhodu zanke {\bf while}.
V nasprotnem primeru obstaja najkrajša pot z neskončno povezavami,
kar algoritem zazna ob vstopu v $(n+1)$-vi obhod zanke.
Časovna zahtevnost algoritma je tako $O(mn)$,
kjer je $m$ število povezav grafa.

Potek zgornjega algoritma na grafu s slike~\fig je prikazan v tabeli~\tab.
Drevo najkrajših poti je prikazano na sliki~\fig[bf-resitev].

\begin{tabela}
$$
\begin{array}{c|c|c|c|c}
i & u & v & L_{uv} & \text{razdalje in predhodniki} \\ \hline
  &   &   &    & [0, \infty, \infty, \infty, \infty, \infty, \infty, \infty] \\ \hline
1 & a & b &  3 & [0, 3_a, \infty, \infty, \infty, \infty, \infty, \infty] \\
1 & a & f &  6 & [0, 3_a, \infty, \infty, \infty, 6_a, \infty, \infty] \\ \hline
2 & b & c &  7 & [0, 3_a, 10_b, \infty, \infty, 6_a, \infty, \infty] \\
2 & b & e &  9 & [0, 3_a, 10_b, \infty, 12_b, 6_a, \infty, \infty] \\
2 & f & g &  8 & [0, 3_a, 10_b, \infty, 12_b, 6_a, 14_f, \infty] \\ \hline
3 & c & d & -7 & [0, 3_a, 10_b, 3_c, 12_b, 6_a, 14_f, \infty] \\
3 & e & d &  1 & [0, 3_a, 10_b, 3_c, 12_b, 6_a, 14_f, \infty] \\
3 & e & g &  1 & [0, 3_a, 10_b, 3_c, 12_b, 6_a, 13_e, \infty] \\
3 & g & f & -1 & [0, 3_a, 10_b, 3_c, 12_b, 6_a, 13_e, \infty] \\ \hline
4 & d & b &  9 & [0, 3_a, 10_b, 3_c, 12_b, 6_a, 13_e, \infty] \\
4 & d & h &  4 & [0, 3_a, 10_b, 3_c, 12_b, 6_a, 13_e, 7_d] \\
4 & g & f & -1 & [0, 3_a, 10_b, 3_c, 12_b, 6_a, 13_e, 7_d] \\ \hline
5 & h & g & -5 & [0, 3_a, 10_b, 3_c, 12_b, 6_a, 2_h, 7_d] \\ \hline
6 & g & f & -1 & [0, 3_a, 10_b, 3_c, 12_b, 1_g, 2_h, 7_d] \\ \hline
7 & f & g &  8 & [0, 3_a, 10_b, 3_c, 12_b, 1_g, 2_h, 7_d]
\end{array}
$$
\podnaslov{Potek izvajanja algoritma}
\end{tabela}

\begin{slika}
\pgfslika[bf-resitev]
\podnaslov{Drevo najkrajših poti}
\end{slika}
\end{odgovor}
\end{naloga}
