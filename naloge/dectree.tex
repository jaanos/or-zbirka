\begin{naloga}{Sergio Cabello}{Izpit OR 15.3.2017}
\begin{vprasanje}
Imaš odločitveno drevo s slike~\fig,
a nisi prepričan glede vrednosti $p \in [0, 1/3]$.
Pričakovano vrednost želiš maksimizirati.
Poišči optimalne odločitve glede na vrednost $p$.

\begin{slika}
\pgfslika
\podnaslov{Odločitveno drevo}
\end{slika}
\end{vprasanje}

\begin{odgovor}
Izračunajmo najprej pričakovane vrednosti v vozliščih $C$, $D$ in $E$
odločitvenega drevesa s slike~\fig.
\begin{alignat*}{2}
C &= 10 \cdot p + 2 \cdot p - 5 \cdot (1-2p) &&= 22p - 5 \\
D &= 3 \cdot 2p - 2 \cdot (1-2p) &&= 10p - 2 \\
E &= 1 \cdot 3p + 0 \cdot (1 - 3p) &&= 3p
\end{alignat*}
Obravnavajmo najprej odločitev v vozlišču $B$.
Za pot k vozlišču $D$ se odločimo,
če velja $10p - 2 \ge 3p$ oziroma $p \ge 2/7$,
sicer se odločimo za pot k vozlišču $E$.
Obravnavajmo sedaj še odločitev v vozlišču $A$
-- za pot k vozlišču $C$ se odločimo, če velja:
\begin{alignat*}{4}
p < {2 \over 7}: &\quad& 22p - 5 &\ge 3p
&\quad&\Rightarrow \quad & {5 \over 19} \le p &< {2 \over 7} \\
p \ge {2 \over 7}: &\quad& 22p - 5 &\ge 10p-2
&\quad&\Rightarrow \quad & p &\ge {2 \over 7} \ge {1 \over 4}
\end{alignat*}

\needspace{\baselineskip}
Odločamo se torej tako:
\begin{itemize}
\item če je $0 \le p < 5/19$,
se odločimo za pot preko vozlišča $B$ k vozlišču $E$,
pričakovana vrednost je $3p \in [0, 15/19)$; in
\item če je $5/19 \le p \le 1/3$,
se odločimo za pot k vozlišču $C$,
pričakovana vrednost je $22p-5 \in [15/19, 7/3]$.
\end{itemize}
Proces odločanja je prikazan na sliki~\fig[dectree-resitev].

\begin{slika}
\pgfslika[dectree-resitev]
\podnaslov[\res v odvisnosti od vrednosti parametra $p$]{Odločitveno drevo}
\end{slika}
\end{odgovor}
\end{naloga}
