\begin{naloga}{Janoš Vidali}{Izpit OR 2.6.2022}
\begin{vprasanje}
V hotelu z $n$ enakimi sobami so za sezono v trajanju $\ell$ dni
dobili $m$ zahtevkov za rezervacije,
za katere imajo podatke $(a_i, b_i, c_i)$ ($1 \le i \le m$),
pri čemer je $a_i$ najzgodnejši dan prihoda,
$b_i$ naj\-kas\-nej\-ši dan odhoda, $c_i$ pa želeno število nočitev
(pri tem velja $0 \le a_i < b_i \le \ell$ in $1 \le c_i \le b_i - a_i$).
Za vsak zahtevek se lahko odločijo,
da ga sprejmejo ali zavrnejo.
Če sprejmejo $i$-ti zahtevek,
morajo zagotoviti $c_i$ zaporednih nočitev v isti sobi
(tj., če se rezervacija sobe $j$ začne na dan $k$,
potem ji pripadajo nočitve na dni $k+1, k+2, \dots, k+c_i$,
kjer velja $k \ge a_i$, $k+c_i \le b_i$
-- nočitev na dan $k$ lahko pripada drugi rezervaciji,
prav tako se lahko na dan $k+c_i$ začne nova rezervacija).
Seveda je lahko vsaka soba vsako noč oddana le enkrat.
V hotelu želijo sprejeti zahtevke tako, da bo zasedenost čim večja
(tj., maksimizirati želijo vsoto števil zasedenih sob za vsako noč).

Zapiši celoštevilski linearni program,
ki modelira iskanje optimalne rešitve zgornjega problema.
\namig{uvedi spremenljivke,
ki beležijo začetke vsake sprejete rezervacije,
ter spremenljivke, ki beležijo dejansko zasedenost sob.
Ne pozabi vzpostaviti ustrezne povezave med njimi!}
\end{vprasanje}

\begin{odgovor}
Za $i$-ti zahtevek ($1 \le i \le m$),
$j$-to sobo ($1 \le j \le n$)
in $k$-ti dan ($0 \le k \le \ell$)
bomo uvedli spremenljivki $x_{ijk}$ in $y_{ijk}$,
katerih vrednosti interpretiramo kot
\begin{align*}
x_{ijk} &= \begin{cases}
1; &
\text{$i$-temu zahtevku dodelimo $j$-to sobo z začetkom na $k$-ti dan, in}
\\
0  & \text{sicer,}
\end{cases}
\shortintertext{ter}
y_{ijk} &= \begin{cases}
1; &
\text{$j$-to sobo v noči na $k$-ti dan zaseda $i$-ti zahtevek, in}
\\
0  & \text{sicer.}
\end{cases}
\end{align*}
Ker se nobena rezervacija ne more začeti na $\ell$-ti dan
in nobena soba ni zasedena v noči na ničti dan,
spremenljivk $x_{ij\ell}$ in $y_{ij0}$ ($1 \le i \le m$, $1 \le j \le n$)
ne bomo uporabili.

Zapišimo celoštevilski linearni program.
\begin{alignat*}{3}
\max &\ & \sum_{i=1}^m \sum_{j=1}^n \sum_{k=1}^\ell y_{ijk} \quad \text{p.p.} \\
\forall i \in \{1, \dots, m\} \ \forall j \in \{1, \dots, n\} \ \forall k \in \{0, \dots, \ell-1\}:
&\ & 0 \le x_{ijk} \le 1, \quad x_{ijk} &\in \Z \\
\forall i \in \{1, \dots, m\} \ \forall j \in \{1, \dots, n\} \ \forall k \in \{1, \dots, \ell\}:
&\ & 0 \le y_{ijk} \le 1, \quad y_{ijk} &\in \Z
\opis{Vsak zahtevek sprejmemo največ enkrat}
\forall i \in \{1, \dots, m\}:
&\ & \sum_{j=1}^n \sum_{k=0}^{\ell-1} x_{ijk} &\le 1
\opis{Vsaka soba je vsako noč zasedena največ enkrat}
\forall j \in \{1, \dots, n\} \ \forall k \in \{1, \dots, \ell\}:
&\ & \sum_{i=1}^m y_{ijk} &\le 1
\opis{Začetek rezervacije ni prepozen}
\forall i \in \{1, \dots, m\} \ \forall j \in \{1, \dots, n\} \\ \forall k \in \{b_i-c_i+1, \dots, \ell-1\}:
&\ & x_{ijk} &= 0
\opis{Soba ni rezervirana do najzgodnejšega začetka rezervacije}
\forall i \in \{1, \dots, m\} \ \forall j \in \{1, \dots, n\} \ \forall k \in \{1, \dots, a_i\}:
&\ & y_{ijk} &= 0
\opis{Nočitve so rezervirane $c_i$ dni po začetku rezervacije}
\forall i \in \{1, \dots, m\} \ \forall j \in \{1, \dots, n\} \ \forall k \in \{0, \dots, b_i-c_i\}:
&\ & \sum_{h=1}^{c_i} y_{i,j,k+h} &\ge c_i x_{ijk}
\opis{Začetek rezervacije je največ $c_i$ dni pred nočitvijo}
\forall i \in \{1, \dots, m\} \ \forall j \in \{1, \dots, n\} \ \forall k \in \{a_i+1, \dots, \ell\}:
&\ & \sum_{h=1}^{\min\{c_i, k\}} x_{i,j,k-h} &\ge y_{ijk}
\end{alignat*}
\end{odgovor}
\end{naloga}
