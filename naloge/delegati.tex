\begin{naloga}{Bašić, Gajser}{Kolokvij OR 5.4.2012}
\begin{vprasanje}
Na kongresu se je zbralo $n$ delegatov.
Problem je v tem, da ne govorijo vsi skup\-ne\-ga jezika.
Za vsakega delegata poznamo seznam jezikov, ki jih govori,
skupaj s stopnjo znanja za vsak posamični jezik.
Stopnja znanja je število med $1$ in $100$.
Stopnja $1$ pomeni, da oseba jezik popolnoma obvlada,
stopnja $100$ pa pomeni, da pozna zgolj nekaj osnovnih fraz.
Recimo, da bi oseba $A$ rada osebi $B$ posredovala neko sporočilo.
Če znata skupni jezik, se lahko pogovorita neposredno.
Lahko pa oseba $A$ pošlje sporočilo preko enega ali več posrednikov.

\begin{enumerate}[(a)]
\item Radi bi, da oseba $B$ prejme sporočilo v najkrajšem možnem času
(pri čemer lahko sporočilo potuje preko enega ali več posrednikov).
Če osebi $X$ in $Y$ govorita skupen jezik
ter sta stopnji obvladovanja tega jezika
$s_X$ za osebo $X$ in $s_Y$ za osebo $Y$,
prenos sporočila traja $\max\{s_X, s_Y\}$ časovnih enot.
(Če dve osebi ne obvladata dovolj dobro skupnega jezika,
si lahko pomagata z opisovanjem pojmov, mahanjem rok, risanjem ipd.)
Formuliraj zgornjo nalogo
kot problem iskanja najkrajše poti v ustreznem grafu.

\item Dan je naslednji seznam delegatov:

\smallskip
\makebox[\textwidth][r]{
\begin{small}
\begin{tabular}{c|c}
Delegat & Jeziki \\ \hline
Frank & (angleščina, $5$), (španščina, $10$), (ruščina, $80$) \\
Ivan & (ruščina, $5$), (španščina, $20$), (angleščina, $95$) \\
Paul-Henri & (francoščina, $5$), (nemščina, $85$), (angleščina, $95$) \\
Brigitte & (nizozemščina, $10$), (nemščina, $15$) \\
Andrej & (slovenščina, $5$), (nemščina, $10$), (latinščina, $90$) \\
Wolfgang & (nemščina, $5$), (angleščina, $90$) \\
Jafar &
(arabščina, $5$), (ruščina, $10$), (nizozemščina, $30$), (francoščina, $80$)
\end{tabular}
\end{small}
}

\bigskip
Kako lahko Frank najhitreje posreduje informacijo Wolfgangu?
\end{enumerate}
\end{vprasanje}

\begin{odgovor}
\begin{enumerate}[(a)]
\item Skonstruiramo graf, v katerem vsakemu delegatu priredimo vozlišče;
če delegata $X$ in $Y$ govorita skupni jezik
pri stopnjah obvladovanja $s_X$ in $s_Y$,
med njima postavimo povezavo z utežjo $\max\{s_X, s_Y\}$
(če govorita več skupnih jezikov,
potem zadostuje ena sama povezava med $X$ in $Y$,
in sicer tista z najmanjšo utežjo).
Na tem grafu nato poiščemo najcenejšo pot od $A$ do $B$,
kar lahko storimo z Dijkstrovim algoritmom,
saj so vse uteži nenegativne.

\item Po zgornjem postopku zgrajeni graf je prikazan na sliki~\fig.
Postopek izvajanja Dijkstrovega algoritma z začetkom pri Franku
je prikazan v tabeli~\tab,
iz katere je razvidno, da bo Frankovo sporočilo najhitreje
(v $75$ časovnih enotah) do\-seg\-lo Wolfganga,
če ga v španščini preda Ivanu,
ta ga v ruščini preda Jafarju,
ta ga v nizozemščini preda Brigitte,
ki ga nazadnje v nemščini preda Wolfgangu.
\end{enumerate}
%
\begin{slika}
\pgfslika
\podnaslov{Graf}
\end{slika}
%
\begin{tabela}
$$
\begin{array}{ccccccc}
\text{Frank} & \text{Ivan} & \text{Paul-Henri} & \text{Brigitte} & \text{Andrej} & \text{Wolfgang} & \text{Jafar} \\ \hline
* & 20_{\text{Frank}} & 95_{\text{Frank}} &&& 90_{\text{Frank}} & 80_{\text{Frank}} \\
& * &&&&& 30_{\text{Ivan}} \\
&&& 60_{\text{Jafar}} &&& * \\
&&& * & 75_{\text{Brigitte}} & 75_{\text{Brigitte}} & \\
&&&& * && \\
&&&&& * & \\
&& * &&&& \\
\end{array}
$$
\podnaslov[\res{}(b)]{Potek izvajanja algoritma}
\end{tabela}
\end{odgovor}
\end{naloga}
