\begin{naloga}{?}{Kolokvij OR 4.5.2012}
\begin{vprasanje}
Na kongresu se je zbralo $n$ delegatov.
Problem je v tem, da ne govorijo vsi skup\-ne\-ga jezika.
Za vsakega delegata poznamo seznam jezikov, ki jih govori,
skupaj s stopnjo znanja za vsak posamični jezik.
Stopnja znanja je število med $1$ in $100$.
Stopnja $1$ pomeni, da oseba jezik popolnoma obvlada,
stopnja $100$ pa pomeni, da pozna zgolj nekaj osnovnih fraz.
Recimo, da bi oseba $A$ rada osebi $B$ posredovala neko sporočilo.
Če znata skupni jezik, se lahko pogovorita neposredno.
Lahko pa oseba $A$ pošlje sporočilo preko enega ali več posrednikov.

\begin{enumerate}[(a)]
\item Radi bi, da oseba $B$ prejme sporočilo v najkrajšem možnem času
(pri čemer lahko sporočilo potuje preko enega ali več posrednikov).
Če osebi $X$ in $Y$ govorita skupen jezik
ter sta stopnji obvladovanja tega jezika
$s_X$ za osebo $X$ in $s_Y$ za osebo $Y$,
prenos sporočila traja $\max\{s_X, s_Y\}$ časovnih enot.
(Če dve osebi ne obvladata dovolj dobro skupnega jezika,
si lahko pomagata z opisovanjem pojmov, mahanjem rok, risanjem ipd.)
Formuliraj zgornjo nalogo
kot problem iskanja najkrajše poti v ustreznem grafu.

\item Dan je naslednji seznam delegatov:

\smallskip
\makebox[\textwidth][r]{
\begin{small}
\begin{tabular}{c|c}
Delegat & Jeziki \\ \hline
Frank & (angleščina, $5$), (španščina, $10$), (ruščina, $80$) \\
Ivan & (ruščina, $5$), (španščina, $20$), (angleščina, $95$) \\
Paul-Henri & (francoščina, $5$), (nemščina, $85$), (angleščina, $95$) \\
Brigitte & (nizozemščina, $10$), (nemščina, $15$) \\
Andrej & (slovenščina, $5$), (nemščina, $10$), (latinščina, $90$) \\
Wolfgang & (nemščina, $5$), (angleščina, $90$) \\
Jafar &
(arabščina, $5$), (ruščina, $10$), (nizozemščina, $30$), (francoščina, $80$)
\end{tabular}
\end{small}
}

\bigskip
Kako lahko Frank najhitreje posreduje informacijo Wolfgangu?
\end{enumerate}
\end{vprasanje}
\begin{odgovor}
\end{odgovor}
\end{naloga}
