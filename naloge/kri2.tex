\begin{naloga}{Blaž Jelenc}{Izpit OR 24.6.2014}
\begin{vprasanje}
Pri podatkih iz naloge~\nal[kri] poišči še odgovora na naslednji vprašanji.
\begin{enumerate}[(a)]
\item Koliko enot krvi vsake krvne skupine ostane neporabljenih?
\item Koliko enot ustrezne krvne skupine naj na kliniki še naročijo,
da bodo po\-zdra\-vi\-li vse paciente?
\end{enumerate}
\end{vprasanje}

\begin{odgovor}
\begin{enumerate}[(a)]
\item Iz slike~\fig[kri-resitev3] vidimo,
da ostaneta neporabljeni dve enoti krvi.
Ker sta povezavi $s \to k_O$ in $s \to k_{AB}$ v minimalnem prerezu
ter obstaja cikel $s \to k_A \to o_{AB} \gets k_{AB} \gets s$,
po katerem lahko pretok povečamo za največ $2$,
sledi, da sta neporabljeni enoti krvi iz krvnih skupin A ali AB.

\item Iz slike~\fig[kri-resitev3] vidimo,
da na kliniki potrebujejo kri še za eno osebo.
Ker obstaja cikel $t \gets o_O \gets k_O \to o_B \to t$,
po katerem lahko pretok povečamo za največ $1$,
ter sta povezavi $s \to k_O$ in $s \to k_{AB}$ v minimalnem prerezu
in povezava $k_B \to o_B$ ne predstavlja omejitve pri povečevanju pretoka,
sledi, da morajo na kliniki naročiti enoto krvi iz krvne skupine O ali B.
\end{enumerate}
\end{odgovor}
\end{naloga}
