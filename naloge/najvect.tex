\begin{naloga}{Janoš Vidali}{Kolokvij OR 3.6.2021}
\begin{vprasanje}
Dan je usmerjen utežen graf $G = (V, E)$ s cenami povezav $c_e$ ($e \in E$)
ter njegovo vozlišče $s$ in celo število $t$ ($0 \le t < n = |V|$).
V grafu $G$ želimo poiskati najkrajše poti
od vozlišča $s$ do ostalih vozlišč grafa,
ki vsebujejo največ $t$ povezav.

\begin{enumerate}[(a)]
\item Predlagaj čim učinkovitejši algoritem za reševanje danega problema.
Kakšna je njegova časovna zahtevnost?
\opozorilo{pazi, da bo mogoče iz izhoda rekonstruirati vse želene poti.}

\item S svojim algoritmom poišči najkrajše poti
od vozlišča $a$ do ostalih vozlišč v grafu s slike~\fig,
ki vsebujejo največ $4$ povezave.
Iz rešitve naj bo jasno, kako poteka izvajanje algoritma.
\end{enumerate}
%
\begin{slika}
\makebox[\textwidth][c]{
\pgfslika
}
\podnaslov[\nal{}(b)]{Graf}
\end{slika}
\end{vprasanje}

\begin{odgovor}
\begin{enumerate}[(a)]
\item Uporabili bomo prirejeno različico Bellman-Fordovega algoritma,
pri katerem bomo naredili le $t$ obhodov glavne zanke.
\begin{small}
\begin{algorithmic}
\Function{OmejeniBellmanFord}{$G = (V, E), s \in V, c : E \to \R, t \in \{0, 1, \dots, |V| - 1\}$}
    \State $d[0, \dots, t] \gets$ seznam slovarjev z vrednostjo $\infty$ za vsako vozlišče $v \in V$
    \State $\pred[1, \dots, t] \gets$ seznam slovarjev z vrednostjo $\Null$
	za vsako vozlišče $v \in V$
    \State $d[0][s] \gets 0$
    \For{$i = 1, 2, \dots, t$}
        \For{$u \in V$}
            \State $d[i][u] \gets d[i-1][u]$
            \State $\pred[i][u] \gets u$
        \EndFor
        \For{$uv \in E$}
            \If{$d[i][v] > d[i-1][u] + c_{uv}$}
                \State $d[i][v] \gets d[i-1][u] + c_{uv}$
                \State $\pred[i][v] \gets u$
            \EndIf
        \EndFor
    \EndFor
    \State \Return $(d, \pred)$
\EndFunction
\end{algorithmic}
\end{small}
Časovna zahtevnost algoritma je $O(tm)$,
kjer je $m$ število povezav grafa.
Da rekonstruiramo najkrajšo pot od $s$ do vozlišča $u \in V$
z največ $t$ povezavami,
poiščemo tako zaporedje $(v_0, v_1, \dots, v_t)$,
da velja $v_t = u$ in $v_{i-1} = \pred[i][v_i]$ ($1 \le i \le t$).
Ustrezna pot je potem $v_0 v_1 \dots v_h$,
kjer je $h$ najmanjši tak indeks, da velja $v_h = v_{h+1} = \dots = v_t$.

\item Potek izvajanja zgornjega algoritma je prikazan v tabeli~\tab,
kjer so prikazani le koraki,
ko algoritem najde krajšo pot do vozlišča
v primerjavi s prejšnjim obhodom glavne zanke.
Iz tabele lahko rekonstruiramo sledeče najkrajše poti z največ $4$ povezavami:
\begin{itemize}
\item $a$, dolžina $0$
\item $ab$, dolžina $2$
\item $abc$, dolžina $3$
\item $abcd$, dolžina $4$
\item $abcde$, dolžina $5$
\item $aef$, dolžina $8$
\end{itemize}
\end{enumerate}
%
\begin{tabela}
$$
\begin{array}{c|cccccc}
i & a & b & c & d & e & f \\ \hline
0 & 0 &&&&& \\
1 && 2_a &&& 7_a & \\
2 &&& 3_b &&& 8_e \\
3 &&&& 4_c && \\
4 &&&&& 6_d &
\end{array}
$$
\podnaslov[\res{}(b)]{Potek reševanja}
\end{tabela}
\end{odgovor}
\end{naloga}
