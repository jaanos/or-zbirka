\begin{naloga}{Janoš Vidali}{Izpit OR 10.7.2017}
\begin{vprasanje}
Proti računalniškemu programu igraš Texas hold 'em poker.
Pravila igre tukaj niso pomembna.
Ker imaš dostop do kode programa, poznaš logiko, po kateri se ravna.
V trenutni igri si vložil $30$ žetonov, enako tudi nasprotnik.
Nasprotnik z verjetnostjo $0.6$ meni, da so odprte karte ugodnejše zate,
z verjetnostjo $0.4$ pa, da so ugodnejše zanj
(sam si ne ustvariš nobenega mnenja).
V prvem primeru je verjetnost, da so dejansko tvoje karte boljše, enaka $0.8$,
v drugem pa le $0.1$.

Nasprotnik se bo sedaj odločil, ali naj vloži še $10$ žetonov.
Sam se lahko nato odločiš, ali boš vložil $0$, $10$ ali $20$ žetonov
(skupni vložek bo torej $30$, $40$ ali $50$ žetonov).
Če je tvoj vložek manjši od nasprotnikovega,
je igra izgubljena in izgubiš do sedaj vloženo.
Če je tvoj vložek enak na\-sprot\-ni\-ko\-ve\-mu,
z nasprotnikom pogledata karte in tako določita zmagovalca.
Če je tvoj vložek višji od nasprotnikovega,
ima ta možnost odstopiti (tako pridobiš nasprotnikov vložek),
ali pa izenačiti, nakar se zmagovalec določi na podlagi kart.
Če zmagaš, pridobiš nasprotnikov vložek,
če izgubiš, pa izgubiš svojega.

V spodnji tabeli so zbrane verjetnosti dogodkov
v odvisnosti od nasprotnikovega mnenja glede kart.
Verjetnosti navedenih dogodkov pri istem mnenju so med seboj neodvisne.
\begin{center}
\makebox[\textwidth][c]{
\begin{small}
\begin{tabular}{r|cc}
dogodek $\setminus$ nasprotnikovo mnenje
& ugodnejše karte zate & za nasprotnika \\ \hline
dejansko imaš boljše karte & 0.8 & 0.1 \\
nasprotnik vloži $10$ žetonov po razkritju karte & 0.3 & 0.8 \\
nasprotnik izenači skupni vložek $40$ žetonov & 0.2 & 0.7 \\
nasprotnik izenači skupni vložek $50$ žetonov & 0.1 & 0.8
\end{tabular}
\end{small}
}
\end{center}
Na primer:
\begin{multline*}
\Pr[\text{nasprotnik izenači skupni vložek $40$ žetonov} \ | \\
| \ \text{nasprotnikovo mnenje je ``ugodnejše karte zate''}] = 0.2 .
\end{multline*}

Kakšne bodo tvoje odločitve,
da bo tvoj pričakovani dobiček po koncu igre čim večji?
Nariši odločitveno drevo
in odločitve sprejmi na podlagi izračunanih verjetnosti.
Pričakovani dobiček tudi izračunaj.
\end{vprasanje}

\begin{odgovor}
Naj bo $A$ dogodek, da nasprotnik na začetku vloži $10$ žetonov,
$B_n$ dogodek, da odgovorimo z $n$ žetoni ($n \in \{0, 10, 20\}$),
$C$ dogodek, da nasprotnik za tem izenači,
in $D$ dogodek, da dobimo igro.
Izračunajmo ustrezne pogojne verjetnosti.

\begin{alignat*}{2}
P(A) &= 0.6 \cdot 0.3 + 0.4 \cdot 0.8 &&= 0.5 \\
P(\overline{A}) &= 0.6 \cdot 0.7 + 0.4 \cdot 0.2 &&= 0.5 \\
P(C \;|\; A \cap B_{20})
&= {0.6 \cdot 0.3 \cdot 0.1 + 0.4 \cdot 0.8 \cdot 0.8 \over 0.5} &&= 0.548 \\
P(\overline{C} \;|\; A \cap B_{20})
&= {0.6 \cdot 0.3 \cdot 0.9 + 0.4 \cdot 0.8 \cdot 0.2 \over 0.5} &&= 0.452 \\
P(C \;|\; \overline{A} \cap B_{10})
&= {0.6 \cdot 0.7 \cdot 0.2 + 0.4 \cdot 0.2 \cdot 0.7 \over 0.5} &&= 0.28 \\
P(\overline{C} \;|\; \overline{A} \cap B_{10})
&= {0.6 \cdot 0.7 \cdot 0.8 + 0.4 \cdot 0.2 \cdot 0.3 \over 0.5} &&= 0.72 \\
P(C \;|\; \overline{A} \cap B_{20})
&= {0.6 \cdot 0.7 \cdot 0.1 + 0.4 \cdot 0.2 \cdot 0.8 \over 0.5} &&= 0.212 \\
P(\overline{C} \;|\; \overline{A} \cap B_{20})
&= {0.6 \cdot 0.7 \cdot 0.9 + 0.4 \cdot 0.2 \cdot 0.2 \over 0.5} &&= 0.788 \\
P(D \;|\; A \cap B_{10})
&= {0.6 \cdot 0.3 \cdot 0.8 + 0.4 \cdot 0.8 \cdot 0.1 \over 0.5} &&= 0.352 \\
P(\overline{D} \;|\; A \cap B_{10})
&= {0.6 \cdot 0.3 \cdot 0.2 + 0.4 \cdot 0.8 \cdot 0.9 \over 0.5} &&= 0.648 \\
P(D \;|\; A \cap B_{20} \cap C)
&= {0.6 \cdot 0.3 \cdot 0.1 \cdot 0.8 + 0.4 \cdot 0.8 \cdot 0.8 \cdot 0.1
\over 0.5 \cdot 0.548} &&\approx 0.146 \\
P(\overline{D} \;|\; A \cap B_{20} \cap C)
&= {0.6 \cdot 0.3 \cdot 0.1 \cdot 0.2 + 0.4 \cdot 0.8 \cdot 0.8 \cdot 0.9
\over 0.5 \cdot 0.548} &&\approx 0.854 \\
P(D \;|\; \overline{A} \cap B_0)
&= {0.6 \cdot 0.7 \cdot 0.8 + 0.4 \cdot 0.2 \cdot 0.1 \over 0.5} &&= 0.688 \\
P(\overline{D} \;|\; \overline{A} \cap B_0)
&= {0.6 \cdot 0.7 \cdot 0.2 + 0.4 \cdot 0.2 \cdot 0.9 \over 0.5} &&= 0.312 \\
P(D \;|\; \overline{A} \cap B_{10} \cap C)
&= {0.6 \cdot 0.7 \cdot 0.2 \cdot 0.8 + 0.4 \cdot 0.2 \cdot 0.7 \cdot 0.1
\over 0.5 \cdot 0.28} &&= 0.52 \\
P(\overline{D} \;|\; \overline{A} \cap B_{10} \cap C)
&= {0.6 \cdot 0.7 \cdot 0.2 \cdot 0.2 + 0.4 \cdot 0.2 \cdot 0.7 \cdot 0.9
\over 0.5 \cdot 0.28} &&= 0.48 \\
P(D \;|\; \overline{A} \cap B_{20} \cap C)
&= {0.6 \cdot 0.7 \cdot 0.1 \cdot 0.8 + 0.4 \cdot 0.2 \cdot 0.8 \cdot 0.1
\over 0.5 \cdot 0.212} &&\approx 0.377 \\
P(\overline{D} \;|\; \overline{A} \cap B_{20} \cap C)
&= {0.6 \cdot 0.7 \cdot 0.1 \cdot 0.2 + 0.4 \cdot 0.2 \cdot 0.8 \cdot 0.9
\over 0.5 \cdot 0.212} &&\approx 0.623
\end{alignat*}
S pomočjo zgoraj izračunanih verjetnosti
lahko narišemo odločitveno drevo s slike~\fig.
Opazimo, da se nam v vsakem primeru izplača višati za $10$ žetonov
(tj., če nasprotnik na začetku vloži $10$ žetonov,
odgovorimo z $20$ žetoni).
Pričakovani dobiček je $10.364$ žetonov.

\begin{slika}
\makebox[\textwidth][c]{
\pgfslika
}
\podnaslov{Odločitveno drevo}
\end{slika}
\end{odgovor}
\end{naloga}
