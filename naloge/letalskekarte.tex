\begin{naloga}{Blaž Jelenc}{Kolokvij OR 17.4.2014}
\begin{vprasanje}
Na letalu je $100$ sedežev.
Dane so naslednje verjetnosti za število potnikov,
ki ne pridejo na vkrcanje,
pri čemer je $P(i)$ verjetnost, da natanko $i$ potnikov ne pride na vkrcanje:
$$
P(0) = 0.25, \quad P(1) = 0.5, \quad P(2) = 0.25 .
$$
\begin{enumerate}[(a)]
\item Koliko letalskih kart naj proda letalska družba,
če vsaka prodana karta prinese $100 €$ dobička,
vsak nezadovoljen potnik, ki ne dobi sedeža, pa $200 €$ izgube?

\item Za $200 €$ lahko izvedemo predhodno analizo,
ki nam napove število potnikov, ki jih ne bo na vkrcanju.
Zanesljivost analize je podana z verjetnostmi
$$
(p_{ij})_{i,j=0}^2 = \begin{bmatrix}
0.7 & 0.1 & 0 \\
0.2 & 0.8 & 0.1 \\
0.1 & 0.1 & 0.9 \\
\end{bmatrix} ,
$$
kjer je $p_{ij}$ verjetnost,
da analiza napove odpoved $i$ potnikov v primeru,
ko imamo dejansko $j$ odpovedi.
Ali naj letalska družba izvede analizo pred prodajo kart?
\end{enumerate}
\end{vprasanje}

\begin{odgovor}
\begin{enumerate}[(a)]
\item Naj bo $X_k$ slučajna spremenljivka,
ki označuje dobiček pri prodaji $k$ kart ($k \in \{100, 101, 102\}$).
Izračunajmo pričakovane dobičke
\begin{alignat*}{2}
E(X_{100}) &= 100 \cdot 100 € &&= 10\,000 € \\
E(X_{101}) &= 101 \cdot 100 € - 0.25 \cdot 200 € &&= 10\,050 € \\
E(X_{102}) &= 102 \cdot 100 € - (0.25 \cdot 2 + 0.5) \cdot 200 € &&= 10\,000 €
\end{alignat*}
Letalska družba naj torej proda $101$ karto.

\item Naj bo $A_i$ dogodek, da analiza napove odpoved $i$ potnikov,
in $B_j$ dogodek, da imamo dejansko $j$ dopovedi ($i, j = 0, 1, 2$).
Poznamo pogojne verjetnosti $P(A_i \;|\; B_j) = p_{ij}$.
Izračunajmo najprej verjetnosti $P(A_j)$.
\begin{alignat*}{2}
P(A_0) &= 0.7 \cdot 0.25 + 0.1 \cdot 0.5 + 0 \cdot 0.25 &&= 0.225 \\
P(A_1) &= 0.2 \cdot 0.25 + 0.8 \cdot 0.5 + 0.1 \cdot 0.25 &&= 0.475 \\
P(A_2) &= 0.1 \cdot 0.25 + 0.1 \cdot 0.5 + 0.9 \cdot 0.25 &&= 0.3
\end{alignat*}
Izračunajmo še verjetnosti posameznega števila odpovedi
ob različnih izidih analize.
\begin{alignat*}{2}
P(B_0 \;|\; A_0) &= {0.7 \cdot 0.25 \over 0.225} &&= {7 \over 9} \\
P(B_1 \;|\; A_0) &= {0.1 \cdot 0.5 \over 0.225} &&= {2 \over 9} \\
P(B_2 \;|\; A_0) &= {0 \cdot 0.25 \over 0.225} &&= 0 \\
P(B_0 \;|\; A_1) &= {0.2 \cdot 0.25 \over 0.475} &&= {2 \over 19} \\
P(B_1 \;|\; A_1) &= {0.8 \cdot 0.5 \over 0.475} &&= {16 \over 19} \\
P(B_2 \;|\; A_1) &= {0.1 \cdot 0.25 \over 0.475} &&= {1 \over 19} \\
P(B_0 \;|\; A_2) &= {0.1 \cdot 0.25 \over 0.3} &&= {1 \over 12} \\
P(B_1 \;|\; A_2) &= {0.1 \cdot 0.5 \over 0.3} &&= {1 \over 6} \\
P(B_2 \;|\; A_2) &= {0.9 \cdot 0.25 \over 0.3} &&= {3 \over 4}
\end{alignat*}
Sedaj lahko izračunamo pričakovane dobičke ob različnih izidih analize.
\begin{alignat*}{2}
E(X_{100} \;|\; A_0) &= 100 \cdot 100 € &&= \underline{10\,000 €} \\
E(X_{101} \;|\; A_0) &= 101 \cdot 100 € - {7 \over 9} \cdot 200 €
&&\approx 9\,944.44 € \\
E(X_{102} \;|\; A_0) &= 102 \cdot 100 €
- \left({7 \over 9} \cdot 2 + {2 \over 9}\right) \cdot 200 €
&&\approx 9\,844.44 € \\
E(X_{100} \;|\; A_1) &= 100 \cdot 100 € &&= 10\,000 € \\
E(X_{101} \;|\; A_1) &= 101 \cdot 100 € - {2 \over 19} \cdot 200 €
&&\approx \underline{10\,078.95 €} \\
E(X_{102} \;|\; A_1) &= 102 \cdot 100 €
- \left({2 \over 19} \cdot 2 + {16 \over 19}\right) \cdot 200 €
&&\approx 9\,989.47 € \\
E(X_{100} \;|\; A_2) &= 100 \cdot 100 € &&= 10\,000 € \\
E(X_{101} \;|\; A_2) &= 101 \cdot 100 € - {1 \over 12} \cdot 200 €
&&\approx 10\,083.33 € \\
E(X_{102} \;|\; A_2) &= 102 \cdot 100 €
- \left({1 \over 12} \cdot 2 + {1 \over 6}\right) \cdot 200 €
&&\approx \underline{10\,133.33 €}
\end{alignat*}
Najboljše izbire v vsakem primeru so podčrtane.
Ker v nobenem primeru pričakovan dobiček
ne preseže pričakovanega dobička brez analize za več kot $200 €$,
se nam torej analize ne izplača izvesti.
\end{enumerate}
\end{odgovor}
\end{naloga}
