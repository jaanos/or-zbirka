\begin{naloga}{?}{Izpit OR 26.6.2012}
\begin{vprasanje}
Nori profesor Boltežar stanuje v stolpnici z $n$ nadstropji,
oštevilčenimi od $1$ do $n$.
Nori stanovalci tega bloka radi mečejo cvetlične lončke z balkonov.
Boltežar bi rad ugotovil, katero je najvišje nadstropje,
s katerega lahko pade cvetlični lonček, ne da bi se razbil.
Jasno je, da če se lonček razbije pri padcu iz $k$-tega nadstropja,
potem se razbije tudi pri padcu s $(k+1)$-tega nadstropja.
Če bi Boltežar imel le en cvetlični lonček,
bi ga lahko metal po vrsti od najnižjega nadstropja navzgor,
dokler se ne bi razbil.
V najslabšem primeru bi lonček torej vrgel $n$ krat
(možno je, da bi lonček preživel tudi padec iz najvišjega nadstropja).

Ker ima Boltežar doma $k$ cvetličnih lončkov,
lahko do rezultata pride tudi z manjšim številom metov.
S pomočjo dinamičnega programiranja bi rad po\-is\-kal strategijo metanja,
ki bi minimizirala število potrebnih metov v najslabšem primeru.
\begin{enumerate}[(a)]
\item Napiši rekurzivne enačbe za opisani problem.
\item Napiši algoritem, ki reši opisani problem.
Oceni tudi njegovo časovno zahtevnost v odvisnosti od $n$ in $k$.
\end{enumerate}

\end{vprasanje}
\begin{odgovor}
\end{odgovor}
\end{naloga}
