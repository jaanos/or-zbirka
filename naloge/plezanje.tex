\begin{naloga}{David Gajser}{Izpit OR 28.8.2013}
\begin{vprasanje}
Za šestimi gorami poteka tekmovanje v plezanju na goro Lep Trikotnik.
Vsak odsek te gore ima prirejeno težavnost od $-4$ do $4$.
Goro lahko ponazorimo s številskim trikotnikom
$$
\begin{array}{ccccccccc}
                &&&& a_{11} \\
            &&& a_{21} && a_{22} \\
        && a_{31} && a_{32} && a_{33} \\
     & \iddots && \vdots && \vdots && \ddots \\
a_{n1} && a_{n2} && \cdots && a_{n,n-1} && a_{nn}
\end{array} ,
$$
kjer so $a_{ij} \in \{-4, -3, \dots, 3, 4\}$ težavnosti odsekov.
{\em Veljavna pot} v takem trikotniku je pot od vznožja do vrha,
ki gre zmeraj desno navzgor ali levo navzgor
-- t.j., zaporedje $(a_{n-i, j_i})_{i=0}^{n-1}$,
kjer je $1 \le j_i \le n-i$ in $j_i \in \{j_{i-1}-1, j_{i-1}\}$
za vsak $i = 0, 1, \dots, n-1$.
{\em Teža} veljavne poti je vsota težavnosti odsekov,
po katerih poteka pot (torej $\sum_{i=0}^{n-1} a_{n-i, j_i}$).

Na tekmovanje sta se med drugimi prijavila tudi gornika Marin in Mirko.
V okviru svojih sposobnosti želita izbrati kar najtežjo pot do vrha.
Oba zmoreta plezati po odsekih težavnosti $3$ ali manj.

\begin{enumerate}[(a)]
\item Marin ni dovolj izkušen, da bi lahko plezal po odsekih težavnosti $4$.
Opiši, kako lahko najdemo najtežjo veljavno pot za Marina,
oz.~mu povemo, da s svojim znanjem ne more priti do vrha po veljavni poti!

\item Gornik Mirko je izkušenejši od Marina.
Odseke težavnosti $4$ lahko prepleza,
a ne more preplezati dveh zaporednih odsekov te težavnosti.
Opiši, kako lahko najdemo najtežjo veljavno pot za Mirka,
oz.~mu povemo, da s svojim znanjem ne more priti do vrha po veljavni poti!
\end{enumerate}
\end{vprasanje}

\begin{odgovor}
Definirajmo
$$
a'_{ij} = \begin{cases}
-\infty & a_{ij} = 4, \text{in} \\
a_{ij} & \text{sicer} .
\end{cases} \quad (1 \le j \le i \le n)
$$
\begin{enumerate}[(a)]
\item Naj bo $p_{ij}$ največja teža za Marina veljavne poti
od vznožja do $a_{ij}$.
Določimo začetne pogoje in rekurzivne enačbe.
\begin{align*}
p_{nj} &= a'_{nj} && (1 \le j \le n) \\
p_{ij} &= a'_{ij} + \max\{p_{i+1,j}, p_{i+1,j+1}\} && (1 \le j \le i \le n-1)
\end{align*}
Vrednosti $p_{ij}$ računamo
v padajočem leksikografskem vrstnem redu glede na $(i, j)$.
Če Marin s svojim znanjem ne more priti do vrha po veljavni poti,
potem velja $p_{11} = -\infty$,
sicer pa težo najtežje veljavne poti dobimo kot $p^* = p_{11}$.
Samo pot $(a_{n-i, j_i})_{i=0}^{n-1}$ lahko dobimo tako,
da izračunamo
\begin{align*}
j_{n-1} &= 1 \\
j_i &= \begin{cases}
j_{i+1} & p_{n-i,j_{i+1}} \ge p_{n-i,j_{i+1}+1}, \text{in} \\
j_{i+1} + 1 & \text{sicer}
\end{cases}
& (0 \le i \le n-2)
\end{align*}
v padajočem vrstnem redu glede na indeks $i$.

\item Naj bosta $q_{ij}$ in $r_{ij}$
največji teži za Mirka veljavne poti od vznožja do $a_{ij}$,
če dovolimo $a_{ij} = 4$ oziroma če tega ne dovolimo.
Določimo začetne pogoje in rekurzivne enačbe.
\begin{align*}
q_{nj} &= a_{nj} && (1 \le j \le n) \\
r_{nj} &= a'_{nj} && (1 \le j \le n) \\
q_{ij} &= a_{ij} + \max\{r_{i+1,j}, r_{i+1,j+1}\} && (1 \le j \le i \le n-1) \\
r_{ij} &= a'_{ij} + \max\{q_{i+1,j}, q_{i+1,j+1}, r_{i+1,j}, r_{i+1,j+1}\}
&& (1 \le j \le i \le n-1)
\end{align*}
Vrednosti $q_{ij}$ in $r_{ij}$ računamo
v padajočem leksikografskem vrstnem redu glede na $(i, j)$.
Če Mirko s svojim znanjem ne more priti do vrha po veljavni poti,
potem velja $q_{11} = r_{11} = -\infty$,
sicer pa težo najtežje veljavne poti
dobimo kot $q^* = \max\{q_{11}, r_{11}\}$.
Samo pot $(a_{n-i, j_i})_{i=0}^{n-1}$ lahko dobimo tako,
da izračunamo
\begin{multline*}
\begin{aligned}
j_{n-1} &= 1 \\
j_i &= \begin{cases}
j_{i+1} & a_{n-i-1,j_{i+1}} < 4 \land {} \\[-1mm]
&\quad r_{n-i-1,j_{i+1}} - a_{n-i-1,j_{i+1}} \in \{q_{n-i,j_{i+1}}, r_{n-i,j_{i+1}}\}, \\
j_{i+1} & a_{n-i-1,j_{i+1}} = 4 \land r_{n-i,j_{i+1}} \ge r_{n-i,j_{i+1}+1},
\text{in} \\
j_{i+1} + 1 & \text{sicer}
\end{cases}
\end{aligned} \\
(0 \le i \le n-2)
\end{multline*}
v padajočem vrstnem redu glede na indeks $i$.
\end{enumerate}
\end{odgovor}
\end{naloga}
