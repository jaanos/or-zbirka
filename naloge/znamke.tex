\begin{naloga}{Janoš Vidali}{Izpit OR 28.8.2024}
\begin{vprasanje}
Tilnov hobi je zbiranje redkih znamk.
V svojo zbirko bi rad dodal znamko $t$.
Pri svojih prijateljih filatelistih je poizvedel,
katere znamke so pripravljeni zamenjati ob doplačilu
in tako sestavil seznam možnih menjav
(npr.,
Ana je pripravljena Tilnu dati svojo znamko $a$
v zameno za znamko $c$ in doplačilo $x$,
Boris je pri\-prav\-ljen Tilnu dati svojo znamko $b$
v zameno za znamko $d$, pri čemer Tilnu plača znesek $y$, itd.).
Tilen ima znamko $s$,
ki jo je pripravljen zamenjati
in potem dobljene znamke zamenjevati naprej,
dokler ne dobi znamke $t$
(ostalih znamk iz svoje zbirke pa ni pri\-prav\-ljen zamenjati).
Zanima ga, kakšno zaporedje menjav naj izvede,
da bo pri tem plačal čim manj oziroma čim več zaslužil.

\begin{enumerate}[(a)]
\item Predstavi problem v jeziku grafov
in predlagaj čim bolj učinkovit algoritem za reševanje zgornjega problema.
Kako naj Tilen ravna glede na izid algoritma?
Kak\-šna je njegova časovna zahtevnost?

\item S svojim algoritmom reši problem na grafu s slike~\fig.
Iz odgovora naj bo jasen potek algoritma
(tj., za vsako spremembo naj bo jasno, v katerem koraku se zgodi).
\end{enumerate}

\begin{slika}
\pgfslika
\podnaslov{Graf}
\end{slika}
\end{vprasanje}

\begin{odgovor}
\begin{enumerate}[(a)]
\item Definirajmo usmerjen graf $G$,
kjer vsako znamko predstavimo z vozliščem,
za vsako možno zamenjavo tipa
``nekdo je pripravljen Tilnu dati svojo znamko $u$
v zameno za znamko $v$ ob doplačilu $w$''
pa v grafu postavimo povezavo od $u$ do $v$ z utežjo $w$
(če ob zamenjavi Tilen zasluži,
to smatramo kot negativno doplačilo).
V grafu $G$ potem iščemo najcenejšo pot od $s$ do $t$.

Ker so lahko uteži na povezavah grafa $G$ tudi negativne,
za iskanje najkrajše poti uporabimo Bellman-Fordov algoritem.
Ta nam lahko vrne najcenejšo pot,
ki predstavlja iskano zaporedje menjav znamk.
Če pot od $s$ do $t$ ne obstaja,
potem tudi ustrezno zaporedje menjav ne obstaja.
Če pa algoritem zazna obstoj negativnega cikla,
to pomeni, da lahko Tilen z menjavami na tem ciklu zasluži poljubno
(pri čemer znamke na ciklu zamenjajo lastnika,
tako da moramo za to predpostaviti,
da bodo novi lastniki sprejeli iste menjave).

Časovna zahtevnost algoritma je $O((m + n) n)$,
kjer je $m$ število povezav v grafu (tj., število možnih menjav),
$n$ pa število vozlišč (tj., število znamk, ki jih obravnavamo).
Da bi algoritem tekel čim hitreje,
lahko za vsak urejen par vozlišč v grafu ohranimo le eno povezavo
(če ta obstaja),
in sicer tisto z najmanjšo utežjo.
V primeru, ko so vse uteži nenegativne,
lahko uporabimo tudi Dijkstrov algoritem,
ki teče v času $O(n^2)$ oziroma $O(m \log n)$.

\item Potek izvajanja zgoraj opisanega algoritma je prikazan v tabeli~\tab,
iz katere razberemo,
da je najcenejša pot $s - g - c - a - t$,
njena cena pa je $3 + (-4) + 6 + 1 = 6$.
\end{enumerate}
%
\begin{tabela}
$$
\begin{array}{c|ccccccccc}
i & s & a & b & c & d & e & f & g & t \\ \hline
0 & 0 &&&&&&&& \\
1 &&&&& 7_s &&& 3_s & 10_s \\
2 &&&& -1_g && 2_d &&& \\
3 && 5_c &&&&& 1_e && \\
4 &&& 11_a &&&&&& 6_a \\
5 &&&&&&&&&
\end{array}
$$
\podnaslov[\res{}(b)]{Potek izvajanja algoritma}
\end{tabela}
\end{odgovor}
\end{naloga}
