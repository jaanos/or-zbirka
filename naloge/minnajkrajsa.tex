\begin{naloga}{Janoš Vidali}{Kolokvij OR 5.6.2024}
\begin{vprasanje}
Dan je povezan neusmerjen graf $G = (V, E)$
s cenami povezav $t_{uv} \in \R$ ($uv \in E$)
ter njegovo vozlišče $r \in V$.
Za vsako vozlišče $u \in V$ želimo izmed najkrajših poti med $r$ in $u$
(tj., izmed poti z najmanjšim številom povezav)
poiskati tako, ki ima najmanjšo ceno
(kjer je cena poti enaka vsoti cen njenih povezav).
Pri reševanju problema si bomo pomagali z iskanjem v širino.

\begin{enumerate}[(a)]
\item Zapiši rekurzivne enačbe za določitev cen iskanih poti ter pojasni,
v katerem vrst\-nem redu računamo spremenljivke,
ter kako iz danih podatkov pridemo do tega vrstnega reda.

\item Natančno opiši algoritem (z besedami ali psevdokodo),
ki s pomočjo iskanja v širino
(metoda {\sc bfs} -- ni potrebno opisovati njenega delovanja)
in zgornjih rekurzivnih enačb reši dani problem.
Kakšna je časovna zahtevnost tvojega algoritma?
Razloži še, kako lahko iz dobljenih vrednosti
določimo iskano pot od $r$ do izbranega vozlišča $u$.

\item Uporabi svoj algoritem,
da rešiš problem za graf $G = (V, E)$ s slike~\fig
z začetkom v vozlišču $a \in V$
(tj., za vsako vozlišče grafa določi najcenejšo izmed najkrajših poti iz $a$).
Vrednosti $t_{uv}$ ($uv \in E$) so prikazane pri povezavah.

\item Denimo,
da namesto samo med najkrajšimi potmi iz vozlišča $r$
za vsako vozlišče $u \in V$ iščemo najcenejšo pot od $r$ do $u$ izmed poti,
ki sestojijo iz največ $d(r, u) + 1$ povezav
(kjer je $d(r, u)$ razdalja med $r$ in $u$ v grafu $G$ brez upoštevanja uteži,
tj., najmanjše število povezav v poteh med $r$ in $u$).
Iščemo torej najcenejše poti od $r$ do ostalih vozlišč,
ki imajo kvečjemu eno povezavo več od najkrajše poti.
Dopolni svoje rekurzivne enačbe tako, da bodo reševale ta problem.
\namig{definiraj nove spremenljivke
za obravnavo poti z eno povezavo več od najkrajše poti do vsakega vozlišča.}
\end{enumerate}
%
\begin{slika}
\pgfslika
\podnaslov[\nal{}(c)]{Graf}
\end{slika}
\end{vprasanje}

\begin{odgovor}
\begin{enumerate}[(a)]
\item Definirajmo spremenljivko $x_v$ ($v \in V$)
kot najmanjšo ceno najkrajše poti od $r$ do $v$ v grafu $G$.
Zapišimo začetni pogoj in rekurzivne enačbe.
\begin{align*}
x_r &= 0 \\
x_v &= \min\{x_u + t_{uv} \mid u \sim v, d(r, u) < d(r, v)\} & (v \in V \setminus \{r\})
\end{align*}
Tukaj smo z $d(u, v)$ označili razdaljo med vozliščema $u$ in $v$ v grafu $G$.
Vred\-no\-sti $x_v$ ($v \in V$) računamo naraščajoče po vrednosti $d(r, v)$
-- v tem vrstnem redu obiskujemo vozlišča pri iskanju v širino
z začetkom v vozlišču $r$.

\item Zapišimo algoritem, ki kliče funkcijo {\sc bfs} iz naloge~\res[bfs].
\needspace{3\baselineskip}
\begin{small}
\begin{algorithmic}
\Function{NajcenejšeNajkrajšePoti}{$G = (V, E), r, t : E \to \R$}
	\State $x \gets$ prazen slovar
	\State $d \gets$ prazen slovar
	\State $\pred \gets$ prazen slovar
	\Function{Visit}{$v, w$}
		\If{$w = \Null$} \hfill ni predhodnika, $v = r$
			\State $d[v] \gets 0$
			\State $x[v] \gets 0$
			\State $\pred[v] \gets \Null$
		\Else \hfill uporabimo rekurzivno enačbo
			\State $d[v] \gets d[w] + 1$
			\State $x[v] \gets \infty$
			\For{$u \in \Adj(G, v)$}
				\If{$u \in d \land d[u] < d[v]$}
					\State $c \gets x[u] + t_{uv}$
					\If{$c < x[v]$}
						\State $x[v] \gets c$
						\State $\pred[v] \gets u$
					\EndIf
				\EndIf
			\EndFor
		\EndIf
	\EndFunction
	\State {\sc bfs}$(G, \{r\}, \visit)$
	\State \Return $(x, \pred)$
\EndFunction
\end{algorithmic}
\end{small}
Funkcija $\visit$ se kliče na vsakem vozlišču enkrat
in tako pregleda vsakega soseda vsakega vozlišča.
Časovna zahtevnost algoritma je tako $O(m)$,
kjer je $m$ število povezav v grafu.
Iskano pot od $r$ do izbranega vozlišča $u$
lahko določimo s klicem funkcije {\sc RekonstruirajPot}$(\pred, u)$
iz naloge~\res[topo].

\item Izračunajmo vrednosti $x_v$ in $\pred_v$ ($v \in V \setminus \{a\}$).
\begin{alignat*}{4}
x_c &= 0 + 3 &&= 3 &\qquad \pred_c &= a \\
x_d &= 0 + 5 &&= 5 &\qquad \pred_d &= a \\
x_b &= 3 + 2 &&= 5 &\qquad \pred_b &= c \\
x_f &= 3 + 4 &&= 7 &\qquad \pred_f &= c \\
x_g &= \min\{3+2, \underline{5 + (-1)}\} &&= 4 &\qquad \pred_g &= d \\
x_e &= \min\{\underline{5+3}, 7+7\} &&= 8 &\qquad \pred_e &= b \\
x_h &= 7 + 9 &&= 16 &\qquad \pred_h &= f \\
x_i &= \min\{7+10, \underline{4 + (-1)}\} &&= 3 &\qquad \pred_i &= g
\end{alignat*}
Drevo najcenejših najkrajših poti je prikazano na sliki~\fig[minnajkrajsa-resitev].
%
\begin{slika}
\pgfslika[minnajkrajsa-resitev]
\podnaslov[\nal{}(c)]{Drevo najcenejših najkrajših poti}
\end{slika}

\item Definirajmo še spremenljivki $y_v$ in $z_v$ ($v \in V$)
kot najmanjšo ceno poti od $r$ do $v$ v grafu $G$
z eno povezavo več od najkrajše poti
ter kot najmanjšo ceno poti od $r$ do $v$ v grafu $G$
z največ eno povezavo več od najkrajše poti.
Zapišimo začetni pogoj in rekurzivne enačbe.
\begin{alignat*}{3}
y_r &= \infty \\
y_v &= \min&\big(&\{y_u + t_{uv} \mid u \sim v, d(r, u) < d(r, v)\} \\
&&\cup &\{x_u + t_{uv} \mid u \sim v, d(r, u) = d(r, v)\}\big) &\quad (v &\in V \setminus \{r\}) \\
z_v &= \min&&\{x_v, y_v\} &\quad (v &\in V)
\end{alignat*}
Najprej izračunamo spremenljivke $x_v$ ($v \in V$) kot prej,
nato pa v istem vrst\-nem redu še spremenljivke $y_v$ in $z_v$.

\end{enumerate}
\end{odgovor}
\end{naloga}
