\begin{naloga}[problem prevoza in skladiščenja dobrin]{?}{Vaje OR 16.3.2016}
\begin{vprasanje}
V Evropski uniji je na voljo $n$ skladišč,
pri čemer znašajo stroški najema $i$-tega skladišča $f_i$
(ne glede na zasedenost),
vsako skladišče pa lahko hrani enoto dobrine.
Imamo $m$ strank, ki jim dostavljamo dobrine,
pri čemer $c_{ij}$ ($1 \le i \le n$, $1 \le j \le m$)
predstavlja strošek dostave dobrine stranki $j$ iz skladišča $i$.
Predpostavimo tudi, da ima vsaka stranka določeno potrebo $d_j$,
ki ponazarja število enot dobrine, ki jo potrebuje.
V katerih skladiščih naj hranimo dobrine,
da bodo skupni stroški najema in dostave čim manjši?
Na smiseln način modeliraj opisani problem z linearnim programom.
\end{vprasanje}

\begin{odgovor}
Za vsako skladišče, ki ga uporabimo,
želimo vedeti, kateri stranki bomo dostavili dobrino.
Za $i$-to skladišče ($1 \le i \le n$) in $j$-to stranko ($1 \le j \le m$)
bomo uvedli spremenljivko $x_{ij}$,
katere vrednost interpretiramo kot
$$
x_{ij} = \begin{cases}
1, & \text{če bomo iz $i$-tega skladišča dostavili $j$-ti stranki, in} \\
0  & \text{sicer.}
\end{cases}
$$
Zapišimo celoštevilski linearni program.
\begin{alignat*}{2}
\min \ \sum_{i=1}^n \sum_{j=1}^m (f_i + c_{ij}) x_{ij} && \text{p.p.} \\
\forall i \in \{1, \dots, n\} \ \forall j \in \{1, \dots, m\}: &\ &
0 \le x_{ij} &\le 1, \quad x_{ij} \in \Z
\opis{Vsako skladišče uporabimo največ enkrat}
\forall i \in \{1, \dots, n\}: &\ & \sum_{j=1}^m x_{ij} &\le 1
\opis{Zadostimo potrebe vsake stranke}
\forall j \in \{1, \dots, m\}: &\ & \sum_{i=1}^n x_{ij} &\ge d_j
\end{alignat*}
\end{odgovor}
\end{naloga}
