\begin{naloga}{Juvan, Orbanić}{Izpit OR 5.2.2010}
\begin{vprasanje}
Neusmerjeno omrežje $G$ s pozitivnimi razdaljami na povezavah
razdelimo na dve disjunktni množici vozlišč $A$ in $B$,
ki pokrivata vsa vozlišča.
Naj bo $e$ najkrajša povezava z enim krajiščem v $A$ in drugim v $B$.
Pokaži ali ovrzi:
\begin{enumerate}[(a)]
\item Vsaka najkrajša pot med kakim vozliščem iz $A$
in kakim vozliščem iz $B$ vsebuje povezavo $e$.
\item Obstaja najkrajša pot med nekima vozliščema v grafu $G$,
ki vsebuje povezavo $e$.
\item Za vsak par vozlišč $a \in A$ in $b \in B$ obstaja najkrajša pot,
ki vsebuje povezavo $e$.
\end{enumerate}
\end{vprasanje}

\begin{odgovor}
Trditvi (a) in (c) sta napačni.
Protiprimer dobimo,
če vzamemo graf s slike~\fig[trditve]
ter množici $A = \{a, c\}$ in $B = \{b\}$.
Potem je $e = bc$,
toda nobena najkrajša pot med vozliščema $a$ in $b$
ne vsebuje povezave $e$.

Trditev (b) je pravilna,
saj lahko vzamemo najkrajšo pot med krajiščema povezave $e$.
Vsaka pot med tema vozliščema, ki ne vsebuje povezave $e$,
namreč vsebuje vsaj eno povezavo, ki je vsaj tako dolga kot $e$,
zaradi pozitivnosti dolžin povezav pa je taka pot
vsaj tako dolga kot pot,
ki sestoji samo iz povezave $e$.
\end{odgovor}
\end{naloga}
