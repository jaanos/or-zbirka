\begin{naloga}{Batagelj, Kaufman}{\cite[Naloga~10.7]{bk}}
\begin{vprasanje}
Tovarna Mitsuzuki načrtuje,
da bo naslednje leto izdelala $10\,000$ stereo televizorjev.
Odločijo se, da zvočnikov ne bodo izdelovali sami,
ampak jih bodo (po dva za vsak televizor) naročili
pri ponudniku najkvalitetnejših zvočnikov,
ki je predložil naslednji cenik:
\begin{center}
\begin{tabular}{c|c}
število zvočnikov v enem naročilu & cena zvočnika \\
\hline
      1 --  1\,999 & $150 €$ \\
 2\,000 --  4\,999 & $135 €$ \\
 5\,000 --  7\,999 & $125 €$ \\
 8\,000 -- 19\,999 & $120 €$ \\
20\,000 ali več    & $115 €$ \\
\end{tabular}
\end{center}
Poleg tega naročilo pošiljke stane Mitsuzuki $500 €$,
letni stroški skladiščenja vsakega zvočnika pa znašajo $20\%$ njegove cene.
Koliko zvočnikov naj vsakič naročijo
in kakšni so skupni stroški naročil za naslednje leto?
\end{vprasanje}

\begin{odgovor}
V tovarni bodo naročili $\nu = 20\,000$ zvočnikov,
pri čemer jih posamezno naročilo stane $K = 500 €$.
Pri naročilu velikosti vsaj $n_i$
je cena posameznega zvočnika enaka $c_i$,
cena skladiščenja pa je $s_i = 0.2 \cdot c_i$ ($1 \le i \le 5$),
pri čemer imamo
$$
\begin{array}{rrrrrrl}
(n_i)_{i=1}^5 = ( & 1, & 2\,000, & 5\,000, & 8\,000, & 20\,000 )
&\quad \text{in} \\
(c_i)_{i=1}^5 = ( & 150 €, & 135 €, & 125 €, & 120 €, & 115 € ) & .
\end{array}
$$
Ker zvočnikov ne proizvajajo sami in primanjkljaj ni dovoljen,
velja $\lambda = p = \infty$ in torej $\alpha = \nu$, $\beta = 1$.

Naj bo $M_i$ optimalna velikost naročila,
če lahko zvočnike kupujemo po ceni $c_i$ ($1 \le i \le 5$).
\begin{gather*}
M_i = \sqrt{2K \alpha \over s_i \beta} \\
(M_i)_{i=1}^5 = (816.497, 860.663, 894.427, 912.871, 932.505)
\end{gather*}
Opazimo, da $M_i \ge n_i$ velja samo pri $i = 1$.
Za velikost optimalnega naročila $M^*_i$ po ceni $c_i$
torej velja $M^*_1 = M_1$ in $M^*_i = n_i$ ($2 \le i \le 5$).
Za naročila velikosti $M^*_i$ ($1 \le i \le 5$)
izračunajmo skupne stroške naročanja in skladiščenja $S_i$.
\begin{gather*}
S_i = {M^*_i s_i \over 2} + K {\nu \over M^*_i} + \nu c_i \\
(S_i)_{i=1}^5 = (3\,024\,495, 2\,732\,000, 2\,564\,000, 2\,496\,250, 2\,530\,500)
\end{gather*}
Opazimo, da najmanjše stroške dosežemo,
če naročamo po $M^* = M^*_4 = 8\,000$ zvočnikov po ceni $c^* = c_4 = 120 €$.
Izračunajmo še interval naročanja.
$$
\tau^* = {M^* \over \nu} = {8\,000 \over 20\,000} = 0.4
$$
Naročamo torej na $0.4$ leta oziroma na $146$ dni.
\end{odgovor}
\end{naloga}
