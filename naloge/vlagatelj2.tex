\begin{naloga}{Janoš Vidali}{Izpit OR 27.8.2019}
\begin{vprasanje}
Vlagatelj ima na voljo $m$ kapitala,
del katerega bo vložil v eno izmed dveh konkurenčnih podjetij,
ki razvijata podobna produkta
(v obe ne more vložiti),
preostanek pa bo namenil za marketinško kampanjo,
ki bo promovirala produkt izbranega podjetja.
Naj bodo $x_1$, $x_2$ in $x_3$ količine denarja
(te so lahko poljubna nenegativna realna števila),
ki jih bo vložil v prvo oziroma drugo podjetje in v marketing,
ter določimo
$$
p_i = \begin{cases}
n_i + k_i x_i, & \text{če $x_i \ge a_i$, in} \\
0 & \text{sicer}
\end{cases}
\qquad (i = 1, 2, 3)
$$
kot faktor,
ki ga ustvari vsako izmed njih.
Pričakovani dobiček se izračuna po formuli $d = (p_1 + p_2) p_3$,
pri čemer bo eden od $p_1$ in $p_2$ enak $0$.
Vlagatelj bi rad sredstva porazdelil tako,
da bo pričakovani dobiček čim večji.

\begin{enumerate}[(a)]
\item Zapiši enačbe za reševanje danega problema.
Lahko predpostaviš,
da velja $p_i \ge 0$ za vse vrednosti $x_i$ ($i = 1, 2, 3$).

\item Z zgoraj zapisanimi enačbami reši problem pri podatkih
$m = 10$, $a_1 = 4$, $n_1 = 8$, $k_1 = 4$, $a_2 = 5$, $n_2 = 12$, $k_2 = 2$,
$a_3 = 3$, $n_3 = 4$ in $k_3 = 1$.
\end{enumerate}
\end{vprasanje}

\begin{odgovor}
\begin{enumerate}[(a)]
\item Naj bo $d_i$ ($i = 1, 2$) optimalni dobiček
pri vlaganju v $i$-to podjetje.
Potem velja
$$
d_i = \begin{cases}
\max\set{(n_i + k_i x)(n_3 + k_3 (m-x))}{a_i \le x \le m-a_3},
& \text{če $a_i \le m-a_3$, in} \\
0 & \text{sicer.}
\end{cases}
$$
Optimalni dobiček potem dobimo kot $d^* = \max\{d_1, d_2\}$.

\item Naj bo $q_i(x)$ ($i = 1, 2$) kvadratni polinom,
ki nastopa v zgornjem izrazu.
Da izračunamo $d_1$ in $d_2$,
bomo poiskali maksimum polinomov $q_1$ in $q_2$
v ustreznih mejah.

Vrednost $d_1$ je maksimalna vrednost izraza
$$
q_1(x) = (8 + 4x)(4 + 10 - x) = -4x^2 + 48x + 112
$$
za $x \in [4, 7]$.
Ker ima polinom negativen vodilni člen,
je njegov globalni maksimum tam, kjer ima njegov odvod ničlo.
\begin{align*}
q'_1(x) = -8x + 48 &= 0 \\
x &= 6
\end{align*}
Ker maksimum leži na iskanem intervalu, velja $d_1 = q_1(6) = 256$.

Vrednost $d_2$ je maksimalna vrednost izraza
$$
q_2(x) = (12 + 2x)(4 + 10 - x) = -2x^2 + 16x + 168
$$
za $x \in [5, 7]$.
Ker ima polinom negativen vodilni člen,
je njegov globalni maksimum tam, kjer ima njegov odvod ničlo.
\begin{align*}
q'_2(x) = -4x + 16 &= 0 \\
x &= 4
\end{align*}
Ker maksimum leži levo od iskanega intervala,
leži maksimum na njegovem levem robu,
torej velja $d_2 = q_2(5) = 198$.

Optimalni dobiček je tako $d^* = \max\{256, 198\} = 256$
in ga dosežemo tako, da vložimo $6$ v prvo podjetje in $4$ v marketing.
\end{enumerate}
\end{odgovor}
\end{naloga}
