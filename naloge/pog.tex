\begin{naloga}{David Gajser}{Izpit OR 24.6.2013}
\begin{vprasanje}
Profesionalni programer je vladi Republike Slovenije
prodal algoritem {\sc PoG}.
A na pristojnem ministrstvu so pozabili, kdo je ta algoritem naročil,
in zato ne vedo, kaj naj bi algoritem vračal.
V prospektu so zasledili naslednjo psevdokodo algoritma:
\begin{small}
\begin{algorithmic}
\Require{neusmerjen graf $G = (V, E)$}
\For{$v \in V$}
    \State obarvaj $v$ sivo
\EndFor
\State $Q \gets$ nova vrsta
\State $s \gets 0$
\For{$v \in V$}
    \If{$v$ je siv}
        \State $s \gets s+1$
        \State $Q.\append(v)$
        \While{$\lnot Q.\isEmpty()$}
            \State $w \gets Q.\pop()$
            \For{$u \in \Adj(G, w)$}
                \If{$u$ je siv}
                    \State $Q.\append(u)$
                    \State obarvaj $u$ rumeno
                \EndIf
            \EndFor
        \EndWhile
    \EndIf
\EndFor
\State \Return{$s$}
\end{algorithmic}
\end{small}

\begin{enumerate}[(a)]
\item Algoritem izvedi na grafu s slike~\fig,
tj., zabeleži vsako spremembo barve nekega vozlišča.
Koliko je takrat vrednost števca?

\item Kaj algoritem vrača?
Odgovor utemelji.

\item Kakšna je časovna zahtevnost algoritma?
Odgovor utemelji.
\end{enumerate}

\begin{slika}
\pgfslika
\podnaslov{Graf}
\end{slika}
\end{vprasanje}

\begin{odgovor}
\begin{enumerate}[(a)]
\item V tabeli~\tab je prikazan vrstni red barvanja vozlišč v rumeno barvo,
pri čemer se obarva vozlišče $v$ ali $u$.
Zabeleženo je tudi, kdaj se spremeni vrednost števca $s$.

\item Algoritem vrača število povezanih komponent grafa.
Izvaja namreč pregled v širino,
pri čemer se števec poveča vsakič, ko naleti na vozlišče,
ki ga ni dosegel po predhodnih povečevanjih števca.

\item Časovna zahtevnost algoritma je $O(n + m)$,
kjer je $n$ število vozlišč, $m$ pa število povezav podanega grafa.
Algoritem namreč z zunanjo zanko obišče vsako vozlišče enkrat,
z notranjo zanko pa vsako povezavo dvakrat.
\end{enumerate}
%
\begin{tabela}
$$
\begin{array}{ccccl}
s & v & w & u & Q \\ \hline
1 & a &   &   & [a] \\
  &   & a & e & [e] \\
2 & b &   &   & [b] \\
  &   & b & c & [c] \\
  &   &   & d & [c, d]
\end{array}
$$
\podnaslov{Potek izvajanja algoritma}
\end{tabela}
\end{odgovor}
\end{naloga}
