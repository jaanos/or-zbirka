\begin{naloga}{Blaž Jelenc}{Izpit OR 5.6.2014}
\begin{vprasanje}
Dano je zaporedje števil $c_1, c_2, \dots, c_n$.
Igralca $A$ in $B$ igrata igro,
pri kateri $A$ bodisi z začetka bodisi s konca danega zaporedja
izbere eno (prej še neizbrano) število,
$B$ pa tako izbira dvakrat zapored.
Igrata, dokler števil ne zmanjka.
Zmaga igralec z največjo skupno vsoto izbranih števil.
Denimo, da začne igralec $A$.
S pomočjo dinamičnega programiranja določi maksimalni znesek,
ki ga $A$ lahko doseže, če tudi drugi igralec igra optimalno.
\end{vprasanje}

\begin{odgovor}
Naj bosta $p_{ij}$ in $q_{ij}$ največji oziroma najmanjši znesek,
ki ga lahko doseže igralec $A$ iz zaporedja $c_i, c_{i+1}, \dots, c_j$,
če je na potezi igralec $A$ oziroma igralec $B$.
Določimo začetne pogoje in rekurzivne enačbe.
\begin{align*}
p_{ii} &= c_i && (1 \le i \le n) \\
q_{ii} &= 0 && (1 \le i \le n) \\
q_{i,i+1} &= 0 && (1 \le i \le n-1) \\
p_{ij} &= \max\{c_i + q_{i+1,j}, c_j + q_{i,j-1}\} && (1 \le i < j \le n) \\
q_{ij} &= \min\{p_{i+2,j}, p_{i+1,j-1}, p_{i,j-2}\} && (1 \le i \le j-2 \le n-2)
\end{align*}
Vrednosti $p_{ij}$ in $q_{ij}$ računamo
v leksikografskem vrstnem redu glede na $(j-i, i)$
-- pravzaprav zadostuje,
da izračunamo vrednosti $p_{ij}$ z $j-i \equiv n-1 \pmod{3}$
in vrednosti $q_{ij}$ z $j-i \equiv n-2 \pmod{3}$.
Maksimalni znesek igralca $A$ ob optimalni igri igralca $B$
dobimo kot $p^* = p_{1n}$.
\end{odgovor}
\end{naloga}
