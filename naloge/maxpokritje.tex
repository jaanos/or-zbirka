\begin{naloga}{Janoš Vidali}{Kolokvij OR 31.3.2022}
\begin{vprasanje}
Dana je matrika $M$ dimenzij $2 \times n$,
pri čemer naj bodo $a_i$ in $b_i$ ($1 \le i \le n$)
vrednosti v prvi oziroma drugi vrstici ter $i$-tem stolpcu matrike $M$:
$$
M = \begin{bmatrix}
a_1 & a_2 & \dots & a_{n-1} & a_n \\
b_1 & b_2 & \dots & b_{n-1} & b_n
\end{bmatrix}
$$
Poiskati želimo tako podmnožico $S$ komponent matrike $M$ z največjo vsoto,
tako da je mogoče med poljubnima elementoma v $S$
priti samo s premiki levo, desno, gor ali dol (ne diagonalno!) tako,
da ne zapustimo množice $S$.

{\bf Primer} dopustne (ne nujno optimalne!) rešitve
(elementi $S$ so v okvirčkih):
$$
\begin{bmatrix}
\fbox{$\,\ 4\ \,$} & \fbox{$\,\ 4\ \,$} & -3 & \fbox{$\,\ 4\ \,$} & -3 \\
-1 & \fbox{$-2$} & \fbox{$\ 4\ $} & \fbox{$-4$} & \fbox{$\,\ 2\ \,$}
\end{bmatrix}
$$
Rešitev bi še vedno bila dopustna,
če bi iz $S$ odstranili elementa v prvem oziroma zadnjem stolpcu.
Če bi pa odstranili katerega od elementov
v spodnji vrstici srednjih treh stolpcev,
pa rešitev ne bi bila dopustna,
saj bi množica $S$ razpadla na dva dela,
med katerima ne bi mogli prehajati z dovoljenimi premiki.

\begin{enumerate}[(a)]
\item Zapiši rekurzivne enačbe,
s katerimi poiščemo največjo vsoto elementov v množici $S$.
Razloži, kaj pred\-stav\-lja\-jo spremenljivke,
v kakšnem vrstnem redu jih računamo,
ter kako dobimo optimalno rešitev.
\namig{uporabi različne spremenljivke glede na izbiro komponent v stolpcu.}

\item Oceni časovno zahtevnost algoritma, ki sledi iz zgoraj zapisanih enačb.

\item S svojim algoritmom poišči optimalno rešitev
(tj., največjo vsoto in še samo množico $S$)
na zgornjem primeru.
\end{enumerate}
\end{vprasanje}

\begin{odgovor}
\begin{enumerate}[(a)]
\item Naj bodo $x_i$, $y_i$ in $z_i$ ($1 \le i \le n$)
največje vsote, ki jih dosežemo,
če so najbolj desni elementi v $S$
v prvi, drugi oziroma obeh vrsticah $i$-tega stolpca.
Določimo začetne pogoje in rekurzivne enačbe.
\begin{align*}
x_0 &= y_0 = z_0 = 0 \\
x_i &= a_i + \max\{0, x_{i-1}, z_{i-1}\} & (1 \le i \le n) \\
y_i &= b_i + \max\{0, y_{i-1}, z_{i-1}\} & (1 \le i \le n) \\
z_i &= a_i + b_i + \max\{0, x_{i-1}, y_{i-1}, z_{i-1}\} & (1 \le i \le n)
\end{align*}
Vrednosti $x_i, y_i, z_i$ računamo naraščajoče po indeksu $i$ ($1 \le i \le n$).
Največjo vsoto množice $S$ dobimo kot
$z^* = \max\{x_i, y_i, z_i \mid (0 \le i \le n)\}$.

\item Algoritem izračuna $O(n)$ spremenljivk,
za vsako pa porabi konstantno časa.
Časovna zahtevnost algoritma je torej $O(n)$.

\item Izračunajmo vrednosti $x_i$, $y_i$ in $z_i$ ($1 \le i \le 5$).
\begin{alignat*}{3}
x_1 &&&&&= 4 \\
y_1 &&&&&= -1 \\
z_1 &=&\ 4 + (-1) &&&= 3 \\
x_2 &=&\ 4 &+ \max\{0, \underline{4}, 3\} &&= 8 \\
y_2 &=&\ -2 &+ \max\{0, -1, \underline{3}\} &&= 1 \\
z_2 &=&\ 4 + (-2) &+ \max\{0, \underline{4}, -1, 3\} &&= 6 \\
x_3 &=&\ -3 &+ \max\{0, \underline{8}, 6\} &&= 5 \\
y_3 &=&\ 4 &+ \max\{0, 1, \underline{6}\} &&= 10 \\
z_3 &=&\ -3 + 4 &+ \max\{0, \underline{8}, 1, 6\} &&= 9 \\
x_4 &=&\ 4 &+ \max\{0, 5, \underline{9}\} &&= 13 \\
y_4 &=&\ -4 &+ \max\{0, \underline{10}, 9\} &&= 6 \\
z_4 &=&\ 4 + (-4) &+ \max\{0, 5, \underline{10}, 9\} &&= 10 \\
x_5 &=&\ -3 &+ \max\{0, \underline{13}, 10\} &&= 10 \\
y_5 &=&\ 2 &+ \max\{0, 6, \underline{10}\} &&= 12 \\
z_5 &=&\ -3 + 2 &+ \max\{0, \underline{13}, 6, 10\} &&= 12
\end{alignat*}
Največja vsota množice $S$ je torej
$z^* = \max\{x_i, y_i, z_i \mid (0 \le i \le 5)\} = 13$.
Poglejmo, kateri elementi so v množici $S$.
\begin{align*}
z^* = x_4 &= a_4 + z_3       & a_4 &= 4 \\
      z_3 &= a_3 + b_3 + x_2 & a_3 &= -3,\ b_3 = 4 \\
      x_2 &= a_2 + x_1       & a_2 &= 4 \\
      x_1 &= a_1             & a_1 &= 4
\end{align*}
Imamo torej sledečo optimalno rešitev:
$$
\begin{bmatrix}
\fbox{$\,\ 4\ \,$} & \fbox{$\,\ 4\ \,$} & \fbox{$-3$} & \fbox{$\,\ 4\ \,$} & -3 \\
-1 & -2 & \fbox{$\,\ 4\ \,$} & -4 & 2
\end{bmatrix}
$$
\end{enumerate}
\end{odgovor}
\end{naloga}
