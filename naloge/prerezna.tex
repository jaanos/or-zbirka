\begin{naloga}{Janoš Vidali}{Kolokvij OR 11.6.2018}
\begin{vprasanje}
Dan je povezan neusmerjen enostaven graf $G = (V, E)$
(tj., brez zank in večkratnih povezav).
{\em Prerezno vozlišče} v grafu $G$ je tako vozlišče $u \in V$,
da graf $G - u$
(tj., graf $G$ brez vozlišča $u$ in povezav s krajiščem v $u$)
ni več povezan.
Poiskati želimo seznam prereznih vozlišč grafa $G$.

Pri iskanju si bomo pomagali s preiskovanjem v globino.
Ob prvem obisku vozlišča $u$ s predhodnikom $v$
se tako pokliče funkcija $\previsit(u, v)$,
ob njegovem zadnjem obisku pa funkcija $\postvisit(u, v)$.
Če je $u$ koren preiskovalnega drevesa, potem ima $v$ vrednost $\Null$.
Predpostavi, da imaš v obeh funkcijah dostop do seznama {\sl izhod},
kamor bo treba dodati najdena presečna vozlišča.
Prav tako imata lahko obe funkciji dostop do drugih pomožnih spremenljivk.

Naj bo $\ell_u$ globina vozlišča $u$ v drevesu iskanja v globino
(tj., razdalja od korena do $u$ v drevesu iskanja v globino).
Za vsako vozlišče $u$ definiramo vrednost $p_u$ kot najmanjšo globino vozlišč,
ki so v grafu $G$ sosedna (ali enaka) vozlišču $u$
ali njegovim potomcem v drevesu iskanja v globino.

\begin{enumerate}[(a)]
\item Za graf na sliki~\fig nariši drevo iskanja v globino
(v njem označi tudi povratne povezave, npr.~s črtkano črto)
in določi njegova prerezna vozlišča.
Upoštevaj abecedni vrstni red obiskovanja vozlišč.
Za vsako vozlišče $u$ določi še vrednosti $\ell_u$ in $p_u$.
\namig{vrednosti $p_u$ najprej določi za vozlišča z večjo globino.}

\item Napiši rekurzivno formulo za vrednost $p_u$.
\namig{loči med sosedi $v$ vozlišča $u$ v grafu $G$ (pišeš lahko $u \sim v$)
in njegovimi neposrednimi nasledniki $w$ v preiskovalnem drevesu ($u \to w$).}

\item Natančno opiši funkcijo $\previsit(u, v)$
(z besedami ali psevdokodo),
ki poskrbi za izračun vrednosti $\ell_u$.

\item Natančno opiši funkcijo $\postvisit(u, v)$
(z besedami ali psevdokodo),
ki naj za vozlišče $u$ izračuna vrednost $p_u$ in ugotovi,
ali je $u$ prerezno vozlišče, in ga v tem primeru doda v {\sl izhod}.
Predpostavi, da imaš globine vozlišč že poračunane.
\namig{obravnavaj dve možnosti --
ko je $u$ koren drevesa, in ko $u$ ni koren drevesa.
Kako v vsakem od teh primerov ugotoviš, ali je $u$ prerezno vozlišče?}

\item Oceni časovno zahtevnost celotnega algoritma.
\end{enumerate}

\begin{slika}
\pgfslika
\podnaslov{Graf}
\end{slika}
\end{vprasanje}

\begin{odgovor}
\begin{enumerate}[(a)]
\item Drevo iskanja v globino je prikazano na sliki~\fig[prerezna-resitev],
pri čemer sta pri vsakem vozlišču $u$ prikazani še vrednosti $(\ell_u, p_u)$.

Prerezna vozlišča grafa s slike~\fig so $a$, $b$ in $c$.
Koren drevesa iskanja v globino $a$ je prerezno vozlišče,
ker ima več kot enega naslednika.
Ostala prerezna vozlišča prepoznamo po tem,
da imajo takega naslednika v drevesu iskanja v globino,
da v njegovem poddrevesu ni vozlišč s prečnimi povezavami
do predhodnikov obravnavanega vozlišča
-- drugače povedano,
notranje vozlišče $u$ v drevesu iskanja v globino
je prerezno vozlišče natanko tedaj,
ko obstaja njegov naslednik $w$,
za katerega velja $p_w \ge \ell_u$.

\item Zapišimo rekurzivno formulo.
$$
p_u = \min\left(\set{p_w}{w \in V, u \to w} \cup \set{\ell_w}{w \in V, u \sim w \lor u = w}\right)
$$

\item Zapišimo psevdokodo za funkcijo $\previsit$.
\begin{small}
\begin{algorithmic}
\Function{Previsit}{$u, v$}
    \If{$v = \Null$}
        \State $\ell_u \gets 0$
    \Else
        \State $\ell_u \gets \ell_v + 1$
    \EndIf
\EndFunction
\end{algorithmic}
\end{small}

\item Zapišimo psevdokodo za funkcijo $\postvisit$.
\begin{small}
\begin{algorithmic}
\Function{Postvisit}{$u, v$}
    \If{$(v = \Null \land |\Adj(G, u)| > 1) \lor (v \ne \Null \land \exists w \in \Adj(G, u): p_w \le \ell_u)$}
        \State {\sl izhod}$.\append(u)$
    \EndIf
    \State $p_u \gets \min\left(\set{p_w}{w \in \Adj(G, u) \land \ell_w > \ell_u} \cup \set{\ell_w}{w \in \Adj(G, u) \lor w = u}\right)$
\EndFunction
\end{algorithmic}
\end{small}

\item Iskanje v globino teče v času $O(m + T)$,
pri čemer je $m$ število povezav v grafu,
$T$ pa je čas,
porabljen za klic funkcij $\previsit$ in $\postvisit$ za vsako vozlišče grafa.
Funkcija $\previsit$ teče v konstantnem času,
klic funkcije $\postvisit(u, v)$ pa teče v času $O(d_G(u))$,
kjer je $d_G(u)$ stopnja vozlišča $u$ v grafu $G$.
Ker je vsota stopenj vozlišč grafa enaka dvakratniku števila povezav,
lahko sklenemo, da celoten algoritem teče v času $O(m)$.
\end{enumerate}
%
\begin{slika}
\pgfslika[prerezna-resitev]
\podnaslov{Drevo iskanja v globino}
\end{slika}
\end{odgovor}
\end{naloga}
