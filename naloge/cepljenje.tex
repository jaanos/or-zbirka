\begin{naloga}{Janoš Vidali}{Izpit OR 18.6.2021}
\begin{vprasanje}
Vaščani Gornjih Dol že nestrpno čakajo na cepljenje proti COVID-19,
zato so se odločili stvari vzeti v svoje roke.
Vzpostavili so kontakt s trgovcem,
ki se ukvarja z maloprodajo cepiv na sivem trgu,
in dobili ponudbo za nakup stekleničke cepiva po ceni $250 €$,
pri čemer je v vsaki steklenički dovolj cepiva za $4$ ljudi.

V vasi živi $122$ ljudi,
vendar je zelo verjetno, da se nekateri ne bodo prišli cepit,
prav tako pa je možno, da bodo nekateri prišli na vrsto za cepljenje drugje,
že preden bo cepivo dobavljeno (in se tako ne bodo prišli ponovno cepit).
Tako se odločajo med nabavo $29$, $30$ ali $31$ stekleničk cepiva.
Pri tem so ocenili verjetnosti $p_i$,
da se natanko $i$ ljudi ne odloči za cepljenje
(in tudi drugje niso prišli na vrsto),
in verjetnosti $q_i$, da natanko $i$ ljudi cepivo dobi drugje:
$$
\begin{array}{c|cccc}
i & 0 & 1 & 2 & 3 \\ \hline
p_i & 0.3 & 0.4 & 0.2 & 0.1 \\
q_i & 0.1 & 0.3 & 0.4 & 0.2
\end{array}
$$
Verjetnosti $p_i$ in $q_i$ so med seboj neodvisne.
Računajo, da bodo imeli z vsakim vaščanom, ki ne bo cepljen
(bodisi zato, ker ne bo prišel na cepljenje,
ali pa zato, ker bo zanj zmanjkalo cepiva),
dodatnih $1\,000 €$ stroškov.

Zanima nas torej, koliko stekleničk naj naročijo,
da bodo pričakovani skupni stroški čim manjši.
\end{vprasanje}

\begin{odgovor}
Naj bodo $X_k$ ($29 \le k \le 31$) stroški,
če se vaščani odločijo nabaviti $k$ stekleničk cepiva.
\begin{alignat*}{2}
E(X_{29}) &= 29 \cdot 250 € + (3 q_3 + 4 q_2 + 5 q_1 + 6 q_0) \cdot 1\,000 € &&= 11\,550 € \\
E(X_{30}) &= 30 \cdot 250 € + ((p_0 q_1 + p_1 (1 - q_0)) + 2 \cdot ((p_0 + p_1) q_0 + p_2) + 3 p_3) \cdot 1\,000 € &&= 8\,790 € \\
E(X_{31}) &= 31 \cdot 250 € + (p_1 + 2 p_2 + 3 p_3) \cdot 1\,000 € &&= 8\,850 €
\end{alignat*}
Vidimo, da se vaščanom najbolj izplača kupiti $30$ stekleničk cepiva.
\end{odgovor}
\end{naloga}
