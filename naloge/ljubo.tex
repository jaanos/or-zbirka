\begin{naloga}{Batagelj, Kaufman}{\cite[Naloga~4.3]{bk}}
\begin{vprasanje}
Ljubo prodaja časopise po centru Ljubljane.
Vsak dan se mora odločiti, koliko časopisov naj naroči.
Za vsak naročen časopis plača $0.8 €$, iztrži pa $1 €$.
Za neprodane časopise ne dobi povrnjenega denarja.
Vsak dan je verjetnost, da bo imel $i$ kupcev, enaka $p_i$,
kjer je $p_6 = 0.1$, $p_7 = 0.2$, $p_8 = 0.3$, $p_9 = 0.3$ in $p_{10} = 0.1$.
\begin{enumerate}[(a)]
\item Kakšno odločitev svetuješ Ljubu in zakaj?
\item Kolikšen zaslužek lahko pričakuje v mesecu,
ko časopis izide petindvajsetkrat?
\end{enumerate}
\end{vprasanje}
\begin{odgovor}
\end{odgovor}
\end{naloga}
