\begin{naloga}{Janoš Vidali}{Izpit OR 5.7.2018}
\begin{vprasanje}
Vlagatelj ima na voljo $50$ milijonov evrov sredstev,
ki jih lahko porabi za donosno, a tvegano naložbo.
Ocenjuje, da bi se mu ob uspehu naložbe vložek povrnil petkratno,
verjetnost uspeha pa ocenjuje na $0.6$.
Zaradi tveganja se lahko odloči za zavarovanje naložbe,
pri čemer ima ponudbi dveh zavarovalnic,
ki mu proti plačilu ustrezne premije ponujata povračilo dela vložka,
če bo naložba neuspešna.
Vlagatelj lahko del sredstev obdrži tudi zase
(tj., ga ne porabi za naložbo ali premijo).

Naj bodo torej $x_1, x_2, x_3$ vrednosti v milijonih evrov,
ki zaporedoma pred\-stav\-lja\-jo količine,
ki jih vlagatelj obrži zase, porabi za naložbo,
in plača za zavarovalniško premijo.
Pričakovana vrednost naložbene strategije vlagatelja
(tj., količina denarja, ki jo ima na koncu)
je potem
$$
x_1 + x_2 (0.6 \cdot 5 + 0.4 q(x_3)) ,
$$
kjer $q(x_3)$ predstavlja delež vložka,
ki ga glede na vloženo premijo zavarovalnica povrne ob ne\-uspe\-hu naložbe.

Vlagatelj ima dve ponudbi konkurenčnih zavarovalnic.
Zavarovalnica Zvezna d.z.z.~za premijo v višini $x_3$ milijonov evrov
ponuja povračilo deleža $0.15 x_3$ celotne naložbe v primeru neuspeha,
pri čemer je največja možna premija $4$ milijone evrov.
Zavarovalnica Diskretna d.d.z.~pa ponuja le tri možne premije:
\begin{center}
\begin{tabular}{c|c}
premija & delež povračila ob neuspešni naložbi \\ \hline
$1$ milijon evrov & $0.1$ \\
$2$ milijona evrov & $0.35$ \\
$3$ milijoni evrov & $0.5$
\end{tabular}
\end{center}
Pogodbo smemo skleniti samo pri eni zavarovalnici.

\begin{enumerate}[(a)]
\item Zapiši definicijo funkcije $q(x)$
skupaj z izbiro najugodnejše zavarovalnice pri vsakem $x$.

\item Zapiši rekurzivne formule za določitev strategije vlaganja,
ki nam bo prinesla največji pričakovani dobiček.

\item S pomočjo zgornjih rekurzivnih enačb ugotovi,
kako naj ravna vlagatelj, da bo imel čim večji dobiček.
\end{enumerate}
\end{vprasanje}

\begin{odgovor}
\begin{enumerate}[(a)]
\item Zapišimo definicijo funkcije $q(x)$ za $0 \le x \le 4$.
\begin{align*}
q(x) &= \max\left(\{0.15 x\} \cup \set{0.1}{x = 1} \cup \set{0.35}{x = 2} \cup \set{0.5}{x = 3}\right) \\
&= \begin{cases}
0.35   & x = 2 \text{ (Diskretna d.d.z.)} \\
0.5    & x = 3 \text{ (Diskretna d.d.z.)} \\
0.15 x & \text{sicer (Zvezna d.z.z.)}
\end{cases}
\end{align*}

\item Naj bosta $v_1(x)$ in $v_2(x)$
največji pričakovani vrednosti naložbenih strategij,
pri katerih imamo na voljo $x$ milijonov evrov
oziroma to količino v celoti vložimo v naložbo in zavarovanje.
Zapišimo rekurzivni enačbi.
\begin{align*}
v_2(x) &= \max\set{(x-y) (3 + 0.4 q(y))}{0 \le y \le \min\{4, x\}} \\
v_1(x) &= \max\set{x-y + v_2(y)}{0 \le y \le x}
\end{align*}

\item Najprej izrazimo $v_2(x)$ glede na vrednost $x$ ($0 \le x \le 50$).
\begin{alignat*}{2}
v_2(x) &= \max&\big(&\set{3.14 (x-2)}{x \ge 2} \cup \set{3.2 (x-3)}{x \ge 3} \cup \\
&&&{} \set{(x-y) (3 + 0.06 y)}{0 \le y \le \min\{4, x\}, y \not\in \{2, 3\}}\big)
\end{alignat*}
Naj bo $f(y)$ izraz v zadnjem oklepaju.
Opazimo, da gre za kvadraten polinom v $y$ z negativnim vodilnim členom,
tako da lahko s pomočjo odvajanja poiščemo njegov maksimum.
\begin{align*}
f(y)  &= -0.06 y^2 + (0.06x - 3) y + 3x \\
f'(y) &= -0.12 y + 0.06 x - 3 = 0 \\
y &= x/2 - 25 \le 0
\end{align*}
Opazimo, da je maksimum vedno dosežen za vrednost $y$,
ki ni večja od spod\-nje meje ustreznega intervala,
tako da je znotraj njega maksimum dosežen pri $y = 0$ in znaša $f(0) = 3x$
-- tj., premije ne plačamo.
S primerjavo vseh treh izrazov tako dobimo
$$
v_2(x) = \begin{cases}
3 x        & 0 \le x \le 314/7 \text{ (brez zavarovanja)} \\
3.14 (x-2) & 314/7 < x \le 50 \text{ (Diskretna d.d.z.~s premijo $2$ mio evrov)}
\end{cases}
$$
Optimalno strategijo vlaganja sedaj dobimo tako,
da izračunamo $v_1(50)$.
\begin{alignat*}{2}
v_1(50) &= \max&\big(&\set{50 + 2y}{0 \le y \le 314/7} \cup \\
&&&{} \set{43.72 + 2.14y}{314/7 < x \le 50}\big)
\end{alignat*}
Ker oba izraza naraščata z $y$,
zadostuje preveriti njuni vrednosti pri zgornji meji ustreznega intervala.
Tako opazimo, da največjo vrednost dobimo pri $y = 50$.
Optimalna strategija vlaganja je torej taka,
pri kateri vlagatelj $48$ milijonov evrov vloži v naložbo,
$2$ milijona evrov porabi za premijo pri zavarovalnici Diskretna d.d.z.,
zase pa ne obdrži ničesar.
Pričakovana vrednost naložbene strategije
je tako $v^* = v_1(50) = 150.72$ milijonov evrov.
\end{enumerate}
\end{odgovor}
\end{naloga}
