\begin{naloga}{Juvan, Orbanić}{Izpit OR 5.2.2010}
\begin{vprasanje}
Dana je rekurzija
$$
c_{ij} = \min_{i \le k \le j} \left(c_{i-1,k} + \sum_{u=i}^k a_{iu}\right),
\quad c_{0j} = 1 , \quad (1 \le i \le j \le n)
$$
kjer je $A = (a_{ij})_{i,j=1}^n$ matrika realnih števil.
Pokaži in zelo natančno utemelji,
da lahko izračunaš $c_{nn}$ v času $O(n^3)$.
\end{vprasanje}

\begin{odgovor}
Zapišimo začetne pogoje in rekurzivne enačbe za reševanje danega problema.
\begin{align*}
b_{ii} &= a_{ii} && (1 \le i \le n) \\
b_{ij} &= a_{ij} + b_{i,j-1} && (1 \le i < j \le n) \\[1ex]
c_{0j} &= 1 && (1 \le j \le n) \\
c_{ii} &= c_{i-1,i} + b_{ii} && (1 \le i \le n) \\
c_{ij} &= \min\{c_{i,j-1}, c_{i-1,j} + b_{ij}\} && (1 \le i < j \le n)
\end{align*}
Pokažimo, da zgornje enačbe ustrezajo podani rekurziji.
Najprej dokažimo, da velja
$$
b_{ij} = \sum_{u=i}^j a_{iu} .
$$
Zgornja enakost očitno velja za $j = i$.
Denimo, da velja tudi za $j = j_0 \ge i$.
Potem velja
$$
b_{i,j_0+1} = a_{i,j_0+1} + b_{ij_0} = a_{i,j_0+1} + \sum_{u=i}^{j_0} a_{iu}
= \sum_{u=i}^{j_0+1} a_{iu} ,
$$
tako da po indukciji zgornja trditev velja za vse $j \ge i$.
Nadalje dokažimo, da velja
$$
c_{ij} = \min\set{c_{i-1,k} + b_{ik}}{i \le k \le j} .
$$
Zgornja enakost spet očitno velja za $j = i$.
Denimo, da velja tudi za $j = j_0 \ge i$.
Potem velja
\begin{align*}
c_{i,j_0+1} &= \min\{c_{ij_0}, c_{i-1,j_0+1} + b_{i,j_0+1}\} \\
&= \min\{\min\set{c_{i-1,k} + b_{ik}}{i \le k \le j_0}, c_{i-1,j_0+1} + b_{i,j_0+1}\} \\
&= \min\set{c_{i-1,k} + b_{ik}}{i \le k \le j_0+1} ,
\end{align*}
tako da po indukciji zgornja trditev velja za vse $j \ge i$.
Tako velja
$$
c_{ij} = \min\set{c_{i-1,k} + \sum_{u=i}^k a_{iu}}{i \le k \le j} .
$$
Vrednost $c_{nn}$ je tako mogoče izračunati tako,
da izračunamo vrednosti $b_{ij}$ in $c_{ij}$
za vsak par $(i, j)$ z $1 \le i \le j \le n$
v leksikografskem vrstnem redu.
Ker je mogoče vsako vrednost izračunati v konstantnem času,
je mogoče celoten izračun opraviti v času $O(n^2)$.
\end{odgovor}
\end{naloga}
