\begin{naloga}{?}{Vaje OR 6.4.2016}
\begin{vprasanje}
Dana je matrika $A = (a_{ij})_{i,j=1}^{m,n}$.
Poiskati želimo strnjeno podmatriko matrike $A$
z največjo vsoto komponent.
\begin{enumerate}[(a)]
\item Napiši rekurzivne enačbe za opisani problem.
\item Reši problem za matriko
$$
A = \begin{pmatrix}
 1 & -1 &  2 &  4 \\
-3 & -2 &  8 &  2 \\
-3 &  2 & -2 &  4 \\
 1 & -5 & -1 & -2
\end{pmatrix} .
$$
\item Napiši algoritem, ki reši opisani problem.
Oceni tudi njegovo časovno zahtevnost v odvisnosti od $m$ in $n$.
\end{enumerate}
\end{vprasanje}

\begin{odgovor}
\begin{enumerate}[(a)]
\item Naj bo $p_{hik}$ največja vsota komponent strnjene podmatrike
z vrsticami od $h$-te do $i$-te vrstice matrike $A$,
katere zadnji stolpec je $k$-ti stolpec matrike $A$.
Poleg tega naj bo $s_{ij}$ vsota prvih $i$ komponent
v $j$-tem stolpcu matrike $A$.
Določimo začetne pogoje in rekurzivne enačbe.
\begin{align*}
s_{0j}  &= 0 && (1 \le j \le n) \\
s_{ij}  &= a_{ij} + s_{i-1,j} && (1 \le i \le m, \ 1 \le j \le n) \\
p_{hi0} &= 0 && (1 \le h \le i \le m) \\
p_{hik} &= s_{ik} - s_{h-1,k} + \max\{0, p_{h,i,k-1}\} &&
(1 \le h \le i \le m, \ 1 \le k \le n)
\end{align*}
Najprej za vsak $j$ ($1 \le j \le n$)
izračunamo vrednosti $s_{ij}$ naraščajoče po indeksu $i$,
nato pa za vsaka $h$ in $i$ ($1 \le h \le i \le m$)
izračunamo vrednosti $p_{hik}$ naraščajoče po indeksu $k$.
Največjo vsoto komponent dobimo kot
$p^* = \max\set{p_{hik}}{1 \le h \le i \le m, \ 0 \le k \le n}$.

\item Najprej poračunajmo vrednosti $s_{ij}$.
$$
(s_{ij})_{i,j=1}^4 = \begin{pmatrix}
 1 & -1 &  2 &  4 \\
-2 & -3 & 10 &  6 \\
-5 & -1 &  8 & 10 \\
-4 & -6 &  7 &  8
\end{pmatrix}
$$
Sedaj poračunajmo še vrednosti $p_{hik}$.
$$
\begin{array}{cc|cccc}
\multicolumn{2}{c|}{p_{hik}} & \multicolumn{4}{c}{k} \\ \hline
h & i &  1 &  2 &  3 &  4 \\ \hline
1 & 1 &  1 &  0 &  2 &  6 \\
1 & 2 & -2 & -3 & 10 & 16 \\
1 & 3 & -5 & -1 &  8 & 18 \\
1 & 4 & -4 & -6 &  7 & 15 \\ \hline
2 & 2 & -3 & -2 &  8 & 10 \\
2 & 3 & -6 &  0 &  6 & 12 \\
2 & 4 & -5 & -5 &  5 &  9 \\ \hline
3 & 3 & -3 &  2 &  0 &  4 \\
3 & 4 & -2 & -2 & -3 &  2 \\ \hline
4 & 4 &  1 & -4 & -1 & -2
\end{array}
$$
Opazimo, da je največja vsota komponent enaka $p^* = p_{134} = 18$.
Indeks $j$ prvega stolpca iskane podmatrike je najmanjša taka vrednost,
da za vse $\ell$ ($j \le \ell \le k$) velja $p_{13\ell} > 0$.
V našem primeru je torej $j = 3$
-- iskana podmatrika je torej
$$
A_{1 \dots 3, 3 \dots 4} = \begin{pmatrix}
 2 &  4 \\
 8 &  2 \\
-2 &  4
\end{pmatrix} .
$$

\item Zapišimo algoritem,
ki bo vrnil največjo vsoto komponent strnjene podmatrike skupaj s paroma,
ki določata prvo in zadnjo vrstico oziroma stolpec.
\begin{small}
\begin{algorithmic}
\Function{MaxStrnjenaPodmatrika}{$(a_{ij})_{i,j=1}^{m,n}$}
    \For{$j = 1, \dots, n$} \hfill računanje $s_{ij}$
        \State $s_{0j} \gets 0$ \hfill začetni pogoj
        \For{$i = 1, \dots, m$}
            \State $s_{ij} \gets a_{ij} + s_{i-1,j}$ \hfill rekurzivna enačba
        \EndFor
    \EndFor
    \State $p^*, h^*, i^*, j^*, k^* \gets 0, 0, 0, 0, 0$
        \hfill ničelna rešitev
    \For{$h = 1, \dots, n$} \hfill računanje $p_{hik}$
        \For{$i = h, \dots, n$}
            \State $p_{hi0} \gets 0$ \hfill začetni pogoj
            \State $j \gets 1$ \hfill vsoto začenjamo s prvim stolpcem
            \For{$k = 1, \dots, m$}
                \State $p_{hik} \gets s_{ik} - s_{h-1,k}
                                    + \max\{0, p_{h,i,k-1}\}$
                    \hfill rekurzivna enačba
                \If{$p_{hik} > p^*$}
                        \hfill preverimo, ali imamo boljšo rešitev
                    \State $p^*, h^*, i^*, j^*, k^* \gets p_{hik}, h, i, j, k$
                \ElsIf{$p_{hik} < 0$} \hfill če je vsota negativna,
                    \State $j \gets k+1$ \hfill
                        bomo naslednjo vsoto začeli v naslednjem stolpcu
                \EndIf
            \EndFor
        \EndFor
    \EndFor
    \State \Return $p^*, (h^*, i^*), (j^*, k^*)$
\EndFunction
\end{algorithmic}
\end{small}
Za računanje $s_{ij}$ potrebujemo $O(mn)$ korakov,
za računanje $p_{hik}$ pa $O(mn^2)$ korakov.
Skupna časovna zahtevnost algoritma je torej $O(mn^2)$.
Opazimo, da se v primeru, ko velja $n > m$,
matriko $A$ izplača transponirati,
saj tako dobimo časovno zahtevnost $O(m^2 n)$.
\end{enumerate}
\end{odgovor}
\end{naloga}
