\begin{naloga}{?}{Izpit OR 12.5.2016}
\begin{vprasanje}
Napiši algoritem,
ki na vhod sprejme neusmerjen graf $G = (V, E)$ in povezavo $e \in E$
ter pove, ali obstaja cikel v $G$, ki vsebuje povezavo $e$.
\end{vprasanje}

\begin{odgovor}
Naj bo $e = (s, t)$. 
Naloga nas potem sprašuje, če obstaja pot od $t$ do $s$, 
saj bomo v tem primeru očitno dobili cikel, ki vsebuje $e$.
Ideja algoritma bo torej uporabiti koncept {\sc dfs} algoritma iz naloge~\nal[Dfs], z začetkom v $t$ 
in preveriti, če smo prišli z njim do $s$.

\begin{small}
\begin{algorithmic}
\Function{Cikel}{$G = (V, E)$, $e = (s, t)$}
	\State označeni $\gets$ slovar s ključi $v \in V$ in vrednostmi $\False$
	\State sklad $\gets [t]$
	\While{$\length$(sklad) $> 0$}
		\State $u \gets sklad.\pop()$
		\If{$\neg$ označeni$[u]$}
			\State označeni$[u] \gets \True$
			\For{$v \in \Adj(G, u)$}
				\If{$\neg$ označeni$[v]$}
					\State sklad.$\append(v)$
					\If{$v = s$}
						\State \Return $\True$
					\EndIf
				\EndIf
			\EndFor
		\EndIf
	\EndWhile
	\State \Return $\False$
\EndFunction
\end{algorithmic}
\end{small}
Pripomnimo, da algoritem deluje pravilno za usmerjene grafe, 
pri neusmerjenih pa moramo paziti, 
da ne uporabimo povezave $t \rightarrow s$, 
torej moramo pri pregledu sosedov vozlišča $t$ $s$ ignorirati.
\end{odgovor}
\end{naloga}
