\begin{naloga}{Janoš Vidali}{Izpit OR 10.7.2017}
\begin{vprasanje}
Z električnim vozilom se odpravljamo na počitnice.
Vozilo moramo vsako noč napolniti,
zato smo si pripravili seznam krajev in cestnih povezav med njimi,
ki jih lahko prevozimo v enem dnevu.
Poiskati želimo pot od začetne točke do destinacije,
ki bo imela čim manjše število postankov (tj., bo trajala čim manj dni).

\begin{enumerate}[(a)]
\item Predstavi problem v jeziku grafov
in predlagaj algoritem za njegovo reševanje.

\item Na koncu počitnic razmišljamo o poti nazaj.
Spet bi radi naredili čim manj postankov,
a se pri tem ne želimo ustaviti v nobenem kraju,
kjer smo se ustavili na poti naprej.
Dopolni zgornji algoritem, da bo našel še ustrezno pot nazaj.

\item S pomočjo zgornjih algoritmov poišči najkrajšo pot od LJ do AM
in najkrajšo pot nazaj v grafu s slike~\fig,
ki ne gre čez kraje iz prejšnje poti.
\end{enumerate}

\begin{slika}
\pgfslika
\caption{Graf za nalogi~\nal (brez uteži) in~\nal[pot].}
\end{slika}
\end{vprasanje}

\begin{odgovor}
\begin{enumerate}[(a)]
\item Konstruiramo graf $G$, katerega vozlišča so kraji z našega seznama,
med vozliščema pa je povezava,
če lahko cesto med ustreznima krajema prevozimo v enem dnevu.
Če vozlišči $s$ in $t$ predstavljata začetno točko in destinacijo,
potem lahko na grafu $G$ izvedemo iskanje v širino z začetkom v vozlišču $s$
ter v dobljenem drevesu poiščemo pot do vozlišča $t$.

\item Iz grafa $G$ izpeljemo graf $G'$, ki ga dobimo tako,
da odstranimo notranja vozlišča poti iz prejšnje točke
(tj., $s$ in $t$ pustimo v $G'$).
Potem lahko na grafu $G'$ izvedemo iskanje v širino z začetkom v vozlišču $t$
ter v dobljenem drevesu poiščemo pot do vozlišča $s$.

\item Iskanje v širino na grafu $G$ s slike~\fig
nam da drevo s slike~\fig[pocitnice-resitev1],
s katere je razvidna pot LJ -- WI -- BN -- LU -- AM.
Graf $G'$ torej dobimo tako, da iz $G$ odstranimo vozlišča WI, BN in LU.
Iskanje v širino nam da drevo s slike~\fig[pocitnice-resitev2],
s katere je razvidna povratna pot AM -- BL -- PR -- BS -- BU -- LJ.
\end{enumerate}
%
\begin{slika}[p]
\pgfslika[pocitnice-resitev1]
\podnaslov{Drevo iskanja v širino na grafu $G$}
\end{slika}
%
\begin{slika}[p]
\pgfslika[pocitnice-resitev2]
\podnaslov{Drevo iskanja v širino na grafu $G'$}
\end{slika}
\end{odgovor}
\end{naloga}
