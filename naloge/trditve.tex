\begin{naloga}{Juvan, Orbanić}{Kolokvij OR 19.11.2009}
\begin{vprasanje}
Naj bo $G$ graf z uteženimi povezavami, kjer so uteži nenegativne.
Dokaži ali ovrzi naslednje trditve:
\begin{enumerate}[(a)]
\item Če v grafu obstaja enolična najkrajša povezava,
potem je ta povezava vsebovana v vsakem drevesu najkrajših poti.
\item Če v grafu obstaja enolična najkrajša povezava,
potem je ta vključena v vsako minimalno vpeto drevo.
\item Naj bo $e$ povezava v grafu $G$,
ki je izmed povezav na nekem ciklu $C$ najdaljša.
Če iz $G$ odstranimo povezavo $e$, dobimo graf $G'$.
Pokaži, da je poljubno minimalno vpeto drevo v $G'$
tudi minimalno vpeto drevo v $G$.
\end{enumerate}
\end{vprasanje}

\begin{odgovor}
\begin{enumerate}[(a)]
\item Trditev ni resnična.
Protiprimer je podan na sliki~\fig,
ki z rdečo barvo prikazuje drevo najkrajših poti iz vozlišča $a$
-- to ne vsebuje najkrajše povezave $bc$.

\item Trditev je resnična.
Da jo dokažemo,
predpostavimo, da vpeto drevo $T$ ne vsebuje najkrajše povezave $e$.
Potem graf $T + e$ vsebuje cikel, ki vsebuje povezavo $e' \ne e$.
Po predpostavki je povezava $e'$ daljša od povezave $e$,
tako da je graf $T + e - e'$ vpeto drevo v grafu $G$ z manjšo dolžino kot $T$.
Drevo $T$ tako ne more biti minimalno vpeto drevo.

\item Očitno je, da je vsako vpeto drevo v $G'$ tudi vpeto drevo v $G$.
Naj bo $T$ drevo, ki je vpeto v grafu $G$, ne pa tudi v grafu $G'$.
Drevo $T$ potem vsebuje povezavo $e$.
Naj bo $e'$ povezava iz $C - e$
-- po predpostavki je $T' = T - e + e'$ vpeto drevo v $G$,
katerega dolžina ni večja kot dolžina drevesa $T$.
Ker pa $T'$ ne vsebuje povezave $e$, je tudi vpeto drevo v grafu $G'$.
Dolžina minimalnega vpetega drevesa $T_{\min}$ v $G'$
tako ne presega dolžine nobenega vpetega drevesa v $G$,
zaradi česar lahko sklenemo,
da je $T_{\min}$ tudi minimalno vpeto drevo v $G$.

\end{enumerate}
\begin{slika}
\pgfslika
\podnaslov[\res{}(a)]{Protiprimer}
\end{slika}
\end{odgovor}
\end{naloga}
