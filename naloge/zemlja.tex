\begin{naloga}{Janoš Vidali}{Izpit OR 25.8.2020}
\begin{vprasanje}
Janez je podedoval zemljo, na kateri si želi zgraditi hišo.
Ve, da mora pred začetkom gradnje pridobiti gradbeno dovoljenje,
zaradi nerazumljivih birokratskih postopkov
in nejasne razmejitve parcel pa ne ve,
ali morda potrebuje tudi soglasja sosedov.
Lahko se odloči, da takoj zaprosi za gradbeno dovoljenje
-- ocenjuje,
da bo z verjetnostjo $1/2$ vloga uspešna in bo po $30$ dneh lahko začel graditi,
sicer pa mu bodo vlogo zavrnili in bo moral pridobiti soglasja,
kar bo celoten postopek podaljšalo na $55$ dni.
Lahko pa pred vložitvijo prošnje pridobi soglasja vseh sosedov
-- v tem primeru bo celoten postopek trajal $45$ dni.

Janez pozna tudi odvetnika Bineta,
ki lahko pred sprožitvijo postopkov preuči situacijo,
za kar potrebuje $7$ dni.
Spodaj so zbrane pogojne verjetnosti za Binetovo mnenje glede na to,
ali so soglasja potrebna.

\begin{center}
    \setlength{\extrarowheight}{2pt}
    \begin{tabular}{c|cc}
        $P(\text{Binetovo mnenje} \mid \text{potreba po soglasjih})$
        & so potrebna & niso potrebna \\ \hline
        ugodno & ${1 \over 3}$ & ${3 \over 4}$ \\
        neugodno & ${2 \over 3}$ & ${1 \over 4}$
    \end{tabular}
\end{center}

Kako naj se Janez odloči,
da bo pričakovani čas do začetka gradnje hiše čim krajši?
Nariši odločitveno drevo in ga uporabi pri sprejemanju odločitev.
Izračunaj tudi pričakovani čas.
\end{vprasanje}

\begin{odgovor}
Če se Janez odloči, da bo takoj zaprosil za gradbeno dovoljenje,
bo pričakovano število dni do začetka gradnje enako
$$
{1 \over 2} \cdot 30 + {1 \over 2} \cdot 55 = 42.5,
$$
kar je manj kot $45$ dni, ki bi jih potreboval,
če bi soglasja pridobil pred vlo\-žit\-vi\-jo prošnje.
Brez Binetovega mnenja
se Janezu torej izplača takoj zaprositi za gradbeno dovoljenje.

Izračunajmo še verjetnosti za Binetovo mnenje
ter potrebe po soglasjih ob vsakem mnenju.
\begin{align*}
P(\text{ugodno mnenje}) &=
{1 \over 3} \cdot {1 \over 2} + {3 \over 4} \cdot {1 \over 2} = {13 \over 24} \\
P(\text{neugodno mnenje}) &=
{2 \over 3} \cdot {1 \over 2} + {1 \over 4} \cdot {1 \over 2} = {11 \over 24} \\
P(\text{soglasja so potrebna} \mid \text{ugodno mnenje})
&= {1/3 \cdot 1/2 \over 13/24} = {4 \over 13} \\
P(\text{soglasja niso potrebna} \mid \text{ugodno mnenje})
&= {3/4 \cdot 1/2 \over 13/24} = {9 \over 13} \\
P(\text{soglasja so potrebna} \mid \text{neugodno mnenje})
&= {2/3 \cdot 1/2 \over 11/24} = {8 \over 11} \\
P(\text{soglasja niso potrebna} \mid \text{neugodno mnenje})
&= {1/4 \cdot 1/2 \over 11/24} = {3 \over 11}
\end{align*}
S pomočjo zgoraj izračunanih verjetnosti
lahko narišemo odločitveno drevo s slike~\fig.
Opazimo,
da je pričakovani čas ob pridobitvi Binetovega mnenja večji kot $42.5$ dni,
zato naj Janez takoj zaprosi za gradbeno dovoljenje.

\begin{slika}
\makebox[\textwidth][c]{
\pgfslika
}
\podnaslov{Odločitveno drevo}
\end{slika}
\end{odgovor}
\end{naloga}
