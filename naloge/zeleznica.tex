\begin{naloga}{Janoš Vidali}{Kolokvij OR 5.6.2024}
\begin{vprasanje}
Gradbeno podjetje pripravlja načrt izgradnje novega odseka železniške proge.
Faze gradnje so zbrane v tabeli~\tab.

\begin{enumerate}[(a)]
\item Topološko uredi ustrezni graf in ga nariši.
Za trajanja opravil vzemi pričakovana trajanja po modelu PERT.

\item Določi pričakovano kritično pot in pričakovani čas izgradnje.

\item Katero opravilo je (ob zgornjih predpostavkah) najmanj kritično?
Najmanj kritično je opravilo, katerega trajanje lahko najbolj podaljšamo,
ne da bi vplivali na celotno trajanje projekta.

\item Določi variance trajanj opravil in oceni ve\-rjet\-nost,
da bo projekt trajal manj kot $340$ dni.
\end{enumerate}
%
\begin{tabela}
\makebox[\textwidth][c]{
\begin{tabular}{c|cc|b{2cm}b{2.7cm}b{2cm}}
& Opravilo & Pogoji & Minimalno trajanje
& Najbolj verjetno trajanje & Maksimalno trajanje \\ \hline
$a$ & postavitev stebrov viadukta                       & /         &  84 dni &  90 dni &  96 dni \\
$b$ & gradnja voziščne konstrukcije viadukta            & $a$       & 133 dni & 151 dni & 163 dni \\
$c$ & izkop glavne cevi predora                         & /         & 180 dni & 195 dni & 240 dni \\
$d$ & izkop servisne cevi predora                       & /         & 150 dni & 174 dni & 234 dni \\
$e$ & postavljanje tirov                                & $b, c, d$ &  51 dni &  60 dni &  69 dni \\
$f$ & postavitev električne in komunikacijske napeljave & $d$       &  37 dni &  46 dni &  49 dni \\
$g$ & pregled trase                                     & $e, f$    &  25 dni &  25 dni &  55 dni \\
$h$ & priključitev na obstoječe železniško omrežje      & $e$       &  11 dni &  14 dni &  23 dni
\end{tabular}
}
\podnaslov{Podatki}
\end{tabela}
\end{vprasanje}

\begin{odgovor}
\begin{enumerate}[(a)]
\item Projekt lahko predstavimo z uteženim grafom s slike~\fig,
iz katerega je raz\-vid\-na topološka ureditev
$s, a, c, d, b, f, e, h, g, t$.
Uteži povezav ustrezajo pričakovanim trajanjem opravil,
ki so prikazana v tabeli~\tab[zeleznica-resitev].

\item V tabeli~\tab[zeleznica-resitev]
so podani najzgodnejši začetki opravil in najpoznejši začetki,
da se celotno trajanje projekta ne podaljša.
Pričakovano trajanje projekta je torej $\mu = 330$ dni,
edina kritična pot pa je $s - a - b - e - g - t$,
ki tako vsebuje tudi vsa kritična opravila.

\item Iz tabele~\tab[zeleznica-resitev] je razvidno,
da je najmanj kritično opravilo $f$,
saj lahko njegovo trajanje
podaljšamo za $75$ dni v primerjavi s pričakovanim,
ne da bi vplivali na trajanje celotnega projekta.

\item Izračunajmo standardni odklon trajanja pričakovane kritične poti,
pri čemer uporabimo variance iz tabele~\tab[zeleznica-resitev].
$$
\sigma = \sqrt{4 + 25 + 9 + 25} \approx 7.94
$$
Naj bo $X$ slučajna spremenljivka,
ki meri čas izvajanja pričakovane kritične poti.
Izračunajmo verjetnost končanja v roku $T = 340$ dni.
$$
P(X \le 340) = \Phi\left({T - \mu \over \sigma}\right) = \Phi(1.260) = 0.8962
$$
\end{enumerate}
%
\begin{slika}
\makebox[\textwidth][c]{
\pgfslika
}
\podnaslov{Graf odvisnosti med opravili in kritična pot}
\end{slika}
%
\begin{tabela}
\setlabel{zeleznica-resitev}
\makebox[\textwidth][c]{
\begin{tabular}{c|cccccccccc}
& $s$ & $a$ & $c$ & $d$ & $b$ & $f$ & $e$ & $h$ & $g$ & $t$ \\ \hline
pričakovano && $90$ & $200$ & $180$ & $150$ & $45$ & $60$ & $15$ & $30$ \\
varianca && $4$ & $100$ & $196$ & $25$ & $4$ & $9$ & $4$ & $25$ \\
\hline
najprej & $0$ & $0_s$ & $0_s$ & $0_s$ & $90_a$ & $180_d$ & $240_b$ & $300_e$ & $300_e$ & $330_g$ \\
najkasneje & $0_a$ & $0_b$ & $40_e$ & $60_e$ & $90_e$ & $255_g$ & $240_h$ & $315_t$ & $300_t$ & $330$ \\
\text{razlika} & $0$ & $0^*$ & $40$ & $60$ & $0^*$ & $75$ & $0^*$ & $15$ & $0^*$ & $0$
\end{tabular}
}
\podnaslov{Razporejanje opravil}
\end{tabela}
\end{odgovor}
\end{naloga}
