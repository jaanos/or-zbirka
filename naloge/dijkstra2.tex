\begin{naloga}{Sergio Cabello}{Vaje OR 7.12.2016}
\begin{vprasanje}
Zasnuj različico Dijkstrovega algoritma
za iskanje najkrajše poti med vozliščema $s$ in $t$ v grafu $G$,
ki iskanje hkrati začne v vozliščih $s$ in $t$.
Kdaj naj se iskanje konča in kako naj se poišče rešitev?
\end{vprasanje}

\begin{odgovor}
Zapišimo algoritem, ki poišče najkrajšo pot od $s$ do $t$,
pri čemer se preiskovanje začne v obeh koncih poti hkrati.
Tako bomo v prednostno vrsto vsako vozlišče postavili dvakrat,
preiskovanje pa bomo končali,
ko bomo neko vozlišče drugič odstranili iz vrste.
Iskana pot bo šla bodisi skozi to vozlišče,
bodisi po povezavi med vozliščema,
ki smo ju odstranili iz vrste po tem,
ko sta bili doseženi iz različnih koncev iskane poti.
\begin{small}
\begin{algorithmic}
\Function{Dijkstra2}{$G = (V, E), s, t, L : E \rightarrow \R_{+}$}
	\State $Q \gets$ prednostna vrsta
        s prioriteto $\infty$ za $(v, 0)$ in $(v, 1)$
        za vsako vozlišče $v \in V$
	\State $d_0 \gets$ prazen slovar
	\State $d_1 \gets$ prazen slovar
    \State $\pred_0 \gets$ slovar z vrednostjo $\Null$ za vsak $v \in V$
	\State $\pred_1 \gets$ slovar z vrednostjo $\Null$ za vsak $v \in V$
	\State $Q[s, 0] \gets 0$
	\State $Q[t, 1] \gets 0$
    \State $(u, i), r \gets Q.\pop()$ \hfill vzamemo prvo vozlišče iz vrste
	\While{$(u, 1-i) \in Q$} \hfill nadaljujemo, če je vozlišče še v vrsti
        \State $d_i[u] \gets r$ \hfill zapomnimo si razdaljo
		\For{$v \in \Adj(G, u)$} \hfill posodobimo razdalje sosedom
			\If{$(v, i) \in Q \land Q[v, i] > d_i[u] + L_{uv}$}
				\State $Q[v, i] \gets d_i[u] + L_{uv}$
                \State $\pred_i[v] \gets u$
			\EndIf
		\EndFor
        \If{$Q.\isEmpty()$} \hfill če je vrsta prazna, ni poti od $s$ do $t$
            \State \Return $(\infty, \Null)$
        \EndIf
        \State $(u, i), r \gets Q.\pop()$ \hfill vzamemo naslednje vozlišče
	\EndWhile
    \State $r, u \gets \min(\{(r + d_{1-i}[u], u)\} \cup
                            \set{(d_i[v] + Q[v, 1-i], v)}{v \in d_i})$
        \hfill poiščemo vozlišče,
    \State $v \gets u$ \hfill preko katerega poteka najkrajša pot
    \State {\sl pot}$_0 \gets [u]$
    \While{$\pred_0[u] \ne \Null$} \hfill rekonstruiramo del poti od $s$
        \State $u \gets \pred_0[u]$
        \State {\sl pot}$_0.\append(u)$
    \EndWhile
    \State {\sl pot}$_0.\reverse()$
    \State {\sl pot}$_1 \gets []$
    \While{$\pred_1[v] \ne \Null$} \hfill rekonstruiramo del poti do $t$
        \State $v \gets \pred_1[v]$
        \State {\sl pot}$_1.\append(v)$
    \EndWhile
    \State \Return ($r$, {\sl pot}$_0 +${\sl pot}$_1$)
\EndFunction
\end{algorithmic}
\end{small}
Pokažimo, da algoritem deluje pravilno.
Naj bosta $S$ in $T$ množici vozlišč,
ki jih algoritem ob izteku prve zanke {\bf while} doseže iz $s$ oziroma $t$
(tj., odstrani ustrezni vnos v $Q$)
-- do teh vozlišč torej poznamo dolžino najkrajše poti od $s$ oziroma do $t$.
V preseku množic $S$ in $T$ je natanko eno vozlišče, in sicer $u$.
Za vsako vozlišče $v \not\in S \cup T$
velja $d_G(s, v) \ge d_G(s, u)$ in $d_G(v, t) \ge d_G(u, t)$,
tako da če najkrajša pot od $s$ do $t$ vsebuje vozlišče $v$,
obstaja tudi najkrajša pot, ki vsebuje vozlišče $u$.
Tako vidimo, da najkrajša pot med $s$ in $t$ bodisi vsebuje vozlišče $u$,
bodisi vsebuje povezavo s krajiščema v $S$ in $T$.

Če najkrajša pot med $s$ in $t$ bodisi vsebuje vozlišče $u$,
potem algoritem očitno pravilno najde najkrajšo pot.
Denimo sedaj, da najkrajša pot med $s$ in $t$ vsebuje povezavo $vw$,
kjer je $v \in S$ in $w \in T$.
Očitno tudi najkrajši poti od $s$ do $w$ in od $v$ do $t$
vsebujeta povezavo $vw$
(sicer bi obstajala krajša pot od $s$ do $t$ preko $v$ ali $w$),
zato ob izteku prve zanke {\bf while}
velja $d_G(s, w) = Q[w, 0]$ in $d_G(v, t) = Q[v, 1]$.
Tako algoritem obravnava najkrajšo pot
od $s$ do $t$ preko vozlišča $v$ ali $w$
in zato pravilno najde najkrajšo pot.

Tako kot Dijkstrov algoritem
tudi zgornji algoritem teče v času $O(m \log n)$
(kjer je $n$ število vozlišč in $m$ število povezav podanega grafa
-- ob predpostavki, da je graf povezan),
če za prednostno vrsto uporabimo kopico.
\end{odgovor}
\end{naloga}
