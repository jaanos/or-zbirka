\begin{naloga}{Bašić, Gajser}{Izpit OR 4.9.2012}
\begin{vprasanje}
Antonio se preživlja z graviranjem na zlato.
Ravnokar je dobil naročilo za graviranje
dveh zelo umetelno izdelanih zaročnih prstanov.
Če ne uspe izdelati prstanov v dogovorjenem času,
bo moral plačati $1\,600 €$ kazni.
Prstane lahko naroči pri zlatarju, ki ga je izbral naročnik.
Ta zlatar bo izdelavo vsakega prstana zaračunal po $100 €$,
varovana dostava pa (ne glede na število prstanov) stane $200 €$.
Ker prstani niso čisto homogeni,
je verjetnost, da graviranje uspe, le $50 \%$.
Ves ostali material (neuspeli prstani, opilki ipd.)
bo moral na koncu vrniti zlatarju (tako je določeno v pogodbi).
Ker je naročilo časovno omejeno, ima časa za največ $2$ naročili pri zlatarju.
Koliko prstanov naj vsakič naroči, da bodo pričakovani stroški čim manjši?
\end{vprasanje}

\begin{odgovor}
Naj bodo $s_{ij}$ pričakovani stroški, ki jih bo imel Antonio,
če mora naročiti $i$ prstanov, na voljo pa ima še $j$ naročil.
Zapišimo začetne pogoje in rekurzivne enačbe.
\begin{align*}
s_{ij} &= 0 € & (i \le 0, \ 0 \le j \le 2) \\
s_{i0} &= 1\,600 € & (1 \le i \le 2) \\
s_{ij} &= 200 € + \min\set{h \cdot 100 € + {1 \over 2^h} \sum_{k=0}^h {h \choose k} s_{i-h, j-1}}{h \ge 1} & (1 \le i, j \le 2)
\end{align*}
Izračunajmo najprej vrednosti $s_{11}$ in $s_{21}$,
nato pa še skupne pričakovane stroške $s^* = s_{22}$.
Opazimo, da minimziramo vsoto
linearno naraščajočega in eksponentno padajočega člena,
tako da lahko računanje ustavimo, ko začne ta vsota naraščati.
\begin{alignat*}{2}
s_{11} &= 200 € + \min\{900 €, 600 €, \underline{500 €}, \underline{500 €}, 550 €, \dots\} &&= 700 € \\
s_{21} &= 200 € + \min\{1\,700 €, 1\,400 €, 1\,100 €, 900 €, 800 €, \underline{775 €}, 800 €, \dots\} &&= 975 € \\
s_{22} &\approx 200€ + \min\{937.5 €, 793.75 €, 684.38 €, \underline{635.94 €}, 639.84 €, \dots\} &&\approx 835.94 €
\end{alignat*}
Antonio naj torej naroči $4$ prstane.
Če uspe graviranje samo enega prstana,
naj v drugo naroči še $3$ ali $4$ prstane,
če pa ne uspe graviranje nobenega prstana,
naj v drugo naroči še $6$ prstanov.
Pričakovani stroški bodo tedaj približno $835.94 €$.
\end{odgovor}
\end{naloga}
