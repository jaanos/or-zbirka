\begin{naloga}{David Gajser}{Izpit OR 14.6.2013}
\begin{vprasanje}
Pohorski škratje že 644.~leto zapored
organizirajo tradicionalni tek po rovih od $a$ do $z$,
ki letos poteka na trasi, prikazani na sliki~\fig.

Udeleženci začnejo v točki $a$ in končajo v točki $z$,
zmeraj pa tečejo le v smeri povezav.
Vsakemu škratu posebej pot določijo organizatorji,
zaradi dobre pri\-prav\-lje\-no\-sti pa lahko pričakujemo,
da bodo vsi prišli do cilja.
Ker so škratje zelo težki,
je med tekom po rovu nevarnost, da se le-ta poruši.
Na povezavah je napisano število udeležencev, ki jih vsak rov prenese.
Organizatorji takoj za tem, ko rov preči en tekmovalec manj,
kot je nosilnost rova, rov zaprejo
(npr. rov $x$ se zapre, ko ga preči $(x-1)$-ti tekmovalec
-- $x$-ti tekmovalec ne stopi v rov).

\begin{enumerate}[(a)]
\item Trenutno stanje trase je takšno, da je $x = 1$ in $y = 2$.
Največ koliko udeležencev lahko preteče od $a$ do $z$?
Odgovor utemelji.
Napiši, kako naj organizatorji pripravijo poti za škrate
(tj., koliko škratov so pripravljeni spustiti po kateri poti).

\item Rova $x$ in $y$ lahko škratje dodatno utrdijo,
ostalih ni moč preurejati.
Da lahko povečajo nosilnost rova za enega udeleženca, morajo delati $10$ dni.
Najmanj koliko dni morajo delati in na katerem rovu,
da bo na tradicionalnem teku od $a$ do $z$
lahko sodelovalo kar največ udeležencev?
Odgovor utemelji.
Napiši, kako naj organizatorji v tem primeru pripravijo poti za škrate
(tj., koliko škratov so pripravljeni spustiti po kateri poti).
\end{enumerate}

\begin{slika}
\pgfslika
\podnaslov{Graf}
\end{slika}
\end{vprasanje}

\begin{odgovor}
Delali bomo z omrežjem,
v katerem je kapaciteta povezav za $1$ manjša od nosilnosti ustreznega rova.

\begin{enumerate}[(a)]
\item Začetno omrežje je prikazano na sliki~\fig[rovi-resitev1],
kjer sta prikazani tudi povečujoči poti
$a \to f \to z$ in $a \to c \to g \to z$ s kapaciteto $1$.
Naslednji dve poti,
$a \to e \to f \to g \to z$ in $a \to f \to d \to g \to z$,
prav tako s kapaciteto $1$,
sta prikazani na sliki~\fig[rovi-resitev2].
Tako dobimo omrežje s slike~\fig[rovi-resitev3],
iz katerega je razvidno, da je maksimalni pretok $4$.
Organizatorji lahko torej po trasi spustijo $4$ škrate,
npr. po vsaki od prej omenjenih poti po enega.

\item Iz slike~\fig[rovi-resitev3] je razvidno,
da gre povezava $c \to f$ v nasprotni smeri minimalnega prereza,
tako da s povečevanjem njene kapacitete ne bomo povečali maksimalnega pretoka,
medtem ko je povezava $f \to d$ v minimalnem prerezu.
Za primer $y > 2$ je prikazana je povečujoča pot
$a \to e \to f \to d \to c \to g \to z$,
po kateri lahko pretok povečamo za največ $3$.

Dobljeno omrežje za primer $y \ge 5$ je prikazano na sliki~\fig[rovi-resitev4].
Opazimo,
da lahko v primeru $y = 5$ v prikazanem minimalnem prerezu
povezavi $d \to c$ in $d \to g$ nadomestimo s povezavo $f \to d$.
Tako vidimo, da se nam vrednost $y$ izplača povečati na največ $5$
-- ker je bila predhodna vrednost enaka $2$,
to ustreza $30$ dnem dela.
Tedaj bodo organizatorji lahko spustili še $3$ škrate po prej omenjeni poti,
s čimer bodo lahko število udeležencev povečali na $7$.
\end{enumerate}

\begin{slika}
\pgfslika[rovi-resitev1]
\podnaslov[\res{}(a)]{Omrežje in prvi korak}
\end{slika}
\begin{slika}
\pgfslika[rovi-resitev2]
\podnaslov[\res{}(a)]{Drugi korak}
\end{slika}
\begin{slika}
\pgfslika[rovi-resitev3]
\podnaslov[\res{}(a) ter povečuječa pot pri $y > 2$]%
    {Maksimalni pretok in minimalni prerez}
\end{slika}
\begin{slika}
\pgfslika[rovi-resitev4]
\podnaslov[\res{}(b)]{Maksimalni pretok in minimalni prerez}
\end{slika}
\end{odgovor}
\end{naloga}
