\begin{naloga}{Blaž Jelenc}{Izpit OR 24.6.2014}
\begin{vprasanje}
Tovarna od dobavitelja dobi paket $10$ komponent $A$.
Tovarna sestavlja izdelke $B$,
kjer gre v vsak izdelek $B$ natanko ena komponenta $A$.
Dane so verjetnosti
$$
p_i = P(\text{med $10$ komponentami $A$ je natanko $i$ pokvarjenih}),
$$
kjer je $p_1 = 0.7$, $p_2 = 0.2$ in $p_3 = 0.1$.
Stroški pregleda posamezne komponente so $45 €$.
Stroški popravila izdelka $B$ zaradi vgrajene okvarjene komponente $A$
so enaki $350 €$.
\begin{enumerate}[(a)]
\item Denimo, da ima tovarna na izbiro samo dve možnosti.
Prva je, da pregleda vseh $10$ dobavljenih izdelkov.
Druga možnost pa je, da ne pregleda nobenega dobavljenega izdelka.
Katera odločitev tovarni povzroči manj stroškov?

\item Naj ima sedaj tovarna na voljo drugačni izbiri.
Prva je, da naključno izbere eno izmed $10$ komponent $A$,
jo pregleda in po potrebi zamenja,
ter gre nato direktno v proizvajanje izdelkov $B$.
Druga možnost pa je, da pregleda vseh $10$ izdelkov.
Katera odločitev tovarni povzroči manj stroškov?
\end{enumerate}
\end{vprasanje}
\begin{odgovor}
\end{odgovor}
\end{naloga}
