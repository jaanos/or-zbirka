\begin{naloga}{Blaž Jelenc}{Izpit OR 24.6.2014}
\begin{vprasanje}
Tovarna od dobavitelja dobi paket $10$ komponent $A$.
Tovarna sestavlja izdelke $B$,
kjer gre v vsak izdelek $B$ natanko ena komponenta $A$.
Dane so verjetnosti
$$
p_i = P(\text{med $10$ komponentami $A$ je natanko $i$ pokvarjenih}),
$$
kjer je $p_1 = 0.7$, $p_2 = 0.2$ in $p_3 = 0.1$.
Stroški pregleda posamezne komponente so $45 €$.
Stroški popravila izdelka $B$ zaradi vgrajene okvarjene komponente $A$
so enaki $350 €$.
\begin{enumerate}[(a)]
\item Denimo, da ima tovarna na izbiro samo dve možnosti.
Prva je, da pregleda vseh $10$ dobavljenih izdelkov.
Druga možnost pa je, da ne pregleda nobenega dobavljenega izdelka.
Katera odločitev tovarni povzroči manj stroškov?

\item Naj ima sedaj tovarna na voljo drugačni izbiri.
Prva je, da naključno izbere eno izmed $10$ komponent $A$,
jo pregleda in po potrebi zamenja,
ter gre nato direktno v proizvajanje izdelkov $B$.
Druga možnost pa je, da pregleda vseh $10$ izdelkov.
Katera odločitev tovarni povzroči manj stroškov?
\end{enumerate}
\end{vprasanje}
\begin{odgovor}
Definiramo si spremenljivke in izpišemo znane podatke.
\begin{itemize}
\item[] $X$ \dots strošek
\item[] $C$ \dots izberemo pokvarjen izdelek
\item[] $D_i$ \dots imamo $i$ pokvarenih izdelkov
\item[] $E$ \dots tovarna pregleda vseh $10$ izdelkov
\item[] $F$ \dots izberemo naključen izdelek
\end{itemize}
\begin{align*}
p_1 &= 0,7 & p_2 &= 0,2 & p_3 &= 0,1
\end{align*}

\begin{enumerate}[(a)]
\item Izračunajmo, koliko tovarno stane, da pregleda vse izdelke in popravi pokvarjene. Vemo, da podjetje plača $45 € \cdot 10$ za $10$ pregledov.
\begin{align*}
D_1 &: 450 + 350 = 800 €&
D_2&: 450 + 2 \cdot 350 = 1150 €\\
D_3&: 450 + 3 \cdot 350 = 1500 €
\end{align*}
Pričakovan strošek, če podjetje pregleda vseh $10$ izdelkov je torej
$$
E(X|E) = 0,7 \cdot 800 + 0,2 \cdot 1150 + 0,1 \cdot 1500 = 940 €
$$
Sedaj si poglejmo še koliko znaša pričakovan strošek, če tovarna ne pregleda nobenega izdelka. 
$$
E(X|\neg E) = 0,7 \cdot 350 + 0,2 \cdot 2 \cdot 350 + 0,1 \cdot 3 \cdot 350 = 490 €
$$
Če dobljena stroška primerjamo, lahko vidimo, da bima podjetje manjši strošek, če ne pregleda nobenega dobavljenega izdelka.

\item Strošek za pregled vseh komponent je enak, kot v primeru (a) in sicer $940$. \\
Izračunajmo še pričakovani strošek, če tovarna pogleda eno naključno komponento. Priblem ločimo na tri dele: imamo $1$ pokvarjen izdelek, $2$ pokvarena ali $3$. Vsakega od teh primerov ločimo še na dva dela: ali izberemo pokvarjen izdelek ali ne.
\begin{itemize}

\item[] $D_1$: 
\begin{align*}
&P(C)  = \frac{1}{10} &X &= 45 €\\
&P(\neg C) = \frac{9}{10} & X &= 45 + 350 = 395 €\\
&E(X|D_1)= \frac{1}{10} \cdot 45 + \frac{9}{10} \cdot 395 =  360 €
\end{align*}
\item[] $D_2: E(X|D_2) = \frac{2}{10} \cdot 395 + \frac{8}{10} \cdot 745 = 675 €$
\item[] $D_3: E(X|D_3) = \frac{3}{10} \cdot 745 + \frac{7}{10} \cdot 1095 = 990 €$
\end{itemize}
Sedaj lahko izračunamo pričakovan strošek, če tovarna izbere naključen izdelek, ga pregleda in po potrebi popravi:
$$
E(X|F) = 0,7 \cdot 360 + 0,2 \cdot 675 + 0,1 \cdot 990 = 486 €
$$

Vidimo, da velja $E(X|F) < E(X|E)$, zato je za tovarno bolje, da izbere eno naključno komponento, ki jo pregleda, kot če pregleda vseh $10$.

\end{enumerate}
\end{odgovor}
\end{naloga}
