\begin{naloga}{Janoš Vidali}{Izpit OR 5.7.2018}
\begin{vprasanje}
V trgovini s pohištvom vsak mesec prodajo $20$ sedežnih garnitur.
Z vsakim naročilom imajo $750 €$ stroškov,
pri tem pa vse naročeno blago dobijo hkrati.
Skladiščenje posamezne sedežne garniture jih stane $10 €$ na mesec,
primanjkljaj za en kos pa jih stane $50 €$ na mesec.

\begin{enumerate}[(a)]
\item Kako pogosto naj v trgovini naročajo sedežne garniture,
da bodo stroški čim manjši?
Kako veliko skladišče morajo imeti?
Izračunaj tudi enotske stroške.

\item Izračunaj največji dovoljeni primanjkljaj.
Koliko sedežnih garnitur naj vsakič naročijo?

\item V trgovini dobijo ponudbo za uporabo drugega skladišča,
v katerem imajo za po\-sa\-mez\-no se\-dež\-no garnituro
le $6 €$ stroškov na mesec,
vendar lahko hrani največ $40$ sedežnih garnitur.
Kako naj organizirajo naročanje,
če namesto prvotnega uporabljajo to skladišče?
Ali se jim ga splača uporabiti?
\namig{izpelji formulo za enotske stroške
in poišči optimalen interval naročanja pri fiksni velikosti skladišča.}
\end{enumerate}
\end{vprasanje}

\begin{odgovor}
Imamo sledeče podatke (pri časovni enoti $1$ mesec).
\begin{align*}
\nu &= 20 &
K &= 750 € \\
s &= 10 € &
p &= 50 €
\end{align*}
Pri teh podatkih izračunajmo še
$$
\beta = 1 + {s \over p} = 1 + {10 \over 50} = 1.2 .
$$
Ker artiklov ne proizvajamo
(tj., ob dostavi imamo na voljo celotno količino naročila),
vzamemo $\lambda = \infty$ in torej $\alpha = \nu$.

\begin{enumerate}[(a)]
\item Izračunajmo optimalno dolžino cikla $\tau^*$,
največjo zalogo $M^*$ in enotske stroš\-ke $S^*$.
\begin{alignat*}{3}
\tau^* &= \sqrt{2K \beta \over s \alpha}
&&= \sqrt{1\,800 \over 200} &&= 3 \\
M^* &= \sqrt{2K \alpha \over s \beta}
&&= \sqrt{30\,000 \over 12} &&= 50 \\
S^* &= \sqrt{2Ks \alpha \over \beta}
&&= \sqrt{300\,000 \over 1.2} € &&= 500 €
\end{alignat*}
Optimalna dolžina cikla je torej $3$ mesece,
pri tem pa mora trgovina imeti skladišče za $50$ sedežnih garnitur.
Mesečni stroški nabave in skladiščenja so $60 €$.

\item Izračunajmo še največji primanjkljaj $m^*$ in velikost naročila $q^*$.
\begin{alignat*}{3}
m^* &= \tau^* \alpha - M^* &&= 3 \cdot 20 - 50 &&= 10 \\
q^* &= \tau^* \nu &&= 3 \cdot 20 &&= 60
\end{alignat*}

\item Drugo skladišče lahko hrani $n' = 40$ sedežnih garnitur,
mesečni strošek hrambe enega kosa pa je $s' = 6 €$.
Najprej preverimo,
kakšna bi bila optimalna velikost skladišča pri takih stroških.
\begin{alignat*}{3}
\beta' &= 1 + {s' \over p} &&= 1 + {6 \over 50} &&= 1.12 \\
M' &= \sqrt{2K \alpha \over s' \beta'}
&&= \sqrt{30\,000 \over 6.72} &&\approx 66.815
\end{alignat*}
Ker je $M' > n'$,
zapišimo formulo za enotske stroške za primer,
ko imamo v skladišču največ $n'$ kosov.
$$
S' = {2K \alpha + (s'+p) n'^2 \over 2 \alpha \tau'}
+ p \left({\alpha \tau' \over 2} - n'\right)
= 500 € \cdot \tau' - 2\,000 € + {2\,990 € \over \tau'}
$$
Ker iščemo dolžino intervala, kjer dosežemo minimalne stroške, poiščimo,
kje ima odvod zgornjega izraza po $\tau'$ ničlo za $\tau' > 0$.
\begin{align*}
0 € &= 500 € - {2\,990 € \over \tau'^2} \\
\tau' &= \sqrt{2\,990 \over 500} \approx 2.445
\end{align*}
Optimalna dolžina cikla ob uporabi drugega skladišča je torej $2.445$ meseca.
Vstavimo v zgornjo formulo, da dobimo mesečne stroške.
$$
S' = 500 € \cdot 2.445 - 2\,000 € + {2\,990 € \over 2.445} \approx 445.404 €
$$
Ker so stroški nižji kot pri prvem skladišču,
se trgovini izplača sprejeti ponudbo.
\end{enumerate}
\end{odgovor}
\end{naloga}
