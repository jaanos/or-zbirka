\begin{naloga}{Janoš Vidali}{Izpit OR 3.6.2019}
\begin{vprasanje}
V trgovini z oblačili sestavljajo ponudbo za poletno sezono.
Odločijo se lahko za nakup kolekcij $n$ različnih proizvajalcev,
pri čemer za posamezen izvod kolekcije $i$-tega proizvajalca ($1 \le i \le n$)
plačajo ceno $c_i$ evrov,
od nje pa si obetajo zaslužek $d_i$ evrov.
Kolekcijo $i$-tega proizvajalca lahko kupijo v največ $k_i$ izvodih
(vsak izvod je nedeljiva enota),
z njeno prisotnostjo v ponudbi (ne glede na količino)
pa si obetajo povečano vidnost in s tem dodaten zaslužek $e_i$ evrov
(npr.~iz prodaje stalne ponudbe in starih zalog).

Za nakup kolekcij imajo na voljo $C$ evrov,
iz marketinških razlogov pa želijo kupiti
kolekcije največ $K$ različnih proizvajalcev.
Poleg tega proizvajalec $p$ zahteva,
da če kupijo kak izvod njegove kolekcije,
je morajo kupiti v vsaj toliko izvodih
kot kolekciji proizvajalcev $q$ in $r$ skupaj
(če pa ne kupijo nobenega izvoda kolekcije proizvajalca $p$, te omejitve ni).
V trgovini želijo maksimizirati pričakovani dobiček
(tj., razliko zaslužka od prodaje s stroški nakupa).

Zapiši celoštevilski linearni program, ki modelira zgoraj opisani problem.
\end{vprasanje}

\begin{odgovor}
Za $i$-to kolekcijo $(1 \le i \le n)$
bomo uvedli spremenljivki $x_i$ in $y_i$,
kjer je $x_i$ število kupljenih izvodov $i$-te kolekcije,
vrednost $y_i$ pa interpretiramo kot
$$
y_i = \begin{cases}
1, & \text{kupimo vsaj en izvod $i$-te kolekcije, in} \\
0  & \text{sicer.}
\end{cases}
$$
Zapišimo celoštevilski linearni program.
\begin{alignat*}{3}
&& \max &\ \sum_{i=1}^n ((d_i - c_i) x_i + e_i y_i) &&\quad \text{p.p.}
\opis{Omejitev števila izvodov}
\forall i \in \{1, \dots, n\}: && 0 \le x_i &\le k_i, &&\quad x_i \in \Z \\
\forall i \in \{1, \dots, n\}: && 0 \le y_i &\le 1, &&\quad y_i \in \Z
\opis{Povezava med $x_i$ in $y_i$}
\forall i \in \{1, \dots, n\}: && y_i \le x_i &\le k_i y_i
\opis{Omejitev kapitala za nakup}
&& \sum_{i=1}^n c_i x_i &\le C
\opis{Omejitev števila proizvajalcev}
&& \sum_{i=1}^n y_i &\le K
\opis{Omejitev proizvajalca $p$}
&& x_q + x_r - x_p &\le (k_q + k_r) (1 - y_p)
\end{alignat*}
\end{odgovor}
\end{naloga}
