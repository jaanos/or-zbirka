\begin{naloga}{Janoš Vidali}{Izpit OR 3.6.2019}
\begin{vprasanje}
V trgovini z oblačili sestavljajo ponudbo za poletno sezono.
Odločijo se lahko za nakup kolekcij $n$ različnih proizvajalcev,
pri čemer za posamezen izvod kolekcije $i$-tega proizvajalca ($1 \le i \le n$)
plačajo ceno $c_i$ evrov,
od nje pa si obetajo zaslužek $d_i$ evrov.
Kolekcijo $i$-tega proizvajalca lahko kupijo v največ $k_i$ izvodih
(vsak izvod je nedeljiva enota),
z njeno prisotnostjo v ponudbi (ne glede na količino)
pa si obetajo povečano vidnost in s tem dodaten zaslužek $e_i$ evrov
(npr.~iz prodaje stalne ponudbe in starih zalog).

Za nakup kolekcij imajo na voljo $C$ evrov,
iz marketinških razlogov pa želijo kupiti
kolekcije največ $K$ različnih proizvajalcev.
Poleg tega proizvajalec $p$ zahteva,
da če kupijo kak izvod njegove kolekcije,
je morajo kupiti v vsaj toliko izvodih
kot kolekciji proizvajalcev $q$ in $r$ skupaj
(če pa ne kupijo nobenega izvoda kolekcije proizvajalca $p$, te omejitve ni).
V trgovini želijo maksimizirati pričakovani dobiček
(tj., razliko zaslužka od prodaje s stroški nakupa).

Zapiši celoštevilski linearni program, ki modelira zgoraj opisani problem.
\end{vprasanje}

\begin{odgovor}
\end{odgovor}
\end{naloga}
