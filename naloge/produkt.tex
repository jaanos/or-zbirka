\begin{naloga}{?}{Vaje OR 30.3.2016}
\begin{vprasanje}
Podjetje je razvilo produkt,
za katerega je konkurenca pripravljena plačati $15 \ME$.
Če se odločijo samostojno prodajati produkt,
jih vzpostavitev proizvodnje stane $6 \ME$,
za vsak uspešno prodan produkt pa dobijo $600 €$.
Računajo, da bi z ve\-rjet\-nost\-jo $0.5$ investicija uspela
in bi prodali $100000$ produktov,
z verjetnostjo $0.5$ pa bi projekt propadel
in bi prodali zgolj $10000$ produktov.
Podjetje se lahko odloči tudi za neodvisno raziskavo trga.
Ta stane $1 \ME$,
z verjetnostjo $2/3$ pa bo pravilno napovedala uspeh projekta
(ne glede na to, ali bi ta uspel ali ne).
Kako naj se podjetje odloči?

\end{vprasanje}
\begin{odgovor}
\end{odgovor}
\end{naloga}
