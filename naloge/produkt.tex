\begin{naloga}{?}{Vaje OR 30.3.2016}
\begin{vprasanje}
Podjetje je razvilo produkt,
za katerega je konkurenca pripravljena plačati $15 \ME$.
Če se odločijo samostojno prodajati produkt,
jih vzpostavitev proizvodnje stane $6 \ME$,
za vsak uspešno prodan produkt pa dobijo $600 €$.
Računajo, da bi z ve\-rjet\-nost\-jo $0.5$ investicija uspela
in bi prodali $100000$ produktov,
z verjetnostjo $0.5$ pa bi projekt propadel
in bi prodali zgolj $10000$ produktov.
Podjetje se lahko odloči tudi za neodvisno raziskavo trga.
Ta stane $1 \ME$,
z verjetnostjo $2/3$ pa bo pravilno napovedala uspeh projekta
(ne glede na to, ali bi ta uspel ali ne).
Kako naj se podjetje odloči?
\end{vprasanje}

\begin{odgovor}
Če se podjetje odloči za samostojno prodajo,
bo ob uspehu imelo $100000 \cdot 600 € - 6 \ME = 54 \ME$ dobička,
ob neuspehu pa $10000 \cdot 600 € - 6 \ME = 0 \ME$.
Ker je pričakovani dobiček enak
$$
{1 \over 2} \cdot 54 \ME + {1 \over 2} \cdot 0 \ME = 27 \ME,
$$
kar je več kot $15 \ME$,
kolikor bi zaslužili ob prodaji konkurenci,
se podjetju (v odsotnosti raziskave) bolj izplača samostojna prodaja.

Zanesljivost raziskave je tako v primeru uspeha kot neuspeha enaka $2/3$,
torej
\begin{align*}
P(\text{raziskava napove uspeh} \;|\; \text{investicija uspe})
&= {2 \over 3} \\
P(\text{raziskava napove uspeh} \;|\; \text{investicija ne uspe})
&= {1 \over 3}
\end{align*}
Izračunajmo še verjetnosti napovedi raziskave
ter uspešnosti investicije ob vsaki napovedi.
\begin{align*}
P(\text{raziskava napove uspeh}) &=
{2 \over 3} \cdot {1 \over 2} + {1 \over 3} \cdot {1 \over 2} = {1 \over 2} \\
P(\text{raziskava napove neuspeh}) &=
{1 \over 3} \cdot {1 \over 2} + {2 \over 3} \cdot {1 \over 2} = {1 \over 2} \\
P(\text{investicija uspe} \;|\; \text{raziskava napove uspeh})
&= {2/3 \cdot 1/2 \over 1/2} = {2 \over 3} \\
P(\text{investicija ne uspe} \;|\; \text{raziskava napove uspeh})
&= {1/3 \cdot 1/2 \over 1/2} = {1 \over 3} \\
P(\text{investicija uspe} \;|\; \text{raziskava napove neuspeh})
&= {1/3 \cdot 1/2 \over 1/2} = {1 \over 3} \\
P(\text{investicija ne uspe} \;|\; \text{raziskava napove neuspeh})
&= {2/3 \cdot 1/2 \over 1/2} = {2 \over 3}
\end{align*}
S pomočjo zgoraj izračunanih verjetnosti
lahko narišemo odločitveno drevo s slike~\fig{}.
Opazimo, da rezultat raziskave ne vpliva na našo odločitev o prodaji,
zato se za raziskavo ne odločimo
in tako pričakujemo dobiček $27 \ME$.

\begin{slika}
\makebox[\textwidth][c]{
\pgfslika
}
\podnaslov{Odločitveno drevo}
\end{slika}
\end{odgovor}
\end{naloga}
