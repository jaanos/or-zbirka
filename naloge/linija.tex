\begin{naloga}{Janoš Vidali}{Izpit OR 11.6.2018}
\begin{vprasanje}
Avtobusno podjetje želi uvesti novo avtobusno linijo.
Dan je neusmerjen enostaven graf $G = (V, E)$,
katerega vozlišča predstavljajo postajališča,
povezave pa ceste med njimi.
Za vsako povezavo $uv \in E$ poznamo še čas $c_{uv} \in \N$,
v katerem avtobus pride od $u$ do $v$.

Dan je še seznam postajališč $p_1, p_2, \dots p_n \in V$,
ki jih linija mora obiskati v tem vrstnem redu
-- začeti mora v $p_1$ in končati v $p_n$,
na poti med dvema postajališčema iz seznama
pa lahko obišče tudi druga postajališča.
Linija lahko vsako postajališče obišče največ enkrat
(ko doseže končno postajališče, se vrne po isti poti do začetnega).
Skupno trajanje vožnje (v eno smer) je lahko največ $T$.
Podjetje želi določiti linijo tako, da bo obiskala čim več postaj.

Zapiši celoštevilski linearni program, ki modelira zgoraj opisani problem.
\namig{poskrbi, da linija nima ciklov.}
\end{vprasanje}

\begin{odgovor}
Naj bo $m$ število vozlišč grafa $G$.
Očitno bo linija obiskala največ $m$ postaj,
zato bomo za postajališče $u \in V$ in $i \in \{1, \dots, m\}$
uvedli spremenljivko $x_{ui}$,
za vsako povezavo $uv \in E$ pa še spremenljivko $y_{uv}$.
Njihove vrednosti interpretiramo kot
\begin{align*}
x_{ui} &= \begin{cases}
1; & \text{postajališče $u$ obiščemo $i$-to po vrsti, in} \\
0  & \text{sicer;}
\end{cases} \\
\text{ter} \quad
y_{uv} &= \begin{cases}
1; & \text{postajališči $u$ in $v$ obiščemo zaporedoma, in} \\
0  & \text{sicer.}
\end{cases}
\end{align*}
\needspace{\baselineskip}
Zapišimo celoštevilski linearni program.
\begin{alignat*}{3}
&& \max &\ \sum_{u \in V} \sum_{i=1}^n x_{ui} & \text{p.p.} \\
\forall u \in V \ \forall i \in \{1, \dots, m\}: &\ &
0 \le x_{ui} &\le 1, &\quad x_{ui} &= \Z \\
\forall uv \in E: &\ &
0 \le y_{uv} &\le 1, &\quad y_{uv} &= \Z
\opis{Vsako postajališče obiščemo največ enkrat}
\forall u \in V: &\ & \sum_{i=1}^m x_{ui} &\le 1
\opis{Naenkrat obiščemo največ eno postajališče}
\forall i \in \{1, \dots, m\}: &\ & \sum_{u \in V} x_{ui} &\le 1
\opis{Indeksov ne izpuščamo}
\forall i \in \{2, \dots, m\}: &\ &
\sum_{u \in V} x_{ui} &\le \sum_{u \in V} x_{u,i-1}
\opis{Nesosedna postajališča niso zaporedna}
\forall uv \not\in E \ \forall i \in \{2, \dots, m\}: &\ &
x_{u,i-1} + x_{ui} + x_{v,i-1} + x_{vi} &\le 1
\opis{Začnemo v postajališču $p_1$}
&& x_{p_1,1} &= 1
\opis{Postajališča iz seznama si sledijo v podanem vrstnem redu}
\forall i \in \{1, \dots, m\} \ \forall j \in \{2, \dots, n\}: &\ &
x_{p_{j-1},i} &\le \sum_{h=i+1}^m x_{p_j,h}
\opis{Končamo v postajališču $p_n$}
\forall i \in \{1, \dots, m\}: &\ &
(m-i) x_{p_n,i} + \sum_{u \in V} \sum_{h=i+1}^m x_{uh} &\le m-i
\opis{Časovna omejitev}
\forall uv \in E \ \forall i \in \{2, \dots, m\}: &\ &
x_{u,i-1} + x_{ui} + x_{v,i-1} + x_{vi} &\le y_{uv} + 1 \\
&& \sum_{uv \in E} c_{uv} y_{uv} &\le T
\end{alignat*}
\end{odgovor}
\end{naloga}
