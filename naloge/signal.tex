\begin{naloga}{Janoš Vidali}{Kolokvij OR 19.4.2019}
\begin{vprasanje}
Gradimo avtocesto skozi puščavo in želimo zagotoviti,
da bo v celoti pokrita z mobilnim signalom.
Cesta je ravna in dolga $n$ milj ($n \in \Z$),
na vsako miljo pa imamo možnost postaviti bazno postajo z dosegom $1$ miljo.
Cena postavitve bazne postaje na $i$-ti milji
je podana s parametrom $a_i$ ($0 \le i \le n$).
Predpostaviš lahko, da so vse cene pozitivne.
Pri obstoječi infrastrukturi je pokrita le začetna točka avtoceste
(tj., milja $0$).
Poiskati želimo torej čim cenejšo postavitev baznih postaj,
da je vsaka točka avtoceste pokrita s signalom.

\medskip
\noindent
{\bf Primer:} če postavimo postaji na milji $0$ in $3$,
potem je interval $(1, 2)$ nepokrit.
Če pa npr.~postajo namesto na milji $0$ postavimo na milji $1$,
smo tako v celoti pokrili interval $[0, 4]$.

\begin{enumerate}[(a)]
\item Zapiši rekurzivne enačbe za reševanje danega problema.
Razloži, kaj pred\-stav\-lja\-jo spremenljivke,
v kakšnem vrstnem redu jih računamo,
ter kako dobimo optimalno rešitev.

\item Oceni časovno zahtevnost algoritma, ki sledi iz zgoraj zapisanih enačb.

\item S svojim algoritmom poišči optimalno postavitev postaj za $n = 10$ ter
$$
(a_i)_{i=0}^{10} = (4,  6,  1, 10, 14, 21, 15,  6, 10,  3,  2) .
$$

\item Denimo, da lahko postavimo tudi večje bazne postaje z dosegom $2$ milji,
pri čemer je cena postavitve take postaje na $i$-ti milji
podana s parametrom $b_i$ ($0 \le i \le n$).
Zapiši rekurzivne enačbe, ki bodo upoštevale tudi to možnost.

\item Problem iz točke (d) reši s podatki iz točke (c) in
$$
(b_i)_{i=0}^{10} = (10, 12, 3, 18, 24, 25, 20, 11, 16,  7,  4) .
$$
\end{enumerate}
\end{vprasanje}

\begin{odgovor}
\begin{enumerate}[(a)]
\item Naj bo $p_i$ najmanjša cena postavitve baznih postaj
do $i$-te milje tako,
da se v celoti pokrije interval $[0, i]$.
Določimo začetne pogoje in rekurzivne enačbe.
\begin{align*}
p_{-1} &= p_0 = 0 \\
p_i &= \min\{a_{i-1} + p_{i-2}, a_i + p_{i-1}\} && (1 \le i \le n)
\end{align*}
Vrednosti $p_i$ računamo naraščajoče po indeksu $i$ ($0 \le i \le n$).
Najmanjšo ceno postavitve baznih postaj dobimo s $p^* = p_n$.

\item Za izračun vrednosti $p_i$ za posamezen $i$
potrebujemo konstantno mnogo časa.
Ker ta izračun opravimo $(n-1)$-krat,
je torej časovna zahtevnost ustreznega algoritma $O(n)$.

\item Izračunajmo vrednosti $p_i$ ($1 \le i \le 10$).
\begin{alignat*}{2}
p_1    &= \min\{\underline{4+0}, 6+0\}      &&= 4  \\
p_2    &= \min\{6+0, \underline{1+4}\}      &&= 5  \\
p_3    &= \min\{\underline{1+4}, 10+5\}     &&= 5  \\
p_4    &= \min\{\underline{10+5}, 14+5\}    &&= 15 \\
p_5    &= \min\{\underline{14+5}, 21+15\}   &&= 19 \\
p_6    &= \min\{21+15, \underline{15+19}\}  &&= 34 \\
p_7    &= \min\{\underline{15+19}, 6+34\}   &&= 34 \\
p_8    &= \min\{\underline{6+34}, 10+34\}   &&= 40 \\
p_9    &= \min\{10+34, \underline{3+40}\}   &&= 43 \\
p_{10} &= \min\{\underline{3+40}, 2+43\}    &&= 43
\end{alignat*}
Cena postavitve baznih postaj je torej $p^* = p_{10} = 43$.
Poglejmo, kam moramo postaviti bazne postaje.
\begin{align*}
p_{10} &= a_9 + p_8 && \text{postaja na $9$} \\
p_8    &= a_7 + p_6 && \text{postaja na $7$} \\
p_6    &= a_6 + p_5 && \text{postaja na $6$} \\
p_5    &= a_4 + p_3 && \text{postaja na $4$} \\
p_3    &= a_2 + p_1 && \text{postaja na $2$} \\
p_1    &= a_0       && \text{postaja na $0$}
\end{align*}

\item Naj bo $q_i$ najmanjša cena postavitve baznih postaj
(pri čemer dovolimo tudi večje postaje)
do $i$-te milje tako,
da se v celoti pokrije interval $[0, i]$.
Določimo začetne pogoje in rekurzivne enačbe.
\begin{alignat*}{3}
&& b_{-1} &= \infty \\
q_{-3} = q_{-2} &=& q_{-1} = q_0 &= 0 \\
q_i &=&{} \min\{b_{i-2} &+ \min\{q_{i-4},\ q_{i-3}\}, \\
    & &         b_{i-1} &+ \min\{q_{i-3},\ q_{i-2}\}, \\
    & &         b_i &+ \min\{q_{i-2},\ q_{i-1}\}, \\
    & &         a_{i-1} &+ q_{i-2}, \\
    & &         a_i &+ q_{i-1}\} & (1 \le i \le n)
\end{alignat*}
Vrednosti $q_i$ računamo naraščajoče po indeksu $i$ ($0 \le i \le n$).
Najmanjšo ceno postavitve baznih postaj dobimo s $q^* = q_n$.

\item Izračunajmo vrednosti $q_i$ ($1 \le i \le 10$).
\begin{alignat*}{2}
q_1    &= \min\{\infty+0, 10+0, 12+0, \underline{4+0}, 6+0\}    &&= 4  \\
q_2    &= \min\{10+0, 12+0, \underline{3+0}, 6+0, 1+4\}         &&= 3  \\
q_3    &= \min\{12+0, \underline{3+0}, 18+3, 1+4, 10+3\}        &&= 3  \\
q_4    &= \min\{\underline{3+0}, 18+3, 24+3, 10+3, 14+3\}       &&= 3  \\
q_5    &= \min\{18+3, 24+3, 25+3, \underline{14+3}, 21+3\}      &&= 17 \\
q_6    &= \min\{24+3, 25+3, \underline{20+3}, 21+3, 15+17\}     &&= 23 \\
q_7    &= \min\{25+3, \underline{20+3}, 11+17, 15+17, 6+23\}    &&= 23 \\
q_8    &= \min\{\underline{20+3}, 11+17, 16+23, 6+23, 10+23\}   &&= 23 \\
q_9    &= \min\{11+17, 16+23, 7+23, 10+23, \underline{3+23}\}   &&= 26 \\
q_{10} &= \min\{16+23, 7+23, 4+23, \underline{3+23}, 2+26\}     &&= 26
\end{alignat*}
Cena postavitve baznih postaj je torej $q^* = q_{10} = 26$.
Poglejmo, kam moramo postaviti bazne postaje.
\begin{align*}
q_{10} &= a_9 + q_8 && \text{manjša postaja na $9$} \\
q_8    &= b_6 + q_4 && \text{večja postaja na $6$}  \\
q_4    &= b_2 + q_0 && \text{večja postaja na $2$}
\end{align*}
\end{enumerate}
\end{odgovor}
\end{naloga}
