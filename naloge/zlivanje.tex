\begin{naloga}{David Gajser}{Izpit OR 28.8.2013}
\begin{vprasanje}
Asistent pri predmetu Operacijske raziskave je dobil idejo,
kako bi predelal algoritem, ki z zlivanjem uredi seznam celih števil.
Namesto,
da na začetku seznam razdelimo na dva (približno) enako velika seznama,
ga razdelimo na tri (približno) enako velike sezname.
Le-te rekurzivno uredimo ter združimo (``zlijemo''), da dobimo urejen seznam.

Napiši psevdokodo algoritma časovne zahtevnosti $O(n \log n)$,
ki uresniči asistentovo idejo.
V psevdokodi lahko uporabiš algoritem {\sc Zlij},
ki sprejme urejena seznama celih števil $A$ in $B$ dolžin $m$ in $n$
ter vrne urejen seznam števil iz $A$ in $B$ dolžine $m+n$ v času $O(m+n)$.
Psevdokode algoritma {\sc Zlij} ni potrebno pisati.
Utemelji, da je časovna zahtevnosti tvojega algoritma res $O(n \log n)$.
\end{vprasanje}

\begin{odgovor}
Zapišimo algoritem.
\begin{small}
\begin{algorithmic}
\Function{Zlivanje}{$S[0 \dots n-1]$ seznam števil}
    \If{$S.\length() \le 1$}
        \State \Return $S$
    \EndIf
    \State $k \gets \left\lfloor {n \over 3} \right\rfloor$
    \State $\ell \gets \left\lfloor {2n \over 3} \right\rfloor$
    \State $A \gets \text{\sc Zlivanje}(S[0 \dots k-1])$
    \State $B \gets \text{\sc Zlivanje}(S[k \dots \ell-1])$
    \State $C \gets \text{\sc Zlij}(A, B)$
    \State $D \gets \text{\sc Zlivanje}(S[\ell \dots n-1])$
    \State $E \gets \text{\sc Zlij}(C, D)$
    \State \Return $E$
\EndFunction
\end{algorithmic}
\end{small}
Časovno zahtevnost algoritma lahko opišemo z rekurzivno zvezo
$$
T(n) = 3T\left({n \over 3}\right) + O(n) .
$$
Po krovnem izreku je časovna zahtevnost algoritma $T(n) = O(n \log n)$.
\end{odgovor}
\end{naloga}
