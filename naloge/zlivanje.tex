\begin{naloga}{David Gajser}{Izpit OR 28.8.2013}
\begin{vprasanje}
Asistent pri predmetu Operacijske raziskave je dobil idejo,
kako bi predelal algoritem, ki z zlivanjem uredi seznam celih števil.
Namesto,
da na začetku seznam razdelimo na dva (približno) enako velika seznama,
ga razdelimo na tri (približno) enako velike sezname.
Le-te rekurzivno uredimo ter združimo (``zlijemo''), da dobimo urejen seznam.

Napiši psevdokodo algoritma časovne zahtevnosti $O(n \log n)$,
ki uresniči asistentovo idejo.
V psevdokodi lahko uporabiš algoritem {\sc Zlij},
ki sprejme urejena seznama celih števil $A$ in $B$ dolžin $m$ in $n$
ter vrne urejen seznam števil iz $A$ in $B$ dolžine $m+n$ v času $O(m+n)$.
Psevdokode algoritma {\sc Zlij} ni potrebno pisati.
Utemelji, da je časovna zahtevnosti tvojega algoritma res $O(n \log n)$.
\end{vprasanje}
\begin{odgovor}
\end{odgovor}
\end{naloga}
