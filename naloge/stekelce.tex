\begin{naloga}{Bašić, Gajser}{Kolokvij OR 31.5.2012}
\begin{vprasanje}
Samostojna umetnica z nadimkom Stekelce se preživlja z barvanjem na steklo.
Tokrat je dobila naročilo
za izdelavo dveh zelo posebnih pobarvanih steklenih kroglic.
Plačilo je dobila vnaprej, stroške pa bo imela le z nabavo stekla,
saj ima barv in čopičev že dovolj na zalogi.
Če ne uspe izdelati obeh kroglic v dogovorjenem času,
bo morala plačati $800 €$ kazni.
Steklene kroglice lahko naroči v steklarni, in sicer jo vsaka stane $100 €$.
Steklarna ima stroške vsakič, ko zažene proizvodnjo teh kroglic,
zato ji za vsako naročilo (ne glede na število naročenih kroglic)
zaračuna dodatnih $300 €$.
Kljub temu, da mora Stekelce zmeraj plačati vse dobljene kroglice,
je za vsako le $50 \%$ verjetnost, da bo zanjo uporabna.
Neuporabne kroglice lahko vrže kar v smeti,
odvečne uporabne kroglice pa zanjo prav tako niso vredne nič
in jih bo brez stroškov zavrgla.
\begin{enumerate}[(a)]
\item Ker je naročilo časovno omejeno,
ima Stekelce čas za največ dve naročili v steklarni.
Po koliko kroglic naj vsakič naroči, da bodo pričakovani stroški čim manjši?
(Število naročenih kroglic pri drugem naročilu
bo seveda odvisno od števila dobrih kroglic pri prvem naročilu.)

\item Kaj pa,
če steklarna v prvi seriji za naročilo namesto $300 €$ zahteva le $150 €$,
v drugi seriji pa stroški naročila ostanejo $300 €$?
\end{enumerate}
\end{vprasanje}

\begin{odgovor}
\begin{enumerate}[(a)]
\item Naj bodo $s_{ij}$ pričakovani stroški, ki bodo nastali,
če Stekelce pri $i$-tem naročilu ($i = 1, 2$)
potrebuje še $j$ kroglic ($j = 0, 1, 2$).
Določimo začetne pogoje in rekurzivne enačbe.
\begin{align*}
s_{22} &= \min\left(\{k\} \cup
    \set{z_2 + cx + (p + x(1-p)) p^{x-1} k}{x \ge 2, x \in \Z}\right) \\
s_{21} &= \min\left(\{k\} \cup
    \set{z_2 + cx + p^x k}{x \ge 1, x \in \Z}\right) \\
s_{12} &= \min\left(\{s_{22}\} \cup
    \set{z_1 + cx + p^x s_{22} + x(1-p) p^{x-1} s_{21}}%
        {x \ge 1, x \in \Z}\right)
\end{align*}
Tukaj $c$ predstavlja ceno posamezne kroglice,
$z_i$ ($i = 1, 2$) stroške zagona izdelave pri $i$-tem naročilu,
$k$ kazen ob neizstavitvi naročila,
in $p$ verjetnost, da bo posamezna kroglica neuporabna.
Pričakovani stroški so $s^* = s_{12}$.

Izračunajmo zgornje vrednosti
pri $c = 100 €$, $z_1 = z_2 = 300 €$, $k = 800 €$ in $p = 0.5$.
Opazimo, da imajo izrazi na desni natanko en lokalni minimum (za $x \in \R$),
tako da zadostuje obravnavati le tiste vrednosti $x$,
kjer vrednost izraza še pada.
\begin{alignat*}{2}
s_{22} &= \min\{800 €, 1\,100 €, 1\,000 €, 950 €, 950 €, \dots\} &&= 800 € \\
s_{21} &= \min\{800 €, 800 €, 700 €, 700 €, \dots\} &&= 700 € \\
s_{12} &= \min\{800 €, 1\,150 €, 1\,050 €, 962.5 €, 925 €, 934.375 €, \dots\}
&&= 800 €
\end{alignat*}
Kroglic se ji torej ne izplača naročati,
saj bo do najmanjših pričakovanih stroškov prišlo,
če enostavno plača kazen.

\item Rekurzivne enačbe iz prejšnje točke rešimo še za primer,
ko velja $z_1 = 150 €$.
Ker so ostali parametri nespremenjeni,
zadostuje, da znova izračunamo le vrednost $s_{12}$.
$$
s_{12} = \min\{800 €, 1\,000 €, 900 €, 812.5 €, 775 €, 784.375 €, \dots\}
= 775 €
$$
Pri prvem naročilu naj torej naroči $4$ kroglice.
Če bo le ena uporabna, naj drugič naroči še $2$ ali $3$ kroglice,
če pa nobena ne bo uporabna, naj plača kazen.
\end{enumerate}
\end{odgovor}
\end{naloga}
