\begin{naloga}{Janoš Vidali}{Vaje OR 21.2.2018}
\begin{vprasanje}
Naj bo $A[1 \dots n][1 \dots n]$ matrika (tj., seznam seznamov)
dimenzij $n \times n$.
Dan je spodnji program:
\begin{small}
\begin{algorithmic}
\For{$i = 1, \dots, n$}
    \For{$j = i+1, \dots, n$}
        \State $A[i][j] \gets A[j][i]$
    \EndFor
\EndFor
\end{algorithmic}
\end{small}

\begin{enumerate}[(a)]
\item Kaj počne zgornji program?
\item Oceni število korakov, ki jih opravi zgornji program,
v odvisnosti od parametra $n$.
\end{enumerate}
\end{vprasanje}

\begin{odgovor}
\begin{enumerate}[(a)]
\item Program prepiše vnose v matriki $A$ nad diagonalo
na ustrezno mesto pod diagonalo tako,
da je po izvedbi programa matrika $A$ simetrična.
\item Kot korak bomo upoštevali posamezno izvedbo notranje zanke {\bf for},
kjer kopiramo vrednost v matriki na drugo mesto
-- ob predpostavki, da je velikost vnosov omejena
(npr.~32-bitna cela števila),
bo trajanje take operacije omejeno s konstanto.
Preštejmo število takih korakov:
$$
\sum_{i=1}^n \sum_{j=i+1}^n 1 = \sum_{i=1}^n (n-i) = \sum_{i=0}^{n-1} i =
{n(n-1) \over 2}
$$

Lahko bi seveda upoštevali še korake,
ki so potrebni za vzdrževanje števcev zank
(inicializacija števca, povečevanje števca, preverjanje konca zanke),
a bi spet dobili kvadratni polinom v $n$.
Tako lahko rečemo, da je število korakov omejeno z $O(n^2)$.
\end{enumerate}
\end{odgovor}
\end{naloga}
