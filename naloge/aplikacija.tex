\begin{naloga}{Janoš Vidali}{Izpit OR 10.7.2017}
\begin{vprasanje}
V podjetju imajo na voljo $m$ milijonov evrov sredstev,
ki jih bodo vložili v razvoj nove aplikacije.
Denar bodo porazdelili med tri skupine.
Naj bodo $x_1$, $x_2$ in $x_3$ količine denarja (v milijonih evrov),
ki jih bodo dodelili razvijalcem, oblikovalcem in marketingu.
Vrednosti $x_1, x_2, x_3$ niso nujno cela števila.
Razvijalci morajo dobiti vsaj $a_1$ milijonov evrov,
potencial, ki ga ustvarijo, pa je $p_1 = n_1 + k_1 x_1$.
Oblikovalci morajo dobiti vsaj $a_2$ milijonov evrov,
potencial, ki ga ustvarijo, pa je $p_2 = n_2 + k_2 x_2$.
Marketing mora dobiti vsaj $a_3$ milijonov evrov,
ustvari pa faktor $p_3 = n_3 + k_3 x_3$.
Pričakovani dobiček v milijonih evrov
se izračuna po formuli $d = (p_1 + p_2) p_3$.
V podjetju bi radi sredstva porazdelili med skupine tako,
da bo pričakovani dobiček čim večji.

\begin{enumerate}[(a)]
\item Zapiši rekurzivne enačbe za reševanje danega problema.
\item Z zgoraj zapisanimi enačbami reši problem
pri podatkih $m = 15$, $a_1 = 4$, $n_1 = 3$, $k_1 = 1.5$,
$a_2 = 3$, $n_2 = 4$, $k_2 = 2$, $a_3 = 2$, $n_3 = 0.4$ in $k_3 = 0.3$.
\end{enumerate}
\end{vprasanje}
\begin{odgovor}
\end{odgovor}
\end{naloga}
