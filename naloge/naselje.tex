\begin{naloga}{Janoš Vidali}{Kolokvij OR 3.6.2019}
\begin{vprasanje}
Izdelati želimo terminski plan za izgradnjo manjšega stanovanjskega naselja.
V tabeli~\tab so zbrana opravila pri gradnji.

\begin{enumerate}[(a)]
\item Topološko uredi ustrezni graf in ga nariši.
Za trajanja opravil vzemi pričakovana trajanja po modelu PERT.

\item Določi pričakovano kritično pot in čas izdelave.

\item Katero opravilo je (ob zgornjih predpostavkah) najmanj kritično?
Najmanj kritično je opravilo, katerega trajanje lahko najbolj podaljšamo,
ne da bi vplivali na celotno trajanje izvedbe.

\item Določi variance trajanj opravil in oceni ve\-rjet\-nost,
da bo grad\-nja trajala manj kot $180$ dni.
\end{enumerate}
%
\begin{tabela}
\makebox[\textwidth][c]{
\begin{tabular}{c|cc|b{2cm}b{2.7cm}b{2cm}}
& Opravilo & Pogoji & Minimalno trajanje
& Najbolj verjetno trajanje & Maksimalno trajanje \\ \hline
$a$ & pridobitev gradbenega dovoljenja   & /      & 14 dni & 14 dni & 23 dni \\
$b$ & pridobitev uporabnega dovoljenja   & $f, g$ & 18 dni & 21 dni & 30 dni \\
$c$ & izgradnja komunalne infrastrukture & $a$    & 40 dni & 45 dni & 59 dni \\
$d$ & postavitev temeljev                & $a, i$ & 28 dni & 37 dni & 46 dni \\
$e$ & izgradnja podpornega zidu          & $a$    &  7 dni &  8 dni & 12 dni \\
$f$ & izgradnja severne stolpnice        & $c, d$ & 63 dni & 78 dni & 99 dni \\
$g$ & izgradnja južne stolpnice          & $d, e$ & 75 dni & 84 dni & 99 dni \\
$h$ & izgradnja otroškega igrišča        & $a, i$ & 20 dni & 29 dni & 29 dni \\
$i$ & odstranitev rastja                 & /      & 19 dni & 22 dni & 25 dni
\end{tabular}
}
\podnaslov{Podatki}
\end{tabela}
\end{vprasanje}

\begin{odgovor}
\begin{enumerate}[(a)]
\item Projekt lahko predstavimo z uteženim grafom s slike~\fig,
iz katerega je razvidna topološka ureditev $s, a, i, c, d, e, h, f, g, b, t$.
Uteži povezav ustrezajo pričakovanim trajanjem opravil,
ki so prikazana v tabeli~\tab[naselje-resitev].

\item V tabeli~\tab[naselje-resitev]
so podani najzgodnejši začetki opravil in najpoznejši začetki,
da se celotno trajanje gradnje ne podaljša.
Pričakovano trajanje gradnje je torej $\mu = 166$ dni,
edina kritična pot pa je $s - i - d - g - b - t$,
ki tako vsebuje tudi vsa kritična opravila.

\item Iz tabele~\tab[naselje-resitev] je razvidno,
da je najmanj kritično opravilo $h$,
saj lahko njegovo trajanje
podaljšamo za $116.5$ dni v primerjavi s pričakovanim,
ne da bi vplivali na trajanje celotnega projekta.

\item Izračunajmo standardni odklon trajanja pričakovane kritične poti.
$$
\sigma = \sqrt{4 + 9 + 2.25 + 1} = 4.03
$$
Naj bo $X$ slučajna spremenljivka,
ki meri čas izvajanja pričakovane kritične poti.
Izračunajmo verjetnost končanja v roku $T = 180$ dni.
$$
P(X \le 180) = \Phi\left({T - \mu \over \sigma}\right) = \Phi(3.473) = 0.9997
$$
\end{enumerate}
%
\begin{slika}
\makebox[\textwidth][c]{
\pgfslika
}
\podnaslov{Graf odvisnosti med opravili in kritična pot}
\end{slika}
%
\begin{tabela}
\setlabel{naselje-resitev}
$$
\begin{array}{c|ccccccccccc}
& s & a & i & c & d & e & h & f & g & b & t \\ \hline
\text{pričakovano} && 15.5 & 22 & 46.5 & 37 & 8.5 & 79 & 85 & 27.5 & 22 \\
\text{varianca} && 2.25 & 4 & 10.03 & 9 & 0.69 & 36 & 16 & 2.25 & 1 \\
\hline
\text{najprej} & 0 & 0_s & 0_s & 15.5_a & 22_i & 15.5_a & 22_i & 62_c & 59_d & 144_g & 166_b \\
\text{najkasneje} & 0_i & 3_c & 0_d & 18.5_f & 22_g & 50.5_g & 138.5_t & 65_b & 59_b & 144_t & 166 \\
\text{razlika} & 0 & 3 & 0^* & 3 & 0^* & 35 & 116.5 & 3 & 0^* & 0^* & 0
\end{array}
$$
\podnaslov{Razporejanje opravil}
\end{tabela}
\end{odgovor}
\end{naloga}
