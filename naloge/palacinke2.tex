\begin{naloga}{Janoš Vidali}{Vaje OR 14.12.2016}
\begin{vprasanje}
Določi razpored opravil iz naloge~\nal[palacinke],
pri čemer en kuhar prevzame opravila na kritični poti,
drugi pa naj čim kasneje začne in čim prej konča.
\end{vprasanje}

\begin{odgovor}
Prvi kuhar bo prevzel opravila $c, a, e, f, h, i$,
drugi pa opravila $b, j, d, g$.
Iz tabele~\tab[palacinke-resitev] je razvidno,
da ga najbolj omejuje razporeditev opravilo $g$
-- najhitreje ga lahko začne $17$ minut po začetku priprave,
in tedaj ga konča $18$ minut po začetku.
Taka razporeditev mu dovoljuje,
da z opravilom $d$ začne $15$ minut po začetku priprave
in ga konča do začetka opravila $g$.
Opravili $b$ in $j$ lahko tako opravi pred tem
-- začne torej $11$ minut po začetku priprave
in obe zaključi (vrstni red ni pomemben),
preden se loti opravila $d$.
Drugi kuhar lahko tako zaporedoma izvede opravila $b, j, d, g$,
z začetkom $11$ minut po začetku priprave in s koncem $7$ minut kasneje
(torej brez prestanka).
\end{odgovor}
\end{naloga}
