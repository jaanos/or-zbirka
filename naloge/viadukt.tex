\begin{naloga}{Hillier, Lieberman}{\cite[Problem~10.4-4]{hl}}
\begin{vprasanje}
Pri gradbenem podjetju razmišljajo,
da bi se prijavili na razpis za prenovo av\-to\-cest\-ne\-ga viadukta.
Identificirali so pet nalog, ki so opisane v tabeli~\tab.

Če bodo izbrani za izvedbo del, si obetajo zaslužek v višini $250\,000 €$.
Če del ne bodo končali v roku $11$ tednov,
bodo morali plačati pogodbeno kazen v višini $500\,000 €$.
\begin{enumerate}[(a)]
\item Za vsako opravilo določi pričakovano trajanje in varianco.
\item Topološko uredi ustrezni graf in ga nariši.
\item Določi pričakovano kritično pot ter trajanje izvedbe.
\item Oceni verjetnost, da se bo projekt zaključil v $11$ tednih.
Naj se podjetje prijavi na razpis?
\end{enumerate}

\begin{tabela}
\makebox[\textwidth][c]{
\begin{tabular}{c|c|c|c|c}
naloga & najkrajše trajanje & najbolj verjetno trajanje
& najdaljše trajanje & predhodna opravila \\
\hline
$a$ & $3$ tedni & $4$ tedni  & $5$ tednov & / \\
$b$ & $2$ tedna & $2$ tedna  & $2$ tedna  & $a$ \\
$c$ & $3$ tedni & $5$ tednov & $6$ tednov & $b$ \\
$d$ & $1$ teden & $3$ tedni  & $5$ tednov & $a$ \\
$e$ & $2$ tedna & $3$ tedni  & $5$ tednov & $b, d$
\end{tabular}
}
\podnaslov{Podatki}
\end{tabela}
\end{vprasanje}

\begin{odgovor}
\begin{enumerate}[(a)]
\item Pričakovana trajanja in variance
so prikazane v tabeli~\tab[viadukt-resitev].

\item Projekt lahko predstavimo z uteženim grafom s slike~\fig,
iz katerega je raz\-vid\-na topološka ureditev $s, b, d, c, e, t$.

\item V tabeli~\tab[viadukt-resitev]
so podani pričakovani najzgodnejši začetki opravil in najpoznejši začetki,
da se celotno trajanje priprave ne podaljša.
Pričakovano trajanje projekta je torej $\mu = 10.83$ tednov,
kritična pot pa je $s - a - b - c - t$.

\item Izračunajmo standardni odklon trajanja pričakovane kritične poti.
$$
\sigma = \sqrt{{1 \over 9} + 0 + {1 \over 4}} = 0.601
$$
Naj bo $X$ slučajna spremenljivka,
ki meri čas izvajanja pričakovane kritične poti.
Izračunajmo verjetnost končanja v roku $T = 11$ tednov.
$$
P(X \le 11) = \Phi\left({T - \mu \over \sigma}\right) = \Phi(0.277) = 0.609
$$
Ob predpostavki,
da bodo dela zaključena in bodo tako prejeli izplačilo $250\,000 €$,
je pričakovan dobiček enak
$$
250\,000 € - (1 - P(X \le 11)) \cdot 500\,000 € = 54\,622.18 € .
$$
Ker je pričakovani dobiček pozitiven, se podjetju izplača prijaviti na razpis.
\end{enumerate}
%
\begin{tabela}
\setlabel{viadukt-resitev}
$$
\begin{array}{r|ccccccc}
& s & a & b & d & c & e & t \\ \hline
\text{pričakovano trajanje} &
& 4 & 2 & 3 & {29 \over 6} & {19 \over 6} & \\
\text{varianca} &
& {1 \over 9} & 0 & {4 \over 9} & {1 \over 4} & {1 \over 4} & \\ \hline
\text{najzgodnejši začetek} &
0 & 0_s & 4_a & 4_a & 6_b & 7_d & 10.83_c \\
\text{najpoznejši začetek} &
0_a & 0_b & 4_c & 4.67_e & 6_t & 7.67_t & 10.83 \\
\text{razlika} &
0 & 0^* & 0^* & 0.67 & 0^* & 0.67 & 0
\end{array}
$$
\podnaslov{Razporejanje opravil}
\end{tabela}
%
\begin{slika}
\makebox[\textwidth][c]{
\pgfslika
}
\podnaslov{Graf odvisnosti med opravili in kritična pot}
\end{slika}
\end{odgovor}
\end{naloga}
