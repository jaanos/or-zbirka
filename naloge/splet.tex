\begin{naloga}{Janoš Vidali}{Izpit OR 31.1.2017}
\begin{vprasanje}
Izdelati želimo terminski plan za izdelavo spletne aplikacije.
V tabeli~\tab so zbrana opravila pri izdelavi.

\begin{enumerate}[(a)]
\item Topološko uredi ustrezni graf in ga nariši.
\item Določi kritična opravila in kritično pot ter čas izdelave.
\item Katero opravilo je najmanj kritično?
Najmanj kritično je opravilo, katerega trajanje lahko najbolj podaljšamo,
ne da bi vplivali na trajanje izdelave.
\end{enumerate}

\begin{tabela}
\makebox[\textwidth][c]{
\begin{tabular}{c|l|c|c}
opravilo & opis & trajanje & pogoji \\
\hline
$a$ & natančna opredelitev funkcionalnosti & 15 dni & / \\
$b$ & programiranje uporabniškega vmesnika & 40 dni & $k$ \\
$c$ & programiranje skrbniškega vmesnika & 25 dni & $a, m$ \\
$d$ & programiranje strežniškega dela & 30 dni & $a, m$ \\
$e$ & integracija uporabniškega vmesnika s strežnikom & 20 dni & $b, d$ \\
$f$ & alfa testiranje & 20 dni & $c, e$ \\
$g$ & beta testiranje & 30 dni & $f, h$ \\
$h$ & pridobivanje testnih uporabnikov & 45 dni & $a$ \\
$i$ & vnos zadnjih popravkov & 10 dni & $g$ \\
$j$ & izdelava uporabniške dokumentacije & 35 dni & $b$ \\
$k$ & dizajniranje uporabniškega vmesnika & 15 dni & $a$ \\
$\ell$ & nabava računalniške opreme & 20 dni & / \\
$m$ & postavitev strežnikov & 10 dni & $\ell$ \\
\end{tabular}
}
\podnaslov{Podatki}
\end{tabela}
\end{vprasanje}

\begin{odgovor}
\begin{enumerate}[(a)]
\item Projekt lahko predstavimo z uteženim grafom s slike~\fig,
iz katerega je raz\-vid\-na topološka ureditev
$s, a, \ell, h, k, m, b, d, c, j, e, f, g, i, t$.

\item V tabeli~\tab[splet-resitev] so podani
najzgodnejši začetki opravil in najpoznejši začetki,
da se celotno trajanje projekta ne podaljša.
Izdelava bo torej trajala $150$ dni,
edina kritična pot pa je $s - a - k - b - e - f - g - i - t$,
ki tako vsebuje tudi vsa kritična opravila.

\item Iz tabele~\tab[splet-resitev] je razvidno,
da je najmanj kritično opravilo $h$,
saj lahko njegovo trajanje podaljšamo za $50$ dni,
ne da bi vplivali na trajanje celotnega projekta.
\end{enumerate}
%
\begin{slika}
\makebox[\textwidth][c]{
\pgfslika
}
\podnaslov{Graf odvisnosti med opravili in kritična pot}
\end{slika}
%
\begin{tabela}
\setlabel{splet-resitev}
\makebox[\textwidth][c]{
\begin{tabular}{c|ccccccccccccccc}
& $s$ & $a$ & $\ell$ & $h$ & $k$ & $m$ & $b$ & $d$
& $c$ & $j$ & $e$ & $f$ & $g$ & $i$ & $t$ \\ \hline
najprej & $0$ & $0_s$ & $0_s$ & $15_a$ & $15_a$ & $20_\ell$ & $30_k$ & $30_m$
& $30_m$ & $70_b$ & $70_b$ & $90_e$ & $110_f$ & $140_g$ & $150_i$ \\
najkasneje & $0_a$ & $0_k$ & $10_m$ & $65_g$ & $15_b$ & $30_d$ & $30_e$
& $40_e$ & $65_f$ & $115_t$ & $70_f$ & $90_g$ & $110_i$ & $140_t$ & $150$ \\
razlika & $0$ & $0^*$ & $10$ & $50$ & $0^*$ & $10$ & $0^*$ & $10$ & $35$
& $45$ & $0^*$ & $0^*$ & $0^*$ & $0^*$ & $0$
\end{tabular}
}
\podnaslov{Razporejanje opravil}
\end{tabela}
\end{odgovor}
\end{naloga}
