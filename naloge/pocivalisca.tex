\begin{naloga}{Janoš Vidali}{Izpit OR 28.8.2018}
\begin{vprasanje}
Pri direkciji za ceste načrtujejo nov avtocestni odsek dolžine $M$ kilometrov.
Ob cesti želijo zgraditi počivališča tako,
da je razdalja med dvema zaporednima počivališčema največ $K$ kilometrov.
Prav tako mora biti prvo počivališče največ $K$ kilometrov od začetka,
zadnje pa največ $K$ kilometrov od konca avtocestnega odseka.
Naj bodo $x_1 < x_2 < \dots < x_n$ možne lokacije počivališč
(v kilometrih od začetka avtocestnega odseka),
in $c_i$ ($1 \le i \le n$) cena izgradnje počivališča na lokaciji $x_i$.
Postavitev počivališč želijo izbrati tako,
da bo skupna cena izgradnje čim manjša.

\begin{enumerate}[(a)]
\item Zapiši rekurzivne enačbe za reševanje danega problema.
Razloži, kaj pred\-stav\-lja\-jo spremenljivke,
v kakšnem vrstnem redu jih računamo, ter kako dobimo optimalno rešitev.

\item Oceni časovno zahtevnost algoritma, ki sledi iz zgoraj zapisanih enačb.

\item S pomočjo rekurzivnih enačb reši zgornji problem za podatke
\begin{align*}
M &= 100, & (x_i)_{i=1}^8 &= ( 5, 12, 22, 34, 49, 65, 83, 91), \\
K &= 30,  & (c_i)_{i=1}^8 &= (18, 11, 21, 16, 23, 15, 19, 13).
\end{align*}
\end{enumerate}
\end{vprasanje}

\begin{odgovor}
\begin{enumerate}[(a)]
\item Naj bo $v_i$ najnižja cena izgradnje počivališč
na odseku od začetka avtoceste do lokacije $x_i$,
če zadnje počivališče zgradimo na tej lokaciji.
Zapišimo začetne pogoje in rekurzivne enačbe.
\begin{align*}
v_0 &= x_0 = 0 \\
v_i &= c_i + \min\left(\set{v_j}{0 \le j \le i-1, x_i - x_j \le K}
                       \cup \{\infty\}\right)
\quad (1 \le i \le n)
\end{align*}
Vrednosti $v_i$ računamo naraščajoče po indeksu $i$.
Najnižjo ceno izgradnje počivališč na celotnem odseku dobimo kot
$$
v^* = \min\left(\set{v_j}{0 \le j \le n, M - x_j \le K} \cup \{\infty\}\right) .
$$
Če velja $v^* = \infty$, potem izgradnja počivališč pod danimi pogoji ni mogoča.

\item V algoritmu izračunamo $n$ vrednosti $v_i$,
pri čemer pri vsaki obravnavamo največ $n$ možnosti;
prav tako obravnavamo največ $n$ možnosti pri računanju $v^*$.
Časovna zahtevnost algoritma je torej $O(n^2)$.

\item Izračunajmo vrednosti $v_i$ ($1 \le i \le 8$).
Ker se lokacije zaporednih počivališč razlikujejo za manj kot $K$,
prav tako pa sta prva in zadnja možna lokacija
oddaljeni manj kot $K$ od začetka oziroma konca odseka,
bodo vse vrednosti $v_i$ končne,
enako pa velja tudi za $v^*$.
\begin{alignat*}{3}
x_1 &= 5  &\qquad v_1 &= 18 + \min\{\underline{0}\} &&= 18 \\
x_2 &= 12 &\qquad v_2 &= 11 + \min\{\underline{0}, 18\} &&= 11 \\
x_3 &= 22 &\qquad v_3 &= 21 + \min\{\underline{0}, 18, 11\} &&= 21 \\
x_4 &= 34 &\qquad v_4 &= 16 + \min\{18, \underline{11}, 21\} &&= 27 \\
x_5 &= 49 &\qquad v_5 &= 23 + \min\{\underline{21}, 27\} &&= 44 \\
x_6 &= 65 &\qquad v_6 &= 15 + \min\{\underline{44}\} &&= 59 \\
x_7 &= 83 &\qquad v_7 &= 19 + \min\{\underline{59}\} &&= 78 \\
x_8 &= 91 &\qquad v_8 &= 13 + \min\{\underline{59}, 78\} &&= 72
\end{alignat*}
Ker sta samo zadnji dve lokaciji oddaljeni manj kot $K$ od konca odseka,
je najmanjša cena izgradnje počivališč enaka $v^* = \min\{78, 72\} = 72$.
Re\-kon\-stru\-iraj\-mo še optimalno postavitev.
\begin{align*}
v^* = v_8 &= c_8 + v_6 & \text{počivališče na $x_8 = 91$} \\
v_6 &= c_6 + v_5 & \text{počivališče na $x_6 = 65$} \\
v_5 &= c_5 + v_3 & \text{počivališče na $x_5 = 49$} \\
v_3 &= c_3 + v_0 & \text{počivališče na $x_3 = 22$} \\
\end{align*}
\end{enumerate}
\end{odgovor}
\end{naloga}
