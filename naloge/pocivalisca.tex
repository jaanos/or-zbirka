\begin{naloga}{Janoš Vidali}{Izpit OR 28.8.2018}
\begin{vprasanje}
Pri direkciji za ceste načrtujejo nov avtocestni odsek dolžine $M$ kilometrov.
Ob cesti želijo zgraditi počivališča tako,
da je razdalja med dvema zaporednima počivališčema največ $K$ kilometrov.
Prav tako mora biti prvo počivališče največ $K$ kilometrov od začetka,
zadnje pa največ $K$ kilometrov od konca avtocestnega odseka.
Naj bodo $x_1 < x_2 < \dots < x_n$ možne lokacije počivališč
(v kilometrih od začetka avtocestnega odseka),
in $c_i$ ($1 \le i \le n$) cena izgradnje počivališča na lokaciji $x_i$.
Postavitev počivališč želijo izbrati tako,
da bo skupna cena izgradnje čim manjša.

\begin{enumerate}[(a)]
\item Zapiši rekurzivne enačbe za reševanje danega problema.
Razloži, kaj pred\-stav\-lja\-jo spremenljivke,
v kakšnem vrstnem redu jih računamo, ter kako dobimo optimalno rešitev.

\item Oceni časovno zahtevnost algoritma, ki sledi iz zgoraj zapisanih enačb.

\item S pomočjo rekurzivnih enačb reši zgornji problem za podatke
\begin{align*}
M &= 100, & (x_i)_{i=1}^8 &= ( 5, 12, 22, 34, 49, 65, 83, 91), \\
K &= 30,  & (c_i)_{i=1}^8 &= (18, 11, 21, 16, 23, 15, 19, 13).
\end{align*}
\end{enumerate}
\end{vprasanje}
\begin{odgovor}
\end{odgovor}
\end{naloga}
