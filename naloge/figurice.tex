\begin{naloga}{Janoš Vidali}{Izpit OR 9.6.2025}
\begin{vprasanje}
Mojca je prejela naročilo za izdelavo $100$ figuric,
ki jih bo natisnila s 3D tiskalnikom.
Z enim tiskanjem nastanejo tri figurice,
posamezno tiskanje pa traja $2$ minuti.
Ker se čas uporabe 3D tiskalnika plača in kvaliteta nastalih figuric niha,
se Mojca odloča, ali naj izvede $34$, $35$ ali $36$ tiskanj.
Mojca pozna verjetnosti $p_i$ ($2 \le i \le 7$),
da $i$ figuric ne bo dosegalo standardov kakovosti
in jih torej ne bo mogla vključiti v naročilo
(verjetnosti so neodvisne od števila tiskanj):
$$
\begin{array}{c|cccccc}
	i & 2 & 3 & 4 & 5 & 6 & 7 \\ \hline
	p_i & 0.05 & 0.15 & 0.3 & 0.25 & 0.15 & 0.1 \\
\end{array}
$$
Če ne bo imela dovolj figuric za vključitev v naročilo,
bo Mojca manjkajoče figurice natisnila z drugačnimi nastavitvami
-- tokrat bosta z enim tiskanjem nastali dve figurici,
posamezno tiskanje pa bo trajalo $3$ minute.
V tem primeru bo kvaliteta nastalih figuric zagotovo dosegala standarde.

Zanima nas, koliko tiskanj naj Mojca izvede,
da bo pričakovani skupni čas tiskanja čim manjši.
\end{vprasanje}

\begin{odgovor}
Naj bo $X_k$ ($34 \le k \le 36$) čas uporabe 3D tiskalnika v minutah,
če se Mojca odloči za $k$ tiskanj.
\begin{alignat*}{2}
E(X_{34}) &= 34 \cdot 2 + ((p_3 + p_4) + 2 \cdot (p_5 + p_6) + 3 \cdot p_7) \cdot 3 &&= 72.65 \\
E(X_{35}) &= 35 \cdot 2 + (p_6 + p_7) \cdot 3 &&= 70.75 \\
E(X_{36}) &= 36 \cdot 2 &&= 72
\end{alignat*}
Vidimo, da se Mojci najbolj izplača opraviti $35$ tiskanj.
\end{odgovor}
\end{naloga}
