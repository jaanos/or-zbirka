\begin{naloga}{Janoš Vidali}{Izpit OR 15.12.2016}
\begin{vprasanje}
Oceni časovno zahtevnost algoritma,
ki sledi iz rekurzivnih enačb za nalogo~\nal[blago].
Reši problem za podatke $m = 5, n = 3, k = 4$,
$(a_h)_{h=1}^k = (2, 3, 1, 2)$, $(b_h)_{h=1}^k = (2, 1, 4, 3)$
in $(c_h)_{h=1}^k = (6, 3, 5, 7)$.
\end{vprasanje}

\begin{odgovor}
Algoritem, ki izhaja iz rekurzivnih enačb,
ima časovno zahtevnost $O(mn(m + n + k))$.
Z ustrezno podatkovno strukturo (zgoščena tabela)
je mogoče časovno zahtevnost izboljšati na $O(mn(m + n) + k)$.

Izračunajmo vrednosti $p_{ij}$ ($1 \le i \le m, 1 \le j \le n)$
po rekurzivnih enačbah iz rešitve naloge~\res[blago].
Pri tem upoštevamo, da velja $p_{ij} = p_{ji}$,
tako da bomo izračunali samo primere z $i \ge j$.
\begin{alignat*}{3}
p_{11} &= \max\{0\} &&= 0 \\
p_{21} &= \max\{0 + 0, 0\} &&= 0 \\
p_{22} &= \max\{\underline{6}, 0 + 0, 0\} &&= 6 &\quad \text{izdelek $1$} \\
p_{31} &= \max\{\underline{3}, 0 + 0, 0\} &&=3 &\quad \text{izdelek $2$} \\
p_{32} &= \max\{\underline{7}, 0 + 6, 3 + 3, 0\} &&=7 &\quad \text{izdelek $4$} \\
p_{33} &= \max\{\underline{3 + 7}, 0\} &&= 10 &\quad \text{razrez na $1 \times 3$ in $2 \times 3$} \\
p_{41} &= \max\{\underline{5}, 0 + 3, 0 + 0, 0\} &&=5 &\quad \text{izdelek $3$} \\
p_{42} &= \max\{0 + 7, \underline{6 + 6}, 5 + 5, 0\} &&= 12 &\quad \text{razrez na $2 \times 3$ in $2 \times 2$} \\
p_{43} &= \max\{3 + 10, 7 + 7, \underline{5 + 12}, 0\} &&= 17 &\quad \text{razrez na $4 \times 1$ in $4 \times 2$} \\
p_{51} &= \max\{\underline{0 + 5}, 0 + 3, 0\} &&=5 &\quad \text{razrez na $1 \times 1$ in $4 \times 1$} \\
p_{52} &= \max\{0 + 12, \underline{6 + 7}, 5 + 5, 0\} &&= 13 &\quad \text{razrez na $2 \times 2$ in $3 \times 2$} \\
p_{53} &= \max\{\underline{3 + 17}, 7 + 10, 5 + 13, 0\} &&= 20 &\quad \text{razrez na $1 \times 3$ in $4 \times 3$}
\end{alignat*}
Maksimalni dobiček je torej $p^* = p_{53} = 20$, dosežemo pa ga tako,
da iz prvotnega kosa blaga dimenzij $5 \times 3$
odrežemo kos dimenzij $1 \times 3$ za izdelek s ceno $3$,
nato od preostanka dimenzij $4 \cdot 3$
odrežemo kos dimenzij $4 \times 1$ za izdelek s ceno $5$,
nazadnje pa preostanek $4 \cdot 2$
razrežemo na dva kosa dimenzij $2 \times 2$ za izdelka s ceno $6$.
\end{odgovor}
\end{naloga}
