\begin{naloga}{Alen Orbanić}{Kolokvij OR 19.11.2009}
\begin{vprasanje}
Janez je prekaljen poslovnež, ki hoče vedno zaslužiti kar največ.
A Janez lahko naenkrat prevzame le en posel.
Če se en posel začne med izvajanjem nekega drugega posla,
novega posla ne more prevzeti.
Poslovne priložnosti so predstavljene z grafom.
Vozlišča predstavljajo stanja, v katerih Janez izbira med posli,
izhodne povezave iz vsakega vozlišča pa predstavljajo posle,
ki se jih lahko loti.
Cene povezav predstavljajo dobiček pri poslu oziroma izgubo,
če je cena negativna
(včasih je potrebno sprejeti tudi kak posel, ki nosi izgubo,
da se lahko prebijemo do dobičkonosnega posla \dots).
V vsakem stanju lahko Janez izbere katerega koli izmed poslov,
ki pripadajo izhodnim povezavam.
Tekom svoje poslovne kariere se lahko Janez
v določenem stanju znajde tudi večkrat.
Janez se trenutno nahaja v izbranem vozlišču grafa.
\begin{enumerate}[(a)]
\item Za poljuben graf poslovnih priložnosti ugotovi,
kakšen problem moraš na njem rešiti, da boš lahko pravilno svetoval Janezu,
da bo uresničil svoje poslovne ambicije.
Predlagaj algoritem, s katerim rešiš problem,
in oceni časovno zahtevnost predlagane rešitve.
Janeza med drugim še posebej zanima,
ali mu graf poslovnih priložnosti zagotavlja stalno pridobivanje poslov
in dobičkonosno poslovanje za celo kariero.
\item Za graf poslovnih priložnosti s slike~\fig
izberi ustrezen algoritem in ga izvedi ter izračunaj,
koliko največ lahko zasluži Janez, če ``vstopi v igro'' v stanju $a$ ali $b$.
Upoštevaj lastnosti spodnjega grafa in navedi, kateri algoritem je to.
Ali graf predstavlja poslovne priložnosti,
ki bodo Janezu zagotovile trajno dobičkonosno poslovanje?
\end{enumerate}

\begin{slika}
\pgfslika
\podnaslov{Graf}
\end{slika}
\end{vprasanje}
\begin{odgovor}
\begin{enumerate}[(a)]
\item Naj bo $G=(V,E)$ graf, v katerem iščemo iskano rešitev. Da problem prevedemo na 
iskanje najcenejše poti, v grafu spremenimo predznak vsem povezavam. Z $E'$ označimo
množico novih povezav, z $G'=(V,E')$ pa nov graf. Ideja algoritma je, da na grafu $G'$
najde najcenejšo pot oz. negativen cikel, če ta obstaja. Kot algoritem uporabimo 
predelano verzijo Floyd-Warshallovega algoritma, ki lahko zazna negativen cikel že med 
samim iskanjem poti. 
\\
\begin{small}
\begin{algorithmic}
\Function{FW-MAKSIMALNIDOBIČEK}{$G'=(V, E'), začetek$}
    \State $razdalja \gets$  slovar s ključi oblike $(u,v)$ in vrednostmi
    0 v primeru $u=v$,
    \State{ sicer $\infty$, za vse $u,v \in V$}
	\State $pot \gets$ slovar s ključi oblike $(u,v)$ in vrednostmi $\Null$
    za vse $u,v \in V$
    \For{$u \in V$}
        \For{$ v, r \in \Adj(G, u).items()$}
            \State $razdalja[u, v] \gets r$
            \State $pot[u,v] \gets [u,v]$
        \EndFor
    \EndFor
    \For{$w \in V$}
        \For{$u \in V$}
            \If{$razdalja[u, w] + razdalja[w, u] < 0$} \hfill $\exists$ negativen cikel
                \State {$potZasluzka = pot[zacetek,u][:-1] +$} \hfill sestavimo pot
                \State{$+ pot[u,w][:-1] + pot[w,u][:-2]$}
                    \State{ \Return{($\infty, potZasluzka$)}} \hfill vrne $zaslužek = \infty$ in pot s ciklom
            \EndIf
            \For{$v \in V$}
                \State{ $r \gets razdalja[u, w] + razdalja[w, v]$}
                \If {$r < razdalja[u, v]$}
                    \State{ $razdalja[u, v] = r$}
                    \State{$pot[u,v] = pot[u,w] + pot[w,v][1:]$  }
                \EndIf 
            \EndFor
        \EndFor
    \EndFor \hfill v grafu ni cikla
\State {$zasluzek = $min$(razdalja[zacetek,u]$ \textbf{for} $u \in V)$} \hfill poiščemo najmanjšo vr.
\For{$u \in  V$} \hfill poiščemo najmanjšo pot
    \If {$razdalja[zacetek,u] == zasluzek$}
        \State{$konec = u$}
    \EndIf
\EndFor
\State {$potZasluzka = pot[zacetek,konec]$}
\State{ $zasluzek = -zasluzek$  \hfill spremenimo predznak zasluzka}
\State {\Return {$(potZasluzka, zasluzek)$}}
\EndFunction
\end{algorithmic}
\end{small}

Ker gre za izpeljavo Floyd- Warshallovega algoritma, je časovna zahtevnost
enaka $o(n^3)$, kjer je $n$ število vozlišč.

\item Če algoritem iz točke (a) izvedemo na danem grafu, se lahko prepričamo, 
da je graf acikličen in zato vrne končno pot, ki torej ne predstavlja trajnega
dobičkonosnega poslovanja.  \\
V primeru, da začnemo v stanju a, nam algoritem vrne pot [a, c, d, f] in 
dobiček 10, v primeru, da začnemo v stanju b, pa [b, e, g, h, j] in 
dobiček 11.

\end{enumerate}
\begin{slika}
\pgfslika[poslovnez-resitev1]
\end{slika}

\begin{slika}
\pgfslika[poslovnez-resitev2]
\end{slika}
    


\end{odgovor}
\end{naloga}
