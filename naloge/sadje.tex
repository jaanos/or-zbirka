\begin{naloga}{Janoš Vidali}{Izpit OR 29.8.2017}
\begin{vprasanje}
Distributer ima $A$ zabojev avokadov in $B$ zabojev banan,
ki jih bo prodal $n$ trgovcem.
Trgovec $i$ ($1 \le i \le n$)
plača $a_i$ evrov za zaboj avokadov in $b_i$ evrov za zaboj banan,
skupaj pa bo porabil največ $c_i$ evrov.
Distributer bo zaboje dostavil s tovornjaki,
v katerih je lahko največ $K$ zabojev (ne glede na vsebino).
Če nekemu trgovcu dostavi $t$ zabojev,
bo torej opravljenih $\lceil t/K \rceil$ voženj.
Vsaka vožnja do trgovca $i$ (ne glede na to, koliko je poln tovornjak)
ga stane $d_i$ evrov.
Poleg tega bo trgovec $p$ kupil samo banane ali samo avokade,
trgovec $q$ pa bo kupil vsaj en zaboj avokadov,
če bo tudi trgovec $r$ kupil vsaj en zaboj avokadov.
Distributer želi zaboje razdeliti med trgovce tako,
da bo maksimiziral svoj dobiček
-- torej vsoto cen, ki jih plačajo trgovci,
zmanjšano za stroške dostave.
Lahko predpostavljaš, da so vse cene pozitivne.

Zapiši celoštevilski linearni program, ki modelira zgoraj opisani problem.
\end{vprasanje}

\begin{odgovor}
Za $i$-tega trgovca ($1 \le i \le n$)
bomo uvedli spremenljivke $x_i$, $y_i$ in $z_i$,
pri čemer sta $x_i$ in $y_i$ število zabojev avokadov oziroma banan,
ki jih distributer proda $i$-temu trgovcu,
in $z_i$ število voženj tovornjakov,
ki bodo sadje dostavili do $i$-tega trgovca.
Poleg tega bomo uvedli še spremenljivki $u$ in $v$,
ki ju interpretiramo kot
\begin{align*}
u &= \begin{cases}
1, & \text{trgovec $p$ kupi avokade, in} \\
0  & \text{sicer;}
\end{cases} \\
\text{ter} \quad
v &= \begin{cases}
1, & \text{trgovec $p$ kupi banane, in} \\
0  & \text{sicer.}
\end{cases}
\end{align*}

Zapišimo celoštevilski linearni program.
\begin{alignat*}{3}
\max \ \sum_{i=1}^n (a_i x_i + b_i y_i - d_i z_i) && \text{p.p.} \\
\forall i \in \{1, \dots, n\}: &\ & x_i &\ge 0, & x_i &\in \Z \\
\forall i \in \{1, \dots, n\}: &\ & y_i &\ge 0, & y_i &\in \Z \\
\forall i \in \{1, \dots, n\}: &\ & z_i &\ge 0, & z_i &\in \Z \\
&& 0 \le u &\le 1, & u &\in \Z \\
&& 0 \le v &\le 1, & v &\in \Z
\opis{Skupno število zabojev}
&& \sum_{i=1}^n x_i &\le A \\
&& \sum_{i=1}^n y_i &\le B
\opis{Omejitev porabe trgovca}
\forall i \in \{1, \dots, n\}: &\ & a_i x_i + b_i y_i &\le c_i
\opis{Število tovornjakov}
\forall i \in \{1, \dots, n\}: &\ & x_i + y_i &\le K z_i
\opis{Trgovec $p$ ne kupi obojega}
&& x_p &\le A u \\
&& y_p &\le B v \\
&& u + v &\le 1
\opis{Trgovec $q$ kupi avokade, če jih tudi $r$}
&& x_r &\le A x_q
\end{alignat*}
\end{odgovor}
\end{naloga}
