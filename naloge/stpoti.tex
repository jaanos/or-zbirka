\begin{naloga}{Gabrovšek, Konvalinka}{Kolokvij OR 25.11.2010}
\begin{vprasanje}
Napiši psevdokodo algoritma,
ki v acikličnem usmerjenem grafu prešteje število poti
od danega vozlišča $s$ do vseh vozlišč grafa.
Algoritem nato uporabi na grafu s slike~\fig.

\begin{slika}
\pgfslika
\podnaslov{Graf}
\end{slika}
\end{vprasanje}

\begin{odgovor}
Zapišimo psevdokodo algoritma,
ki kliče funkcijo {\sc Topo} iz naloge~\res[topo]{}(a).
\begin{small}
\begin{algorithmic}
\Function{ŠteviloVsehPoti}{$G = (V, E), s$}
    \State $C \gets$ slovar z vrednostjo $0$ za vsako vozlišče $v \in V$
    \State $C[s] \gets 1$
    \For{$u \in$ {\sc Topo}$(G)$}
        \For{$v \in \Adj(G, u)$}
            \State $C[v] \gets C[v] + C[u]$
        \EndFor
    \EndFor
    \State \Return $C$
\EndFunction
\end{algorithmic}
\end{small}
Potek izvajanja algoritma na grafu s slike~\fig z začetnim vozliščem $s$,
pri čemer je uporabljena topološka ureditev $s, a, b, c, d, t$,
je prikazan v tabeli~\tab.
%
\begin{tabela}
$$
\begin{array}{c|cccccc}
u & s & a & b & c & d & t \\ \hline
  & 1 & 0 & 0 & 0 & 0 & 0 \\
s & 1 & 1 & 0 & 1 & 0 & 0 \\
a & 1 & 1 & 1 & 2 & 0 & 0 \\
b & 1 & 1 & 1 & 3 & 1 & 0 \\
c & 1 & 1 & 1 & 3 & 4 & 3 \\
d & 1 & 1 & 1 & 3 & 4 & 7
\end{array}
$$
\podnaslov{Potek izvajanja algoritma}
\end{tabela}
\end{odgovor}
\end{naloga}
