\begin{naloga}{Boštjan Gabrovšek}{Izpit OR 28.6.2011}
\begin{vprasanje}
Navdušenje nad koncem študijskega leta na ljubljanski univerzi
je botrovalo temu,
da v ambulanti Univerzitetne klinike
$169$ študentov išče nujno zdravniško pomoč.
Vsak od študentov potrebuje eno enoto krvi,
klinika pa ima na zalogi $170$ enot krvi.
V spodnji tabeli je prikazano število enot krvi posamezne krvne skupine,
ki so na zalogi,
in število študentov z dano krvno skupino.
\begin{center}
\begin{tabular}{l|cccc}
Krvna skupina &  A &  B &  O & AB \\ \hline
Zaloga        & 46 & 34 & 45 & 45 \\
Povpraševanje & 39 & 38 & 42 & 50
\end{tabular}
\end{center}
Osebe s krvno skupino A lahko prejmejo kri tipa A ali O.
Osebe tipa B lahko prejmejo kri tipa B ali O.
Osebe tipa O lahko prejmejo samo kri tipa O.
Osebe tipa AB lahko prejmejo katerokoli krvno skupino.
Naša naloga je razporediti enote krvi tako,
da bomo z njo oskrbeli kar se da veliko število študentov.
Iz podatkov sestavi omrežje
in modeliraj nalogo kot problem maksimalnega pretoka.
Največ koliko pacientov lahko oskrbimo?
Utemelji.
\end{vprasanje}
\begin{odgovor}
\end{odgovor}
\end{naloga}
