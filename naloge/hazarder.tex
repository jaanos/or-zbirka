\begin{naloga}{Hillier, Lieberman}{\cite[Problem~11.4-1]{hl}}
\begin{vprasanje}
Igralec na srečo bo odigral tri partije s svojimi prijatelji,
pri čemer lahko vsakič stavi na svojo zmago.
Stavi lahko katerokoli vsoto denarja, ki jo ima na voljo
-- če izgubi partijo, zastavljeno vsoto izgubi, sicer pa tako vsoto pridobi.
Pri vsaki partiji sta verjetnosti zmage in poraza enaki $1/2$.
Na začetku ima $75 €$, na koncu pa želi imeti $100 €$
(ker igra s prijatelji, noče imeti več kot toliko).

Z dinamičnim programiranjem poišči strategijo stavljenja,
ki maksimizira verjetnost, da bo na koncu imel natanko $100 €$.
\end{vprasanje}

\begin{odgovor}
Naj bo $p_i(x)$ največja verjetnost,
da bo igralec na srečo na koncu imel natanko $100 €$,
če bo v $i$-to igro vstopil z $x$ evri.
Opazimo, da če tretjo igro začne z manj kot $50 €$,
je ne bo mogel končati s $100 €$.
Če ima $x$ evrov, kjer je $50 \le x < 100$,
potem bo stavil $100 - x$ evrov in upal na zmago;
pri $x > 100$ pa bo stavil $x - 100$ evrov in upal na poraz.
Če ima natanko $100 €$, potem ne bo stavil in si tako zagotovil,
da bo tudi na koncu imel natanko $100 €$.
Določimo torej začetne pogoje in rekurzivne enačbe.
\begin{align*}
p_3(x) &= \begin{cases}
0 & 0 \le x < 50 \\
{1 \over 2} & 50 \le x < 100 \\
1 & x = 100 \\
{1 \over 2} & x > 100
\end{cases} \\
p_2(x) &= \max\set{p_3(x-y) + p_3(x+y) \over 2}{0 \le y \le x} \\
p_1(x) &= \max\set{p_2(x-y) + p_2(x+y) \over 2}{0 \le y \le x}
\end{align*}
Maksimalno verjetnost dobimo kot $p^* = p_1(75)$.

Izračunajmo sedaj $p_2(x)$ in $p^* = p_1(75)$.
\begin{alignat*}{2}
p_2(x) &= {1 \over 2} \begin{cases}
0+0 & 0 \le x < 25 \\
\max\left\{0+0, \underline{0+{1 \over 2}}\right\} & 25 \le x < 50 \\
\max\left\{\underline{{1 \over 2}+{1 \over 2}}, 0+{1 \over 2},
           0+1\right\} & x = 50 \\
\max\left\{\underline{{1 \over 2}+{1 \over 2}}, 0+{1 \over 2},
           0+1, 0+{1 \over 2}\right\} & 50 < x < 75 \\
\max\left\{{1 \over 2}+{1 \over 2}, \underline{{1 \over 2}+1},
           0+{1 \over 2}\right\} & x = 75 \\
\max\left\{{1 \over 2}+{1 \over 2}, \underline{{1 \over 2}+1},
           {1 \over 2}+{1 \over 2}, 0+{1 \over 2}\right\} & 75 < x < 100 \\
\max\left\{\underline{1+1}, {1 \over 2}+{1 \over 2},
           0+{1 \over 2}\right\} & x = 100 \\
\max\left\{{1 \over 2}+{1 \over 2}, \underline{1+{1 \over 2}},
           {1 \over 2}+{1 \over 2}, 0+{1 \over 2}\right\} & x > 100
\end{cases}
&&= \begin{cases}
0 & 0 \le x < 25 \\
{1 \over 4} & 25 \le x < 50 \\
{1 \over 2} & 50 \le x < 75 \\
{3 \over 4} & 75 \le x < 100 \\
1 & x = 100 \\
{3 \over 4} & x \ge 100
\end{cases} \\
p_1(75) &= {1 \over 2} \max\left\{\underline{{3 \over 4}+{3 \over 4}},
                                  {1 \over 2}+{3 \over 4},
                                  {1 \over 2}+1,
                                  {1 \over 4}+{3 \over 4},
                                  0+{3 \over 4}\right\} &&= {3 \over 4}
\end{alignat*}

\needspace{\baselineskip}
\noindent
Maksimalna verjetnost je torej $p^* = {3 \over 4}$.
Poiščimo strategijo, pri kateri jo dobimo.
\begin{align*}
p_1(75) &= {p_2(75-0) + p_2(75+0) \over 2}
&& \text{v prvi igri ne stavimo} \\
p_2(75) &= {p_3(75-25) + p_3(75+25) \over 2}
&& \text{v drugi igri stavimo $25 €$} \\
p_3(50) &= {1 \over 2}
&& \text{če izgubimo, v tretji stavimo $25 €$} \\
p_3(100) &= 1
&& \text{če zmagamo, v tretji ne stavimo}
\end{align*}
Pri drugi optimalni strategiji v prvi igri stavimo $25 €$,
v drugi igri pa ne glede na izid prve igre ne stavimo.
V tretji igri potem ravnamo enako kot v zgornji strategiji.
\end{odgovor}
\end{naloga}
