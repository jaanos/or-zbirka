\begin{naloga}{?}{Kolokvij OR 25.11.2010}
\begin{vprasanje}
\begin{enumerate}[(a)]
\item Dokaži, da za poljubni konstanti $a, b \in \R$, kjer je $b > 0$,
velja ${(n + a)}^b = O(n^b)$.

\item Naj bo $f$ naraščajoča funkcija.
Ali velja $g(n) = O(f(g(n)))$?

\item Dokaži,
da če $T$ zadošča pogoju $T(n) = 2T(\lceil n/2 \rceil) - 13$ za $n \ge 2$,
potem je $T(n) = O(n)$.
\end{enumerate}
\end{vprasanje}

\begin{odgovor}
\begin{enumerate}[(a)]
\item Dokažimo, da velja $(n + a)^b = \Omega(n^b)$.
Dokazati potrebujemo torej, da obstajata neki linearni funkciji $F$ in $G$, ter konstanta $n_0 \in \mathbb{N}$,
za katere velja $$F(n)^b \leq (n + a)^b \leq G(n)^b,$$ za vsak $n > n_0$.
Ker  je $b > 0$, je $x^{\frac{1}{b}}$ monotono naraščajoča za $x>0$,
torej jo lahko brez škode za splošnost uporabimo na zgornji neenakosti.
Dobimo $$F(n) \leq (n + a) \leq G(n).$$
Določimo $F(n) = \frac{n}{2}$ in $G(n) = 2n$.
Dobimo pogoja
\begin{align*}
\frac{n}{2} &\leq n + a \\
n + a &\leq 2n
\end{align*}
Ker moramo paziti, da so vsi členi znotraj neenakosti pozitivni, dodamo še pogoj $n + a > 0$.
Za dovolj visok $n$, bodo veljali vsi ti pogoji, torej za $n_0$ izberemo $\max(-a, 2a, a)$.

\item Po premisleku hitro ugotovimo, da za poljubno funkcijo $f$ to ne velja,
saj ta lahko konkretno spremeni hitrost naraščanja funkcije $g$.
Za protiprimer vzemimo $g(n) = n$ in $f(n) = \log(n)$ in poglejmo 
$\lim_{n \rightarrow \infty} (c \cdot \log(n) / n)$, pri poljubni izbiri $c > 0$.
Z razširitvijo funkcij na $\mathbb{R}$ in L'Hospitalovim pravilom ugorovimo, 
da bo limita enaka 0, ne glede na izbrano konstantno $c$, 
torej $n$ v neskončnosti ne moremo omejiti z $O(\log(n))$.

\item Rešitev naloge je klasična uporaba krovnega izreka, a enostavno jo rešimo tudi brez tega.
Ker velja $T(n) < 2T(n/2 + 1) + 1$ in prišteta konstanta znotraj $T$ ne vpliva na hitrost, 
saj lahko namesto $n$, računamo z $n - 2$, iz česar bi dobili $T(n - 2) < 2T(n/2) + 1$, 
premiki $n$ za konstanto pa na hitrost funkcije v neskončnosti očitno ne vplivajo.
Časovno zahtevnost lahko torej dobimo z uporabo $T(n) = 2T(n/2) + 1$.
Razvijajmo rekurzivno zvezo po korakih.
$$T(n) = 2T(n/2) + 1 = 2 (T(n/4) + 1) + 1 = \dots = 2^k T\left(n / 2^k\right) + \sum_{i=0}^{k} 2^i$$
Privzemimo, da $T(1) = O(1)$, torej mora biti $k$ dovolj visok, da bo veljalo
$n / 2^k = 1$, to pa bo, ko $k = \log_2(n)$.
Rešitev te naloge je potem
$$T(n) = 2^{\log_2(n)} \cdot O(1) + \sum_{i=0}^{\log_2(n)} 2^i = n \cdot O(1) + 2^{\log_2(n) + 1} = O(n) + 2n = O(n),$$
kjer smo pri izračunu vsote upoštevali formulo za geometrijske vsote 
(ali pa dejstvo, da ima polno dvojiško drevo globine $k$ $2^{k+1}$ listov).

\end{enumerate}
\end{odgovor}
\end{naloga}
