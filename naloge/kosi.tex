\begin{naloga}{David Gajser}{Kolokvij OR 9.5.2013}
\begin{vprasanje}
Profesionalni programer je gospodu Velikozobcu prodal algoritem {\sc Koši}.
A Velikozobec je že v letih in je pozabil, kaj naj bi ta algoritem vračal.
V prospektu je zasledil naslednjo psevdokodo algoritma:
\begin{small}
\begin{algorithmic}
\Require{$s[0 \dots n-1]$ tabela tromestnih števil}
\State $T \gets$ tabela velikosti $10$,
    katere elementi so prazne vrste (podatkovne strukture)
\For{$i = 0, 1, \dots, n-1$}
    \For{$i = 0, 1, \dots, 9$}
        \If{zadnja števka $s[i] = j$}
            \State $T[j].\append(s[i])$
        \EndIf
    \EndFor
\EndFor
\State $k \gets 0$
\For{$i = 0, 1, \dots, n-1$}
    \While{$T[k].\isEmpty()$}
        \State $k \gets k+1$
    \EndWhile
    \State $s[i] \gets T[k].\pop()$
\EndFor
\For{$i = 0, 1, \dots, n-1$}
    \For{$i = 0, 1, \dots, 9$}
        \If{srednja števka $s[i] = j$}
            \State $T[j].\append(s[i])$
        \EndIf
    \EndFor
\EndFor
\State $k \gets 0$
\For{$i = 0, 1, \dots, n-1$}
    \While{$T[k].\isEmpty()$}
        \State $k \gets k+1$
    \EndWhile
    \State $s[i] \gets T[k].\pop()$
\EndFor
\For{$i = 0, 1, \dots, n-1$}
    \For{$i = 0, 1, \dots, 9$}
        \If{prva števka $s[i] = j$}
            \State $T[j].\append(s[i])$
        \EndIf
    \EndFor
\EndFor
\State $k \gets 0$
\For{$i = 0, 1, \dots, n-1$}
    \While{$T[k].\isEmpty()$}
        \State $k \gets k+1$
    \EndWhile
    \State $s[i] \gets T[k].\pop()$
\EndFor
\State \Return{$s$}
\end{algorithmic}
\end{small}

\begin{enumerate}[(a)]
\item Algoritem izvedi na primeru
$$
s = [146, 145, 359, 100, 359, 486, 200, 367, 980, 909, 257, 100] .
$$
Dovolj je napisati tabelo $s$ po vsaki izmed zunanjih zank.

\item Kakšna je časovna zahtevnost algoritma?

\item Kaj algoritem vrača?

\item Gospod Velikozobec je ugotovil,
da se algoritem na vhodih velikosti $n = 10\,000$
izvede v približno $0.25$ sekundah.
Približno koliko časa potrebuje za izvajanje
na vhodih velikosti $n = 50\,000$?
\end{enumerate}
\end{vprasanje}

\begin{odgovor}
\begin{enumerate}[(a)]
\item Opazimo, da ima algoritem tri dele,
ki se razlikujejo le po tem, katera števka v številih se obravnava.
Vsakič se števila iz tabele $s$ dodajo v ustrezno vrsto,
nato pa se vsebina vrst zlepi
(po vrsti po indeksih tabele $T$)
v novo vsebino tabele $T$.
Zapisali bomo torej vsebino vrst po vsaki od zunanjih zank,
vsebino tabele $s$ po sledeči zanki pa lahko razberemo.
\begin{small}
\begin{align*}
\text{zadnje:} &\ [100, 200, 980, 100], [], [], [], [], [145], [146, 486], [367, 257], [], [359, 359, 909] \\
\text{srednje:} &\ [100, 200, 100, 909], [], [], [], [145, 146], [257, 359, 359], [367], [], [980, 486], [] \\
\text{prve:} &\ [], [100, 100, 145, 146], [200, 257], [359, 359, 367], [486], [], [], [], [], [909, 980]
\end{align*}
\end{small}
Algoritem torej vrne tabelo
$$
[100, 100, 145, 146, 200, 257, 359, 359, 367, 486, 909, 980] .
$$

\item Časovna zahtevnost algoritma je $O(n)$
-- notranje zanke namreč naredijo konstantno število korakov.

\item Algoritem uredi vhodno tabelo po velikosti in jo vrne.
Pri tem upošteva dejstvo,
da je mogoče trimestna števila predstaviti
s končnim številom simbolov iz končne abecede
in tako doseže časovno zahtevnost,
ki je boljša od $\Omega(n \log n)$
(t.i.~{\em bucket sort}).

\item Ker algoritem teče v linearnem času,
bo čas izvajanja premo sorazmeren z velikostjo vhoda.
Pri vhodu velikosti $n = 50\,000$ torej pričakujemo,
da bo algoritem potreboval približno
$$
{50\,000 \over 10\,000} \cdot 0.25 \operatorname{s} = 1.25 \operatorname{s} .
$$
\end{enumerate}
\end{odgovor}
\end{naloga}
