\begin{naloga}{Janoš Vidali}{Vaje OR 7.12.2016}
\begin{vprasanje}
Dan je usmerjen acikličen graf s slike~\fig.

\begin{enumerate}[(a)]
\item Poišči topološko ureditev vozlišč zgornjega grafa.

\item Poišči najkrajšo pot od vozlišča $g$ do vozlišča $e$.

\item Poišči najdaljšo pot od vozlišča $g$ do vozlišča $e$.
\end{enumerate}

\begin{slika}
\pgfslika
\podnaslov{Graf}
\end{slika}
\end{vprasanje}
\begin{odgovor}

\begin{enumerate}[(a)]
\item Upoštevamo, da je graf $G = (V, E)$ acikličen,
torej za vsaki vozlišči $u, v \in V$ velja
$$
u \rightarrow v \implies \postlabel(u) > \postlabel(v),
$$
kjer je $\postlabel$ števec, s katerim označimo vozlišče,
ko ga v algoritmu {\sc dfs} docela preiščemo.

Trditev lahko dokažemo z obravnavanjem dveh primerov, in sicer, 
ko z {\sc dfs} pridemo v vozlišče $u$ pred $v$,
in ko pridemo v vozlišče $v$ pred $u$.
V prvem primeru bomo pred določitvijo pooznake vozlišču $u$
morali obiskati vse njegove sosede,
torej tudi $v$, ki mu bomo določili pooznako pred $u$.
V drugem primeru pridemo prej do $v$,
a ker je graf acikličen, od $v$ ne bomo prišli do $u$,
torej bomo spet $v$ obdelali prej.

Trditev nam omogoča zapis krajšega algoritma, ki bo implementiral {\sc dfs}.
Njegova ideja je, da padajoče uredimo vozlišča po njihovih pooznakah, 
in tako dobimo inverzno topološko ureditev.
To dosežemo tako, da po popolni obdelavi vozlišča tega postavimo na sklad.

\begin{small}
\begin{algorithmic}
\Function{Topo}{$G = (V, E)$}
	\State oznaceni $\gets$ slovar z vrednostjo $\False$ za vsako vozlišče iz $V$
	\State toposklad $\gets []$
	\Function{dfs}{$u$}
		\State oznaceni$[u] \gets \True$
		\For{$v \in \Adj(u)$}
			\If{$\lnot$oznaceni$[v]$}
				\State {\sc dfs}$(v)$
			\EndIf
		\EndFor
		\State toposklad$.\append(u)$
	\EndFunction
	\For{$u \in V$}
		\If{$\lnot$oznaceni$[u]$}
			\State {\sc dfs}$(u)$
		\EndIf
	\EndFor
	\State toposklad.\reverse()
	\State \Return toposklad
\EndFunction
\end{algorithmic}
\end{small}

Klic {\sc Topo}$(G)$ nam vrne ureditev vozlišč $[g, a, h, b, c, f, d, e]$.

\item Tukaj bomo uporabili rezultat iz prejšnje točke
in osnoven koncept dinamičnega programiranja.

Označimo z $\ell(u, v)$ utež povezave $u \rightarrow v$.
Naj $d_G(u)$ predstavlja najkrajšo razdaljo
od izbranega vozlišča $s$ do vozlišča $u$.

\begin{align*}
d_G(s) &= 0 \\
d_G(u) &= \min_{v \rightarrow u}(d_G(v) + \ell(v, u))
\qquad (u \in V \setminus \{s\})
\end{align*}

Da bomo imeli vse potrebne $d_G(v)$ definirane pred klicem $d_G(u)$,
poskrbimo z rezultatom iz prejšnje točke,
saj pri topološki ureditvi vodijo vse povezave le naprej,
Po premiku na naslednje vozlišče tako ne potrebujemo kasnejših vozlišč,
pač pa le prejšnja, za katera vrednost že poznamo.

Imamo torej vse, kar potrebujemo za izračun najkrajše poti.
Za beleženje vozlišč, skozi katera vodi ta pot,
bomo uporabili seznam predhodnikov,
ki bo nakazoval, katero vozlišče smo izbrali za predhodnika.
\begin{align*}
\pred(s) &= \Null \\
\pred(u) &= \argmin_{v \rightarrow u}(d_G(v) + \ell(v, u))
\qquad (u \in V \setminus \{s\})
\end{align*}
Na koncu dobimo pot tako,
da sledimo predhodnikom podanega končnega vozlišča $t$,
dokler ne pridemo do $s$.

\begin{small}
\begin{algorithmic}
\Function{najkrajšaPot}{$G = (V, E), s, t$}
	\State topo $\gets$ {\sc Topo}$(G)$
	\State $d_G \gets$ slovar z vrednostjo $\infty$ za vsako vozlišče $v \in V$
	\State $d_G[s] \gets 0$
	\State $\pred \gets$ slovar z vrednostjo $\Null$ za vsako vozlišče $v \in V$
	\For{$u \in \text{topo}$}
		\For{$v \in \Adj(u)$}
			\If{$d_G[v] > d_G[u] + \ell(u, v)$}
				\State $d_G[v] \gets d_G[u] + \ell(u, v)$
				\State $\pred[v] \gets u$
			\EndIf
		\EndFor
	\EndFor
    \State $u \gets t$
	\State pot $\gets [t]$
	\While{$\pred[u] \ne \Null$}
        \State $u \gets \pred[u]$
		\State pot.$\append(u)$
	\EndWhile
    \State pot.$\reverse()$
	\State \Return ($d_G[t]$, pot)
\EndFunction 
\end{algorithmic}
\end{small}

Klic {\sc najkrajšaPot}$(G, g, e)$
vrne razdaljo od vozlišča $g$ do vozlišča $e$
ter seznam vozlišč, skozi katera je algoritem potoval,
da je opravil to pot -- to so $[g, a, b, e]$.
Dolžina najkrajše poti je $-1$.

Pripomnimo, da algoritem deluje pravilno
ne glede na položaj začetnega vozlišča $s$ v topološki ureditvi.
Za vsako vozlišče $u$, ki v topološki ureditvi nastopa pred $s$,
namreč velja $d_G(u) = \infty$,
tako da se ne nastavi kot predhodnik nobenemu vozlišču.
Tako se vozlišču $s$ ne bo nastavil predhodnik.
Algoritem lahko nekoliko pohitrimo tako,
da gledamo vozlišča v topološki ureditvi le od $s$ naprej,
prejšnja pa ignoriramo, saj iz $s$ ne moremo priti do njih.
Prav tako bi se lahko ustavili, ko dosežemo vozlišče $t$.

\begin{tabela}
$$
\begin{array}{c|c|rcrcl|cccccccc|c}
\multicolumn{7}{c}{} & \multicolumn{8}{c}{d_G} \\
u & v & d_G(v) &>& d_G(u) &+& \ell(u, v) & a & b & c & d & e & f & g & h & \pred[v] \\ \hline
g & a & \infty &>& 0 &+& (-1) & -1 & \infty & \infty & \infty & \infty & \infty & 0 & \infty & g\\
g & d & \infty &>& 0 &+& 6 & -1 & \infty & \infty & 6 & \infty & \infty & 0 & \infty & g\\
g & f & \infty &>& 0 &+& 6 & -1 & \infty & \infty & 6 & \infty & 6 & 0 & \infty & g\\
g & h & \infty &>& 0 &+& 6 & -1 & \infty & \infty & 6 & \infty & 6 & 0 & 6 & g\\
a & b & \infty &>& -1 &+& 2 & -1 & 1 & \infty & 6 & \infty & 6 & 0 & 6 & a\\
a & c & \infty &>& -1 &+& (-2) & -1 & 1 & -3 & 6 & \infty & 6 & 0 & 6 & a\\
a & h & 6 &>& -1 &+& 0 & -1 & 1 & -3 & 6 & \infty & 6 & 0 & -1 & a\\
h & b & 1 &>& -1 &+& 3 & -1 & 1 & -3 & 6 & \infty & 6 & 0 & -1 & a\\
h & f & 6 &>& -1 &+& 6 & -1 & 1 & -3 & 6 & \infty & 5 & 0 & -1 & h\\
b & c & -3 &>& 1 &+& 6 & -1 & 1 & -3 & 6 & \infty & 5 & 0 & -1 & a\\
b & e & \infty &>& 1 &+& (-2) & -1 & 1 & -3 & 6 & -1 & 5 & 0 & -1 & b\\
c & d & 6 &>& -3 &+& 2 & -1 & 1 & -3 & -1 & -1 & 5 & 0 & -1 & c\\
c & f & 5 &>& -3 &+& 6 & -1 & 1 & -3 & -1 & -1 & 3 & 0 & -1 & c\\
d & e & -1 &>& -1 &+& 7 & -1 & 1 & -3 & -1 & -1 & 3 & 0 & -1 & b
\end{array}
$$
\podnaslov{Potek izvajanja algoritma {\sc najkrajšaPot}}
\end{tabela}

\item Uporabimo algoritem {\sc najdaljšaPot},
ki deluje na enak način kot {\sc najkrajšaPot},
le da namesto $\infty$ vzame v $d_G$ za začetne vrednosti $-\infty$,
v zanki pa na vsakem koraku preverja pogoj $d_G[v] < d_G[u] + \ell(u, v)$.
Klic {\sc najdaljšaPot}$(G, g, e)$ vrne pot $[g, h, b, c, d, e]$ dolžine $24$.

\begin{tabela}
$$
\begin{array}{c|c|rcrcl|cccccccc|c}
\multicolumn{7}{c}{} & \multicolumn{8}{c}{d_G} \\
u & v & d_G(v) &<& d_G(u) &+& \ell(u, v) & a & b & c & d & e & f & g & h & \pred[v] \\ \hline
g & a & -\infty &<& 0 &+& (-1) & -1 & -\infty & -\infty & -\infty & -\infty & -\infty & 0 & -\infty & g\\
g & d & -\infty &<& 0 &+& 6 & -1 & -\infty & -\infty & 6 & -\infty & -\infty & 0 & -\infty & g\\
g & f & -\infty &<& 0 &+& 6 & -1 & -\infty & -\infty & 6 & -\infty & 6 & 0 & -\infty & g\\
g & h & -\infty &<& 0 &+& 6 & -1 & -\infty & -\infty & 6 & -\infty & 6 & 0 & 6 & g\\
a & b & -\infty &<& -1 &+& 2 & -1 & 1 & -\infty & 6 & -\infty & 6 & 0 & 6 & a\\
a & c & -\infty &<& -1 &+& (-2) & -1 & 1 & -3 & 6 & -\infty & 6 & 0 & 6 & a\\
a & h & 6 &<& -1 &+& 0 & -1 & 1 & -3 & 6 & -\infty & 6 & 0 & 6 & g\\
h & b & 1 &<& 6 &+& 3 & -1 & 9 & -3 & 6 & -\infty & 6 & 0 & 6 & h\\
h & f & 6 &<& 6 &+& 6 & -1 & 9 & -3 & 6 & -\infty & 12 & 0 & 6 & h\\
b & c & -3 &<& 9 &+& 6 & -1 & 9 & 15 & 6 & -\infty & 12 & 0 & 6 & b\\
b & e & -\infty &<& 9 &+& (-2) & -1 & 9 & 15 & 6 & 7 & 12 & 0 & 6 & b\\
c & d & 6 &<& 15 &+& 2 & -1 & 9 & 15 & 17 & 7 & 12 & 0 & 6 & c\\
c & f & 12 &<& 15 &+& 6 & -1 & 9 & 15 & 17 & 7 & 21 & 0 & 6 & c\\
d & e & 7 &<& 17 &+& 7 & -1 & 9 & 15 & 17 & 24 & 21 & 0 & 6 & d
\end{array}
$$
\podnaslov{Potek izvajanja algoritma {\sc najdaljšaPot}}
\end{tabela}

\end{enumerate}

\end{odgovor}
\end{naloga}
