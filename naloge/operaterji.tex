\begin{naloga}{Janoš Vidali}{Izpit OR 19.6.2019}
\begin{vprasanje}
Trg mobilne telefonije je zelo konkurenčen,
zato si mobilni operaterji na vse kriplje
prizadevajo pridobiti stranke svojih konkurentov.
Tako zelo, da lahko v nekaterih primerih
stranka s prehodom h konkurentu celo zasluži.
Da pa vendarle operaterji sami ne bi imeli prevelike izgube,
stranke zadržujejo z različnimi vezavami,
poleg tega pa novim strankam
(tistim, ki še niso imele mobilnega telefona)
krepko zaračunajo priklop.

Stanje trga zberemo v utežen usmerjen graf $G = (V, E)$,
kjer vozlišča pred\-stav\-lja\-jo operaterje,
utež $L_{uv}$ povezave $uv \in E$
pomeni strošek prehoda od operaterja $u$ k operaterju $v$,
utež $c_v$ vozlišča $v \in V$ pa pomeni strošek,
ki nastane tekom trajanja vezave pri operaterju $v$
(ko preidemo k operaterju $v$, imamo torej $c_v$ dodatnih stroškov).
Če povezave $uv$ ni v grafu, potem neposreden prehod od $u$ k $v$ ni mogoč.
Uteži vozlišč so nenegativne,
uteži povezav pa so lahko tudi negativne (tj., če ob prehodu zaslužimo).
Poiskati želimo najcenejše zaporedje prehodov med operaterji
(tj., tako, pri katerem imamo najmanj stroškov),
če začnemo pri operaterju $s$ in končamo pri operaterju $t$.

\begin{enumerate}[(a)]
\item Predlagaj čim bolj učinkovit algoritem za reševanje zgornjega problema.
Kak\-šna je njegova časovna zahtevnost?
Lahko predpostaviš, da velja $c_s = c_t = 0$.

\item Denimo, da trenutno še nimamo mobilnega telefona.
Naši izračuni kažejo,
da je dolgoročno najugodnejša ponudba operaterja {\em Rega},
tako da bi radi s čim manjšimi stroški postali njihov naročnik.
S pomočjo algoritma iz točke (a)
poišči ustrezno zaporedje prehodov na grafu s slike~\fig.
\end{enumerate}

\begin{slika}
\pgfslika
\podnaslov{Graf}
\end{slika}
\end{vprasanje}

\begin{odgovor}
\begin{enumerate}[(a)]
\item Za dani graf $G = (V, E)$, vozlišči $s, t \in V$,
uteži povezav $L_{uv}$ ($uv \in E$) in uteži vozlišč $c_v$ ($v \in V$)
bomo izračunali nove uteži povezav $\ell_{uv} = L_{uv} + c_v$ ($uv \in E$),
ki predstavljajo skupen strošek prehoda od operaterja $u$ k operaterju $v$.
Ker so te uteži še vedno lahko tudi negativne,
bomo v grafu $G$ poiskali najkrajšo pot od $s$ do $t$
s pomočjo Bellman-Fordovega algoritma,
ki teče v času $O(mn)$,
kjer je $n$ število vozlišč in $m$ število povezav grafa $G$.

\item Najprej izračunajmo nove uteži povezav.
\begin{alignat*}{3}
\ell_{\text{{\sl brez}, {\em Bobitel}}}       &=\ &  65 &+ 15 &&= 80 \\
\ell_{\text{{\sl brez}, {\em Sodafon}}}       &=\ &  85 &+ 10 &&= 95 \\
\ell_{\text{{\sl brez}, {\em Rega}}}          &=\ &  95 &+  0 &&= 95 \\
\ell_{\text{{\sl brez}, {\em Fenmobil}}}      &=\ &  50 &+ 30 &&= 80 \\
\ell_{\text{{\sl brez}, {\em $Z^0$}}}         &=\ &  60 &+  5 &&= 65 \\
\ell_{\text{{\sl brez}, {\em TeleJanez}}}     &=\ &  45 &+ 20 &&= 65 \\
\ell_{\text{{\em Bobitel}, {\em Sodafon}}}    &=\ &  10 &+ 10 &&= 20 \\
\ell_{\text{{\em Sodafon}, {\em Rega}}}       &=\ &  -5 &+  0 &&= -5 \\
\ell_{\text{{\em Rega}, {\em Fenmobil}}}      &=\ & -15 &+ 30 &&= 15 \\
\ell_{\text{{\em Fenmobil}, {\em $Z^0$}}}     &=\ & -10 &+  5 &&= -5 \\
\ell_{\text{{\em $Z^0$}, {\em TeleJanez}}}    &=\ &   5 &+ 20 &&= 25 \\
\ell_{\text{{\em TeleJanez}, {\em Bobitel}}}  &=\ &   0 &+ 15 &&= 15 \\
\ell_{\text{{\em TeleJanez}, {\em Fenmobil}}} &=\ & -20 &+ 30 &&= 10 \\
\ell_{\text{{\em Fenmobil}, {\em Sodafon}}}   &=\ &   5 &+ 10 &&= 15 \\
\ell_{\text{{\em Sodafon}, {\em TeleJanez}}}  &=\ &  20 &+ 20 &&= 40
\end{alignat*}
Za izračun najcenejše poti od vozlišča {\sl brez} do vozlišča {\em Rega}
bomo sledili algoritmu {\sc BellmanFord} iz naloge~\res[bf],
pri čemer bomo upoštevali vrstni red povezav iz zgornjega seznama.
Strnjen potek algoritma
(samo koraki, kjer se katera od razdalj spremeni)
je prikazan v tabeli~\tab.
Za najcenejše zaporedje prehodov med operaterji tako dobimo
{\em TeleJanez}, {\em Fenmobil}, {\em Sodafon}, {\em Rega},
skupni strošek pa je $75$.
%
\begin{tabela}
\makebox[\textwidth][c]{
\begin{tabular}{c|ccc|ccccccc}
$i$ & $u$ & $v$ & $\ell_{uv}$ &
{\sl brez} & {\em Bobitel} & {\em Sodafon} & {\em Rega} &
{\em Fenmobil} & {\em $Z^0$} & {\em TeleJanez} \\ \hline
1 & {\sl brez} & {ostali} && $0$ &
$80_{\text{\sl brez}}$ & $95_{\text{\sl brez}}$ & $95_{\text{\sl brez}}$ &
$80_{\text{\sl brez}}$ & $65_{\text{\sl brez}}$ & $65_{\text{\sl brez}}$ \\
1 & {\em Sodafon} & {\em Rega} & $-5$ & $0$ &
$80_{\text{\sl brez}}$ & $95_{\text{\sl brez}}$ & $90_{\text{\em Sodafon}}$ &
$80_{\text{\sl brez}}$ & $65_{\text{\sl brez}}$ & $65_{\text{\sl brez}}$ \\
1 & {\em TeleJanez} & {\em Fenmobil} & $10$ & $0$ &
$80_{\text{\sl brez}}$ & $95_{\text{\sl brez}}$ & $90_{\text{\em Sodafon}}$ &
$75_{\text{\em TeleJanez}}$ & $65_{\text{\sl brez}}$ & $65_{\text{\sl brez}}$ \\
1 & {\em Fenmobil} & {\em Sodafon} & $10$ & $0$ &
$80_{\text{\sl brez}}$ & $90_{\text{\em Fenmobil}}$ & $90_{\text{\em Sodafon}}$ &
$75_{\text{\em TeleJanez}}$ & $65_{\text{\sl brez}}$ & $65_{\text{\sl brez}}$ \\
2 & {\em Sodafon} & {\em Rega} & $-5$ & $0$ &
$80_{\text{\sl brez}}$ & $90_{\text{\em Fenmobil}}$ & $85_{\text{\em Sodafon}}$ &
$75_{\text{\em TeleJanez}}$ & $65_{\text{\sl brez}}$ & $65_{\text{\sl brez}}$ \\
\end{tabular}
}
\podnaslov[\res{}(b)]{Potek izvajanja algoritma}
\end{tabela}
\end{enumerate}
\end{odgovor}
\end{naloga}
