\begin{naloga}{Blaž Jelenc}{Izpit OR 25.8.2014}
\begin{vprasanje}
Na voljo imamo $w$ delavcev, ki jih želimo razporediti med $N$ opravil.
Dane imamo verjetnosti $p_k(j)$,
da bo $k$-to opravilo uspešno opravljeno, če na njem dela $j$ delavcev.

\begin{enumerate}[(a)]
\item S pomočjo dinamičnega programiranja opiši algoritem,
ki izračuna maksimalno verjetnost,
da bo vseh $N$ opravil uspešno opravljenih (opravila so med seboj neodvisna).

\item Izračunaj maksimalno verjetnost uspešno opravljenih opravil
in določi število delavcev na posameznem opravilu
za $N = 3$, $w = 5$ in verjetnosti iz spodnje tabele:
\begin{center}
\begin{tabular}{c|cccccc}
$j$ & $0$ & $1$ & $2$ & $3$ & $4$ & $5$ \\ \hline
$p_1(j)$ & $0$ & $0.5$ & $0.6$ & $0.7$ & $0.8$  & $0.85$ \\
$p_2(j)$ & $0$ & $0.4$ & $0.7$ & $0.8$ & $0.9$  & $0.95$ \\
$p_3(j)$ & $0$ & $0.3$ & $0.6$ & $0.8$ & $0.85$ & $0.85$
\end{tabular}
\end{center}
\end{enumerate}
\end{vprasanje}

\begin{odgovor}
\begin{enumerate}[(a)]
\item Naj bo $q_{kj}$ maksimalna verjetnost,
da bo prvih $k$ opravil uspešno opravljenih,
če na njih dela $j$ delavcev.
Zapišimo začetne pogoje in rekurzivne enačbe.
\begin{align*}
q_{1j} &= p_1(j) & (0 \le j \le w) \\
q_{kj} &= \max\set{q_{k-1,i} p_k(j-i)}{0 \le i \le j} & (2 \le k \le N, 0 \le j \le w)
\end{align*}
Vrednosti $1_{kj}$ računamo v leksikografskem vrstnem redu
glede na indeksa $(k, j)$.
Maksimalno verjetnost dobimo kot $q^* = q_{Nw}$.

\item Izračunajmo vrednosti $q_{2j}$ ($0 \le j \le w$) in $q^* = q_{35}$.
\begin{alignat*}{2}
q_{20} &= 0 \cdot 0 &&= 0 \\
q_{21} &= \max\{0 \cdot 0.4, 0.5 \cdot 0\} &&= 0 \\
q_{22} &= \max\{0 \cdot 0.7, \underline{0.5 \cdot 0.4}, 0.6 \cdot 0\} &&= 0.2 \\
q_{23} &= \max\{0 \cdot 0.8, \underline{0.5 \cdot 0.7}, 0.6 \cdot 0.4, 0.7 \cdot 0\} &&= 0.35 \\
q_{24} &= \max\{0 \cdot 0.9, 0.5 \cdot 0.8, \underline{0.6 \cdot 0.7}, 0.7 \cdot 0.4, 0.8 \cdot 0\} &&= 0.42 \\
q_{25} &= \max\{0 \cdot 0.95, 0.5 \cdot 0.9, 0.6 \cdot 0.8, \underline{0.7 \cdot 0.7}, 0.8 \cdot 0.4, 0.85 \cdot 0\} &&= 0.49 \\
q_{35} &= \max\{0 \cdot 0.85, 0 \cdot 0.85, 0.2 \cdot 0.8, \underline{0.35 \cdot 0.6}, 0.42 \cdot 0.3, 0.49 \cdot 0\} &&= 0.21
\end{alignat*}
Maksimalna verjetnost uspešno opravljenih opravil
je torej $q^* = q_{35} = 0.21$.
Poiščimo še optimalno razporeditev delavcev.
\begin{align*}
q^* = q_{35} &= q_{23} p_3(2) & \text{$2$ delavca za tretje opravilo} \\
q_{23} &= q_{11} p_2(2) & \text{$2$ delavca za drugo opravilo} \\
q_{11} &= p_1(1) & \text{$1$ delavec za prvo opravilo}
\end{align*}
\end{enumerate}
\end{odgovor}
\end{naloga}
