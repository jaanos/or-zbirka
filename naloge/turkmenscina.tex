\begin{naloga}{David Gajser}{Izpit OR 24.6.2015}
\begin{vprasanje}
Tajni agent slovenske obveščevalne službe
nam je prinesel besedilo v neznanem tujem jeziku,
brez ločil in brez presledkov.
V bistvu smo dobili en zelo dolg niz znakov (recimo dolžine $n$).
Ker v obveščevalni službi sumijo, da je besedilo v turkmenščini,
so nam prinesli elektronski seznam veljavnih turkmenskih besed
(ki so ga dobili iz skeniranja več turkmenskih spletnih strani
ter iz slovarja njihovega knjižnega jezika).
Ta seznam besed nam pomaga, da lahko hitro preverimo,
ali je dana beseda veljavna.

Naša naloga je, da preverimo,
ali lahko besedilo razdelimo na zaporedje veljavnih turkmenskih besed.
Reši nalogo s pomočjo dinamičnega programiranja.
\end{vprasanje}

\begin{odgovor}
Naj bo $s = (a_i)_{i=1}^n$ dani niz in $W$ množica veljavnih turkmenskih besed,
spremenljivka $v_j$ ($0 \le j \le n$) pa nam pove,
ali je mogoče niz $s_j = (a_i)_{i=1}^j$
razbiti na zaporedje veljavnih turkmenskih besed.
Zapišimo začetni pogoj in rekurzivne enačbe.
\begin{align*}
v_0 &= \top \\
v_j &= \exists w \in W : (|w| \le j \land v_{j-|w|} \land s_j = s_{j-|w|} \| w)
\quad (1 \le j \le n)
\end{align*}
Vrednosti $v_j$ računamo v naraščajočem vrstnem redu glede na indeks $j$.
Informacijo o tem, ali je mogoče niz $s$
razbiti na zaporedje veljavnih turkmenskih besed,
dobimo kot $d^* = d_n$.
\end{odgovor}
\end{naloga}
