\begin{naloga}{Janoš Vidali}{Kolokvij OR 23.4.2018}
\begin{vprasanje}
Pri izvedbi projekta bo potrebno narediti $n$ nalog,
ki jih bomo dodelili delavcem.
Vsako nalogo bo opravil natanko en delavec,
vsak delavec pa lahko hkrati izvaja samo eno nalogo.
Naj bo $t_i \in \N$ število časovnih enot,
ki ga posamezen delavec potrebuje za dokončanje naloge $i$
($1 \le i \le n$).
Vsaka naloga mora biti opravljena v enem kosu
(če torej začnemo z nalogo $i$ v času $s$,
bo ta dokončana v času $s+t_i$,
brez možnosti prekinitve in kasnejšega dokončanja).
Celoten projekt mora biti dokončan v času $T$.
Dana je še množica parov $S$,
kjer $(i, j) \in S$ pomeni,
da se naloga $j$ ne sme začeti, preden se zaključi naloga $i$
(lahko se pa $j$ začne izvajati v trenutku, ko se $i$ konča).

Delavce bomo najeli preko podjetja za posredovanje dela,
to pa nam zaračuna fiksno ceno na najetega delavca
(za ustrezna plačila delavcem bodo tako poskrbeli oni).
Minimizirati želimo torej število najetih delavcev.
Zapiši celoštevilski linearni program, ki modelira zgoraj opisani problem.
\end{vprasanje}

\begin{odgovor}
Očitno bomo potrebovali največ $n$ delavcev,
zato bomo za $h$-tega delavca ($1 \le h \le n$),
$i$-to nalogo ($1 \le i \le n$)
in $k$-to časovno enoto ($1 \le k \le T$)
uvedli spremenljivko $x_{hik}$,
katere vrednost interpretiramo kot
$$
x_{hik} = \begin{cases}
1; &
\text{$h$-ti delavec začne izvajati $i$-to nalogo v $k$-ti časovni enoti, in}
\\
0  & \text{sicer.}
\end{cases}
$$
Poleg tega bomo uvedli še spremenljivko $r$,
ki šteje število delavcev.

Zapišimo celoštevilski linearni program,
pri čemer uporabimo oznako $T_i = T-t_i+1$
za zadnji možen čas začetka $i$-te naloge.
\begin{alignat*}{3}
&& \min &\ r & \text{p.p.} \\
\forall h, i \in \{1, \dots, n\} \ \forall k \in \{1, \dots, T\}: &\ &
0 \le x_{hik} &\le 1, &\quad x_{hik} &\in \Z \\
&& r &\ge 0, &\quad r &\in \Z
\opis{Vsako nalogo opravi natanko en delavec in jo pravočasno konča}
\forall i \in \{1, \dots, n\}: &\ &
\sum_{h=1}^n \sum_{k=1}^{T_i} x_{hik} &= 1
\opis{Vsak delavec lahko hkrati izvaja samo eno nalogo}
\forall h, i \in \{1, \dots, n\} \ \forall k \in \{1, \dots, T_i\}: &\ &
(t_i-1) x_{hik} + \sum_{j=1}^n \sum_{\ell=k}^{k+t_i-1} x_{hj\ell} &\le t_i
\opis{Vrstni red opravljanja nalog}
\forall (i, j) \in S \ \forall k \in \{1, \dots, T_i\}: &\ &
\sum_{h=1}^n \left(x_{hik} - \sum_{\ell=k+t_i}^{T_j} x_{hj\ell}\right) &\le 0
\opis{Štetje delavcev}
\forall h, i \in \{1, \dots, n\} \ \forall k \in \{1, \dots, T\}: &\ &
h x_{hik} &\le r
\end{alignat*}
\end{odgovor}
\end{naloga}
