\begin{naloga}{Janoš Vidali}{Kolokvij OR 2.6.2022}
\begin{vprasanje}
Na Fakulteti za matematiko in fiziko bo izvedena prenova računalniškega omrežja.
V Računalniškem centru so tako identificirali opravila,
ki jih bodo pri tem izvedli -- ta so zbrana v tabeli~\tab.

\begin{enumerate}[(a)]
\item Topološko uredi ustrezni graf in ga nariši.
Za trajanja opravil vzemi pričakovana trajanja po modelu PERT.

\item Določi pričakovano kritično pot in čas izdelave.

\item Katero opravilo je (ob zgornjih predpostavkah) najmanj kritično?
Najmanj kritično je opravilo, katerega trajanje lahko najbolj podaljšamo,
ne da bi vplivali na celotno trajanje izvedbe.

\item Določi variance trajanj opravil in oceni ve\-rjet\-nost,
da bo projekt trajal manj kot $105$ dni.
\end{enumerate}
%
\begin{tabela}
\makebox[\textwidth][c]{
\begin{tabular}{c|cc|b{2cm}b{2.7cm}b{2cm}}
& Opravilo & Pogoji & Minimalno trajanje
& Najbolj verjetno trajanje & Maksimalno trajanje \\ \hline
$a$ & nabava omrežne opreme                      & $b$, $d$ & 30 dni & 45 dni & 60 dni \\
$b$ & pregled fizičnega stanja omrežja           & /        & 18 dni & 20 dni & 40 dni \\
$c$ & popis računalnikov v omrežju               & /        & 20 dni & 20 dni & 26 dni \\
$d$ & načrtovanje prenovljene sheme omrežja      & $c$      &  6 dni &  7 dni &  8 dni \\
$e$ & zamenjava osrednjih usmerjevalnikov        & $a$      &  2 dni &  3 dni &  4 dni \\
$f$ & polaganje novih kablov                     & $d$      &  5 dni &  8 dni & 11 dni \\
$g$ & postavitev vzporednega brezžičnega omrežja & $a$, $f$ & 12 dni & 18 dni & 30 dni \\
$h$ & preklop na novo brezžično omrežje          & $e$, $g$ &  1 dan &  2 dni &  3 dni \\
$i$ & testiranje nove postavitve                 & $h$      &  7 dni &  9 dni &  14 dni \\
$j$ & odpravljanje napak                         & $i$      &  1 dan &  7 dni & 10 dni \\
$k$ & odstranitev stare opreme                   & $h$      &  8 dni & 10 dni & 12 dni
\end{tabular}
}
\podnaslov{Podatki}
\end{tabela}
\end{vprasanje}

\begin{odgovor}
\begin{enumerate}[(a)]
\item Projekt lahko predstavimo z uteženim grafom s slike~\fig,
iz katerega je raz\-vid\-na topološka ureditev
$s, b, c, d, a, f, e, g, h, i, k, j, t$.
Uteži povezav ustrezajo pričakovanim trajanjem opravil,
ki so prikazana v tabeli~\tab[omrezjefmf-resitev].

\item V tabeli~\tab[omrezjefmf-resitev]
so podani najzgodnejši začetki opravil in najpoznejši začetki,
da se celotno trajanje projekta ne podaljša.
Pričakovano trajanje projekta je torej $\mu = 110$ dni,
edina kritična pot pa je $s - c - d - a - g - h - i - j - t$,
ki tako vsebuje tudi vsa kritična opravila.

\item Iz tabele~\tab[omrezjefmf-resitev] je razvidno,
da je najmanj kritično opravilo $f$,
saj lahko njegovo trajanje
podaljšamo za $37$ dni v primerjavi s pričakovanim,
ne da bi vplivali na trajanje celotnega projekta.

\item Izračunajmo standardni odklon trajanja pričakovane kritične poti,
pri čemer uporabimo variance iz tabele~\tab[omrezjefmf-resitev].
$$
\sigma = \sqrt{1 + 0.11 + 25 + 9 + 0.11 + 1.36 + 2.25} \approx 6.231
$$
Naj bo $X$ slučajna spremenljivka,
ki meri čas izvajanja pričakovane kritične poti.
Izračunajmo verjetnost končanja v roku $T = 105$ dni.
$$
P(X \le 105) = \Phi\left({T - \mu \over \sigma}\right) = \Phi(-0.802) = 0.2119
$$
\end{enumerate}
%
\begin{slika}
\makebox[\textwidth][c]{
\pgfslika
}
\podnaslov{Graf odvisnosti med opravili in kritična pot}
\end{slika}
%
\begin{tabela}
\setlabel{omrezjefmf-resitev}
\makebox[\textwidth][c]{
\begin{tabular}{c|ccccccccccccc}
& $s$ & $b$ & $c$ & $d$ & $a$ & $f$ & $e$ & $g$ & $h$ & $i$ & $k$ & $j$ & $t$ \\ \hline
pričakovano && $23$ & $21$ & $7$ & $45$ & $8$ & $3$ & $19$ & $2$ & $9.5$ & $10$ & $6.5$ \\
varianca && $13.44$ & $1$ & $0.11$ & $25$ & $1$ & $0.11$ & $9$ & $0.11$ & $1.36$ & $0.44$ & $2.25$ \\
\hline
najprej & $0$ & $0_s$ & $0_s$ & $21_c$ & $28_d$ & $28_d$ & $73_a$ & $73_a$ & $92_g$ & $94_h$ & $94_h$ & $103.5_i$ & $110_j$ \\
najkasneje & $0_c$ & $5_a$ & $0_d$ & $21_a$ & $28_g$ & $65_g$ & $89_h$ & $73_h$ & $92_i$ & $94_j$ & $100_t$ & $103.5_t$ & $110$ \\
\text{razlika} & $0$ & $5$ & $0^*$ & $0^*$ & $0^*$ & $37$ & $16$ & $0^*$ & $0^*$ & $0^*$ & $6$ & $0^*$ & $0$
\end{tabular}
}
\podnaslov{Razporejanje opravil}
\end{tabela}
\end{odgovor}
\end{naloga}
