\begin{naloga}{David Gajser}{Izpit OR 14.6.2013}
\begin{vprasanje}
Fenko, brat škrata Bolfenka, je najboljši alkimist, kar jih je družina imela.
Preživlja se z izdelovanjem zlata iz granodiorita\footnote{
Granodiorit je zelo razširjena magmatska kamnina na Pohorju,
znana tudi kot pohorski tonalit.
}.
Postopek izdelave zlata je naslednji:
ob 23:00 mora nastaviti sveže izkopan granodiorit
($1$ kg svežega granodiorita je vreden $2$ gulda\footnote{
Guld je denarna enota, ki jo uporabljajo pohorski škratje.
})
na točno določeno mesto pod milo nebo\footnote{
Kam je treba nastaviti granodiorit, je odvisno tudi od položaja planetov.
}.
Če nastavljene kamnine noben škrat, srna ali človek ne vidi,
se le-ta v dveh urah spremeni v zlato
(za $1$ kg granodiorita dobi $10$ g zlata).
Verjetnost, da se to zgodi (tj., da nastane zlato), je $2/5$.
To zlato lahko proda škratu zlatarju za gulde
(in to je edini možni način porabe zlata).
Za $10$ g zlata dobi $4$ gulde.
Granodiorit, ki je enkrat že bil nastavljen,
se ne bo nikoli več spremenil v zlato.
Fenko lahko kilogram ``porabljenega'' granodiorita proda za $1$ guld.

Predpostavimo, da Fenko zapravlja denar le za izdelavo zlata iz granodiorita,
da zmeraj nastavi celoštevilsko mnogo kilogramov granodiorita
in ima zmeraj celoštevilsko mnogo guldov
(npr.~s tremi guldi lahko kupi največ $1$ kg granodiorita, en guld mu ostane).

Trenutno ima Fenko $5$ guldov, nič zlata in nič granodiorita.
Koliko kilogramov granodiorita
naj v naslednjih dveh nočeh nastavi pod milo nebo,
da bo imel kar največje pričakovano število guldov?
V odgovoru napiši,
koliko granodiorita naj nastavi prvo noč in koliko drugo noč.
Odgovor bo odvisen od tega,
ali se mu ponoči granodiorit spremeni v zlato ali ne.
\end{vprasanje}

\begin{odgovor}
 \begin{align*} 
 X &\dots \text{število guldov po drugi noči} \\ 
 Y_i &\dots \text{število guldov pred $i$-to nočjo ($i \in \{1, 2\}$)} \\ 
 Z_i &\dots \text{kilogrami granodiorita pred $i$-to nočjo ($i \in \{1, 2\}$)} \\ 
 \end{align*} 

druga noč:
\begin{alignat*}{2}
E(X \mid  Y_2 = 1, Z_2 = 4) &= 0.4 \cdot 17 + 0.6 \cdot 5 &&= 9.8 \\
E(X \mid  Y_2 = 1, Z_2 = 1) &= 0.4 \cdot 5 + 0.6 \cdot 2 &&= 3.2 \\
E(X \mid  Y_2 = 1, Z_2 = 3) &= 0.4 \cdot 13 + 0.6 \cdot 4 &&= 7.6 \\
E(X \mid  Y_2 = 0, Z_2 = 2) &= 0.4 \cdot 8 + 0.6 \cdot 2 &&= 4.4 \\
\end{alignat*}
prva noč:
\begin{alignat*}{2}
E(X \mid  Y_1 = 1, Z_2 = 2) &= 0.4 \cdot 9.8 + 0.6 \cdot 3.2 &&= 5.84 \\
E(X \mid  Y_1 = 3, Z_2 = 1) &= 0.4 \cdot 7.6 + 0.6 \cdot 4.4 &&= 5.68 \\
\end{alignat*}

\begin{alignat*}{2}
E(X \mid  Y_2 = k-2x, Z_2 = x) &= 0.4 \cdot (k - 2x + 4x) + 0.6 \cdot (k - 2x + x) &&= k + 0.2x \\
\end{alignat*}
Ker je k + 0.2x večji od k, vedno kupi maksimalno količino granodiorita.

Na začetku bo torej kupil 2kg granodiorita, za kar bo porabil 4 G, ter oba nastavil prvo noč. Če uspe, bo pridobil 20 g zlata, ki 
jih proda in kupi 4 kg granodiorita, ki jih nastavi drugo noč.
Če mu prvo noč ne uspe, potem proda granodiorit za 2 G, ter kupi 1 kg granodiorita, ki ga nastavi drugo noč.

\begin{slika}
\makebox[\textwidth][c]{
\pgfslika
}
\podnaslov{Odločitveno drevo}
\end{slika}
\end{odgovor}
\end{naloga}
