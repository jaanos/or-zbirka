\begin{naloga}{David Gajser}{Izpit OR 14.6.2013}
\begin{vprasanje}
Fenko, brat škrata Bolfenka, je najboljši alkimist, kar jih je družina imela.
Preživlja se z izdelovanjem zlata iz granodiorita\footnote{
Granodiorit je zelo razširjena magmatska kamnina na Pohorju,
znana tudi kot pohorski tonalit.
}.
Postopek izdelave zlata je naslednji:
ob 23:00 mora nastaviti sveže izkopan granodiorit
($1$ kg svežega granodiorita je vreden $2$ gulda\footnote{
Guld je denarna enota, ki jo uporabljajo pohorski škratje.
})
na točno določeno mesto pod milo nebo\footnote{
Kam je treba nastaviti granodiorit, je odvisno tudi od položaja planetov.
}.
Če nastavljene kamnine noben škrat, srna ali človek ne vidi,
se le-ta v dveh urah spremeni v zlato
(za $1$ kg granodiorita dobi $10$ g zlata).
Verjetnost, da se to zgodi (tj., da nastane zlato), je $2/5$.
To zlato lahko proda škratu zlatarju za gulde
(in to je edini možni način porabe zlata).
Za $10$ g zlata dobi $4$ gulde.
Granodiorit, ki je enkrat že bil nastavljen,
se ne bo nikoli več spremenil v zlato.
Fenko lahko kilogram ``porabljenega'' granodiorita proda za $1$ guld.

Predpostavimo, da Fenko zapravlja denar le za izdelavo zlata iz granodiorita,
da zmeraj nastavi celoštevilsko mnogo kilogramov granodiorita
in ima zmeraj celoštevilsko mnogo guldov
(npr.~s tremi guldi lahko kupi največ $1$ kg granodiorita, en guld mu ostane).

Trenutno ima Fenko $5$ guldov, nič zlata in nič granodiorita.
Koliko kilogramov granodiorita
naj v naslednjih dveh nočeh nastavi pod milo nebo,
da bo imel kar največje pričakovano število guldov?
V odgovoru napiši,
koliko granodiorita naj nastavi prvo noč in koliko drugo noč.
Odgovor bo odvisen od tega,
ali se mu ponoči granodiorit spremeni v zlato ali ne.
\end{vprasanje}

\begin{odgovor}
Najprej uvedemo slučajne spremenljivke $X$, $Y_i$ in $Z_i$ ($i \in \{1, 2\}$),
s katerimi bomo računali v nadaljevanju.
\begin{align*}
X &\dots \text{število guldov po drugi noči} \\
Y_i &\dots \text{število guldov pred $i$-to nočjo} & (i &\in \{1, 2\}) \\
Z_i &\dots \text{kilogrami granodiorita pred $i$-to nočjo} & (i &\in \{1, 2\})
\end{align*}

Odločitev pred drugo nočjo bomo obravnavali glede na število guldov,
ki jih imamo na voljo po prvi noči.
$$
E(X \mid  Y_2 = k-2x, Z_2 = x) = 0.4 \cdot (k - 2x + 4x) + 0.6 \cdot (k - 2x + x) = k + 0.2x
$$
Spremenljivka $k$ tu predstavlja število guldov po prvi noči,
$x$ pa število kilogramov kupljenega granodiorita pred drugo nočjo.
Ker je $k + 0.2x$ večji od $k$,
Fenko vedno kupi maksimalno količino granodiorita.
Ne glede na izid iz prve noči torej vemo,
da po prvi noči kupi maksimalno število granodiorita.

Obravanavamo torej štiri primere.
Če v prvi noči kupi $2$ kg granodiorita ter izdelava uspe in proda zlato,
ima torej $k = 9$ guldov, s katerimi kupi $x = 4$ kg granodiorita,
en guld pa mu ostane;
če izdelava ne uspe, potem proda granodiorit in ima torej $k = 3$ gulde,
s katerimi kupi $x = 1$ kg granodiorita, en guld pa mu ostane.
Če pa v prvi noči kupi $1$ kg granodiorita ter izdelava uspe in proda zlato,
ima torej $k = 7$ guldov, s katerimi kupi $x = 3$ kg granodiorita,
en guld pa mu ostane;
če izdelava ne uspe, potem proda granodiorit in ima torej $k = 4$ gulde,
s katerimi kupi $x = 2$ kg granodiorita.
\begin{alignat*}{2}
E(X \mid  Y_2 = 1, Z_2 = 4) &= 0.4 \cdot 17 + 0.6 \cdot 5 &&= 9.8 \\
E(X \mid  Y_2 = 1, Z_2 = 1) &= 0.4 \cdot 5 + 0.6 \cdot 2 &&= 3.2 \\
E(X \mid  Y_2 = 1, Z_2 = 3) &= 0.4 \cdot 13 + 0.6 \cdot 4 &&= 7.6 \\
E(X \mid  Y_2 = 0, Z_2 = 2) &= 0.4 \cdot 8 + 0.6 \cdot 2 &&= 4.4
\end{alignat*}
Izračunajmo pričakovne vrednosti še za prvo noč.
Obravnavali bomo dva primera:
v prvem kupi $2$ kg granodiorita, v drugem pa $1$ kg granodiorita.
\begin{alignat*}{2}
E(X \mid  Y_1 = 1, Z_1 = 2) &= 0.4 \cdot 9.8 + 0.6 \cdot 3.2 &&= 5.84 \\
E(X \mid  Y_1 = 3, Z_1 = 1) &= 0.4 \cdot 7.6 + 0.6 \cdot 4.4 &&= 5.68
\end{alignat*}

Na začetku bo torej kupil $2$ kg granodiorita, za kar bo porabil $4$ gulde,
ter oba nastavil prvo noč.
Če uspe, bo pridobil $20$ g zlata, ki jih proda,
in kupi $4$ kg granodiorita, ki jih nastavi drugo noč.
Če mu prvo noč ne uspe,
potem proda granodiorit za $2$ gulda ter kupi $1$ kg svežega granodiorita,
ki ga nastavi drugo noč.
Celoten proces odločanja je prikazan na sliki~\fig.

\begin{slika}
\makebox[\textwidth][c]{
\pgfslika
}
\podnaslov{Odločitveno drevo}
\end{slika}
\end{odgovor}
\end{naloga}
