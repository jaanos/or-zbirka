\begin{naloga}{Janoš Vidali}{Izpit OR 22.6.2020}
\begin{vprasanje}
Večja skupina prijateljev želi preživeti počitnice na jadrnici.
Ker pa je na sami jadrnici le manjše število ležišč,
so se odločili, da bodo prenočevali na kopnem vzdolž poti
(vključno z začetnim in končnim krajem).
Sestavili so usmerjen acikličen graf $G = (V, E)$
z $n$ vozlišči in $m$ povezavami,
v katerem vozlišča predstavljajo kraje, kjer se lahko ustavijo,
vozlišči $u, v \in V$ pa sta povezani z usmerjeno povezavo $uv \in E$,
če lahko v enem dnevu plujejo od kraja $u$ do kraja $v$.
Začetna in končna točka poti sta predstavljeni z vozliščema $s, t \in V$.
Poleg tega za vsako vozlišče $u \in V$ poznajo še število $k_u$,
ki predstavlja, koliko oseb lahko prespi v kraju $u$.
Sestaviti želijo tako pot od $s$ do $t$,
da bo lahko na jadrnici potovalo čim več prijateljev
(tj., maksimizirati želijo minimalno vrednost $k_u$
vseh obiskanih krajev $u$ na poti).
Dolžina poti pri tem ni pomembna.

\begin{enumerate}[(a)]
\item Natančno opiši čim učinkovitejši algoritem (z besedami ali psevdokodo)
za reševanje zgornjega problema
in določi njegovo časovno zahtevnost.
\item Uporabi svoj algoritem,
da v grafu $G = (V, E)$ s slike~\fig poiščeš tako pot od KP do DU,
da bo lahko na jadrnici potovalo čim večje število prijateljev.
Vrednosti $k_u$ ($u \in V$) so zapisane ob vsakem vozlišču.
Natančno zabeleži, kaj se zgodi v vsakem koraku algoritma.
\end{enumerate}

\begin{slika}
\pgfslika
\podnaslov[\nal{}(b)]{Graf}
\end{slika}
\end{vprasanje}

\begin{odgovor}
\begin{enumerate}[(a)]
\item Zgledovali se bomo po algoritmu {\sc NajkrajšaPot} iz naloge~\res[topo].
Zapišimo algoritem,
ki kliče funkciji {\sc Topo} in {\sc RekonstruirajPot} iz iste naloge,
poleg tega pa kliče tudi funkcijo {\sc AdjIn},
ki vrne množico vozlišč v podanem usmerjenem grafu
s povezavo do podanega vozlišča.

\needspace{\baselineskip}
\begin{small}
\begin{algorithmic}
\Function{Jadrnica}{$G = (V, E), s, t, (k_u)_{u \in V}$}
	\State $c \gets$ slovar z vrednostjo $-\infty$ za vsako vozlišče $v \in V$
	\State $\pred \gets$ slovar z vrednostjo $\Null$ za vsako vozlišče $v \in V$
	\For{$u \in$ {\sc Topo}$(G)$}
		\For{$v \in$ {\sc AdjIn}$(G, u)$}
			\If{$c[v] > c[u]$}
				\State $c[v] \gets c[u]$
				\State $\pred[v] \gets u$
			\EndIf
		\EndFor
        \If{$c[u] > k_u$}
            \State $c[u] \gets k_u$
        \EndIf
	\EndFor
    \State \Return {\sc RekonstruirajPot}$(\pred, t)$
\EndFunction
\end{algorithmic}
\end{small}
Algoritem opravi topološko urejanje grafa,
nato vsako povezavo pregleda enkrat,
nazadnje pa opravi še obratni prehod po najdeni poti,
tako da je njegova časovna zahtevnost $O(m)$.

\item Graf s slike~\fig ima topološko ureditev
KP, PU, RI, ML, ZD, DO, ŠI, ST, HV, DU.
Izračunane vrednosti v tabelah $c$ in {\sl pred} v algoritmu iz točke (a)
so prikazane v tabeli~\tab,
iz katere je razvidno,
da gre lahko na jadrnico največ $15$ prijateljev,
kar je mogoče doseči z izbiro poti
KP -- PU -- ZD -- DO -- HV -- DU.
\end{enumerate}
%
\begin{tabela}
\begin{tabular}{cccccccccc}
KP & PU & RI & ML & ZD & DO & ŠI & ST & HV & DU \\
\hline
$20$ & $18_\text{KP}$ & $14_\text{PU}$ & $12_\text{PU}$ & $16_\text{PU}$ &
$15_\text{ZD}$ & $13_\text{ZD}$ & $13_\text{ŠI}$ & $15_\text{DO}$ & $15_\text{HV}$
\end{tabular}
\podnaslov[\res{}(b)]{Izračunane vrednosti v algoritmu}
\end{tabela}
\end{odgovor}
\end{naloga}
