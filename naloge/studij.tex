\begin{naloga}{Janoš Vidali}{Izpit OR 29.8.2017}
\begin{vprasanje}
Peter zaključuje študij na Fakulteti za alternativno znanost.
Opravil je že vse obvezne predmete,
za pristop k zaključnemu izpitu pa mora opraviti še nekaj izbirnih predmetov.
To bi rad storil čim hitreje.
V tabeli~\tab so našteti izbirni predmeti
skupaj s trajanjem (v tednih, od pristopa do uspešnega opravljanja)
in predmeti, h katerim lahko pristopi po uspešnem opravljanju.
K predmetom $a$, $c$ in $f$ lahko pristopi že takoj,
za pristop k ostalim predmetom pa zadostuje, če je opravljen eden od pogojev.

\begin{enumerate}[(a)]
\item Topološko uredi ustrezni graf in ga nariši.
\item Katere predmete naj Peter opravi,
da bo lahko čim prej pristopil k zaključnemu izpitu?
Koliko časa bo za to potreboval?
Natančno opiši postopek iskanja odgovora.
\end{enumerate}

\begin{tabela}
\makebox[\textwidth][c]{
\begin{tabular}{c|lcc}
Oznaka & Predmet & Trajanje & Zadosten pogoj za \\ \hline
$a$ & Alternativna zgodovina   &  5 & $i, j$ \\
$b$ & Astrološki praktikum     &  7 & $d, g$ \\
$c$ & Diskretna numerologija   &  9 & $b$    \\
$d$ & Filozofija magije        &  4 & $z$    \\
$e$ & Kvantno pravo            &  8 & $b, h$ \\
$f$ & Postmoderna ekonomija    &  4 & $e$    \\
$g$ & Telepatija in telekineza &  5 & $z$    \\
$h$ & Teorija antigravitacije  &  4 & $h$    \\
$i$ & Teorije zarote           &  5 & $h$    \\
$j$ & Ufologija II             & 10 & $k$    \\
$k$ & Uvod v kriptozoologijo   &  6 & $z$    \\ \hline
$z$ & Zaključni izpit          &  / & /
\end{tabular}
}
\podnaslov{Podatki}
\end{tabela}
\end{vprasanje}

\begin{odgovor}
\begin{enumerate}[(a)]
\item Ustrezni graf je prikazan na sliki~\fig,
iz katere je razvidna topološka ureditev
$s, c, f, a, e, i, j, b, h, k, d, g, z$.
Vozlišče $s$ je bilo dodano kot izhodiščna točka
s povezavami dolžine $0$ do predmetov,
h katerim lahko Peter pristopi takoj.

\item S pomočjo algoritma {\sc NajkrajšaPot} iz naloge~\res[topo]{}(b)
poiščemo razdalje od vozlišča $s$ do ostalih vozlišč grafa
-- te so zbrane v tabeli~\tab[studij-resitev].
Iz izhoda lahko rekonstruiramo
najkrajšo pot $s - a - i - h - g - z$ dolžine $19$,
ki je prikazana tudi na sliki~\fig.
Peter naj torej zaporedoma opravi predmete $a$, $i$, $h$ in $g$,
kar mu bo omogočilo, da po $19$ tednih pristopi k zaključnemu izpitu.
\end{enumerate}
%
\begin{slika}
\makebox[\textwidth][c]{
\pgfslika
}
\podnaslov{Graf in najkrajša pot}
\end{slika}

\begin{tabela}
\setlabel{studij-resitev}
$$
\begin{array}{ccccccccccccc}
s & c & f & a & e & i & j & b & h & k & d & g & z \\ \hline
0 & 0_s & 0_s & 0_s & 4_f & 5_a & 5_a & 9_c & 10_i & 15_j & 16_b & 14_h & 19_g
\end{array}
$$
\podnaslov[\res{}(b)]{Izračunane vrednosti v algoritmu}
\end{tabela}
\end{odgovor}
\end{naloga}
