\begin{naloga}{Bašić, Gajser}{Kolokvij OR 5.4.2012}
\begin{vprasanje}
Na divjem žuru matematikov se je zbralo $n$ ljudi.
Organizator je priskrbel $p$ različnih vrst sendvičev,
a ne v neomejenih količinah.
Na voljo imajo $u_i$ sendvičev sorte $i$ ($1 \le i \le p$).
Prav tako imajo na voljo $q$ različnih sokov;
$v_j$ predstavlja število sokov sorte $i$ ($1 \le j \le q$).
Za vsakega udeleženca zabave je znano,
katere sokove in sendviče ima rad.

\begin{enumerate}[(a)]
\item Zanima nas, ali lahko hrano in pijačo razdelimo tako,
da dobi vsak udeleženec zabave en sendvič, ki ga ima rad,
in en sok, ki ga ima rad.
Nalogo formuliraj kot problem maksimalnega pretoka v ustreznem grafu.
Jasno napiši, kako iz dobljenega maksimalnega toka ugotovimo,
ali želena razdelitev hrane in pijače obstaja.

\item S pomočjo formulacije iz prejšnje točke
reši nalogo za naslednje konkretne podatke:
\begin{center}
\begin{tabular}{c|ccc|c}
Vrsta soka & Količina & \quad & Vrsta sendviča & Količina \\
\cline{1-2} \cline{4-5}
Ananas          & 1 && Pohanček & 1 \\
Črni ribez      & 1 && Šoferski & 2 \\
Rdeča pomaranča & 2 && Kraški   & 2 \\
Jagoda          & 2 && Vegi     & 1 \\
Borovnica       & 2
\end{tabular}

\bigskip
\begin{tabular}{c|c|c}
Ime & Seznam sokov & Seznam sendvičev \\ \hline
Damir   & ananas, jagoda                & pohanček, šoferski       \\
Jasna   & ananas, črni ribez, jagoda    & kraški, vegi             \\
Bernard & črni ribez, jagoda            & vegi, pohanček, šoferski \\
Matjaž  & ananas, rdeča pomaranča       & kraški, vegi             \\
Sanela  & borovnica, črni ribez, jagoda & vegi                     \\
Patrik  & borovnica, jagoda             & kraški, vegi
\end{tabular}
\end{center}
\end{enumerate}
\end{vprasanje}

\begin{odgovor}
\begin{enumerate}[(a)]
\item Sestavimo omrežje $G = (V, E)$, kjer je množica vozlišč
$$
V = \{s, t\} \cup \set{w_i}{1 \le i \le p}
    \cup \set{x_j}{1 \le j \le q} \cup \set{y_k, z_k}{1 \le k \le n},
$$
v množici povezav $E$ pa imamo povezave
$s \to w_i$ s kapaciteto $u_i$ ($1 \le i \le p$),
$s \to x_j$ s kapaciteto $v_j$ ($1 \le j \le q$),
$w_i \to y_k$ s kapaciteto $1$, če je osebi $k$ všeč sendvič $i$,
$x_j \to z_k$ s kapaciteto $1$, če je osebi $k$ všeč sok $j$,
ter $y_k \to t$ in $z_k \to t$ s kapaciteto $1$ ($1 \le k \le n$).
Želena razdelitev hrane in pijače obstaja natanko tedaj,
ko so v dobljenem maksimalnem pretoku
vse povezave $y_k \to t$ in $z_k \to t$ ($1 \le k \le n$) zasičene.

\item Ustrezno omrežje je prikazano na sliki~\fig[divjizur-resitev1],
kjer je prikazana tudi začetna izbira povečujočih poti
za Ford-Fulkersonov algoritem.
Oznake v indeksih voz\-lišč ustrezajo prvim črkam
v imenih sendvičev, sokov in oseb (brez strešic na šumnikih).
Nadaljevanje postopka je prikazano na sliki~\fig[divjizur-resitev2],
maksimalni pretok pa je prikazan na sliki~\fig[divjizur-resitev3].
Ker v dobljenem maksimalnem pretoku povezava $y_M \to t$ ni zasičena,
sledi, da ne obstaja želena razdelitev sendvičev.
Ker so pa vse povezave $x_j \to z_k$ zasičene
($j \in \{A, B, C, J, R\}$, $k \in \{B, D, J, M, P, S\}$),
sledi, da želena razdelitev sokov obstaja.
\end{enumerate}

\begin{slika}[p]
\makebox[\textwidth][c]{
\pgfslika[divjizur-resitev1]
}
\podnaslov[\res{}(b)]{Omrežje in prvi korak}
\end{slika}
\begin{slika}[p]
\makebox[\textwidth][c]{
\pgfslika[divjizur-resitev2]
}
\podnaslov[\res{}(b)]{Drugi korak}
\end{slika}
\begin{slika}[p]
\makebox[\textwidth][c]{
\pgfslika[divjizur-resitev3]
}
\podnaslov[\res{}(b)]{Maksimalni prerez in minimalni pretok}
\end{slika}
\end{odgovor}
\end{naloga}
