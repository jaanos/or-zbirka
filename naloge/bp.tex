\begin{naloga}{Alen Orbanić}{Izpit OR 28.6.2010}
\begin{vprasanje}
Na naftni ploščadi podjetja BP
je prišlo do nekontroliranih izpustov nafte iz vrtine v zalivu.
Po vseh mogočih podvigih, da bi zamašili vrtino,
se je ob\-upa\-no vodstvo podjetja pod pritiski predsednika bližnje države
začelo dobivati z ruskimi strokovnjaki
za mašenje vrtin s pomočjo jedrskih eksplozij pod vodstvom
prof.~dr.~Mikhaila Razturoviča Totalkova.
Vodstvo BP ocenjuje, da bo ekipa dr.~Totalkova
z verjetnostjo $p = 3/7$ uspešno zamašila vrtino.
Tako bo podjetje imelo sicer samo $2.8$ milijarde \$ dobička,
sicer pa bi zaradi upada dobička in povzročene škode
utrpeli izgubo $700$ milijonov \$.

\begin{enumerate}[(a)]
\item Modeliraj problem v okviru teorije odločanja (stanja, odločitve).
Kakšno odločitev svetuješ vodstvu BP?
Kako je odločitev odvisna od verjetnosti $p$?
Nariši graf, ki prikazuje optimalni pričakovani dobiček v odvisnosti od $p$.

\item Za dodatnih $90\,000$ \$ lahko podjetje BP naroči študijo
pri kitajskem ljudskem inštitutu za naftne vrtine iz mesta Lanzhou,
ki ocenjuje uspešnost rizičnih projektov,
in bi ocenilo, ali bo ekipa dr.~Totalkova uspešna.
Statistični povzetek uspešnosti raziskav inštituta v preteklosti je naslednji:
\begin{center}
\begin{tabular}{l|cc}
$P(\text{rez.~raziskave} \;|\; \text{rez.~projekta})$ &
\multicolumn{2}{c}{Rezultat raziskave inštituta} \\
Rezultat projekta & pozitiven & negativen \\ \hline
uspešen   &  $7/9$ & $2/9$ \\
neuspešen &  $1/3$ & $2/3$
\end{tabular}
\end{center}
Izračunaj $\EVPI$ in $\EVE$ ter komentiraj,
ali je smiselno izvesti dodatno raz\-iska\-vo.

\item Nariši drevo odločitev in ugotovi,
ali naj podjetje naroči raziskavo in če jo,
kako naj se odloči glede mašenja vrtine.
\end{enumerate}
\end{vprasanje}

\begin{odgovor}
\begin{enumerate}[(a)]
\item Imamo eno samo odločitev
-- ali naj izvedemo mašenje po metodi dr.~Totalkova.
Če ga ne izvedemo, je dobiček $0 \$$,
če pa ga, pa preidemo v stanje,
v katerem imamo z verjetnostjo $p$ dobiček $2\,800 \MD$,
z verjetnostjo $1-p$ pa $-700 \MD$.
V tem primeru je pričakovani dobiček enak
$$
p \cdot 2\,800 \MD - (1-p) \cdot 700 \MD
=  p \cdot 3\,500 \MD - 700 \MD .
$$
Za $p < 1/5$ torej pričakujemo izgubo
in se podjetju mašenja ne izplača izvesti,
sicer pa podjetje pričakuje dobiček, tako da se izvedba mašenja izplača.
Graf pričakovanega dobička je prikazan na sliki~\fig[bp-graf].
Pri podani vrednosti $p = 3/7$ se podjetju torej izplača izvesti mašenje,
pri čemer pričakuje dobiček $800 \MD$.

\item EVPI izračunamo tako,
da od pričakovanega dobička pri znanem izidu
odštejemo pričakovani dobiček pri neznanem izidu:
$$
\EVPI = 3/7 \cdot 2\,800 \MD + 4/7 \cdot 0 \MD - 800 \MD = 400 \MD
$$
Ker je EVPI večji od cene dodatnega testiranja, izračunajmo še EVE.
Najprej določimo verjetnosti:
\begin{align*}
P(\text{raziskava pozitivna}) =
{7 \over 9} \cdot {3 \over 7} + {1 \over 3} \cdot {4 \over 7}
&= {11 \over 21} \\
P(\text{raziskava negativna}) =
{2 \over 9} \cdot {3 \over 7} + {2 \over 3} \cdot {4 \over 7}
&= {10 \over 21} \\
P(\text{uspešen projekt} \mid \text{raziskava pozitivna}) =
{7/9 \cdot 3/7 \over 11/21} &= {7 \over 11} \\
P(\text{neuspešen projekt} \mid \text{raziskava pozitivna}) =
{1/3 \cdot 4/7 \over 11/21} &= {4 \over 11} \\
P(\text{uspešen projekt} \mid \text{raziskava negativna}) =
{2/9 \cdot 3/7 \over 10/21} &= {1 \over 5} \\
P(\text{neuspešen projekt} \mid \text{raziskava negativna}) =
{2/3 \cdot 4/7 \over 10/21} &= {4 \over 5}
\end{align*}
Sedaj lahko izračunamo EVE:
\begin{multline*}
\EVE = (7/11 \cdot 2\,800 \MD - 4/11 \cdot 700 \MD) \cdot 11/21 + \\
(1/5 \cdot 0 \MD + 4/5 \cdot 0 \MD) \cdot 10/21 - 800 \MD = 0 \MD
\end{multline*}
Ker je EVE manjši od cene raziskave, se zanjo ne odločimo.

\item Odločitveno drevo je prikazano na sliki~\fig[bp-drevo].
Za raziskavo se ne bomo odločili,
pač pa bomo poskusili zamašiti vrtino po metodi dr.~Totalkova.
\end{enumerate}
%
\begin{slika}[p]
\pgfslika[bp-graf]
\podnaslov[\res{}(a)]{Graf pričakovanega dobička}
\end{slika}
%
\begin{slika}[p]
\makebox[\textwidth][c]{
\pgfslika[bp-drevo]
}
\podnaslov[\res{}(c)]{Odločitveno drevo}
\end{slika}
\end{odgovor}
\end{naloga}
