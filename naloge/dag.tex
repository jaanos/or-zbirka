\begin{naloga}{Janoš Vidali}{Izpit OR 5.7.2018}
\begin{vprasanje}
Dan je utežen usmerjen acikličen graf s slike~\fig.

\begin{enumerate}[(a)]
\item Poišči topološko ureditev grafa s slike~\fig.

\item Poišči najcenejšo pot od vozlišča $S$ do vozlišča $T$
v grafu s slike~\fig.

\item Naj bo $G = (V, E)$ usmerjen acikličen graf
z nenegativno uteženimi povezavami
ter $s, t \in V$ njegovi vozlišči.
Algoritem $\A$ se po grafu $G$ sprehaja po naslednjem pravilu:
začne v vozlišču $s$,
v vsakem koraku pa se iz vozlišča $u$ premakne
v njegovega izhodnega soseda $v$ z verjetnostjo
$$
p_{uv} = {\ell_{uv} \over \sum_{u \to w} \ell_{uw}} ,
$$
kjer je $\ell_{uv}$ teža povezave od $u$ do $v$.
Algoritem $\A$ se ustavi, ko doseže vozlišče $t$.

Natančno opiši (z besedami ali psevdokodo),
kako bi v času $O(m)$ (kjer je $m = |V| + |E|$)
za vsako vozlišče $u \in V$ določil verjetnost $q_u$,
da algoritem $\A$ obišče vozlišče $u$.
Verjetnosti za graf s slike~\fig ni potrebno računati.
\end{enumerate}

\begin{slika}
\pgfslika
\podnaslov{Graf}
\end{slika}
\end{vprasanje}
\begin{odgovor}
\begin{enumerate}[(a)]
\item Topološko ureditev dobimo z uporabo algoritma iz naloge~\nal[topo].
Dobimo $[s, b, a, d, h, e, c, f, i, g, t]$. 
Graf v tej ureditvi si lahko ogledamo na sliki ~\fig[dag_res].
\begin{slika}
\pgfslika[dag_res]
\podnaslov[\res{}(a)]{Predstavitev topološko urejenega grafa}
\end{slika}

\item Najkrajše razdalje do vsakega vozlišča $v \in V$, 
$d_v$  bomo poiskali s pomočjo dinamičnega programiranja in rekurzivne zveze
\begin{align*}
d_v &= \min\{d_v,  d_u + \ell_{uv}| \text{ za } u \rightarrow v\} \\
d_s &= 0,
\end{align*}
Vrednosti $d_v$ bomo pridobivali po vstnem redu, ki ga določa topološka ureditev.
\begin{small}
\begin{algorithmic}
\Function{dagRazdalje}{$G = (V, E), \ell, s, t$}
	\State topo $\gets$ {\sc Topo}$(G)$
	\State $D \gets$ slovar s ključi $v \in V$ in vrednostmi $\infty$
	\State $D[s] \gets 0$
	\For{$u \in$ topo}
		\For{$v \in \Adj(G, u)$}
			\State $D[v] \gets \min(D[v], D[u] + \ell_{uv})$
		\EndFor
	\EndFor
	\State \Return $D[t]$
\EndFunction
\end{algorithmic}
\end{small}
Po klicu {\sc dagRazdalje}($G, \ell, s, t$) dobimo 11.

\item Ker se algoritem odloča neodvisno od svojih prejšnjih odločitev so vsi dogodki neodvisni.
Verjetnost, da pridemo do vozlišča $v$ bo vsota vseh možnih 
verjetnosti dogodkov potovanja po (različnih) poteh, ki se končajo v $v$.
Verjetnost potovanja po določeni poti pa bo produkt verjetnosti potovanja po vsaki vsebujoči povezavi.
Za verjetnosti $q_v$ v dag grafu torej velja naslednja zveza
\begin{align*}
q_v &= \sum_{u \rightarrow v}p_{uv} q_u \\
q_s &= 1.
\end{align*}
Spet bomo iskali vrednosti $q_v$ po topološkem vrstnem redu.
\begin{small}
\begin{algorithmic}
\Function{dagVerjetnosti}{$G, s$}
	\State topo $\gets$ {\sc Topo}$(G)$
	\State $q_v \gets 0$ za $v \in V \setminus \{s\}$
	\State $q_s \gets 1$
	\For{$u \in$ topo}
		\For{$v \rightarrow u$}
			\State $q_v = q_v + q_u p_{uv}$
		\EndFor
	\EndFor
\EndFunction
\end{algorithmic}
\end{small}
Časovna zahtevnost algoritma je $O(m)$, 
saj gremo po vsaki povezavi preko vsakega vozlišča enkrat, 
pri čemer pa moramo seveda paziti, da ne izračunamo vsote, 
s katero je definirana $p_{uv}$ vsakič v notranjii zanki, 
saj jo lahko raje vsak obhod enkrat v zunanji.
\end{enumerate}
\end{odgovor}
\end{naloga}
