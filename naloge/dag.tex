\begin{naloga}{Janoš Vidali}{Izpit OR 5.7.2018}
\begin{vprasanje}
Dan je utežen usmerjen acikličen graf s slike~\fig.

\begin{enumerate}[(a)]
\item Poišči topološko ureditev grafa s slike~\fig.

\item Poišči najcenejšo pot od vozlišča $s$ do vozlišča $t$
v grafu s slike~\fig.

\item Naj bo $G = (V, E)$ usmerjen acikličen graf
z nenegativno uteženimi povezavami
ter $s, t \in V$ njegovi vozlišči.
Algoritem $\A$ se po grafu $G$ sprehaja po naslednjem pravilu:
začne v vozlišču $s$,
v vsakem koraku pa se iz vozlišča $u$ premakne
v njegovega izhodnega soseda $v$ z verjetnostjo
$$
p_{uv} = {\ell_{uv} \over \sum_{u \to w} \ell_{uw}} ,
$$
kjer je $\ell_{uv}$ teža povezave od $u$ do $v$.
Algoritem $\A$ se ustavi, ko doseže vozlišče $t$.

Natančno opiši (z besedami ali psevdokodo),
kako bi v času $O(m)$ (kjer je $m = |V| + |E|$)
za vsako vozlišče $u \in V$ določil verjetnost $q_u$,
da algoritem $\A$ obišče vozlišče $u$.
Verjetnosti za graf s slike~\fig ni potrebno računati.
\end{enumerate}

\begin{slika}
\pgfslika
\podnaslov{Graf}
\end{slika}
\end{vprasanje}

\begin{odgovor}
\begin{enumerate}[(a)]
\item Topološko ureditev dobimo z uporabo algoritma {\sc Topo}
iz naloge~\res[topo]{}(a).
Dobimo $[s, b, a, d, h, e, c, f, i, g, t]$. 
Graf v tej ureditvi si lahko ogledamo na sliki~\fig[dag-resitev].

\item Najkrajšo pot od vozlišča $s$ do vozlišča $t$
lahko izračunamo z algoritmom {\sc NajkrajšaPot} iz naloge~\res[topo]{}(b).
Ta nam vrne pot $s - a - d - h - i - t$ dolžine $11$.
Delovanje algoritma si lahko ogledamo v tabeli~\tab.

\begin{tabela}
\begin{tabular}{cc|cc}
$u$ & $v$ & $\pred[v]$ & $d_G[v]$ \\ \hline
$ s$ & $a$ & $s$ & $2$ \\
$ s$ & $b$ & $s$ & $3$ \\
$ s$ & $c$ & $s$ & $5$ \\
$ b$ & $a$ & $s$ & $2$ \\
$ b$ & $c$ & $s$ & $5$ \\
$ b$ & $e$ & $b$ & $9$ \\
$ b$ & $f$ & $b$ & $7$ \\
$ a$ & $d$ & $a$ & $8$ \\
$ d$ & $e$ & $b$ & $9$ \\
$ d$ & $g$ & $d$ & $14$ \\
$ d$ & $h$ & $d$ & $9$ \\
$ h$ & $e$ & $b$ & $9$ \\
$ h$ & $g$ & $d$ & $14$ \\
$ h$ & $i$ & $h$ & $10$ \\
$ h$ & $t$ & $h$ & $12$ \\
$ e$ & $c$ & $s$ & $5$ \\
$ e$ & $f$ & $b$ & $7$ \\
$ e$ & $g$ & $e$ & $13$ \\
$ g$ & $t$ & $h$ & $12$ \\
$ c$ & $f$ & $b$ & $7$ \\
$ f$ & $i$ & $h$ & $10$ \\
$ i$ & $t$ & $i$ & $11$ \\
\end{tabular}
\podnaslov[\res{}(b)]{Potek izvajanja algoritma {\sc NajkrajšaPot}}
\end{tabela}

\item Ker se algoritem odloča neodvisno od svojih prejšnjih odločitev,
so vsi dogodki neodvisni.
Verjetnost, da pridemo do vozlišča $v$,
bo vsota verjetnosti vseh dogodkov potovanja po (različnih) poteh,
ki se končajo v $v$.
Verjetnost potovanja po določeni poti pa bo produkt verjetnosti potovanja po vsaki vsebujoči povezavi.
Za verjetnosti $q_v$ v usmerjenem acikličnem grafu torej velja zveza
\begin{align*}
q_v &= \sum_{u \rightarrow v}p_{uv} q_u \\
q_s &= 1.
\end{align*}
Vrednosti $q_v$ bomo iskali po topološkem vrstnem redu.
\begin{small}
\begin{algorithmic}
\Function{dagVerjetnosti}{$G = (V, E), s, \ell : E \to \R_+$}
	\State $q \gets$ slovar z vrednostjo $0$ za vsako vozlišče $v \in V$
	\State $q[s] \gets 1$
	\For{$u \in$ {\sc Topo}$(G)$}
        \State $p \gets \sum_{w \in \Adj_G(u)} \ell_{uw}$
		\For{$v \in \Adj_G(u)$}
			\State $q[v] \gets q[v] + q[u] \ell_{uv} / p$
		\EndFor
	\EndFor
    \State \Return $q$
\EndFunction
\end{algorithmic}
\end{small}
Časovna zahtevnost algoritma je $O(m)$, 
saj gremo po vsaki povezavi preko vsakega vozlišča enkrat, 
pri čemer pa moramo seveda poskrbeti,
da vsote v števcu definicije vrednosti $p_{uv}$ ne računamo vsakič sproti,
pač pa to storimo pred posodabljanjem verjetnosti sosedom vozlišča $u$.
\end{enumerate}
%
\begin{slika}
\pgfslika[dag-resitev]
\podnaslov[\res{}(a)]{Predstavitev topološko urejenega grafa}
\end{slika}
\end{odgovor}
\end{naloga}
