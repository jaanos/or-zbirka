\begin{naloga}{Janoš Vidali}{Kolokvij OR 5.6.2023}
\begin{vprasanje}
V kriminalni združbi so se odločili,
da bodo iz bencinskih črpalk v okoliških mestih poskusili ukrasti čim več bencina.
V ta namen so si priskrbeli tovornjak s predelano posodo za gorivo,
ki lahko gorivo preusmeri v cisterno.
V vsakem mestu, kjer se bodo ustavili,
bodo na bencinski črpalki načrpali vnaprej določeno količino goriva,
nato pa se odpeljali, ne da bi plačali.
Tovornjak se trenutno nahaja v enem izmed mest
-- ker v njem trenutno ni nič bencina,
bodo torej prvi rop izpeljali kar v tem mestu,
nato pa se bodo lahko z načrpanim gorivom peljali naprej
(lahko predpostaviš, da jim ga za tem ne bo zmanjkalo).
Seveda bodo iz vsakega mesta, kjer bodo kradli,
za njimi poslali šerifove može,
tako da se sme med vsakima obiskanima mestoma
tovornjak torej peljati le po najkrajši poti
(sicer tvegajo, da jih bodo ujeli).
Pot želijo končati v eni od varnih točk,
kjer lahko tovornjak neopazno izgine in se tako izogne temu,
da bi ga ujeli
(lahko pa tudi peljejo naprej in pot končajo v drugi varni točki).
Cilj je torej, da v eno od varnih točk pripeljejo čim večjo količino bencina.

Cestno omrežje je predstavljeno z neusmerjenim grafom $G = (V, E)$,
tako da lahko množico $V$ zapišemo kot unijo disjunktnih množic $A$ in $B$,
kjer vozlišča iz množice $A$ predstavljajo mesta z bencinskimi črpalkami,
vozlišča iz množice $B$ pa varne točke.
Poleg tega poznamo še nenegativne dolžine povezav med vozlišči
v kilometrih $\ell_{uv}$ ($uv \in E$)
in nenegativne uteži vozlišč $t_v$ ($v \in V$),
ki predstavljajo količino bencina v litrih,
ki jih lahko ukradejo v danem mestu,
ter začetno mesto $r \in A$
in porabo $p$ v litrih na kilometer.
Lahko predpostaviš,
da velja $t_r > 0$ in $t_v = 0$ za vsak $v \in B$.

Če povzamemo, iščemo torej tako pot $u_0 u_1 \dots u_s$ v grafu $G$,
da je $u_i u_{i+1} \dots u_j$ najkrajša pot med vozliščema $u_i$ in $u_j$
($1 \le i \le j \le s$)
ter $u_0 = r$ in $u_s \in B$,
ki maksimizira vrednost $\sum_{i=1}^s (t_{u_i} - p \ell_{u_{i-1} u_i})$.

\begin{enumerate}[(a)]
\item Predpostavimo, da je graf $G$ drevo in da vzamemo vozlišče $r$ za njegov koren.
Zapiši rekurzivne enačbe za reševanje danega problema ter pojasni,
v katerem vrstnem redu računamo spremenljivke,
kako iz podatkov pridemo do tega vrstnega reda,
ter kako dobimo optimalno rešitev.
Kakšna je časovna zahtevnost celotnega algoritma?

\item Uporabi algoritem iz točke ()a),
da rešiš problem za drevo s slike~\fig
pri vrednosti $p = 0.1$.
Vrednosti $\ell_{uv}$ ($uv \in E$) in $t_v$ ($v \in V$)
so prikazane pri povezavah oziroma vozliščih.
Vozlišča (mesta) iz množice $A$ so prikazane z belo barvo,
vozlišča (varne točke) iz množice $B$ pa s sivo barvo.
Treba je torej poiskati pot, ki naj jo prevozijo,
in količino bencina, ki jo bodo pripeljali v varno točko.

\item Opiši algoritem, ki reši dani problem za splošen neusmerjen graf $G$
(tj., ne nujno drevo).
Kakšna je njegova časovna zahtevnost?
\namig{prevedi problem na problem iz točke (a).}
\end{enumerate}

\begin{slika}
\makebox[\textwidth][c]{
\pgfslika
}
\podnaslov[\nal{}(b)]{Drevo}
\end{slika}
\end{vprasanje}

\begin{odgovor}
\begin{enumerate}[(a)]
\item Naj oznaka $u \to v$ pomeni,
da je vozlišče $v$ neposredni potomec vozlišča $u$.
Uvedimo še konstante
$$
z_u = \begin{cases}
-\infty & u \in A, \\
0 & u \in B.
\end{cases}
$$
Spremenljivka $x_u$ ($u \in V$) bo podala največjo količino bencina,
ki jo lahko prepeljemo od vozlišča $u$ do nekega njegovega potomca $v \in B$.
Zapišimo rekurzivno enačbo.
$$
x_u = t_u + \max\left(\{z_u\} \cup \{x_v - p \ell_{uv} \mid u \to v\}\right) .
$$
Vrednosti $x_u$ računamo od listov v smeri proti korenu $r$,
optimalno vrednost pa dobimo kot $x^* = x_r$.
Ustrezni vrstni red lahko dobimo tako,
da na drevesu opravimo iskanje v širino ali globino iz korena $r$
in izračunamo topološko ureditev glede na dobljeno tabelo prednikov vozlišč.
Časovna zahtevnost celotnega algoritma je potem $O(n)$,
kjer je $n$ število vozlišč v drevesu
-- tako drevo ima namreč $n-1$ povezav,
in vsako od njih obravnavamo konstantno mnogo krat.

\item Izračunajmo vrednosti $x_u$ ($u \in V$).
\begin{alignat*}{2}
x_k &= 0 + 0 &&= 0 \\
x_j &= 0 + 0 &&= 0 \\
x_i &= 32 + \max\{-\infty, \underline{0 - 15.3}\} &&= 16.7 \\
x_h &= 47 + \max\{-\infty, \underline{0 - 3.2}\} &&= 43.8 \\
x_g &= 0 + \max\{\underline{0}, 16.7 - 18.7\} &&= 0 \\
x_f &= 123 - \infty &&= -\infty \\
x_e &= 0 + 0 &&= 0 \\
x_d &= 51 + \max\{-\infty, \underline{43.8 - 6.2}\} &&= 88.6 \\
x_c &= 179 + \max\{-\infty, -\infty - 3.7, \underline{0 - 1.4}\} &&= 177.6 \\
x_b &= 89 + \max\{-\infty, \underline{88.6 - 9.3}, 0 - 0.5\} &&= 168.3 \\
x_a &= 0 + \max\{0, \underline{177.6 - 10.2}\} &&= 167.4 \\
x_r &= 145 + \max\{-\infty, \underline{167.4 - 7.4}, 168.3 - 8.8\} &&= 305.0
\end{alignat*}
Pripeljemo lahko torej $x^* = x_r = 305$ litrov bencina.
Poglejmo, katero pot bomo opravili.
\begin{align*}
x^* = x_r &= t_r + x_a - p \ell_{ra} && r \to a \\
x_a &= t_a + x_c - p \ell_{ac}       && a \to c \\
x_c &= t_c + x_g - p \ell_{cg}       && c \to g \\
x_g &= t_g + z_g                     && \text{konec pri $g$}
\end{align*}
Tovornjak bo torej opravil pot $r \to a \to c \to g$.

\item Na grafu $G$ izvedemo Dijkstrov algoritem z začetkom v vozlišču $r$
-- tako dobimo drevo najkrajših poti iz vozlišča $r$.
Na dobljenem drevesu lahko potem uporabimo rekurzivne enačbe iz točke (a),
da poiščemo želeno pot.
Časovna zahtevnost celotnega algoritma je tako $O(m \log n)$,
če v Dijkstrovem algoritmu uporabimo kopico,
oziroma $O(n^2)$,
če uporabimo seznam,
pri čemer je $m$ število povezav v grafu.
\end{enumerate}
\end{odgovor}
\end{naloga}
