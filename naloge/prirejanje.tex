\begin{naloga}{Janoš Vidali}{Kolokvij OR 3.6.2021}
\begin{vprasanje}
{\em Prirejanje} v grafu $G = (V, E)$ je taka množica $M \subseteq E$,
da je vsako vozlišče $u \in V$ krajišče največ ene povezave iz $M$.

Dano je drevo $T = (V, E)$ in uteži povezav $c_e$ ($e \in E$).
V drevesu $T$ želimo najti {\em najtežje prirejanje}
-- torej tako množico povezav $M \subseteq E$,
ki maksimizira vsoto njihovih uteži, torej vrednost $\sum_{e \in M} c_e$.

\begin{enumerate}[(a)]
\item Napiši rekurzivne enačbe za reševanje problema
najtežjega prirejanja v drevesu $T$.
Razloži, kaj pred\-stav\-lja\-jo spremenljivke,
v kakšnem vrstnem redu jih računamo, ter kako dobimo optimalno rešitev.
Predpostaviš lahko,
da imaš poleg seznamov sosedov in uteži povezav drevesa $T$
na voljo tudi tabelo $\pred$ (glej nalogo~\nal[neodvisna]{}(a)),
ki za vsako vozlišče $v \in V$ določa njegovega prednika,
če za koren izbereš vozlišče $r \in V$.
Koren $r$ je lahko izbran poljubno, zanj pa velja $\pred[r] = \Null$.
\namig{za vsako vozlišče uporabi dve spremenljivki
-- eno za primer, ko je vozlišče krajišče povezave iz prirejanja,
in eno za primer, ko ni.}

\item Natančno opiši postopek (z besedami ali psevdokodo),
ki iz zgoraj izračunanih vred\-no\-sti
sestavi najtežje prirejanje v $T$.

\item Oceni časovno zahtevnost algoritma iz točk (a) in (b).

\item S svojim algoritmom poišči najtežje prirejanje
na drevesu s slike~\fig.
Iz re\-šit\-ve naj bo jasno, kako poteka izvajanje algoritma.
\end{enumerate}
%
\begin{slika}
\makebox[\textwidth][c]{
\pgfslika
}
\podnaslov[\nal{}(d)]{Drevo}
\end{slika}
\end{vprasanje}

\begin{odgovor}
\begin{enumerate}[(a)]
\item Naj bosta $x_u$ in $y_u$ teži najtežjega prirejanja $M$
v poddrevesu s korenom $u$ (glede na slovar $\pred$),
za katerega velja, da ne vsebuje nobene povezave s krajiščem $u$,
oziroma da tako povezavo lahko vsebuje.
Potem velja
\begin{align*}
x_u &= \sum_{\substack{v \sim u \\ v \ne \pred[u]}}\!\!\!\! y_v
\qquad \text{in} \\
y_u &= x_u + \max\left(\{0\} \cup \{c_{uv} + x_v - y_v \mid v \sim u, v \ne \pred[u]\}\right).
\end{align*}
Vrednosti $x_u$ in $y_u$ računamo od listov v smeri proti korenu $r$
(tj., v topološki ureditvi glede na $\pred$).
Težo najtežje neodvisne množice dobimo kot $c^* = y_r$.

\item Da iz vrednosti $x_u$ in $y_u$ ($u \in V$)
izluščimo najtežje prirejanje $M$ v drevesu $T$,
se sprehodimo od korena proti listom (npr. z iskanjem v širino)
in za posamezno vozlišče $u$,
za katerega velja $x_u \ne y_u$
in v $M$ nimamo povezave do njegovega predhodnika,
v množico $M$ dodamo povezavo do enega od naslednikov $v$,
za katerega velja $y_u + y_v - x_u - x_v = c_{uv}$.
\begin{small}
\begin{algorithmic}
\Function{NajtežjePrirejanje}%
        {$T = (V, E), r, c : E \to \R, (x_u)_{u \in V}, (y_u)_{u \in V}$}
    \State $M \gets$ prazna množica
    \Function{Visit}{$u, w$}
        \If{$x_u \ne y_u \land (w = \Null \lor uw \not\in M)$}
            \State $v \gets$ eno izmed vozlišč iz $\Adj_G(u) \setminus \{w\}$ z $y_u + y_v - x_u - x_v = c_{uv}$
            \State $M.\add(uv)$
        \EndIf
    \EndFunction
    \State {\sc bfs}$(T, \{r\}, \visit)$
    \State \Return $S$
\EndFunction
\end{algorithmic}
\end{small}

\item Naj bo $n$ število vozlišč drevesa $T$.
Potem ima $T$ natanko $n-1$ povezav.
Za izračun najtežjega prirejanja naredimo tri preglede v širino,
pri čemer v vsakem vozlišču porabimo konstantno mnogo časa.
Časovna zahtevnost celotnega algoritma je torej $O(n)$.

\item Izračunajmo vrednosti $x_u$ in $y_u$ ($u \in V$),
če za koren vzamemo vozlišče $a$.
\begin{alignat*}{4}
x_i &&&= 0 &\qquad y_i &&&= 0 \\
x_j &&&= 0 &\qquad y_j &&&= 0 \\
x_k &&&= 0 &\qquad y_k &&&= 0 \\
x_d &&&= 0 &\qquad y_d &&&= 0 \\
x_e &= 0+0 &&= 0 &\qquad
y_e &= 0 + \max\{0, 4, \underline{9}\} &&= 9 \\
x_f &&&= 0 &\qquad y_f &&&= 0 \\
x_g &&&= 0 &\qquad
y_g &= 0 + \max\{0, \underline{1}\} &&= 1 \\
x_h &&&= 0 &\qquad y_h &&&= 0 \\
x_b &= 0+9+0 &&= 9 &\qquad
y_b &= 9 + \max\{0, \underline{8}, -1, 4\ &&= 17 \\
x_c &= 1+0 &&= 1 &\qquad
y_c &= 1 + \max\{0, 1, \underline{4}\} &&= 5 \\
x_a &= 17+5 &&= 22 &\qquad
y_a &= 22 + \max\{0, -7, \underline{5}\} &&= 27
\end{alignat*}
Teža najtežjega prirejanja $M$ je torej $c^* = y_a = 27$.
Poglejmo, katere povezave so v množici $M$.
\begin{align*}
c^* = y_a &= c_{ac} + x_a + x_c - y_c && ac \in M \\
y_b &= c_{bd} + x_b + x_d - y_d && bd \in M \\
x_c &= y_g + y_h \\
y_e &= c_{ej} + x_e + x_j - y_j && ej \in M \\
x_g &= c_{gk} + x_g + x_k - y_k && gk \in M
\end{align*}
Najtežje prirejanje je torej $M = \{ac, bd, ej, gk\}$.
\end{enumerate}
\end{odgovor}
\end{naloga}
