\begin{naloga}{Gabrovšek, Konvalinka}{Kolokvij OR 25.11.2010}
\begin{vprasanje}
Otoke z grafa na sliki~\fig želimo povezati z mostovi,
tako da bo skupna cena grad\-nje čim manjša.
Cena gradnje mostu je milijon evrov na kilometer
(razdalje v kilometrih so označene na sliki;
na povezavah, ki niso narisane,
zaradi tehničnih preprek ne moremo zgraditi mostov).
Med katerimi otoki naj zgradimo mostove?
Pri iskanju odgovora uporabi primeren algoritem
in zapiši vse vmesne rezultate.
Kakšen je odgovor,
če je za vsak zgrajeni most potrebno dobiti potrdilo inšpektorjev,
ti pa za vsak pregled zaračunajo milijon evrov?

\begin{slika}
\pgfslika
\podnaslov{Graf}
\end{slika}
\end{vprasanje}

\begin{odgovor}
S Primovim algoritmom bomo poiskali najlažje vpeto drevo v grafu s slike~\fig.
Algoritem deluje tako,
da začne z drevesom z enim samim (poljubno izbranim) vozliščem,
nato pa v vsakem koraku v drevo doda najcenejšo izmed povezav,
ki imajo natanko eno krajišče v drevesu,
dokler v drevesu niso vsa vozlišča.
Potek izvajanja algoritma je prikazan v tabeli~\tab,
najcenejše vpeto drevo s ceno $39$
pa je prikazano na sliki~\fig[otoki-resitev].
Skupna cena gradnje je torej $39$ milijonov evrov.

Če mora vsak most pridobiti še potrdilo inšpektorjev,
se rešitev ne spremeni,
saj ima vsako vpeto drevo enako število povezav.
Cena gradnje se v tem primeru poveča na $45$ milijonov evrov.
%
\begin{tabela}
$$
\begin{array}{c|c|l}
\text{dodano vozlišče} & \text{teža drevesa} & \text{vrsta povezav} \\ \hline
a &  0 & ad: 5, ac: 7 \\
d &  5 & df: 6, ac: 7, dc: 9, de: 15 \\
f & 11 & ac: 7, dc: 9, fe: 10, fg: 11, de: 15 \\
c & 18 & ce: 7, cb: 8, fe: 10, fg: 11, de: 15 \\
e & 25 & eb: 5, cb: 8, eg: 9, fg: 11 \\
b & 30 & eg: 9, fg: 11 \\
g & 39 &
\end{array}
$$
\podnaslov{Potek izvajanja algoritma}
\end{tabela}
%
\begin{slika}
\pgfslika[otoki-resitev]
\podnaslov{Najcenejše vpeto drevo}
\end{slika}
\end{odgovor}
\end{naloga}
