\begin{naloga}{Janoš Vidali}{Kolokvij OR 12.4.2021}
\begin{vprasanje}
Janez si je ravnokar odgovoril na vprašanje iz naloge~\nal[zemlja],
ko je ugotovil,
da so uradi zaradi epidemiološke situacije zaprti
in da postopka ne bo mogel začeti v naslednjih $t$ dneh,
kjer je $t \in [7, 15]$
(vrednost $t$ bo znana v kratkem).
Na srečo lahko Janez v tem času še vedno pridobi soglasja sosedov,
za kar potrebuje $15$ dni,
prav tako lahko mu lahko pomaga Bine,
ki za izdelavo svojega mnenja še vedno potrebuje $7$ dni.
Po izteku $t$ dni bo lahko Janez sprožil postopek
(seveda bo pred tem počakal na Binetovo mnenje oziroma soglasja sosedov,
če se za to odloči),
ki bo trajal $55$ dni, če se izkaže,
da so soglasja potrebna in jih pred tem še ni pridobil,
oziroma $30$ dni sicer.
Vse verjetnosti ostajajo nespremenjene.

Nariši odločitveno drevo
in ga uporabi pri sprejemanju odločitev v odvisnosti od vrednosti parametra $t$.
\end{vprasanje}

\begin{odgovor}
Ker je $t \le 15$, bo trajanje celotnega postopka v primerih,
da se Janez pred sprožitvijo postopka odloči za pridobitev soglasij,
ostalo nespremenjeno glede na trajanje iz naloge~\res[zemlja].
Ker velja tudi $t \ge 7$,
bo celoten postopek v primeru, da soglasja niso potrebna, trajal $30+t$ dni,
v nasprotne primeru pa $55+t$ dni
-- oboje neodvisno od tega, ali Janez vpraša Bineta za mnenje.

S pomočjo zgoraj izračunanih trajanj
in verjetnosti z odločitvenega drevesa na sliki~\fig[zemlja]
lahko narišemo odločitveno drevo s slike~\fig.
Opazimo,
da je v primerih,
če Janez ne pridobi Binetovega mnenja ali če je to neugodno,
pričakovani čas trajanja celotnega postopka
ob takojšnji prošnji za gradbeno dovoljenje
večji kot v primeru, če pred tem pridobi soglasja.
Če pa Janez od Bineta pridobi ugodno mnenje,
se mu bolj izplača takoj zaprositi za gradbeno dovoljenje,
natanko tedaj, ko velja $t < 14.31$.
Izračunajmo torej pričakovano trajanje v vozlišču $B$:
$$
E(B) = {13 \over 24} E(D) + {11 \over 24} \cdot 52 =
\begin{cases}
44.25 + {13 \over 24} t & \text{če $t < 14.31$} \\
52 & \text{če $t \ge 14.31$}
\end{cases}
$$
Poglejmo si, za katere vrednosti $t$ se Janezu izplača prositi Bineta za mnenje.
\begin{align*}
44.25 + {13 \over 24} t &< 45 \\
t &< {18 \over 13} < 7
\end{align*}
Vidimo, da zgornja neenakost ne velja za noben $t \in [7, 15]$,
tako da se Janezu torej ne izplača pridobiti Binetovega mnenja,
pač pa naj najprej pridobi soglasja sosedov.
Celoten postopek bo tako trajal 45 dni.
\begin{slika}
\makebox[\textwidth][c]{
\pgfslika
}
\podnaslov{Odločitveno drevo}
\end{slika}
\end{odgovor}
\end{naloga}
