\begin{naloga}{Sergio Cabello}{Izpit OR 15.3.2017}
\begin{vprasanje}
Za zaporedje števil $x_1, x_2, \dots x_m$ pravimo, da je {\em oscilirajoče},
če velja $x_i < x_{i+1}$ za vse sode $i$ in $x_i > x_{i+1}$ za vse lihe $i$.
S pomočjo dinamičnega programiranja zasnuj algoritem,
ki v polinomskem času izračuna dolžino najdaljšega oscilirajočega podzaporedja
zaporedja celih števil $a_1, a_2, \dots a_n$.
\end{vprasanje}

\begin{odgovor}
Naj bosta $p_i$ in $q_i$ dolžini najdaljših oscilirajočih podzaporedij
lihe oziroma sode dolžine zaporedja celih števil $a_1, a_2, \dots a_i$,
ki se končata z elementom $a_i$.
Določimo rekurzivne enačbe.
\begin{align*}
p_i &= 1 + \max\left(\set{q_j}{1 \le j \le i-1 \land a_j < a_i} \cup \{0\}\right)
&& (1 \le i \le n) \\
q_i &= \max\left(\set{1 + p_j}{1 \le j \le i-1 \land a_j > a_i} \cup \{0\}\right)
&& (1 \le i \le n)
\end{align*}
Vrednosti $p_i$ in $q_i$ računamo naraščajoče glede na indeks $i$.
Maksimalno dolžino oscilirajočega podzaporedja dobimo kot
$\ell^* = \max\set{p_i, q_i}{1 \le i \le n}$.
Algoritem, ki sledi iz zgornjih enačb, poišče rešitev v času $O(n^2)$.

Časovno zahtevnost je mogoče sicer nekoliko izboljšati.
Opazimo namreč, da v $i$-tem koraku
iščemo največji taki vrednosti $q_j$ in $p_j$ ($j < i$),
za kateri je vrednost $a_j$ manjša oziroma večja od trenutne vrednosti $a_i$.
Če torej za vsako možno vrednost $p_i$ in $q_i$
hranimo največjo oziroma najmanjšo vrednost $a_i$,
pri kateri je ta dosežena,
lahko omejimo število korakov pri izračunu posamezne vrednosti
na dolžino najdaljšega oscilirajočega podzaporedja.
\begin{small}
\begin{algorithmic}
\Function{OscilirajočePodzaporedje}{$(a_i)_{i=1}^n$}
	\State $r \gets$ prazen slovar največjih in najmanjših koncev podzaporedij
	\State $r[-1] \gets \infty$ \hfill pogoji za ustavitev iskanja
	\State $r[0] \gets -\infty$ \hfill prejšnjega člena podzaporedja
	\State $r[1] \gets a_1$ \hfill začetni pogoji
	\State $p^* \gets p_1 \gets 1$
	\State $q^* \gets q_1 \gets 0$
	\For{$i = 2, 3, \dots, n$}
		\State $r[i] \gets (-1)^i \cdot \infty$
			\hfill začetna vrednost za največjo možno dolžino
		\State $h \gets q^*$ \hfill računanje $p_i$
		\While{$r[h] \ge a_i$} \hfill iskanje največjega $q_j$
			\State $h \gets h - 2$ \hfill $h$ ostane sod
		\EndWhile
		\State $p_i \gets h + 1$
		\If{$a_i > r[p_i]$} \hfill posodobitev vrednosti za $p_i$
			\State $r[p_i] \gets a_i$
		\EndIf
		\If{$p_i > p^*$} \hfill posodobitev največje lihe dolžine
			\State $p^* \gets p_i$
		\EndIf
		\State $h \gets p^*$ \hfill računanje $q_i$
		\While{$r[h] \le a_i$} \hfill iskanje največjega $p_j$
			\State $h \gets h - 2$ \hfill $h$ ostane lih
		\EndWhile
		\State $q_i \gets h + 1$
		\If{$a_i < r[q_i]$} \hfill posodobitev vrednosti za $q_i$
			\State $r[q_i] \gets a_i$
		\EndIf
		\If{$q_i > q^*$} \hfill posodobitev največje sode dolžine
			\State $q^* \gets q_i$
		\EndIf
	\EndFor
	\State \Return $\max\{p^*, q^*\}$
\EndFunction
\end{algorithmic}
\end{small}
Tak algoritem ima časovno zahtevnost $O(n\ell^*)$,
kjer je $\ell^*$ dolžina najdaljšega oscilirajočega podzaporedja\footnote{%
Algoritmom, pri katerih je časovna zahtevnost odvisna od izhoda,
pravimo {\em output-sensitive} algoritmi.
V tem primeru velja $\ell^* = O(n)$,
tako da tudi v najslabšem primeru zahtevnost algoritma ne preseže $O(n^2)$.%
}
(tj., vrnjena vrednost).
Če bi želeli poiskati še samo oscilirajoče podzaporedje,
bi si morali tekom algoritma za vsako vrednost $p_i$ in $q_i$
beležiti še indeks predhodnega elementa v podzaporedju,
za vsako vrednost $r[h]$ pa še indeks $j$,
za katerega velja $a_j = r[h]$ in $p_j = h$ oziroma $q_j = h$.
\end{odgovor}
\end{naloga}
