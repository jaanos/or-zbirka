\begin{naloga}{Janoš Vidali}{Izpit OR 27.8.2019}
\begin{vprasanje}
V gozdu živi veverica, ki rada skače med drevesi.
Za vsak par dreves v gozdu je izračunala ve\-rjet\-nost,
da bo skok med njima uspešen
(verjetnosti so med seboj neodvisne).
Veverica se trenutno nahaja na drevesu $s$,
rada pa bi prišla do drevesa $t$ s takim zaporedjem skokov,
da bo verjetnost, da bodo vsi skoki uspešni, čim večja.

\begin{enumerate}[(a)]
\item Predstavi problem v jeziku grafov
in predlagaj čim bolj učinkovit algoritem za reševanje zgornjega problema.
Kak\-šna je njegova časovna zahtevnost?

\item S pomočjo zgornjega algoritma poišči ustrezno pot
od $s$ do $t$ na grafu s slike~\fig.
Če med dvema vozliščema ni povezave,
je verjetnost uspešnega skoka enaka $0$.
\end{enumerate}

\begin{slika}
\pgfslika
\podnaslov{Graf}
\end{slika}
\end{vprasanje}

\begin{odgovor}
\begin{enumerate}[(a)]
\item Gozd lahko predstavimo kot poln graf,
kjer vsakemu drevesu ustreza eno vozlišče,
povezava med vozliščema pa je utežena z verjetnostjo,
da bo skok med drevesoma uspešen.
Na takem grafu lahko izvedemo algoritem {\sc DijkstraProb}
iz naloge~\res[zaklad]
in tako dobimo najvarnejše poti od vozlišča $s$ do vseh ostalih vozlišč.
Časovna zahtevnost algoritma, kjer za prednostno vrsto uporabimo slovar,
je $O(n^2)$, kjer je $n$ število dreves.

Povezave s ceno $0$ lahko izločimo iz grafa.
Če je takih povezav veliko,
se morda bolj izplača v algoritmu za prednostno vrsto uporabiti kopico,
saj tako dosežemo časovno zahtevnost $O(m \log n)$,
kjer je $m$ število povezav v grafu
(tj., število parov dreves z neničelno verjetnostjo uspešnosti skoka).

\item Potek izvajanja zgoraj opisanega algoritma je prikazan v tabeli~\tab,
iz katere razberemo,
da je najvarnejša pot $s - a - d - f - g - e - t$,
z verjetnostjo, da vsi skoki uspejo, enako $0.729$.
\end{enumerate}
%
\begin{tabela}
$$
\begin{array}{cccccccccccc}
s & a & b & c & d & e & f & g & h & i & j & t \\ \hline
1 & 0 & 0 & 0 & 0 & 0 & 0 & 0 & 0 & 0 & 0 & 0 \\
* & 0.9_s &&& 0.8_s && 0.5_s && 0.8_s &&& \\
& * & 0.63_a && 0.9_a &&&&&&& \\
&&&& * & 0.54_d & 0.81_d &&&&& \\
&&&&&& * & 0.81_f && 0.81_f && \\
&&&&& 0.81_g && * &&& 0.81_g & 0.648_g \\
&& 0.729_e & 0.567_e && * &&&&&& 0.729_e \\
&&&&&&&&& * && \\
&&&&&&&&&& * & \\
&&&&&&&& * &&& \\
&& * & 0.5832_b &&&&&&&& \\
&&&&&&&&&&& * \\
\end{array}
$$
\podnaslov[\res{}(b)]{Potek izvajanja algoritma}
\end{tabela}
\end{odgovor}
\end{naloga}
