\begin{naloga}{Hillier, Lieberman}{\cite[Problem~19.3-2]{hl}}
\begin{vprasanje}
V trgovini vsak teden prodajo $600$ škatel toaletnega papirja.
Vsako naročilo stane $25 €$, za posamezno škatlo pa trgovina plača $3 €$.
Cena skladiščenja posamezne škatle je $0.05 €$ na teden.
\begin{enumerate}[(a)]
\item Denimo, da primanjkljaj ni dovoljen.
Kako pogosto naj trgovina naroča toaletni papir?
Kako velika naj bodo naročila?
\item Kaj pa, če dovolimo primanjkljaj,
ki nas stane $2 €$ na teden za vsako škatlo?
\end{enumerate}

\end{vprasanje}
\begin{odgovor}
\end{odgovor}
\end{naloga}
