\begin{naloga}{Janoš Vidali}{Izpit OR 27.6.2025}
\begin{vprasanje}
Med $m$ delavcev želimo razdeliti $n$ nalog
in pri tem doseči čim večjo učinkovitost.
Za $i$-tega delavca ($1 \le i \le m$) in $j$-to nalogo ($1 \le j \le n$)
poznamo čas $T_{ij}$,
v katerem bi $i$-ti delavec opravil $j$-to nalogo.
Vsak delavec lahko hkrati opravlja samo eno nalogo,
vsako posamezno nalogo pa mora v celoti opraviti en delavec.
Naloge želimo razdeliti tako,
da bo skupni čas izvajanja nalog čim manjši
-- tj., minimizirati želimo najdaljši čas,
ki ga za izvedbo dodeljenih nalog potrebuje posamezen delavec.

\begin{enumerate}[(a)]
\item Zapiši celoštevilski linearni program, ki modelira zgoraj opisani problem.

\item Svojemu celoštevilskemu linearnemu programu dodaj sledeča pogoja.
\begin{itemize}
\item Posamezen delavec lahko izvaja največ dve od nalog $a, b, c, d$.
\item Delavcu $u$ moramo dodeliti največje število nalog
(tj., vsaj toliko nalog kot ostalim delavcem -- ne glede na trajanje).
\end{itemize}
\end{enumerate}
\end{vprasanje}

\begin{odgovor}
Za $i$-tega delavca ($1 \le i \le m$) in $j$-to nalogo ($1 \le j \le n$)
bomo uvedli spremenljivko $x_{ij}$,
katere vrednost interpretiramo kot
$$
x_{ij} = \begin{cases}
1 & \text{$i$-temu delavcu dodelimo $j$-to nalogo, in} \\
0 & \text{sicer.}
\end{cases}
$$
Poleg tega bomo uvedli še spremenljivko $t$,
ki bo navzdol omejena s časom opravljanja nalog pri vsakem delavcu.

\begin{enumerate}[(a)]
\item Zapišimo celoštevilski linearni program.
\begin{alignat*}{2}
&& \min \ t &\quad \text{p.p.} \\
\forall i \in \{1, \dots, m\} \ \forall j \in \{1, \dots, n\}: &\ &
0 \le x_{ij} &\le 1, \quad x_{ij} \in \Z
\opis{Vsako nalogo dodelimo natanko enemu delavcu}
\forall j \in \{1, \dots, n\}: &\ &
\sum_{i=1}^m x_{ij} &= 1
\opis{Skupni čas izvajanja nalog posameznega delavca}
\forall i \in \{1, \dots, m\}: &\ & \sum_{j=1}^n T_{ij} x_{ij} &\le t
\end{alignat*}

\item Zapišimo še dodatni omejitvi.
\odstraniprostor
\begin{alignat*}{2}
\opis{Posamezen delavec lahko izvaja največ dve od nalog $a$, $b$, $c$ in $d$}
\forall i \in \{1, \dots, m\}: &\ & x_{ia} + x_{ib} + x_{ic} + x_{id} &\le 2
\opis{Delavcu $u$ dodelimo največ nalog}
\forall i \in \{1, \dots, m\}: &\ & \sum_{j=1}^n (x_{uj} - x_{ij}) &\ge 0
\end{alignat*}
\end{enumerate}
\end{odgovor}
\end{naloga}

