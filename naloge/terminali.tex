\begin{naloga}{Alen Orbanić}{Izpit OR 15.9.2010}
\begin{vprasanje}
Na sliki~\fig je podano omrežje enosmernih prehodov
med letališkimi terminali skupaj s prepustnostmi (v $1\,000$ ljudeh na uro).

\begin{enumerate}[(a)]
\item Načrtovalce zanima,
koliko največ potnikov na uro lahko preide od terminala $f$ do terminala $b$.
Predlagaj ustrezen algoritem in ga izvedi na grafu.

\item V izrednih razmerah dva terminala proglasijo za vstopna
(na njih letala le pristajajo),
ostale terminale pa za izstopne (letala le vzletajo).
Tako morajo potniki nujno prehajati od vstopnih k izstopnim terminalom.
Strateška odločitev je, da je vstopni terminal $f$,
zaradi redundance pa želijo, da bi bila vstopna terminala dva.
Ostali terminali so torej izstopni.
Katerega izmed terminalov (poleg $f$) se splača še proglasiti za vstopnega,
da bo pretočnost med vstopnimi in izstopnimi terminali kar največja?
\end{enumerate}

\begin{slika}
\pgfslika
\podnaslov{Omrežje}
\end{slika}
\end{vprasanje}

\begin{odgovor}
\begin{enumerate}[(a)]
\item S Ford-Fulkersonovim algoritmom poiščemo maksimalni $(f, b)$-tok
na omrež\-ju s slike~\fig.
Postopek reševanja je prikazan na slikah~%
\fig[terminali-resitev1],~\fig[terminali-resitev2] in~\fig[terminali-resitev3].
Dobimo maksimalni pretok $8+3 = 5+3+3 = 11$.

\item Zadostuje, da poiščemo tak par terminalov,
ki maksimizira vsoto kapacitet izstopnih povezav
(brez morebitnih povezav med izbranima terminaloma).
Če je eden od izbranih vstopnih terminalov $f$,
potem največjo skupno kapaciteto $20$ dosežemo,
če za drugi vstopni terminal izberemo $c$ ali $e$.
\end{enumerate}

\begin{slika}
\pgfslika[terminali-resitev1]
\podnaslov[\res{}(a)]{Prvi korak}
\end{slika}
\begin{slika}
\pgfslika[terminali-resitev2]
\podnaslov[\res{}(a)]{Drugi korak}
\end{slika}
\begin{slika}
\pgfslika[terminali-resitev3]
\podnaslov[\res{}(a)]{Maksimalni pretok in minimalni prerez}
\end{slika}
\end{odgovor}
\end{naloga}
