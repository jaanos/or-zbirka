\begin{naloga}{Janoš Vidali}{Izpit OR 4.6.2020}
\begin{vprasanje}
Dana sta neutežen neusmerjen graf $G = (V, E)$
z $n$ vozlišči in $m$ povezavami
ter vozlišče $s \in V$.
Za vsako vozlišče $v \in V$ nas zanima,
koliko najkrajših poti od $s$ do $v$ je v grafu $G$
(tj., koliko je takšnih poti od $s$ do $v$,
katerih dolžina je enaka razdalji med $s$ in $v$ v grafu $G$).

\begin{enumerate}[(a)]
\item Čim bolj natančno opiši algoritem,
ki reši zgornji problem v času $O(m)$.

\item Uporabi svoj algoritem,
da za vsako vozlišče v grafu s slike~\fig
poiščeš število najkrajših poti od vozlišča $i$.
Natančno zabeleži, kaj se zgodi v vsakem koraku algoritma.
\end{enumerate}

\begin{slika}
\pgfslika
\podnaslov{Graf}
\end{slika}
\end{vprasanje}

\begin{odgovor}
\begin{enumerate}[(a)]
\item Pomagali si bomo z iskanjem v širino.
Zapišimo algoritem, ki kliče funkcijo {\sc bfs} iz naloge~\res[bfs].
\begin{small}
\begin{algorithmic}
\Function{ŠteviloNajkrajših}{$G = (V, E), s$}
	\State {\sl število} $\gets$ slovar s ključi $v \in V$
                                 in vrednostmi $0$
	\State {\sl globina} $\gets$ slovar s ključi $v \in V$
                                 in vrednostmi $\infty$
    \State {\sl število}$[s] \gets 1$
    \State {\sl globina}$[s] \gets 0$
	\Function{Visit}{$u, v$}
        \For{$w \in \Adj(G, u)$}
            \If{{\sl globina}$[w] = \infty$}
                \State {\sl globina}$[w] \gets$ {\sl globina}$[u] + 1$
            \EndIf
            \If{{\sl globina}$[w] =$ {\sl globina}$[u] + 1$}
                \State {\sl število}$[w] \gets$ {\sl število}$[w] +$ {\sl število}$[u]$
            \EndIf
        \EndFor
	\EndFunction
	\State {\sc bfs}$(G, \{s\}, \visit)$
	\State \Return {\sl globina}
\EndFunction
\end{algorithmic}
\end{small}
Vidimo,
da s klicanjem funkcije {\sc Visit} na vsakem vozlišču
vsako povezavo obravnavamo največ dvakrat,
tako da je časovna zahtevnost algoritma $O(m)$.

\item Potek izvajanja algoritma je prikazan v tabeli~\tab,
pri čemer je bil upoštevan abecedni vrstni red
obiskovanja sosedov posameznega vozlišča.
Končni rezultat lahko preberemo iz zadnje vrstice tabele.
\end{enumerate}
%
\begin{tabela}
$$
\begin{array}{c|ccccccccc|ccccccccc}
& \multicolumn{9}{c|}{\text{\sl globina}}
& \multicolumn{9}{c}{\text{\sl število}} \\
u & a & b & c & d & e & f & g & h & i & a & b & c & d & e & f & g & h & i \\
\hline
& \infty & \infty & \infty & \infty & \infty & \infty & \infty & \infty & 0
& 0 & 0 & 0 & 0 & 0 & 0 & 0 & 0 & 1 \\
i & \infty & \infty & \infty & \infty & \infty & 1 & 1 & 1 & 0
& 0 & 0 & 0 & 0 & 0 & 1 & 1 & 1 & 1 \\
f & \infty & \infty & 2 & \infty & 2 & 1 & 1 & 1 & 0
& 0 & 0 & 1 & 0 & 1 & 1 & 1 & 1 & 1 \\
g & \infty & \infty & 2 & 2 & 2 & 1 & 1 & 1 & 0
& 0 & 0 & 2 & 1 & 1 & 1 & 1 & 1 & 1 \\
h & \infty & \infty & 2 & 2 & 2 & 1 & 1 & 1 & 0
& 0 & 0 & 2 & 1 & 2 & 1 & 1 & 1 & 1 \\
c & 3 & 3 & 2 & 2 & 2 & 1 & 1 & 1 & 0
& 2 & 2 & 2 & 1 & 2 & 1 & 1 & 1 & 1 \\
e & 3 & 3 & 2 & 2 & 2 & 1 & 1 & 1 & 0
& 2 & 4 & 2 & 1 & 2 & 1 & 1 & 1 & 1 \\
d & 3 & 3 & 2 & 2 & 2 & 1 & 1 & 1 & 0
& 3 & 4 & 2 & 1 & 2 & 1 & 1 & 1 & 1 \\
a & 3 & 3 & 2 & 2 & 2 & 1 & 1 & 1 & 0
& 3 & 4 & 2 & 1 & 2 & 1 & 1 & 1 & 1 \\
b & 3 & 3 & 2 & 2 & 2 & 1 & 1 & 1 & 0
& 3 & 4 & 2 & 1 & 2 & 1 & 1 & 1 & 1
\end{array}
$$
\podnaslov[\res{}(b)]{Potek izvajanja algoritma}
\end{tabela}
\end{odgovor}
\end{naloga}
