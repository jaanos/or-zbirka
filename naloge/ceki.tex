\begin{naloga}{Boštjan Gabrovšek}{Kolokvij OR 24.1.2011}
\begin{vprasanje}
Podjetje dobiva po pošti čeke iz celotne Evrope.
Čas potovanja čeka je odvisen od kraja, od koder je bil ček poslan,
in od kraja, kamor ček potuje.
Na primer, ček, poslan iz Vzhodne Evrope v Berlin, v povprečju potuje 5 dni
-- podjetje mora torej čakati 5 dni,
preden lahko ček unovči in razpolaga z denarjem.
Iz Severne Evrope je v povprečju dnevno poslanih za $70.000 €$ čekov,
iz Zahodne Evrope za $50.000 €$, iz Vzhodne $60.000 €$,
iz Južne pa $40.000 €$.
Podjetje želi obračunavati čeke kar se da hitro,
saj izgubi $20\%$ vrednosti čeka na letni ravni.
Podjetje lahko postavi podružnice (in s tem poštne predale)
v Londonu, Berlinu, Budimpešti in/ali Madridu.
Strošek vzdrževanja ene podružnice znaša $50.000 €$ letno.
Časi potovanja čekov so razvidni iz spodnje tabele:
\begin{center}
\begin{tabular}{c|cccc}
& London & Berlin & Budimpešta & Madrid \\ \hline
Severna Evropa & 2 & 6 & 8 & 6 \\
Zahodna Evropa & 6 & 2 & 5 & 5 \\
Vzhodna Evropa & 8 & 5 & 2 & 5 \\
Južna Evropa   & 8 & 5 & 5 & 2
\end{tabular}
\end{center}
Primer: če Severna Evropa pošilja čeke v Budimpešto,
bo v obtoku v pov\-preč\-ju za $560.000 €$ ($= 8 \cdot 70.000 €$) čekov,
kar z $20\%$ izgubo pomeni $112.000 €$ izgube na letni ravni.
Zanima nas, kje naj podjetje postavi podružnice, da bodo stroški čim manjši.
Zastavi nalogo kot celoštevilski linearni program.
\end{vprasanje}
\begin{odgovor}
\end{odgovor}
\end{naloga}
