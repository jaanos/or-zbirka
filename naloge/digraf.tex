\begin{naloga}{?}{Vaje OR 20.4.2016}
\begin{vprasanje}
Kako moramo spremeniti strukturi
iz nalog~\nal[matgraf] in~\nal[sosgraf],
da bosta predstavljali digrafe?
\end{vprasanje}

\begin{odgovor}
Strukturo iz naloge~\res[matgraf] spremenimo tako,
da lahko hrani poljubno dvojiško matriko (torej ne nujno simetrične).
To naredimo tako, da popravimo metodi $\AddEdge$ in $\DelEdge$,
in sicer tako, da spreminjamo samo vrednost $G.A[u][v]$.
Časovne zahtevnosti metod in prostorska zahtevnost strukture
se pri tem ne spremenijo.

Strukturo iz naloge~\res[matgraf] spremenimo tako,
da relacija, določena s slo\-va\-rjem $G.E$, ni nujno simetrična.
To naredimo tako, da popravimo metodi $\AddEdge$ in $\DelEdge$,
in sicer tako, da spreminjamo samo množico $G.E[u]$.
Tudi tukaj se časovne in prostorske zahtevnosti ne spremenijo.

V digrafih lahko ločimo sosednost po vhodnih in izhodnih povezavah.
V ta namen bi lahko za vsako imeli svojo metodo.
Pri matrični predstavitvi lahko samo zamenjamo indeksa v metodi $\Adj$,
če pa želimo ohraniti časovno zahtevnost pri predstavitvi s seznami sosedov,
pa bi morali poleg slovarja za izhodne povezave
hraniti še slovar za vhodne povezave.
\end{odgovor}
\end{naloga}
