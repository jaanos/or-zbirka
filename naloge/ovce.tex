\begin{naloga}{Blaž Jelenc}{Izpit OR 24.6.2014}
\begin{vprasanje}
Kmet redi ovce, pri čemer se vsako leto njihovo število podvoji:
če ima na začetku leta $N$ ovac, jih ima ob koncu leta $2N$.
Denimo, da ima kmet $k_0$ ovac na začetku leta $0$.
Ob prehodu iz leta $i-1$ v leto $i$ se vsakič odloči,
koliko ovac bo prodal, pri čemer je profit enak $p_i$ za vsako prodano ovco.
Ob začetku leta $n$ hoče imeti prodane vse ovce.

\begin{enumerate}[(a)]
\item S pomočjo dinamičnega programiranja opiši algoritem,
ki za dane podatke $n, k_0$ in $p_1, p_2, \dots, p_n$
izračuna maksimalni profit, ki ga lahko kmet doseže po $n$ letih.

\item Reši nalogo za $n = 3$, $k_0 = 2$,
$p_1 = 100 €$, $p_2 = 130 €$ in $p_3 = 120 €$.
\end{enumerate}
\end{vprasanje}
\begin{odgovor}
\end{odgovor}
\end{naloga}
