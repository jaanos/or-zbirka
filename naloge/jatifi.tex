\begin{naloga}{Alen Orbanić}{Kolokvij OR 26.1.2010}
\begin{vprasanje}
Jatifi Bajrami se ukvarja s preprodajo pomaranč.
Naslednja sezona je razdeljena na $n$ mesecev.
Za vsak mesec so znana predvidena povpraševanja $r_1, \dots, r_n$.
Nabavni stroški v mesecu $i$ so sestavljeni iz fiksnih stroškov $K_i$
ter stroškov za nabavo sadja.
Na začetku meseca $i$ stane škatla pomaranč $c_i$.
Če Jatifi takrat kupi $m$ škatel,
je za nabavo sadja torej moral odšteti $K_i + mc_i$ denarnih enot.
Če se Jatifi zaradi nižje cene v nekem mesecu
odloči nakupiti vnaprej za več mesecev,
mora plačati skladiščenje za škatle pomaranč za preostale mesece,
pri čemer je cena skladiščenja za škatlo v mesecu $i$ enaka $h_i$.
Jatifi kupuje izključno, ko ve, da so mu zaloge popolnoma pošle,
in od tega ne odstopa, saj se mu zdi škoda,
da bi v skladišču hranil stare pomaranče.
Pomaranče prodaja ves čas po isti ceni.
Jatifi bi rad maksimiziral dobiček in zadovoljil celotno povpraševanje.

\begin{enumerate}[(a)]
\item Predlagaj rešitev problema s pomočjo dinamičnega programiranja.
Pri tem so vhodni podatki $n$ mesecev, $r = (r_i)_{i=1}^n$,
$K = (K_i)_{i=1}^n$, $c = (c_i)_{i=1}^n$ ter $h = (h_i)_{i=1}^n$.
Kakšno časovno zahtevnost ima tvoj algoritem?

\item Naj bo $n = 4$, $r = (20, 40, 15, 10)$, $K = (2, 3, 4, 2)$,
$c = (3, 4, 3, 4)$ ter $h = (1, 1, 1, 2)$.
Poišči optimalno strategijo kupovanja pomaranč.
\end{enumerate}
\end{vprasanje}

\begin{odgovor}
Ker je predvideno povpraševanje znano vnaprej,
bo zadostovalo, da minimiziramo stroške,
saj je prodajna cena fiksna.

\begin{enumerate}[(a)]
\item Naj bodo $s_{ij}$ najmanjši skupni stroški,
ki jih lahko ima Jatifi za nabavo pomaranč do $j$-tega meseca,
če zadnjič kupuje v $i$-tem mesecu.
Poleg tega naj bo $H_{ij}$ skupna cena skladiščenja škatle,
kupljene v mesecu $i$ in prodane v mesecu $j$.
Zapišimo začetne pogoje in rekurzivne enačbe.
\begin{align*}
c_0 = h_0 &= \infty \\
s_{00} &= 0 \\
H_{jj} &= 0 & (0 \le j \le n) \\
s_{jj} &= K_j + r_j c_j + \min\set{s_{i,j-1}}{0 \le i \le j-1} & (1 \le j \le n) \\
H_{ij} &= h_{j-1} + H_{i,j-1} & (0 \le i < j \le n) \\
s_{ij} &= s_{i,j-1} + r_j (c_i + H_{ij}) & (0 \le i < j \le n)
\end{align*}
Vrednosti $H_{ij}$ in $s_{ij}$ računamo
v leksikografskem vrstnem redu glede na in\-dek\-sa $(i, j)$,
pri čemer upoštevamo $0 \cdot \infty = 0$
(na ta način se izognemo plačilu fiks\-nih stroškov
za morebitne začetne mesece brez povpraševanja).
Najmanjše skupne stroške za celotno obdobje
dobimo kot $s^* = \min\set{s_{in}}{0 \le i \le n}$.

Algoritem izračuna $O(n^2)$ vrednosti,
pri čemer za izračun vrednosti $s_{jj}$ ($1 \le j \le n$) porabi $O(j)$ korakov,
ostale vrednosti pa izračuna v konstantnem času.
Časovna zahtevnost algoritma je tako $O(n^2)$.

\item Ker velja $r_1 > 0$, imamo $H_{0j} = s_{0j} = \infty$ ($1 \le j \le n$).
Izračunajmo še vrednosti $H_{ij}$ in $s_{ij}$ ($1 \le i \le j \le n$).
\begin{alignat*}{3}
H_{11} &= 0 &\qquad s_{11} &= 2 + 20 \cdot 3 + 0 &&= 62 \\
H_{12} &= 1 &\qquad s_{12} &= 62 + 40 \cdot (3 + 1) &&= 222 \\
H_{13} &= 2 &\qquad s_{13} &= 222 + 15 \cdot (3 + 2) &&= 297 \\
H_{14} &= 3 &\qquad s_{14} &= 297 + 10 \cdot (3 + 3) &&= 357 \\
H_{22} &= 0 &\qquad s_{22} &= 3 + 40 \cdot 4 + 62 &&= 225 \\
H_{23} &= 1 &\qquad s_{23} &= 225 + 15 \cdot (4 + 1) &&= 300 \\
H_{24} &= 2 &\qquad s_{24} &= 300 + 10 \cdot (4 + 2) &&= 360 \\
H_{33} &= 0 &\qquad s_{33} &= 4 + 15 \cdot 3 + \min\{\underline{222}, 225\} &&= 271 \\
H_{34} &= 1 &\qquad s_{34} &= 271 + 10 \cdot (3 + 1) &&= 311 \\
H_{44} &= 0 &\qquad s_{44} &= 2 + 10 \cdot 4 + \min\{297, 300, \underline{271}\} &&= 313
\end{alignat*}
Najmanjši skupni stroški za celotno obdobje so torej $s^* = 311$.
Poiščimo še optimalno strategijo kupovanja pomaranč.
\begin{align*}
s^* = s_{34} &= K_3 + r_3 c_3 + r_4 (c_3 + H_{34}) + s_{12} & \text{kupi v 3.~mesecu za 3.~in 4.~mesec} \\
s_{12} &= K_1 + r_1 c_1 + r_2 (c_1 + H_{12}) + s_{00} & \text{kupi v 1.~mesecu za 1.~in 2.~mesec}
\end{align*}
\end{enumerate}
\end{odgovor}
\end{naloga}
