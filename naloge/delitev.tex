\begin{naloga}{Janoš Vidali}{Izpit OR 24.8.2021}
\begin{vprasanje}
Imamo $n$ predmetov s cenami $p_i$ ($1 \le i \le n$),
ki jih želimo razporediti v $m$ košar tako,
da bo delitev čim bolj enakomerna
-- minimizirati želimo največjo razliko
v skupnih cenah predmetov v dveh košarah.

\begin{enumerate}[(a)]
\item Zapiši celoštevilski linearni program, ki modelira zgornji problem.
Lahko predpostaviš,
da so cene $p_i$ ($1 \le i \le n$) nenegativna cela števila.
\namig{vpelji spremenljivko,
ki bo omejena z razlikami skupnih cen v dveh košarah.
Pazi na to,
da absolutna vrednost ni linearna funkcija
in se ne sme uporabljati v linearnem programu!}

\item Zgornjemu celoštevilskemu linearnemu programu dodaj omejitve,
ki modelirajo sledeče dodatne pogoje.
\begin{itemize}
\item Predmetov $a$ in $b$ ne smemo dodeliti v isto košaro.
\item V košaro, kamor dodelimo predmet $c$,
ne smemo dodeliti nobenega dražjega predmeta.
\item Če predmeta $d$ in $e$ dodelimo v isto košaro,
moramo tja dodeliti tudi predmet $f$.
\end{itemize}
\end{enumerate}
\end{vprasanje}

\begin{odgovor}
Za $i$-ti predmet ($1 \le i \le n$) in $j$-to košaro ($1 \le j \le m$)
bomo uvedli spremenljivko $x_{ij}$,
katere vrednost interpretiramo kot
$$
x_{ij} = \begin{cases}
1 & \text{$i$-ti predmet dodelimo v $j$-to košaro, in} \\
0 & \text{sicer.}
\end{cases}
$$
Poleg tega bomo uvedli še spremenljivko $t$,
ki bo navzdol omejena z največjo razliko skupnih cen v dveh košarah.

\begin{enumerate}[(a)]
\item Zapišimo celoštevilski linearni program.
\begin{alignat*}{2}
&& \min \ t &\quad \text{p.p.} \\
\forall i \in \{1, \dots, n\} \ \forall j \in \{1, \dots, m\}: &\ &
0 \le x_{ij} &\le 1, \quad x_{ij} \in \Z
\opis{Vsak predmet dodelimo v natanko eno košaro}
\forall i \in \{1, \dots, n\}: &\ &
\sum_{j=1}^m x_{ij} &= 1
\opis{Razlika skupnih cen dveh košar}
\forall j, k \in \{1, \dots, m\}: &\ & \sum_{i=1}^n p_i (x_{ij} - x_{ik}) &\le t
\end{alignat*}

\item Zapišimo še dodatne omejitve.
\odstraniprostor
\begin{alignat*}{2}
\opis{$a$ in $b$ nista v isti košari}
\forall j \in \{1, \dots, m\}: &\ & x_{aj} + x_{bj} &\le 1
\opis{$c$ je najdražji v svoji košari}
\forall i \in \{1, \dots, n\} \ \forall j \in \{1, \dots, m\}: &\ & p_i (x_{ij} + x_{cj} - 1) &\le p_c
\opis{Če sta $d$ in $e$ skupaj, je zraven tudi $f$}
\forall j \in \{1, \dots, m\}: &\ & x_{dj} + x_{ej} &\le x_{fj} + 1
\end{alignat*}
\end{enumerate}
\end{odgovor}
\end{naloga}

