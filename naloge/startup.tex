\begin{naloga}{Janoš Vidali}{Kolokvij OR 31.3.2022}
\begin{vprasanje}
Skupina $m$ vlagateljev pripravlja ponudbe
za prevzem deležev $n$ zagonskih podjetij.
Naj bo $k_i$ skupni kapital, ki ga ima $i$-ti vlagatelj na voljo,
in $c_{ij}$ znesek, ki ga je pripravljen vložiti v $j$-to podjetje,
da pridobi delež $p_{ij} > 0$ ($1 \le i \le m$, $1 \le j \le n$)
-- odkupil bo natanko tak delež po ponujeni ceni, ali pa ničesar.
Poleg tega naj bo $q_j$ ($1 \le j \le n$) največji delež $j$-tega podjetja,
ki ga je vlagateljem pripravljen odprodati njegov lastnik
(deleže podjetja lahko odproda nobenemu, enemu ali več vlagateljem).
Vlagatelji želijo pripraviti tako ponudbo,
ki bo upoštevala njihove želje in omejitve
ter bo maksimizirala najmanjše število podjetij,
katerih delež bo odkupil posamezen vlagatelj
(če delež nekega podjetja odkupi več vlagateljev,
se pri vsakem šteje enkrat)
-- tj., želimo, da tisti vlagatelj,
ki odkupi delež v najmanjšem številu podjetij,
to stori pri čim večjem številu podjetij.

\begin{enumerate}[(a)]
\item Zapiši celoštevilski linearni program, ki modelira zgornji problem.
\namig{maksimiziraj spremenljivko, ki je navzgor omejena s številom podjetij,
katerih delež odkupi posamezen vlagatelj.}

\item Vlagatelji so pripravili svojo ponudbo,
a lastniki zagonskih podjetij z njo niso najbolj zadovoljni,
zato bi radi sestavili protiponudbo, ki bo upoštevala tudi njihove pogoje.
Lastnik $j$-tega podjetja ($1 \le j \le n$)
ocenjuje njegovo vrednost na $v_j$ in tako ne bo sprejel ponudbe,
s katero bi delež podjetja prodal pod sorazmerno ceno
(tj., če želijo vlagatelji skupaj odkupiti delež $r$,
potem morajo skupaj plačati vsaj $r v_j$);
prav tako ne bo sprejel ponudbe,
s katero ne bi prejel skupnega zneska vsaj $u_j$
(lahko pa ne odproda nobenega deleža in tako ostane brez dobička).

Zgornji celoštevilski linearni program dopolni z omejitvami,
ki bodo upoštevale tudi želje last\-ni\-kov zagonskih podjetij.
\end{enumerate}
\end{vprasanje}

\begin{odgovor}
Za $i$-tega vlagatelja in $j$-to zagonsko podjetje
($1 \le i \le m$, $1 \le j \le n$)
bomo uvedli spremenljivko $x_{ij}$,
katere vrednost interpretiramo kot
$$
x_{ij} = \begin{cases}
1 & \text{$i$-ti vlagatelj kupi delež v $j$-tem podjetju, in} \\
0 & \text{sicer.}
\end{cases}
$$

\begin{enumerate}[(a)]
\item Zapišimo celoštevilski linearni program.
\begin{alignat*}{3}
&& \max \ t &&&\quad \text{p.p.} \\
\forall i \in \{1, \dots, m\} \ \forall j \in \{1, \dots, n\}: &\ &
0 \le x_{ij} &\le 1, &&\quad x_{ij} \in \Z
\opis{Omejitev kapitala za vsakega vlagatelja}
\forall i \in \{1, \dots, m\}: &\ &\sum_{j=1}^n c_{ij} x_{ij} &\le k_i
\opis{Omejitev deleža, ki ga proda vsak lastnik}
\forall j \in \{1, \dots, n\}: &\ & \sum_{i=1}^m p_{ij} x_{ij} &\le q_j
\opis{Spodnja meja za število podjetij, v katera vloži vsak vlagatelj}
\forall i \in \{1, \dots, m\}: &\ & \sum_{j=1}^n x_{ij} &\ge t
\end{alignat*}

\item Za $j$-to podjetje ($1 \le j \le n$) bomo uvedli še spremenljivko $y_j$,
katere vrednost interpretiramo kot
$$
y_j = \begin{cases}
1 & \text{vsaj en vlagatelj kupi delež v $j$-tem podjetju, in} \\
0 & \text{sicer.}
\end{cases}
$$
Zapišimo še dodatne omejitve.
\odstraniprostor
\begin{alignat*}{2}
\opis{Nobeno podjetje ne proda deleža pod cenitvijo}
\forall j \in \{1, \dots, n\}: &\ & v_j \sum_{i=1}^m p_{ij} x_{ij} &\le \sum_{i=1}^m c_{ij} x_{ij}
\opis{Lastnik vsakega podjetja prejme želeni znesek, ali pa ne proda ničesar}
\forall j \in \{1, \dots, n\}: &\ & \sum_{i=1}^m c_{ij} x_{ij} &\ge u_j y_j \\
\forall i \in \{1, \dots, m\} \ \forall j \in \{1, \dots, n\}: &\ & y_j &\ge x_{ij}
\end{alignat*}
\end{enumerate}
\end{odgovor}
\end{naloga}

